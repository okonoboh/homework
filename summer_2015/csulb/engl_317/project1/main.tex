\documentclass[9pt]{article}

\usepackage{amssymb}
\usepackage{amsmath}
\usepackage{amsfonts}
\usepackage{comment}
\usepackage{fancyhdr}
\usepackage{mathrsfs}
\usepackage{enumitem}
\usepackage{setspace}
\usepackage{framed}
\onehalfspacing

\usepackage{tikz}
\usepackage{tabularx}

\voffset = -50pt
%\textheight = 700pt
\addtolength{\textwidth}{60pt}
\addtolength{\evensidemargin}{-30pt}
\addtolength{\oddsidemargin}{-30pt}
%\setlength{\headheight}{44pt}

\newcommand{\qed}{\hfill \ensuremath{\Box}}


\newcommand*\circled[1]{\tikz[baseline=(char.base)]{
            \node[shape=circle,draw,inner sep=2pt] (char) {#1};}}

\newcommand{\Z}{\mathbb{Z}}
\newcommand{\I}{\mathbb{I}}
\newcommand{\M}{\mathbb{M}}
\newcommand{\Q}{\mathbb{Q}}
\newcommand{\R}{\mathbb{R}}
\newcommand{\C}{\mathbb{C}}
\newcommand{\D}{\displaystyle}
%\setcounter{section}{-1}

\begin{document}
\begin{center}
California State University, Long Beach. Summer 2015.
\end{center}
\begin{framed}
\begin{tabularx}{\textwidth}{@{}llll@{}}
\textbf{Course}: & ENGL 317, Technical Communication & \textbf{Classroom}: & LA2-202 \\ \\
\textbf{Instructor}: & Ilan Mitchell-Smith  & \textbf{Office} : & MHB - 506 \\ \\
\textbf{Class Period}: & Tue/Thur 06:00 P.M. - 09:45 P.M. \\ \\
\textbf{Office Hours}: & Tue/Thur \\
& 04:00 P.M. - 05:00 P.M., \\
& 10:00 P.M. - 10:30 P.M., and \\
& by appointment \\ \\
\textbf{Email}: & Ilan.MitchellSmith@csulb.edu \\ \\
\textbf{Textbook}: &\multicolumn{3}{l@{}}{Markel, Mike. Practical Strategies for Techincal Communication} \\
&\multicolumn{3}{l@{}}{1st edition. Boston: Bedford/St. Martin's, 2013. } \\
&\multicolumn{3}{l@{}}{ISBN : 978-1-4576-0940-4 (paperback),}  \\
\textbf{Other} \\
\textbf{Resources}: & \multicolumn{3}{l@{}}{Online Dictionary: http://www.dictionary.com} \\
& \multicolumn{3}{l@{}}{Online Writing Lab: http://owl.english.purdue.edu}
\end{tabularx}
\end{framed}
\begin{enumerate}
%%%%%%%%%%%%%%%%%%%%%%%%%%%%%%%%%%%%%Prob1%%%%%%%%%%%%%%%%%%%%%%%%%%%%%%%%%%%%%%
   \item \textbf{Course Description.} The technical writer is one who is aware
         of the difference between content (the information that an individual
         or a business is trying to convey) and form (the actual words, layout,
         formatting, tone, style, etc., that might be chosen to best convey that
         content). \\
         
         Technical writing skills are those that allow a writer to translate
         content concisely and clearly for a specific user or reader. Technical
         documents range from emails, memos, business letters, brochures, and
         newsletters, to manuals, proposals, and analytical reports. Documents
         such as these might explain a problem, describe a product, report on
         an experiment, or propose a project; additionally, they might need to
         do so across the discipline-specific jargon of a number of professions.
         
   \item \textbf{a) Grade Distribution.} 
         \begin{center}
            \begin{tabular}{@{}|c|c|@{}}
               \hline Project 1 & 20\% \\ \hline
               Project 2 & 20\% \\ \hline
               Project 3 & 20\% \\ \hline
               Project 4 & 20\% \\ \hline
               Participation & 10\% \\ \hline
               Peer critiques, quizzes, and in class-work & 10\% \\ \hline
            \end{tabular}
         \end{center}
         
         \textbf{b) Grading Scale.} 
         \begin{center}
            \begin{tabular}{@{}|c|c|@{}} \hline
               90\%-100\% & A \\ \hline
               80\%-89\% & B \\ \hline
               70\%-79\% & C \\ \hline
               60\%-69\% & D \\ \hline
               59\% and below & F \\ \hline
            \end{tabular}
         \end{center}          
   \item \textbf{Participation.} Active participation is crucial in this class.
         A respectful exchange of ideas in this classroom offers each of you a
         greater opportunity to become better thinkers, speakers, writers, and
         listeners. But please keep in mind that active participants come to
         class every day, on time, with their cell phone turned off (or on
         vibrate).
   \item \textbf{Attendance and Absences.} Attendance in this course is
         mandatory. You must attend class meetings regularly and be on time. If
         you are late, it is your responsibility to speak with me after class
         to make sure you have been marked present. If you know in advance that
         you will be absent or late, please notify me by e-mail before class.
         Note that if you miss more than 1 class meeting---without prior
         arrangement---your final grade will be lowered by ten full points (one
         full letter grade). In other words, you cannot receive an ``A" if you
         miss 3 classes. FOr every two absences beyond the first, your final
         grade will be lowered another 10 points. You are responsible for all
         assignments, whether you are absent or not.
         
   \item \textbf{Late Work / Make-up Policy.} For each class period that an essay
         assignment is late, a whole letter grade will be deducted from the
         assignment's score, unless you have a valid excuse, as defined by
         university policy. Writing assignments three days beyond their due date
         will not be accepted. If for a serious and compelling reason you are
         unable to meet a deadline, please meet with me in advance so that we
         can work out some sort of resolution.
         
   \item \textbf{Withdrawal Policy.} It is your responsibility to withdraw from
         this course if you are unable to complete its requirements. If you
         choose not to complete this course or have accumulated excessive
         absences, you should officially withdraw as soon as possible. For more
         information and exact dates on the university withdrawal policy, see
         the My CSULB.
   \item \textbf{Counseling and Psychological Services.} This is located in
         Brotman Hall, Room 226, and available by phone all day at (562)985-4001.
   \item \textbf{Student Health Services.} If you wish to visit a doctor
         on-campus or need a vaccine, the Student Health Services offers such
         services to registered students. Feel free to contact them at
         (562)985-4771.
   \item \textbf{Accommodations.} If you have a verified disability that
         requires special accommodations, please see me as soon as possible to
         to best meet your needs. Additionally, for information regarding the
         services available to you as a student with disabilities, you should
         contact Disabled Student Services in Brotman Hall or call them at
         (562)985-5401
   \item \textbf{Academic Integrity.} Cheating or plagiarism will be taken 
         seriously and will result in a course grade of ``F". Depending on the
         gravity of the offense, the matter may be forwarded to the Office for
         Judicial Affairs with recommendations of probation, suspension, or
         expulsion. Plagiarism is defined as the act of using the ideas or work
         of another person or persons as if they were your own without giving
         credit to the source. Such an act is not plagiarism if it is
         ascertained that the ideas were arrived at through independent
         reasoning or logic or where the thought or idea is common knowledge. \\
         
         Examples of plagiarism include, but are not limited to, the following:
         the submission of a work, either in part or in whole, completed by
         another; failure to give credit for ideas, statements, facts or
         conclusions which rightfully belong to another; in written work,
         failure to use quotation marks when quoting directly from another
         whether it be a paragraph, a sentence, or even a part thereof; close
         and lengthy paraphrasing of another's writing or programming. If you
         are in doubt about the extent of acceptable paraphrasing, then you
         should consult the instructor. Acknowledgement of an original author or
         source must be made through appropriate referencing and citations.
         
         Please see the ``Academic
         Action" section of the course catalog: \\
         \verb|http://www.csulb.edu/divisions/aa/catalog/current/academic_information/|\\
         \verb|cheating_plagiarism.html|
   \item \textbf{Educational Use of Student Papers.} Occasionally, I retain copies
   of student work completed for this course in order to share it with this class
   or future sections of this course. All selections of student work used in this
   way will be anonymous. If you absolutely object, please let me know so I may
   honor your request.
         
\end{enumerate}
\end{document}
