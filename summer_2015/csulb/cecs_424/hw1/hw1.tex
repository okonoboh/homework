\documentclass[9pt]{article}

\usepackage{amssymb}
\usepackage{amsmath}
\usepackage{amsfonts}
\usepackage{comment}
\usepackage{fancyhdr}
\usepackage{mathrsfs}
\usepackage{enumitem}


\usepackage{tikz}

\voffset = -50pt
%\textheight = 700pt
\addtolength{\textwidth}{60pt}
\addtolength{\evensidemargin}{-30pt}
\addtolength{\oddsidemargin}{-30pt}
%\setlength{\headheight}{44pt}

\pagestyle{fancy}
\fancyhf{} % clear all fields
\fancyhead[R]{%
  \scshape
  \begin{tabular}[t]{@{}r@{}}
  CECS 424, Summer 2015\\Section 1 (11171)\\
  HW \#1, DUE: 2015, July 13
  \end{tabular}}
\fancyhead[L]{%
  \scshape
  \begin{tabular}[t]{@{}r@{}}
  JOSEPH OKONOBOH\\Computer Science\\Cal State Long Beach
  \end{tabular}}
\fancyfoot[C]{\thepage}

\newcommand{\qed}{\hfill \ensuremath{\Box}}


\newcommand*\circled[1]{\tikz[baseline=(char.base)]{
            \node[shape=circle,draw,inner sep=2pt] (char) {#1};}}

\newcommand{\Z}{\mathbb{Z}}
\newcommand{\I}{\mathbb{I}}
\newcommand{\M}{\mathbb{M}}
\newcommand{\Q}{\mathbb{Q}}
\newcommand{\R}{\mathbb{R}}
\newcommand{\C}{\mathbb{C}}
\newcommand{\D}{\displaystyle}
%\setcounter{section}{-1}

\begin{document}
\begin{enumerate}
%%%%%%%%%%%%%%%%%%%%%%%%%%%%%%%%%%%%%Prob1%%%%%%%%%%%%%%%%%%%%%%%%%%%%%%%%%%%%%%
   \item Consider the veracity or falsehood of each of the following statements.
         For bonus, argue for those that you believe are true while providing a
         counterexample for those that you believe are false.

         \begin{enumerate}[label=\protect\circled{\arabic*}]
            \item There is an integral domain with 6 elements.\\

                  Let $k$ be a positive integer. Let $\;\bar{} : \Z \to \Z_k$
                  be the mod function. Thus, $e.g.,$ if $k = 7$, then
                  $\overline{25} = 4$. This leads naturally to a homomorphism
                  $\bar{} : \Z[x] \to \Z_k[x]$. Thus, $e.g.,$ if $k = 7$, then
                  $\overline{25x^2 + 12} = 4x^2 + 5 = -3x^2 - 2$. Consider the
                  veracity or falsehood of each of the following statements. For
                  those that are true give an argument, for those that are
                  false, give a counterexample. Let $p(x) \in \Z[x]$ be monic.
            \item If $p(x)$ has a root in $\Z$, then $\overline{p}(x)$ has a
                  root in $\Z_k$.
            \item If $\overline{p}(x)$ has a root in $\Z_k$, then $p(x)$ has a
                  root in $\Z$.
            \item If $p(x)$ is irreducible, then so is $\overline{p}(x)$.
            \item If $\overline{p}(x)$ is irreducible, then so is $p(x)$.
         \end{enumerate}
         
      \textbf{Solution.}

      \begin{enumerate}[label=\protect\circled{\arabic*}]
         \item False.

               \textbf{Proof.} Assume to the contrary that $R$ is an integral 
               domain with 6 elements. By Cauchy Theorem we must have an element
               of additive order 2 and an element of additive order 3. Since
               $\gcd(2,3) = 1$, it follows that there exists an element $y$ of
               additive order 6. Let $n$ be the additive order of 1. Now we have
               that $6 \mid n$ because               
               $$0 = \underbrace{1 + \cdots + 1}_{n \text{ times}} = 
                 y(\underbrace{1 + \cdots + 1}_{n \text{ times}}) = 
                 \underbrace{y + \cdots + y}_{n \text{ times}},$$
               so that $n = 6$. Now $1 + 1 + 1$ and $1 + 1$ are nonzero, but
               $$0 = 1 + 1 + 1 + 1 + 1 + 1 = (1 + 1 + 1)(1 + 1),$$ a
               contradiction since we assumed that $R$ was an integral domain;
               thus no integral domain of 6 elements exists. \qed \\
               
         \item True.

               \textbf{Proof.} Let
               $$p(x) = a_0 + a_1x + \cdots + x^n \in \Z[x].$$
               Suppose that $c \in \Z$ is a root of $p(x)$. It
               follows immediately that $\overline{c}$ is also a root of
               $\overline{p}(x)$ because
               \begin{align*}
                  \overline{p}(\overline{c}) &= \overline{a_0} +
                     \overline{a_1}\cdot\overline{c} + \cdots +
                     \overline{c}^n \\
                     &= \overline{a_0 + a_1c + \cdots + c^n} \\
                     &= \overline{p(c)} = \overline{0}.
               \end{align*} \qed
         \item False.

               \textbf{Counterexample.} Let $p(x) = x^2 + 1$. Then
               $\overline{p}(x)$ has a root, $\overline{1}$,  in $\Z_2$ but
               $p(x)$ has no root in $\Z$.
         \item False.

               \textbf{Counterexample.} Let $p(x) = x^2 + 1$. Then
               $p(x)$ is irreducible in $\Z[x]$ but
               $\overline{p}(x) = (x+1)^2$ is not irreducible in $\Z_2[x]$.
         \item True.

               \textbf{Proof.} We shall prove the contrapositive:
               $$p(x) \text{ is not irreducible } \Rightarrow \overline{p}(x)
                 \text{ is not irreducible. }$$
                 
               Suppose $p(x)$ is not irreducible. It follows that there exists
               a nontrivial factorization $p(x) = g(x)h(x)$ where $g(x)$ and
               $h(x)$ are non-constant monic polynomials. Since $g(x)$ is a
               monic polynomial, it follows that the degree of $g(x)$ equals
               the degree of $\overline{g}(x)$; similarly the degree of $h(x)$
               equals the degree of $\overline{h}(x)$. Thus
               $\overline{g}(x)$ and $\overline{h}(x)$ are not units, so that
               $\overline{p}(x) = \overline{g}(x)\overline{h}(x)$ is not
               irreducible. \qed
      \end{enumerate}
%%%%%%%%%%%%%%%%%%%%%%%%%%%%%%%%%%%%%Prob2%%%%%%%%%%%%%%%%%%%%%%%%%%%%%%%%%%%%%%
   \item Consider the integral domain $R = \Z[\sqrt{3}]$. Let
         $A = \left(\begin{tabular}{@{}cc@{}}
                  5 & 3 \\
                  9 & 5
               \end{tabular}\right)$.

         \begin{enumerate}[label=\protect\circled{\arabic*}]
            \item Find a nontrivial unit, and show it has infinite order.
            \item Compute $\D\frac{A}{\left(\begin{tabular}{@{}cc@{}}
                     20 & 6 \\
                     18 & 20
                  \end{tabular}\right)}$ and its reciprocal
                  $\D\frac{\left(\begin{tabular}{@{}cc@{}}
                     20 & 6 \\
                     18 & 20
                  \end{tabular}\right)}{A}$. These elements may not be in the
                  domain, but they are certainly in the field of quotients.
            \item Decide if $A$ and $\left(\begin{tabular}{@{}cc@{}}
                     19 & 11 \\
                     33 & 19
                  \end{tabular}\right)$ are associates.
            \item Is $\left(\begin{tabular}{@{}cc@{}}
                     7789 & 4488 \\
                     13464 & 7789
                  \end{tabular}\right) \equiv \left(\begin{tabular}{@{}cc@{}}
                     57 & 24 \\
                     72 & 57
                  \end{tabular}\right) \text{ mod } A$? Give reasons for your
                  answer.
         \end{enumerate}
         
      \textbf{Solution.}

      \begin{enumerate}[label=\protect\circled{\arabic*}]
         \item The matrix $B = \left(\begin{tabular}{@{}cc@{}}
                  2 & 1 \\
                  3 & 2
               \end{tabular}\right)$ is a unit in $\Z[\sqrt{3}]$ because
               $B^{-1} = \left(\begin{tabular}{@{}rr@{}}
                  2 & $-1$ \\
                  $-3$ & 2
               \end{tabular}\right) \in \Z[\sqrt{3}]$. Let $n$ be a positive 
               integer. Observe that the integer in the first row and first 
               column of $B^n$ will never be less than 2 because all the entries 
               in $B$ are positive integers. Thus $B^n \neq I$, so that
               $|B| = \infty$.
         \item We have
               $$
                  \frac{A}{\left(\begin{tabular}{@{}cc@{}}
                     20 & 6 \\
                     18 & 20
                  \end{tabular}\right)} =
                  \frac{1}{146}\left(\begin{tabular}{@{}cc@{}}
                     23 & 15 \\
                     45 & 23
                  \end{tabular}\right) \text{ and }
                  \frac{\left(\begin{tabular}{@{}cc@{}}
                     20 & 6 \\
                     18 & 20
                  \end{tabular}\right)}{A} = \left(\begin{tabular}{@{}rr@{}}
                     $-23$ & 15 \\
                     45 & $-23$
                  \end{tabular}\right).
               $$
         \item $A$ and $\left(\begin{tabular}{@{}cc@{}}
                  19 & 11 \\
                  33 & 19
               \end{tabular}\right)$ are associates if and only if there exists 
               a unit $X = \left(\begin{tabular}{@{}cc@{}}
                  $a$ & $b$ \\
                  $3b$ & $a$
               \end{tabular}\right)\in\nobreak \Z[\sqrt{3}]$ such that
               $$
                  AX = \left(\begin{tabular}{@{}cc@{}}
                     19 & 11 \\
                     33 & 19
                  \end{tabular}\right).
               $$
               Multiplying $A$ and $X$ and equating corresponding entries will
               yield the equations $3a + 5b = 11$ and $5a + 9b = 19$, and whose
               solution is $a = 2$ and $b = 1$. Since
               $\det(X) = a^2 - 3b^2 = 1$, it follows that $X$ is a unit. Thus
               $A$ and $\left(\begin{tabular}{@{}cc@{}}
                  19 & 11 \\
                  33 & 19
               \end{tabular}\right)$ are associates.
         \item A quick computation will show us that
               $$
                  \left(\begin{tabular}{@{}cc@{}}
                     7789 & 4488 \\
                     13464 & 7789
                  \end{tabular}\right) \equiv \left(\begin{tabular}{@{}cc@{}}
                     57 & 24 \\
                     72 & 57
                  \end{tabular}\right) \text{ mod } A
               $$
               because               
               $$
                  \left(\begin{tabular}{@{}cc@{}}
                     7789 & 4488 \\
                     13464 & 7789
                  \end{tabular}\right) - \left(\begin{tabular}{@{}cc@{}}
                     57 & 24 \\
                     72 & 57
                  \end{tabular}\right) = 
                  \left(\begin{tabular}{@{}cc@{}}
                     7732 & 4464 \\
                     13392 & 7732
                  \end{tabular}\right) = A
                  \left(\begin{tabular}{@{}cc@{}}
                     758 & 438 \\
                     1314 & 758
                  \end{tabular}\right).
               $$               
      \end{enumerate}
%%%%%%%%%%%%%%%%%%%%%%%%%%%%%%%%%%%%%Prob3%%%%%%%%%%%%%%%%%%%%%%%%%%%%%%%%%%%%%%
   \item Consider the following element $\left(\begin{tabular}{@{}ccc@{}}
            0 & 1 & 0 \\
            0 & 0 & 1 \\
            1 & 1 & 0
         \end{tabular}\right)$ of $GL(3, \Z_2)$.

         \begin{enumerate}[label=\protect\circled{\arabic*}]
            \item Compute all of its powers.
            \item How many elements would you have to add for this set of
                  powers to be closed under addition?
            \item Find the characteristic polynomial of each of the powers.
            \item Find the lowest degree polynomial that all of the powers
                  satify.
            \item Have you constructed a field?
            \item[\textbf{Bonus.}]
                  Show that every irreducible cubic over $\Z_2$ has a root among
                  these powers.
         \end{enumerate}
         
      \textbf{Solution.} Let $A = \left(\begin{tabular}{@{}ccc@{}}
         0 & 1 & 0 \\
         0 & 0 & 1 \\
         1 & 1 & 0
      \end{tabular}\right)$

      \begin{enumerate}[label=\protect\circled{\arabic*}]
         \item The powers of $A$ are:
               $$A^1 = \left(\begin{tabular}{@{}ccc@{}}
                  0 & 1 & 0 \\
                  0 & 0 & 1 \\
                  1 & 1 & 0
               \end{tabular}\right), A^2 = \left(\begin{tabular}{@{}ccc@{}}
                  0 & 0 & 1 \\
                  1 & 1 & 0 \\
                  0 & 1 & 1
               \end{tabular}\right), A^3 = \left(\begin{tabular}{@{}ccc@{}}
                  1 & 1 & 0 \\
                  0 & 1 & 1 \\
                  1 & 1 & 1
               \end{tabular}\right), A^4 = \left(\begin{tabular}{@{}ccc@{}}
                  0 & 1 & 1 \\
                  1 & 1 & 1 \\
                  1 & 0 & 1
               \end{tabular}\right)$$
               $$A^5 = \left(\begin{tabular}{@{}ccc@{}}
                  1 & 1 & 1 \\
                  1 & 0 & 1 \\
                  1 & 0 & 0
               \end{tabular}\right), A^6 = \left(\begin{tabular}{@{}ccc@{}}
                  1 & 0 & 1 \\
                  1 & 0 & 0 \\
                  0 & 1 & 0
               \end{tabular}\right), A^7 = \left(\begin{tabular}{@{}ccc@{}}
                  1 & 0 & 0 \\
                  0 & 1 & 0 \\
                  0 & 0 & 1
               \end{tabular}\right).$$
         \item We notice that the set of powers is closed under addition of
               two \textit{distinct} matrices. However, since each matrix added 
               to itself yields the zero matrix, we need to add only the zero
               matrix so that this set of powers is closed under addition.
         \item If we let char$(X)$ denote the charactersitic polynomial of a
               matrix $X$, then it follows that char$(A^7) = x^3 + x^2 + x + 1$,
               $$\text{char}(A) = \text{char}(A^2) =
                 \text{char}(A^4) = x^3 + x + 1, \text{and}$$
               $$\text{char}(A^3) = \text{char}(A^5) =
                 \text{char}(A^6) = x^3 + x^2 + 1.$$
         \item The lowest degree polynomial that $A^7$ satisfies is $x + 1$,
               while the lowest degree polynomial that the remaining powers
               satisfy is their respective characteristic polynomials. Thus
               the lowest degree polynomial that all the powers of $A$ satisfy
               is
               $$(x+1)(x^3+x+1)(x^3+x^2+1) = x^7 + 1.$$
         \item Yes. It is clear that the set of powers of $A$ (including the 0
               matrix) is a commutative ring; since each element in the set of   
               powers of $A$ is a unit, it follows that the set of powers union
               the 0 matrix is a field.
         \item[\textbf{Bonus.}] The only cubics with nontrivial factorizations
               in $\Z_2[x]$ are:
               \begin{equation*}
                  \begin{array}{lcl}
                     x^3 & = & (x)(x)(x) \\
                     x^3 + 1 & = & (x+1)(x^2+x+1) \\
                     x^3 + x & = & x(x+1)^2 \\
                     x^3 + x^2 & = & x^2(x+1) \\
                     x^3 + x^2 + x & = & x(x^2+x+1) \\
                     x^3 + x^2 + x + 1 & = & (x + 1)^3,
                  \end{array}
               \end{equation*}
               so that $x^3 + x + 1$ and $x^3 + x^2 + 1$ are irreducible in
               $\Z_2[x]$. But these irreducibles are the characteristic
               polynomials of $A$ and $A^3$, so it follows by the
               Cayley-Hamilton theorem that $A$ is a root of $x^3 + x + 1$ and
               $A^3$ is a root of $x^3 + x^2 + 1$.
      \end{enumerate}
%%%%%%%%%%%%%%%%%%%%%%%%%%%%%%%%%%%%%Prob4%%%%%%%%%%%%%%%%%%%%%%%%%%%%%%%%%%%%%%
   \item On $\Z_2[x]$. Consider the ring of polynomials $\Z_2[x]$ with
         coefficients in $\Z_2$,
         $$p(x) = a_0 + a_1x + \cdots + a_nx^n.$$

         \begin{enumerate}[label=\protect\circled{\arabic*}]
            \item How many polynomials of degree $n$ are there? \textbf{Hint.}
                  Consider $n = 1, 2, 3, \ldots$.
            \item Consider the function $E : \Z_2[x] \to \Z_2$ that sends any
                  polynomial $p(x)$ to $p(1)$. Decide if it is a (ring)
                  homomorphism or not. Decide if it is one-to-one and onto.
                  Argue your case.
            \item Consider the function $S : \Z_2[x] \to \Z_2[x]$ that sends any
                  polynomial $p(x)$ to $p^2(x)$, it square. Decide if it is a
                  (ring) homomorphism or not. Decide if it is one-to-one and 
                  onto. Argue your case.
            \item Count the number of irreducible quadratics in $\Z_2[x]$.
            \item Count the number of irreducible cubics in $\Z_2[x]$.
            \item Count the number of irreducible quartics in $\Z_2[x]$.
         \end{enumerate}
         
      \textbf{Solution.}

      \begin{enumerate}[label=\protect\circled{\arabic*}]
         \item The coefficient of the $x^n$ term must be 1. We now have two
               choices for each of the remaining $n$ coefficients. Thus there
               are $2^n$ polynomials of degree $n$.
         \item It is clear that $E$ is onto since $E(0) = 0$ and $E(1) = 1$.
               However $E$ is not injective because $E(x) = E(1) = 1$ but
               $x \neq 1$. Now we claim that $E$ is a homomorphism of rings.

               \textbf{Proof.} Consider two elements $q(x), r(x) \in \Z_2[x]$
               where
               $$q(x) = q_0 + q_1x + \cdots + q_nx^n \text{ and }
                 r(x) = r_0 + r_1x + \cdots + r_nx^n.$$
               We have that
               \begin{align*}
                  E(q(x) + r(x)) &= E((q_0+r_0) + (q_1+r_1)x + \cdots +
                     (q_n+r_n)x^n) \\
                     &= (q_0+r_0) + (q_1+r_1) + \cdots + (q_n+r_n) \\
                     &= (q_0 + q_1 + \cdots + q_n)+(r_0 + r_1 + \cdots + r_n) \\
                     &= E(q(x)) + E(r(x)) \text{ and } \\
                  E(q(x)r(x)) &= E\left(q_0r_0 + (q_0r_1+q_1r_0)x + \cdots + 
                     \left(\sum_{i=0}^nq_ir_{n-i}\right)x^n\right) \\
                     &= q_0r_0 + (q_0r_1+q_1r_0) + \cdots + 
                     \left(\sum_{i=0}^nq_ir_{n-i}\right) \\
                     &= q(1)r(1) = E(q(x))E(r(x)),
               \end{align*}
               so that $E$ is a surjective ring homomorphism. \qed
         \item Claim that $S$ is an injective homomorphism of rings.

               \textbf{Proof.} Consider two elements $q(x), r(x) \in \Z_2[x]$
               where
               $$q(x) = q_0 + q_1x + \cdots + q_nx^n \text{ and }
                 r(x) = r_0 + r_1x + \cdots + r_nx^n.$$
               Thus
               \begin{align*}
                  S(q(x) + r(x)) &= (q(x) + r(x))^2  \\
                     &= q(x)^2 + r(x)^2 + 2q(x)r(x) \\
                     &= q(x)^2 + r(x)^2 \\
                     &= S(q(x)) + S(r(x)) \text{ and } \\
                  S(q(x)r(x)) &= (q(x)r(x))^2 \\
                     &= q(x)^2r(x)^2 \\
                     &= S(q(x))S(r(x)),
               \end{align*}
               so that $S$ is a ring homomorphism. Now suppose that
               $S(q(x)) = S(r(x))$; then we have that $q(x)^2 = r(x)^2$, so
               that $(q(x) - r(x))(q(x) + r(x)) = 0$. Since we are in $\Z_2[x]$,
               notice that the additive inverse of every polynomial is itself. 
               Thus we must have that
               $(q(x) - r(x))(q(x) - r(x)) = (q(x) - r(x))(q(x)+r(x)) = 0$. And 
               since $\Z_2[x]$ is an integral domain, it follows that
               $q(x) - r(x) = 0$; i.e., $q(x) = r(x)$ so that $S$ is injective. 
               Clearly $S$ is not surjective since the polynomial $x + 1$ has no 
               preimage under $S$. \qed
         \item The only irreducible quadratic in $\Z_2[x]$ is $x^2 + x + 1$.
         \item We know from the Bonus part of Problem 3 that there are two
               irreducible cubics in $\Z_2[x]$ and they are:
               $$x^3 + x + 1 \text{ and } x^3 + x^2 + 1.$$
         \item There are three irreducible quartics in $\Z_2[x]$ and they are:
               $$x^4 + x + 1, x^4 + x^3 + 1, \text{ and }
                 x^4 + x^3 + x^2 + x + 1.$$
      \end{enumerate}
\end{enumerate}
\end{document}
