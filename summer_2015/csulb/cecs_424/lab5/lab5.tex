\documentclass[9pt]{article}

\usepackage{amssymb}
\usepackage{amsmath}
\usepackage{amsfonts}
\usepackage{comment}
\usepackage{fancyhdr}
\usepackage{mathrsfs}
\usepackage{enumitem}


\usepackage{tikz}

\voffset = -50pt
%\textheight = 700pt
\addtolength{\textwidth}{60pt}
\addtolength{\evensidemargin}{-30pt}
\addtolength{\oddsidemargin}{-30pt}
%\setlength{\headheight}{44pt}

\pagestyle{fancy}
\fancyhf{} % clear all fields
\fancyhead[R]{%
  \scshape
  \begin{tabular}[t]{@{}r@{}}
  CECS 424, Summer 2015\\Section 1 (11171)\\
  LAB \#5, DUE: 2015, August 13
  \end{tabular}}
\fancyhead[L]{%
  \scshape
  \begin{tabular}[t]{@{}r@{}}
  JOSEPH OKONOBOH\\Computer Science\\Cal State Long Beach
  \end{tabular}}
\fancyfoot[C]{\thepage}

\newcommand{\qed}{\hfill \ensuremath{\Box}}


\newcommand*\circled[1]{\tikz[baseline=(char.base)]{
            \node[shape=circle,draw,inner sep=2pt] (char) {#1};}}

\newcommand{\Z}{\mathbb{Z}}
\newcommand{\I}{\mathbb{I}}
\newcommand{\M}{\mathbb{M}}
\newcommand{\Q}{\mathbb{Q}}
\newcommand{\R}{\mathbb{R}}
\newcommand{\C}{\mathbb{C}}
\newcommand{\D}{\displaystyle}
%\setcounter{section}{-1}

\begin{document}
\noindent Due August 13 by the start of lecture

\textbf{Program 1.}
\begin{enumerate}
%%%%%%%%%%%%%%%%%%%%%%%%%%%%%%%%%%%%%%%02%%%%%%%%%%%%%%%%%%%%%%%%%%%%%%%%%%%%%%%
   \item \verb|(defn coord []| \\
         \verb|  (list (rand 1) (rand 1)))| \\ \\
  
         \verb|(defn throw-darts [n]| \\
         \verb|  (take n (repeatedly coord)))| \\
  
         \verb|(defn is-hit? [dart]| \\
         \verb|  (let [x (first dart) y (second dart)]| \\
         \verb|    (<= (Math/sqrt (+ (* x x) (* y y))) 1.0)))| \\ \\
    
         \verb|(defn count-hits [n]| \\
         \verb|  (count (filter is-hit? (throw-darts n))))| \\ \\
  
         \verb|(defn estimate-pi [n]| \\
         \verb|  (float (* 4 (/ (count-hits n) n))))|
   \item The output of \verb|estimate-pi| for these values of \verb|n:|

         \begin{enumerate}
            \item 10
            \item 100
            \item 1000
            \item 100000
            \item The largest number you can enter, such that your computation
                  runs for at most 3-4 seconds.
         \end{enumerate}
         
      \textbf{Answer.}
      
      \begin{enumerate}      
         \item 3.6
         \item 3.32
         \item 3.192
         \item 3.1398
         \item $\approx 2500000$
      \end{enumerate}
%%%%%%%%%%%%%%%%%%%%%%%%%%%%%%%%%%%%%%%02%%%%%%%%%%%%%%%%%%%%%%%%%%%%%%%%%%%%%%%
   \item An explanation of why your estimates for smaller values of \verb|n| are
         so poor.

      \textbf{Answer.} Our method for computing $\pi$ uses a process known as
      random statistical sampling. Since there are an infinite number of points
      in any circle of nonzero radius, a smaller value for $n$ will undermine
      the true value of the $\pi$ because they are statistically insignificant,
      compared to all the other points. However, the more points we sample, the
      more relatively significant our data become, and consequently, the
      more accurate our calculated value of $\pi$ will be. Thus, bigger values
      of $n$ will yield more accurate values of the $\pi$.
\end{enumerate}


\textbf{Program 2.}
\begin{enumerate}
   \item \verb|(take 10| \\
         \verb|  (sort-by last >| \\
         \verb|    (frequencies (re-seq #"[a-zA-Z]+"| \\
         \verb|      (.toLowerCase (slurp "src/cecs424/emma.txt"))))))|
   \item \verb|(| \\
         \verb|   ["to" 5242]| \\
         \verb|   ["the" 5204]| \\
         \verb|   ["and" 4897]| \\
         \verb|   ["of" 4293]| \\
         \verb|   ["i" 3192]| \\
         \verb|   ["a" 3130]| \\
         \verb|   ["it" 2529]| \\
         \verb|   ["her" 2490]| \\
         \verb|   ["was" 2400]| \\
         \verb|   ["she" 2364]| \\
         \verb|)| 
\end{enumerate}
\end{document}
