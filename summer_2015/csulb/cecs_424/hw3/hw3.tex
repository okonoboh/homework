\documentclass[9pt]{article}

\usepackage{amssymb}
\usepackage{amsmath}
\usepackage{amsfonts}
\usepackage{comment}
\usepackage{fancyhdr}
\usepackage{mathrsfs}
\usepackage{enumitem}


\usepackage{tikz}

\voffset = -50pt
%\textheight = 700pt
\addtolength{\textwidth}{60pt}
\addtolength{\evensidemargin}{-30pt}
\addtolength{\oddsidemargin}{-30pt}
%\setlength{\headheight}{44pt}

\pagestyle{fancy}
\fancyhf{} % clear all fields
\fancyhead[R]{%
  \scshape
  \begin{tabular}[t]{@{}r@{}}
  CECS 424, Summer 2015\\Section 1 (11171)\\
  HW \#3, DUE: 2015, July 28
  \end{tabular}}
\fancyhead[L]{%
  \scshape
  \begin{tabular}[t]{@{}r@{}}
  JOSEPH OKONOBOH\\Computer Science\\Cal State Long Beach
  \end{tabular}}
\fancyfoot[C]{\thepage}

\newcommand{\qed}{\hfill \ensuremath{\Box}}


\newcommand*\circled[1]{\tikz[baseline=(char.base)]{
            \node[shape=circle,draw,inner sep=2pt] (char) {#1};}}

\newcommand{\Z}{\mathbb{Z}}
\newcommand{\I}{\mathbb{I}}
\newcommand{\M}{\mathbb{M}}
\newcommand{\Q}{\mathbb{Q}}
\newcommand{\R}{\mathbb{R}}
\newcommand{\C}{\mathbb{C}}
\newcommand{\D}{\displaystyle}
%\setcounter{section}{-1}

\begin{document}
\noindent Complete these problems on a separate sheet of paper. Due July 28 at
the end of lab.
\begin{enumerate}
%%%%%%%%%%%%%%%%%%%%%%%%%%%%%%%%%%%%%%%01%%%%%%%%%%%%%%%%%%%%%%%%%%%%%%%%%%%%%%%
   \item Reading:

         \begin{enumerate}
            \item Chapter 6.4, 6.5 (skip ``Iterators"), 6.6 (particularly tail
                  recursion)
            \item Chapter 7 introduction
            \item Chapter 7.1 (skip ``Orthogonality"), 7.2
         \end{enumerate}
%%%%%%%%%%%%%%%%%%%%%%%%%%%%%%%%%%%%%%%02%%%%%%%%%%%%%%%%%%%%%%%%%%%%%%%%%%%%%%%
   \item Python's \verb|filter| function takes two parameters: a function taking
         a single parameter and returning \verb|True| or \verb|False|, and a
         list of values. \verb|filter| returns a new list of all values in the
         parameter for which the supplied function returns \verb|True|. For
         example: \\

         \verb|def is_even(n):| \\
         \verb|   return % 2 == 0| \\
         \verb|print filter(is_even, range(10))| \\

         will print a list of the even numbers from 0 to 8 inclusive. Using
         \verb|lambda| expressions and \verb|filter|, show how to transform the
         following lists into the desired format. The first answer is given:
         \begin{enumerate}
            \item Input: a list of integers, \verb|input|. \\
                  Output: all even integers in the list.
            \item Input: a list of integers, \verb|input|. \\
                  Output: all integers from the list that are mutiples of 3 or
                  multiples of 5, but not multiples of 3 and 5.
            \item Input: a list of integers, \verb|input|. \\
                  Output: all integers from the list that are prime. You may
                  assume the existence of \verb|is_prime| from Homework 2.
            \item Input: a list of integers, \verb|lists|. (Ex:
                  \verb|[[1, 2], [3, 4, 5, 6], [7], [8], []]|.) \\
                  Output: a list of lists of integers, containing \textbf{only}
                  the lists in the input with at least 2 elements in them.
                  (Example output from above: \verb|[[1, 2], [3, 4, 5, 6]]|.)
                  Hint: you may need multiple \verb|lambda| expressions.
            \item Input: a list of dictionaries, each dictionary containing the
                  keys \verb|`name'| (a string) and \verb|`salary'| (a
                  floating-point number). \\
                  Output: a list of dictionaries, containing \textbf{only} the
                  dictionaries in the input with a \verb|salary| of at least
                  \$50,000.
         \end{enumerate}

      \textbf{Answer.}      
         
      \begin{enumerate}
         \item \verb|filter(lambda n: n % 2 == 0, input)|
         \item \verb|filter(lambda n : (n % 3 == 0 or n % 5 == 0)| \\
               \verb|   and (n % 15 != 0), input)|
         \item \verb|filter(lambda n: is_prime(n), input)|
         \item \verb|filter(lambda n: len(n) >= 2, input)|
         \item \verb|filter(lambda n: n[`salary'] >= 50000, input)|
      \end{enumerate}
%%%%%%%%%%%%%%%%%%%%%%%%%%%%%%%%%%%%%%%03%%%%%%%%%%%%%%%%%%%%%%%%%%%%%%%%%%%%%%%
   \item Explain in your words the difference between \textit{type coercion} and
         \textit{type inference}. Give an example of each in a language of your
         choice.

      \textbf{Answer.} \textit{Type coercion} occurs when the compiler
      implicitly converts an object of one type to another type. For example, 
      consider the following statements in Java: \\

      \verb|   1. int x = 90;| \\
      \verb|   2. double y = x;| \\

      In line 2, the variable $\verb|x|$ will be temporarily promoted(or
      coerced) to type \verb|double| and the result will be assigned to
      \verb|y|. On the other hand, \textit{type inference} occurs when a
      compiler infers the type of an object from the context of the statement.
      For example, the following code in C++ 11,\\

      \verb|01.   #include <iostream>| \\
      \verb|02.   #include <typeinfo>| \\
      \verb|03.| \\
      \verb|04.   int f() {| \\
      \verb|05.      return 78;| \\
      \verb|06.   }| \\
      \verb|07.| \\
      \verb|08.   int main() {| \\
      \verb|09.      auto x = f();| \\
      \verb|10.      std::cout << typeid(x).name() << std::endl;| \\
      \verb|11.| \\
      \verb|12.      return 0;| \\
      \verb|13.   }| \\

      will print ``int" (using Visual Studio 2013) to the standard output, 
      indicating that the type of \verb|x|, inferred from line 9, was an
      \verb|int|.
%%%%%%%%%%%%%%%%%%%%%%%%%%%%%%%%%%%%%%%04%%%%%%%%%%%%%%%%%%%%%%%%%%%%%%%%%%%%%%%
   \item Use what you have learned so far in this course to classify the Python,
         C, and Java languages in the following categories:

         \begin{enumerate}
            \item Compilation: is the language compiled, interpreted, or does it
                  use a virtual machine?
            \item Type system: is the type system static or dynamic? What rules
                  of coercion are available?
            \item Allocation: what allocation methods are used for variables?
                  Give examples.
            \item Scoping: is the language statically or dynamically scoped?
            \item Functions: are functions first or second class citizens?
         \end{enumerate}

      \textbf{Answer.}

      \begin{enumerate}
         \item $$
                  \begin{tabular}{@{}|c|l|@{}} \hline
                     \multicolumn{2}{|c|}{\textbf{Compilation}} \\ \hline
                     \textbf{Python} & Interpreted \\ \hline
                     \textbf{C} & Compiled into machine code \\ \hline
                     \textbf{Java} & Compiled into JVM (Java Virtual Machine)
                     byte code \\ \hline         
                  \end{tabular}
               $$
         \item $$
                  \begin{tabular}{@{}|c|l|@{}} \hline
                     \multicolumn{2}{|c|}{\textbf{Type System}} \\ \hline
                     \textbf{Python} & Dynamically typed \\ \hline
                     \textbf{C} & Statically typed \\ \hline
                     \textbf{Java} & Statically typed \\ \hline         
                  \end{tabular}
               $$

               \textbf{Note.} In Java and C, an integral type may be promoted to
               another type
      \end{enumerate}
%%%%%%%%%%%%%%%%%%%%%%%%%%%%%%%%%%%%%%%05%%%%%%%%%%%%%%%%%%%%%%%%%%%%%%%%%%%%%%%
   \item Consider the following Python program:

         \verb|01.   x = 1| \\
         \verb|02.   def incrementer(base):| \\
         \verb|03.      def my_incr(inc):| \\
         \verb|04.         return base * 2 + inc + x| \\
         \verb|05.      return my_incr| \\
         \verb|06.| \\
         \verb|07.   # Later, perhaps in another scope...?| \\
         \verb|08.   i = incrementer(10)| \\
         \verb|09.   x = 100| \\
         \verb|10.   print i(5)| \\
         \verb|11.   print i(3)|

         \begin{enumerate}
            \item Explain in your own words why the concept of a
                  \textit{closure} is needed to make the example work correctly.
            \item Python uses late binding rules for closures. With that in
                  mind, what is the output of the program? (Try to answer
                  without running it, as you will have to do on exams.)
            \item What would the output of the program be if Python used early
                  binding for closures?
         \end{enumerate}

      \textbf{Answer.}

      \begin{enumerate}
         \item After line 8 is executed, the variable \verb|i| will contain a
               reference to the inner function \verb|my_incr|. Without closure,
               the memory allocated to the variable \verb|base| may have been
               deallocated by the time that \verb|i| is called. This may happen
               since \verb|base| will go out of scope (and thus reclaimed) after 
               \verb|incrementer| returns. But closure ensures that a
               referencing environment---that has a reference to
               \verb|base|---will be created for \verb|i|.
         \item \verb|125| \\
               \verb|123|
         \item \verb|26| \\
               \verb|24|
      \end{enumerate}
%%%%%%%%%%%%%%%%%%%%%%%%%%%%%%%%%%%%%%%06%%%%%%%%%%%%%%%%%%%%%%%%%%%%%%%%%%%%%%%
   \item (Exercise 6.7 from \textit{Programming Language Pragmatics} Is the
         expression \verb|&(&i)| ever valid in C? Why or why not? Use the terms
         \textit{lvalue} and \textit{rvalue} in your answer.

      \textbf{Answer.} No, it is never valid. The adress operator only takes an
      lvalue as an argument; since the operation \verb|&i| returns an rvalue,
      the expression \verb|&(&i)| will always fail.
%%%%%%%%%%%%%%%%%%%%%%%%%%%%%%%%%%%%%%%07%%%%%%%%%%%%%%%%%%%%%%%%%%%%%%%%%%%%%%%
   \item \textit{Syntactic sugar} refers to programming language constructs that
         are designed to make programming in the language easier to read or to
         express, but do not actually add to the computational power of the
         language. For each of the following ``sugar" constructs from C++, give
         the ``de-sugared" form they are equivalent to using simpler operators
         or control structures
         \begin{enumerate}
            \item \verb|a->x| (accesing a member of a class through a pointer)
            \item \verb|b++| (postincrement)
            \item \verb|c *= 5| (multiplication assignment)
            \item \verb|for(int d = 0; d < 10; d++) {| \\
                  \verb|   e();| \\
                  \verb|}| (C-style \verb|for| loop)
            \item \verb|f[i]| (array indexing; see pointer array arithmetic)
         \end{enumerate}

      \textbf{Answer.}

      \begin{enumerate}
         \item \verb|(*a).x|
         \item \verb|b| \\
               \verb|b = b + 1|
         \item \verb|c = c * 5|
         \item \verb|d = 0| \\
               \verb|while(d < 10) {| \\
               \verb|   e();| \\
               \verb|   d;| \\
               \verb|   d = d + 1| \\
               \verb|}|
         \item \verb|*(f + i)|
      \end{enumerate}
%%%%%%%%%%%%%%%%%%%%%%%%%%%%%%%%%%%%%%%08%%%%%%%%%%%%%%%%%%%%%%%%%%%%%%%%%%%%%%%
   \item C++ gives the option to declare integer types as \verb|unsigned|, using
         the topmost bit in the number's binary representation as another data
         bit instead of a sign bit. Answer the following questions about
         \verb|unsigned| values:
         \begin{enumerate}
            \item Suppose the signed version of some integer type has a maximum
                  value of \verb|X|. In terms of \verb|X|, what is the maximum
                  value of an \textit{unsigned} variable of the same type?
            \item What does the largest value an unsigned integer can represent
                  roll over to when you add 1 to it?
            \item Java's designers specifically chose not to allow unsigned
                  types. Give at least one argument in favor of not keeping
                  unsigned types in Java, and at least one for including them.
         \end{enumerate}

      \textbf{Answer.}

      \begin{enumerate}
         \item \verb|2X + 1|
         \item 0
         \item Mixing \verb|unsigned| and \verb|signed| types in a program
               increases the chance for some subtle and hard-to-detect logical 
               errors to creep into the program. For example, consider the
               following code snippet in C: \\

               \verb|01.   #include <stdio.h>| \\
               \verb|02.| \\
               \verb|03.   int main() {| \\
               \verb|04.      int x = -1;| \\
               \verb|05.      unsigned int y = 2200000000u;| \\
               \verb|06.| \\
               \verb|07.      if(x < y) {| \\
               \verb|08.         printf("Good\n");| \\
               \verb|09.      } else {| \\
               \verb|10.         printf("Bad\n");| \\
               \verb|11.      }| \\
               \verb|12.   }| \\

               Since \verb|signed| types are lower than \verb|unsigned| types in
               the integral hierarchy in C, \verb|x| is promoted to
               \verb|unsigned| in line 7, so that the maximum unsigned 32-bit
               value is temporarily assigned to the promoted \verb|x|. Thus the
               program prints the incorrect string. This cannot happen in Java
               since there are no native unsigned types. Additionally, it leads
               to a cleaner design, since the programmer does not have to worry
               about the intricate operations of \verb|(un)signed| types. \\

               However, an \verb|unsigned| type can represent twice the number
               of positive numbers that the \verb|signed| type can. This may be
               useful for values for which negative integers make no sense; for
               example, counting, age e.t.c. 
      \end{enumerate}
\end{enumerate}
\end{document}
