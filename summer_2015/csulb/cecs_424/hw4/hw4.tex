\documentclass[9pt]{article}

\usepackage{amssymb}
\usepackage{amsmath}
\usepackage{amsfonts}
\usepackage{comment}
\usepackage{fancyhdr}
\usepackage{mathrsfs}
\usepackage{enumitem}


\usepackage{tikz}

\voffset = -50pt
%\textheight = 700pt
\addtolength{\textwidth}{60pt}
\addtolength{\evensidemargin}{-30pt}
\addtolength{\oddsidemargin}{-30pt}
%\setlength{\headheight}{44pt}

\pagestyle{fancy}
\fancyhf{} % clear all fields
\fancyhead[R]{%
  \scshape
  \begin{tabular}[t]{@{}r@{}}
  CECS 424, Summer 2015\\Section 1 (11171)\\
  HW \#4, DUE: 2015, August 04
  \end{tabular}}
\fancyhead[L]{%
  \scshape
  \begin{tabular}[t]{@{}r@{}}
  JOSEPH OKONOBOH\\Computer Science\\Cal State Long Beach
  \end{tabular}}
\fancyfoot[C]{\thepage}

\newcommand{\qed}{\hfill \ensuremath{\Box}}


\newcommand*\circled[1]{\tikz[baseline=(char.base)]{
            \node[shape=circle,draw,inner sep=2pt] (char) {#1};}}

\newcommand{\Z}{\mathbb{Z}}
\newcommand{\I}{\mathbb{I}}
\newcommand{\M}{\mathbb{M}}
\newcommand{\Q}{\mathbb{Q}}
\newcommand{\R}{\mathbb{R}}
\newcommand{\C}{\mathbb{C}}
\newcommand{\D}{\displaystyle}
%\setcounter{section}{-1}

\begin{document}
\noindent Complete these problems on a separate sheet of paper. Due July 28 at
the start of lecture.
\begin{enumerate}
%%%%%%%%%%%%%%%%%%%%%%%%%%%%%%%%%%%%%%%01%%%%%%%%%%%%%%%%%%%%%%%%%%%%%%%%%%%%%%%
   \item Reading:

         \begin{enumerate}
            \item Chapter 7.3 (\textit{skim} 7.3.3, 7.3.4)
            \item Chapter 7.4 (skip \textit{Dope Vectors})
            \item \textit{Skim} Chapter 7.5, 7.7 (but \textit{read} 7.7.3,
                  particularly Reference Counts, Mark-and-Sweep, and
                  Generational Collection)
            \item \textit{Skim} Chapter 8.1
            \item Read Chapter 8.2 (skip 8.2.1, 8.2.2, 8.3.3)
            \item Read Chapter 8.3, 8.4
            \item \textit{Skim} Chapter 8.5
            \item \textit{Skim} Chapter 9.1      
         \end{enumerate}
%%%%%%%%%%%%%%%%%%%%%%%%%%%%%%%%%%%%%%%02%%%%%%%%%%%%%%%%%%%%%%%%%%%%%%%%%%%%%%%
   \item (Adapted from Exercise 7.8 from
         \textit{Programming Language Pragmatics}) Suppose we are compiling a
         variable of the following type for a machine with 1-byte characters,
         2-byte shorts, 4-byte integers, and 8-byte doubles, with alignment
         rules that require the address of every primitive data element to be an
         even multiple of the element's size. (So a character can start at byte
         0, 2, 4, \ldots; short can start at 0, 4, 8, \ldots; etc.) Assume that
         the compiler will not reorder the struct's fields. \\

         \verb|struct A {| \\
         \verb|   short s;| \\
         \verb|   int i;| \\
         \verb|   char c;| \\
         \verb|   short t;| \\
         \verb|   char d;| \\
         \verb|   double r;| \\
         \verb|};| \\

         \begin{enumerate}
            \item How many bytes of memory will be consumed by a value of type
                  \verb|A|?
            \item How many bytes will be consumed by a value of type
                  \verb|A[10]|?
            \item Give a reordering of \verb|A|'s fields so that the memory used
                  by a value of type \verb|A| is minimized. (The rules above
                  still apply to the reordering.)
         \end{enumerate}

      \textbf{Answer.}

      \begin{enumerate}
         \item $$
                  \begin{tabular}{@{}|c|c|@{}} \hline
                     \textbf{Field} & \textbf{Starting Address} \\ \hline
                     \verb|s| & \verb|0| \\ \hline
                     \verb|i| & \verb|8| \\ \hline
                     \verb|c| & \verb|12| \\ \hline
                     \verb|t| & \verb|16| \\ \hline
                     \verb|d| & \verb|18| \\ \hline
                     \verb|r| & \verb|32| \\ \hline
                  \end{tabular}
               $$
               So \verb|sizeof(A) = 40| bytes.
         \item kk
         \item \verb|struct A {| \\
               \verb|   double r;| \\
               \verb|   int i;| \\
               \verb|   short s;| \\
               \verb|   short t;| \\
               \verb|   char c;| \\
               \verb|   char d;| \\
               \verb|};| \\
      \end{enumerate}
%%%%%%%%%%%%%%%%%%%%%%%%%%%%%%%%%%%%%%%03%%%%%%%%%%%%%%%%%%%%%%%%%%%%%%%%%%%%%%%
   \item Suppose the struct from Problem 2 is \textit{packed} by the compiler
         without regard for alignment issues.

         \begin{enumerate}
            \item Draw a picture of a value of type \verb|A| when packed. Show
                  the address of where each field begins, assuming the value
                  itself starts at address 0.
            \item Assume a variable \verb|unsigned char* p| points to the
                  address of the first field of a value of type \verb|A|.
                  Suppose you want to multiply the value of \verb|p|'s field
                  \verb|i| by the value of field \verb|s|. Show how to use
                  multiple statements to unpack field \verb|i| to a temporary
                  integer, and then multiply that by field \verb|s|. Assume that
                  two-byte values can be read from any address that is a
                  multiple of 2, but 4-byte values can only be read from
                  addresses that are a multiple of 4. (You will need to read a
                  few bytes of \verb|i| at a time and use clever bit-shift and
                  bit-mask operators to recombine its value into your
                  temporary.)
         \end{enumerate}

      \textbf{Answer.}

      \begin{enumerate}
         \item $$
                  \begin{tabular}{@{}|c|c|@{}} \hline
                     \textbf{Field} & \textbf{Starting Address} \\ \hline
                     \verb|s| & \verb|0| \\ \hline
                     \verb|i| & \verb|2| \\ \hline
                     \verb|c| & \verb|6| \\ \hline
                     \verb|t| & \verb|7| \\ \hline
                     \verb|d| & \verb|11| \\ \hline
                     \verb|r| & \verb|12| \\ \hline
                  \end{tabular}
               $$
         \item \verb|int temp;| \\
               \verb|unsigned char* i = p + 2;| \\
               \verb|temp = 0x0000FFFF & *((short*) i);| \\
               \verb|i = p + 4;| \\
               \texttt{temp |= (0x0000FFFF \& *((short*) i)) << 16;} \\
               \verb|temp *= *((*short) p);|
      \end{enumerate}
%%%%%%%%%%%%%%%%%%%%%%%%%%%%%%%%%%%%%%%04%%%%%%%%%%%%%%%%%%%%%%%%%%%%%%%%%%%%%%%
   \item Given the following matrix:
         \begin{tabular}{@{}|lcr|@{}}
            1 & 2 & 3 \\
            4 & 5 & 6 \\
            7 & 8 & 9
         \end{tabular}

         Assuming that each entry in the matrix is stored as a 4-byte integer,
         give the memory layout of this matrix if it is represented as a
         two-dimensional array in a language that uses

         \begin{enumerate}
            \item row-major ordering for arrays
            \item column-major ordering for arrays
         \end{enumerate}

         Give your answer as a linear contiguous block of memory, indicating the
         address of each value in the matrix, starting at address 0.

      \textbf{Answer.}

      \begin{enumerate}
         \item \textbf{Row-major.}
               $$
                  \begin{tabular}{@{}|c|c|@{}} \hline
                     \textbf{Address} & \textbf{Value} \\ \hline
                     \verb|0|  & \verb|1| \\ \hline
                     \verb|4|  & \verb|2| \\ \hline
                     \verb|8|  & \verb|3| \\ \hline
                     \verb|12| & \verb|4| \\ \hline
                     \verb|16| & \verb|5| \\ \hline
                     \verb|20| & \verb|6| \\ \hline
                     \verb|24| & \verb|7| \\ \hline
                     \verb|28| & \verb|8| \\ \hline
                     \verb|32| & \verb|9| \\ \hline
                  \end{tabular}
               $$
         \item \textbf{Column-major.}
               $$
                  \begin{tabular}{@{}|c|c|@{}} \hline
                     \textbf{Address} & \textbf{Value} \\ \hline
                     \verb|0|  & \verb|1| \\ \hline
                     \verb|4|  & \verb|4| \\ \hline
                     \verb|8|  & \verb|7| \\ \hline
                     \verb|12| & \verb|2| \\ \hline
                     \verb|16| & \verb|5| \\ \hline
                     \verb|20| & \verb|8| \\ \hline
                     \verb|24| & \verb|3| \\ \hline
                     \verb|28| & \verb|6| \\ \hline
                     \verb|32| & \verb|9| \\ \hline
                  \end{tabular}
               $$
      \end{enumerate}
%%%%%%%%%%%%%%%%%%%%%%%%%%%%%%%%%%%%%%%05%%%%%%%%%%%%%%%%%%%%%%%%%%%%%%%%%%%%%%%
   \item Give an example in C++ of a dangling reference/pointer to a stack
         variable. Why is this a concern?

      \textbf{Answer.} Consider the code snippet below:
      
      \verb|01.   int* sum(int, int);| \\
      \verb|02.| \\
      \verb|03.   int main() {| \\
      \verb|04.      int* x = sum(3, 5);| \\
      \verb|05.      /* | \\
      \verb|06.       * perform some operations involving x| \\
      \verb|07.       * | \\
      \verb|08.       */| \\
      \verb|09.      return 0;| \\
      \verb|10.   }| \\
      \verb|11.| \\
      \verb|12.   int* sum(int x, int y) {| \\
      \verb|13.      int s = x + y;| \\
      \verb|14.      return &s;| \\
      \verb|15.   }| \\

      After line 5 has executed, the value of \verb|x|  will be the address of a
      local variable in the function \verb|sum|. This local variable was 
      allocated on the stack, and thus, its memory will be used by other,
      possibly, invoked functions after line 5. This situation could lead to
      subtle logical errors.
%%%%%%%%%%%%%%%%%%%%%%%%%%%%%%%%%%%%%%%06%%%%%%%%%%%%%%%%%%%%%%%%%%%%%%%%%%%%%%%
   \item Does the Java programming language support a true call-by-reference
         model for parameter passing? Explain.

      \textbf{Answer.} No. This is so because a parameter, in Java, does not
      have the ability to change the value of variable that was passed as an
      argument.
\end{enumerate}
\end{document}
