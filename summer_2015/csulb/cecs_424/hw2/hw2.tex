\documentclass[9pt]{article}

\usepackage{amssymb}
\usepackage{amsmath}
\usepackage{amsfonts}
\usepackage{comment}
\usepackage{fancyhdr}
\usepackage{mathrsfs}
\usepackage{enumitem}
\usepackage{url}
\expandafter\def\expandafter\UrlBreaks\expandafter{\UrlBreaks%  save the current one
  \do\a\do\b\do\c\do\d\do\e\do\f\do\g\do\h\do\i\do\j%
  \do\k\do\l\do\m\do\n\do\o\do\p\do\q\do\r\do\s\do\t%
  \do\u\do\v\do\w\do\x\do\y\do\z\do\A\do\B\do\C\do\D%
  \do\E\do\F\do\G\do\H\do\I\do\J\do\K\do\L\do\M\do\N%
  \do\O\do\P\do\Q\do\R\do\S\do\T\do\U\do\V\do\W\do\X%
  \do\Y\do\Z}


\usepackage{tikz}

\voffset = -50pt
%\textheight = 700pt
\addtolength{\textwidth}{60pt}
\addtolength{\evensidemargin}{-30pt}
\addtolength{\oddsidemargin}{-30pt}
%\setlength{\headheight}{44pt}

\pagestyle{fancy}
\fancyhf{} % clear all fields
\fancyhead[R]{%
  \scshape
  \begin{tabular}[t]{@{}r@{}}
  CECS 424, Summer 2015\\Section 1 (11171)\\
  HW \#2, DUE: 2015, July 21
  \end{tabular}}
\fancyhead[L]{%
  \scshape
  \begin{tabular}[t]{@{}r@{}}
  JOSEPH OKONOBOH\\Computer Science\\Cal State Long Beach
  \end{tabular}}
\fancyfoot[C]{\thepage}

\newcommand{\qed}{\hfill \ensuremath{\Box}}


\newcommand*\circled[1]{\tikz[baseline=(char.base)]{
            \node[shape=circle,draw,inner sep=2pt] (char) {#1};}}

\newcommand{\Z}{\mathbb{Z}}
\newcommand{\I}{\mathbb{I}}
\newcommand{\M}{\mathbb{M}}
\newcommand{\Q}{\mathbb{Q}}
\newcommand{\R}{\mathbb{R}}
\newcommand{\C}{\mathbb{C}}
\newcommand{\D}{\displaystyle}
%\setcounter{section}{-1}

\begin{document}
\noindent Complete these problems on a separate sheet of paper. Due July 21 at
the end of lab.
\begin{enumerate}
%%%%%%%%%%%%%%%%%%%%%%%%%%%%%%%%%%%%%%%01%%%%%%%%%%%%%%%%%%%%%%%%%%%%%%%%%%%%%%%
   \item Reading:

         \begin{enumerate}
            \item Chapter 3.3 (\textit{skim} section 3.3.3 and 3.3.4;
                  \textit{ignore} section 3.3.5)
            \item \textit{Skim} Chapter 3.5
            \item Chapter 13 intro
            \item Chapter 13.1
            \item \textit{Skim} Chapter 13.2 (focus on Python section), 13.4
            \item Google's Python introduction:

                  \verb|http://developers.google.com/edu/python/introduction|
            \item Chapter 6 intro
            \item From Chapter 6.1: \textit{skim} 6.1.1; \textit{read} 6.1.2
                  (\textit{stop} after ``Boxing"); \textit{skim} 6.1.5
            \item \textit{Skim} Chapter 6.2 (for the answer to Homework 1
                  question 2)
         \end{enumerate}
%%%%%%%%%%%%%%%%%%%%%%%%%%%%%%%%%%%%%%%02%%%%%%%%%%%%%%%%%%%%%%%%%%%%%%%%%%%%%%%
   \item In the C and C++ programming languages, the memory sizes of fundamental
         data types are not strictly defined; in C++, for example, a character
         is defined as 1 byte in size, but \verb|short| is only defined as ``not
         smaller than \verb|char|", \verb|int| is ``not smaller than
         \verb|short|", etc. Other languages strictly define data type sizes:
         Java specifies \verb|int| as 4 bytes, \verb|long| as 8 bytes, etc.
         \textbf{Why} do some languages not strictly define the sizes of
         fundamental data types? (Focus your answer on C/C++ in particular.) Do
         you consider this a \textbf{strength} or \textbf{weakness} of these
         languages?
         
      \textbf{Answer.} The \verb|word| of a processor is typically the maximum
      number of bits that the processor can transfer in parallel. Thus, for
      example, the \verb|word| of a 32-bit machine will be 32 bits. The idea was
      to let the size of an \verb|int| in C/C++ equal the \verb|word| of a
      machine. Since different machines, for the most part, have varying
      \verb|words|, it would not have made sense to specify the exact size of an
      \verb|int|. However, this has unfortunately led to confusion and
      portability issues because so many programmers assume that an \verb|int|
      is always 4 bytes. In fact, \verb|sizeof(int) = 16 bits| in many legacy
      systems, while it is 32 bits in some modern systems and 64-bits in others.
      Thus this is a severe weakness in the language (or, more accurately, in
      the ISO standard of the language).
%%%%%%%%%%%%%%%%%%%%%%%%%%%%%%%%%%%%%%%03%%%%%%%%%%%%%%%%%%%%%%%%%%%%%%%%%%%%%%%
   \item A revision to the C++ language in 2011, titled \textbf{C++11}, requires
         compilers to provide type aliases \verb|int32_t|, \verb|int64_t|, and
         others, which guarantee the number of bits used for values of those
         types. For example, \verb|int32_t myVar| defines a variable that stores
         a 32-bit integer, in contrast to \verb|int otherVar| which defines an
         integer variable that is ``not smaller than a \verb|short|". With this
         knowledge in mind, does your answer to question 2 change? Why or why
         not?
         
   \textbf{Answer.} The addition of \verb|int| aliases with guaranteed bits to C++ is a
   step in the right direction, for programmers can now write code knowing that
   the result will be homogeneous, at least with respect to the operations on
   these aliases. Nevertheless, programmers may not want to switch to
   these new aliases because of habit, so the varying behavior of an \verb|int|
   datatype is still a weakness of the C/C++ languages.
%%%%%%%%%%%%%%%%%%%%%%%%%%%%%%%%%%%%%%%04%%%%%%%%%%%%%%%%%%%%%%%%%%%%%%%%%%%%%%%
   \item (Exercise 3.9 from \textit{Programming Language Pragmatics}) Consider
         the following fragment of code in C:
         \begin{verbatim}
{  int a, b, c;
   ...
   {  int d, e;
      ...
      {  int f;
         ...
      }
      ...
   }
   ...
   {  int g, h, i;
      ...
   }
   ...
}
         \end{verbatim}
         \begin{enumerate}
            \item Assume that each integer variable occupies 4 bytes. How much
                  total space is required in memory for the variables in this 
                  code fragment?
            \item Because C is statically scoped, we can note that variables
                  \verb|d, e, f| will never be in scope at the same time as
                  \verb|g, h, i|. With this in mind, give a displacement address
                  for each variable in the fragment (in terms of \verb|fp|) so 
                  that the active memory requirement for the variables in the 
                  fragment is \textbf{smaller} than your answer to part (a). 
                  Assume that variable \verb|a| is at \verb|fp - 4|, and that 
                  variables are pushed onto the stack in the order they are 
                  declared, so \verb|a| is the bottom element of the stack.
         \end{enumerate}

   \textbf{Answer.}

   \begin{enumerate}
      \item Since 9 integer variables were declared and since each integer
            variable occupies 4 bytes, we would need $4 \cdot 9 = 36$ bytes
            for the integer variables in this code fragment.
      \item $$
               \begin{tabular}{@{}|c|c|@{}} \hline
                  Variable & Displacement Address \\ \hline
                  \verb|a| & \verb|fp -  4| \\ \hline
                  \verb|b| & \verb|fp -  8| \\ \hline
                  \verb|c| & \verb|fp - 12| \\ \hline
                  \verb|d| & \verb|fp - 16| \\ \hline
                  \verb|e| & \verb|fp - 20| \\ \hline
                  \verb|f| & \verb|fp - 24| \\ \hline
                  \verb|g| & \verb|fp - 16| \\ \hline
                  \verb|h| & \verb|fp - 20| \\ \hline
                  \verb|i| & \verb|fp - 24| \\ \hline
               \end{tabular}
            $$
   \end{enumerate}
%%%%%%%%%%%%%%%%%%%%%%%%%%%%%%%%%%%%%%%05%%%%%%%%%%%%%%%%%%%%%%%%%%%%%%%%%%%%%%%
   \item Give an example (from a language of your choice) of a situation in
         which a variable/value outlives the lifetime of all bindings to it;
         i.e., explain how it is possible for a value to be alive even though
         all bindings to it have been lost.

      \textbf{Answer.} Consider the code snippet in C below:
\begin{verbatim}
1. #include <stdlib.h>
2.
3. int main() {
4.    double PI = 3.14;
5.    double* mean = (double*) malloc(sizeof(double));
6.    mean = &PI;
7.
8.    return 0;
9. }
\end{verbatim}

If line 5 executes successfully, then \verb|sizeof(double)| bytes will be 
allocated on the heap and the address of the first byte will be assigned to the
variable \verb|mean|. But on line 6, the address of the local variable \verb|PI| 
is assigned to \verb|mean|, so that the only binding to the \verb|double| that 
was allocated on the heap will be lost. However, the \verb|double| value will 
remain on the heap. In a more severe case, this can lead to memory leak.
%%%%%%%%%%%%%%%%%%%%%%%%%%%%%%%%%%%%%%%06%%%%%%%%%%%%%%%%%%%%%%%%%%%%%%%%%%%%%%%
   \item (Exercise 3.14 from \textit{Programming Language Pragmatics} Consider
         the following pseudocode:
         \begin{verbatim}
01.   x : integer // global value
02.
03.   procedure set_x(n : integer)
04.      x := n
05.
06.   procedure print_x()
07.      write_integer(x) // prints to screen
08.
09.   procedure first()
10.      set_x(1)
11.      print_x()
12.
13.   procedure second()
14.      x : integer // local
15.      set_x(2)
16.      print_x()
17.
18.   // main
19.   set_x(0)
20.   first()
21.   print_x()
22.   second()
23.   print_x()
         \end{verbatim}

         \begin{enumerate}
            \item What does this program print if the language uses static 
                  scoping? Why?
            \item What does this program print if the language uses dynamic
                  scoping? Why?
         \end{enumerate}

      \textbf{Answer.}

      \begin{enumerate}
         \item {[}Static scoping{]} We note that since the procedures
               \verb|set_x| and \verb|print_x| have no local variable named 
               \verb|x|, both procedures will always refer to the global
               \verb|x|. So following the execution of the code from line 19, we 
               get the following:

               \verb|set_x(0):| \\
               \verb|    global x <- 0| \\ \\
               \verb|first():| \\
               \verb|   set_x(1):| \\
               \verb|      global x <- 1| \\
               \verb|   print_x():| \\
               \verb|      print 1| \\ \\
               \verb|print_x():| \\
               \verb|   print 1| \\ \\
               \verb|second():| \\
               \verb|   declare local int x| \\
               \verb|   set_x(2):| \\
               \verb|      global x <- 2| \\
               \verb|      print_x():| \\
               \verb|         print 2| \\ \\
               \verb|print_x():| \\
               \verb|   print 2|

               From the above, we can see that the code will print:
               \verb|1 1 2 2|
         \item {[}Dynamic scoping{]} By following the execution of the code from 
               line 19, we get the following:

               \verb|set_x(0):| \\
               \verb|    global x <- 0| \\ \\
               \verb|first():| \\
               \verb|   set_x(1):| \\
               \verb|      // Use global x since it is the most recent|
               \verb|binding for x| \\
               \verb|      global x <- 1| \\ \\
               \verb|   print_x():| \\
               \verb|      // Print global x since it is the most recent|
               \verb|binding for x| \\
               \verb|      print 1| \\ \\
               \verb|print_x():| \\
               \verb|   // Print global x since it is the most recent|
               \verb|binding for x| \\
               \verb|   print 1| \\ \\
               \verb|second():| \\
               \verb|   declare local int x| \\
               \verb|   set_x(2):| \\
               \verb|      most recent x <- 2| \\
               \verb|      print_x():| \\
               \verb|      // Print most recent x| \\
               \verb|         print 2| \\ \\
               \verb|   // local int x no longer active| \\ \\
               \verb|print_x():| \\
               \verb|   // Print global x since it is the most recent|
               \verb|binding for x| \\
               \verb|   print 1|

               From the above, we can see that the code will print:
               \verb|1 1 2 1|
      \end{enumerate}
%%%%%%%%%%%%%%%%%%%%%%%%%%%%%%%%%%%%%%%07%%%%%%%%%%%%%%%%%%%%%%%%%%%%%%%%%%%%%%%
   \item Go to \verb|CodingBat.com/python| and complete these coding exercises
         in Python. Print your code to each.
         \begin{enumerate}
            \item In Warmup-1: near\_hundred, not\_string, front\_back
            \item List-2: count\_evens, big\_diff, sum67
            \item Logic-2: make\_bricks, lone\_sum
         \end{enumerate}

      \textbf{Solution.}

   

      \begin{enumerate}
         \item \begin{itemize}
                  \item \textit{Given an int n, return True if it is within 10 
                        of 100 or 200.} \\

                        \verb|def near_hundred(n):| \\
                        \verb|   return abs(100 - n) <= 10 or |
                        \verb|abs(200 - n) <= 10| \\
                  \item \textit{Given a string, return a new string where
                        ``not " has been added to the front. However, if the 
                        string already begins with ``not", return the string 
                        unchanged.} \\

                        \verb|def not_string(str):| \\
                        \verb|   return str if str[0:3] == "not" else |
                        \verb|"not " + str|
                  \item \textit{Given a string, return a new string where the 
                        first and last chars have been exchanged.} \\

                        \verb|def front_back(str):| \\
                        \verb|   l = len(str) - 1| \\
                        \verb|   return str[l] + str[1:l] + str[0] if l > 0 |
                        \verb|else str| \\
               \end{itemize}
         \item \begin{itemize}
                  \item \textit{Return the number of even ints in the given 
                        array.} \\         
   
                        \verb|def count_evens(nums):| \\
                        \verb|   i = 0| \\ \\
                        \verb|   for num in nums:| \\
                        \verb|      if num % 2 == 0:| \\
                        \verb|         i += 1| \\ \\
                        \verb|   return i| \\
                  \item \textit{Given an array length 1 or more of ints, return 
                        the difference between the largest and smallest values 
                        in the array.} \\

                        \verb|def big_diff(nums):| \\
                        \verb|   return max(nums) - min(nums)| \\
                  \item \textit{Return the sum of the numbers in the array, 
                        except ignore sections of numbers starting with a 6 and 
                        extending to the next 7 (every 6 will be followed by at 
                        least one 7). Return 0 for no numbers.} \\

                        \verb|def sum67(nums):| \\
                        \verb|   sum = 0| \\
                        \verb|   ignore = False| \\ \\
                        \verb|   for num in nums:| \\
                        \verb|      if ignore:| \\
                        \verb|         if num == 7:| \\
                        \verb|            ignore = False| \\
                        \verb|      else:| \\
                        \verb|         if num == 6:| \\
                        \verb|            ignore = True| \\
                        \verb|         else:| \\
                        \verb|            sum += num| \\ \\
                        \verb|   return sum| \\
               \end{itemize}
         \item \begin{itemize}
                  \item \textit{We want to make a row of bricks that is goal 
                        inches long. We have a number of small bricks (1 inch 
                        each) and big bricks (5 inches each). Return True if it 
                        is possible to make the goal by choosing from the given 
                        bricks.} \\         
   
                        \verb|def make_bricks(small, big, goal):| \\
                        \verb|   return small >= goal % 5 and small + 5 * big >= goal|\\
                  \item \textit{Given 3 int values, a b c, return their sum. 
                        However, if one of the values is the same as another of 
                        the values, it does not count towards the sum.} \\

                        \verb|def lone_sum(a, b, c):| \\
                        \verb|   l = [a, b, c]| \\
                        \verb|   return sum([x for x in l if |
                        \verb|l.count(x) == 1])|
               \end{itemize}
      \end{enumerate}
%%%%%%%%%%%%%%%%%%%%%%%%%%%%%%%%%%%%%%%08%%%%%%%%%%%%%%%%%%%%%%%%%%%%%%%%%%%%%%%
   \item Solve Problem 10 on ProjectEuler.net using the Python programming 
         language. Print your code and the final numerical solution. Structure 
         your code in this way:
         \begin{enumerate}
            \item A function \verb|is_prime|, which takes a single argument and
                  returns \verb|True| if the argument is a prime number. (Hint:
                  suppose $n$ is composite (not prime); what is the largest
                  integer, in terms of $n$, that could one of $n$'s divisors?)
            \item A \textbf{generator} function \verb|get_primes|, which takes a
                  single argument and \textbf{yields} an infinite sequence of
                  prime numbers starting at the given argument value. For
                  example, \verb|get_primes(7)| would yield
                  \verb|7, 11, 13, 17, ...|
            \item A function \verb|sum_primes|, which iterates over the values
                  generated by \verb|get_primes|, sums each value if it is
                  prime, and breaks when the generated prime is greater than 2
                  million.
         \end{enumerate}

      \textbf{Solution.}

      \begin{verbatim}
from math import sqrt

# n >= 2
def is_prime(n):
   if n == 2:
      return True

   if n % 2 == 0:
      return False

   upper = int(sqrt(n))
   num_divisors = 1
   i = 2

   while i <= upper:
      if n % i == 0:
         num_divisors += 1
         break

      i += 1

   return num_divisors == 1

def get_primes(n):
   while True:
      if is_prime(n):
         yield n
      
      n += 1

def sum_primes(n):
   total = 0

   for prime in get_primes(2):
      if prime < n:
         total += prime 
      else:
         break
      
   return total

print(sum_primes(2000000)) # 142913828922
      \end{verbatim}
\end{enumerate}
\end{document}
