\documentclass[9pt]{article}

\usepackage{amssymb}
\usepackage{amsmath}
\usepackage{amsfonts}
\usepackage{comment}
\usepackage{fancyhdr}
\usepackage{mathrsfs}
\usepackage{enumitem}


\usepackage{tikz}

\voffset = -50pt
%\textheight = 700pt
\addtolength{\textwidth}{60pt}
\addtolength{\evensidemargin}{-30pt}
\addtolength{\oddsidemargin}{-30pt}
%\setlength{\headheight}{44pt}

\pagestyle{fancy}
\fancyhf{} % clear all fields
\fancyhead[R]{%
  \scshape
  \begin{tabular}[t]{@{}r@{}}
  CECS 424, Summer 2015\\Section 1 (11171)\\
  HW \#5, DUE: 2015, August 11
  \end{tabular}}
\fancyhead[L]{%
  \scshape
  \begin{tabular}[t]{@{}r@{}}
  JOSEPH OKONOBOH\\Computer Science\\Cal State Long Beach
  \end{tabular}}
\fancyfoot[C]{\thepage}

\newcommand{\qed}{\hfill \ensuremath{\Box}}


\newcommand*\circled[1]{\tikz[baseline=(char.base)]{
            \node[shape=circle,draw,inner sep=2pt] (char) {#1};}}

\newcommand{\Z}{\mathbb{Z}}
\newcommand{\I}{\mathbb{I}}
\newcommand{\M}{\mathbb{M}}
\newcommand{\Q}{\mathbb{Q}}
\newcommand{\R}{\mathbb{R}}
\newcommand{\C}{\mathbb{C}}
\newcommand{\D}{\displaystyle}
%\setcounter{section}{-1}

\begin{document}
\noindent Complete these problems on a separate sheet of paper. Due August 11 at
the start of lecture.
\begin{enumerate}
%%%%%%%%%%%%%%%%%%%%%%%%%%%%%%%%%%%%%%%01%%%%%%%%%%%%%%%%%%%%%%%%%%%%%%%%%%%%%%%
   \item Reading:

         \begin{enumerate}
            \item \textit{Skim} Chapter 9.2
            \item Read Chapter 9.4
            \item \textit{Skim} Chapter 9.7
            \item Read Chapter 10 introduction
            \item \textit{Skim} Chapter 10.1
            \item Read Chapter 10.2
            \item \textit{Skim} Chapter 10.3 - Clojure is based on Scheme; so
                  much of the material here is relevant, but not all of it.
                  Focus on syntax and ideas.
            \item \textit{Skim} Chapter 10.5, 10.7      
         \end{enumerate}
%%%%%%%%%%%%%%%%%%%%%%%%%%%%%%%%%%%%%%%02%%%%%%%%%%%%%%%%%%%%%%%%%%%%%%%%%%%%%%%
   \item Given the following C++ class:

         \verb|class Animal {| \\
         \verb|public:| \\
         \verb|   int GetAge();| \\
         \verb|   virtual void Speak();| \\
         \verb|   virtual double GetCost();| \\
         \verb|   void SetAge();| \\
         \verb|};|

         \begin{enumerate}
            \item How many function pointers will be compiled into
                  \verb|class Animal|'s \textbf{vtable}?
            \item Which member functions can be \textbf{statically bound}, that
                  is, the compiler can determine at compile time what location
                  to jump to when the function is called?
            \item Which member functions must be \textbf{dynamically bound}?
         \end{enumerate}

      \textbf{Answer.}
      
      \begin{enumerate}
         \item Two.
         \item \verb|GetAge| and \verb|SetAge|.
         \item \verb|Speak| and \verb|GetCost|.
      \end{enumerate}
%%%%%%%%%%%%%%%%%%%%%%%%%%%%%%%%%%%%%%%03%%%%%%%%%%%%%%%%%%%%%%%%%%%%%%%%%%%%%%%
   \item Write a C function that is not referentially transparent.
   
      \textbf{Answer.}
      
      \verb|int not_referentially_transparent(int x);| \\ \\
      \verb|int global = 90;| \\ \\
      \verb|int main() {| \\
      \verb|   return 0;| \\
      \verb|}| \\ \\
      \verb|int not_referentially_transparent(int x) {| \\
      \verb|   ++global;| \\
      \verb|   return global;| \\
      \verb|}|
      
      The function \verb|not_referentially_transparent| is not referentially
      transparent because
      $$\verb|not_referentially_transparent(7) == not_referentially_transparent(7)|$$
      is not true, although \verb|7 == 7| is true.
%%%%%%%%%%%%%%%%%%%%%%%%%%%%%%%%%%%%%%%04%%%%%%%%%%%%%%%%%%%%%%%%%%%%%%%%%%%%%%%
   \item Write the following Clojure functions:

         \begin{enumerate}
            \item \verb|third [coll]:| return the third item of a list, or
                  \verb|nil| if there is none.
            \item \verb|mylast [coll]:| return the last item of a list, or
                  \verb|nil| if the list is empty. Use recursion.
            \item \verb|everyother [coll]:| return a list containing every other
                  element of a collection including the first. Example:
                  \verb|(everyother `(1 2 3 4 5)) -> (1 3 5)|. Use recursion.
            \item \verb|sum [coll]|: return a single value that is the additive
                  sum of all the elemens in the list. Use recursion.
            \item \verb|sumif [f coll]:| return the additive sum of all elements
                  in a list for which some predicate function \verb|f| returns
                  true. Use recursion.
            \item \verb|mybutlast [coll]|: return a list that contains all
                  elements of the given list except the last element. Use
                  recursion.
         \end{enumerate}

      \textbf{Answer.}
      
      \begin{enumerate}
         \item \verb|(defn third [coll]| \\
               \verb|  (first (next (next coll))))|
         \item \verb|(defn mylast [coll]| \\
               \verb|  (if (nil? (first (next coll)))| \\
               \verb|    (first coll)| \\
               \verb|    (mylast (next coll))))|
         \item \verb|(defn everyother [coll]| \\
               \verb|  (if (empty? coll)| \\
               \verb|    nil| \\
               \verb|    (cons (first coll) (everyother (next (next coll))))))|
         \item \verb|(defn sum [coll]| \\
               \verb|  (if (empty? coll)| \\
               \verb|    0| \\
               \verb|    (+ (first coll) (sum (next coll)))))|
         \item \verb|(defn sumif [f coll]| \\
               \verb|  (if (or (empty? coll))| \\
               \verb|    0| \\
               \verb|    (if (f (first coll))| \\
               \verb|      (+ (first coll) (sumif f (next coll)))| \\
               \verb|      (sumif f (next coll)))))|
         \item \verb|(defn mybutlast [coll]| \\
               \verb|  (if (nil? (first (next coll)))| \\
               \verb|    nil| \\
               \verb|    (cons (first coll) (mybutlast (next coll)))))|  
      \end{enumerate}
%%%%%%%%%%%%%%%%%%%%%%%%%%%%%%%%%%%%%%%05%%%%%%%%%%%%%%%%%%%%%%%%%%%%%%%%%%%%%%%
   \item Using \verb|map|, \verb|filter|, and \verb|reduce|, write a single
         Clojure expression (not a function) to compute the following:
         \begin{enumerate}
            \item Given a list of integers, calculate the sum of the squares of
                  the values in the list. Example: given \verb|`(1 2 3 4)|,
                  calculate \verb|30|.
            \item Given a list of maps, where each map has a key \verb|:name|
                  for an employee's name and a key \verb|:salary| for the
                  employee's salary, find the total salary of the employees in
                  the list. Example: given \verb|`({:name "Neal" : salary 100}|
                  \verb| {:name "Jaclyn" :salary 500})|, calculate \verb|600|.
            \item Repeat (b), except find the total salary of all employees with
                  a \verb|:salary| of at least \$50,000.
         \end{enumerate}

      \textbf{Answer.}
      
      \begin{enumerate}
         \item \verb|(reduce + (map #(* %1 %1) myListInt))|
         \item \verb|(reduce + (map #(:salary %1) myListMap))|
         \item \verb|(reduce +| \\
               \verb|  (filter #(>= %1 50000) (map #(:salary %1) myListMap)))|
      \end{enumerate}
\end{enumerate}
\end{document}
