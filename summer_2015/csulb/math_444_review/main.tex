\documentclass[9pt]{article}

\usepackage{amssymb}
\usepackage{amsmath}
\usepackage{amsfonts}
\usepackage{comment}
\usepackage{fancyhdr}
\usepackage{mathrsfs}
\usepackage{enumitem}
\usepackage[retainorgcmds]{IEEEtrantools}
\usepackage{color}


\usepackage{tikz}

\voffset = -50pt
%\textheight = 700pt
\addtolength{\textwidth}{60pt}
\addtolength{\evensidemargin}{-30pt}
\addtolength{\oddsidemargin}{-30pt}
%\setlength{\headheight}{44pt}

\newcommand{\qed}{\hfill \ensuremath{\Box}}

\newcommand{\Z}{\mathbb{Z}}
\newcommand{\I}{\mathbb{I}}
\newcommand{\M}{\mathbb{M}}
\newcommand{\R}{\mathbb{R}}
\newcommand{\D}{\displaystyle}
%\setcounter{section}{-1}

\title{Math 444 Review}
\author{Leslie Rodriguez}
\date{\today}
\setlength\parindent{0pt}
\begin{document}

   \maketitle

   \section*{Chapters 1 and 2.}

   \section*{\textcolor{purple}{Equivalence Relations.}} Let $X$ be a nonempy 
   set. A \textit{relation} on $X$ is a subset of $X \times X$, where
   $X \times X$ is the set of ordered pairs $\{(x, y) : x, y \in X\}$. We write 
   $x \sim y$ if and only if $(x, y)$ is a member of this relation. A relation
   on $X$ that satisfies the following properties:
   \begin{itemize}
      \item \textit{Reflexivity.} For each $x \in X$, we have that $x \sim x$.
      \item \textit{Symmetry.} If $x \sim y$ for some $x, y \in X$, then it
            follows that $y \sim x$.
      \item \textit{Transitivity.} If $x \sim y$ and $y \sim z$ for some
            $x, y, z \in X$, then it follows that $x \sim z$.
   \end{itemize}
   is said to be an \textit{equivalence relation}.

   \section*{Exercise 1.} Prove that if $n$ is a natural number, then
   \begin{equation}
      1^2 + 2^2 + \cdots + n^2 = \D\frac{n(n+1)(2n+1)}{6}. \label{1_1}
   \end{equation}

   \textbf{Proof.} We shall proceed by induction. \\

   \textbf{Base Case.} $n = 1$. Since $1 = \D\frac{1(1 + 1)(2\cdot1+1)}{6}$, it 
   follows that \eqref{1_1} holds whenever $n$ is 1. \\

   \textbf{Inductive Hypothesis.} Suppose that \eqref{1_1} holds for some
   positive integer $k$. That is,
   \begin{equation}
      1^2 + 2^2 + \cdots + k^2 = \D\frac{k(k+1)(2k+1)}{6}. \label{1_2}
   \end{equation}

   To complete the proof, we must now show that \eqref{1_1} holds for $k + 1$.
   Thus
   \begin{align*}
      1^2 + 2^2 + \cdots + k^2 + (k+1)^2 &= \frac{k(k+1)(2k+1)}{6} + (k+1)^2
         &[\text{From }\eqref{1_2}]\\
         &= \frac{k(k+1)(2k+1) + 6(k+1)^2}{6} \\
         &= \frac{(k+1)[k(2k+1) + 6(k+1)]}{6} \\
         &= \frac{(k+1)(2k^2+7k+6)}{6} \\
         &= \frac{(k+1)(k+2)(2k+3)}{6} \\
         &= \frac{(k+1)((k+1)+1)(2(k+1)+1)}{6},
   \end{align*}
   so that \eqref{1_1} holds for $k + 1$; thus it follows by Mathematical
   Induction that \eqref{1_1} for all natural numbers. \qed

   \section*{\textcolor{purple}{Functions.}} Let $f : X \rightarrow Y$ be a 
   function. Then this function is
   \begin{itemize}
      \item \textit{injective (or one-to-one)} if $f(x_1) = f(x_2)$, with
            $x_1, x_2 \in X$, then $x_1 = x_2$.
      \item \textit{surjective (or onto)} if for every $y \in Y$, there exists
            an $x \in X$ such that $f(x) = y$.
      \item \textit{well defined} if $x_1 = x_2$, with $x_1, x_2 \in X$, then
            $f(x_1) = f(x_2)$.
   \end{itemize}

   A function $h : S \rightarrow T$ is called an \textit{identity} on $S$ if
   $h(s) = s$ for all $s \in S$. Now consider the functions
   $f : X \rightarrow Y$ and $g : Y \rightarrow X$. The function $g$ is an
   inverse function of $f$ if $f \circ g$ is an identity on $Y$ and if
   $g \circ f$ is an identity on $X$.
\end{document}
