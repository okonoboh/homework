\documentclass{article}

\usepackage{amssymb}
\usepackage{amsmath}
\usepackage{amsfonts}
\usepackage{comment}
\usepackage{fancyhdr}
\usepackage{mathrsfs}
\usepackage{enumitem}
\usepackage[retainorgcmds]{IEEEtrantools}
\usepackage{color}


\usepackage{tikz}

\voffset = -50pt
%\textheight = 700pt
\addtolength{\textwidth}{60pt}
\addtolength{\evensidemargin}{-30pt}
\addtolength{\oddsidemargin}{-30pt}
%\setlength{\headheight}{44pt}

\newcommand{\qed}{\hfill \ensuremath{\Box}}

\newcommand{\Z}{\mathbb{Z}}
\newcommand{\I}{\mathbb{I}}
\newcommand{\M}{\mathbb{M}}
\newcommand{\R}{\mathbb{R}}
\newcommand{\D}{\displaystyle}
%\setcounter{section}{-1}

\title{Math 444 Review}
\author{Leslie Rodriguez}
\date{\today}
\setlength\parindent{0pt}
\begin{document}

   \maketitle

   \section*{Chapters 1 and 2.}

   \section*{\textcolor{purple}{Equivalence Relations.}} Let $X$ be a nonempy 
   set. A \textit{relation} on $X$ is a subset of $X \times X$, where
   $X \times X$ is the set of ordered pairs $\{(x, y) : x, y \in X\}$. We write 
   $x \sim y$ if and only if $(x, y)$ is a member of this relation. A relation
   on $X$ that satisfies the following properties:
   \begin{itemize}
      \item \textit{Reflexivity.} For each $x \in X$, we have that $x \sim x$.
      \item \textit{Symmetry.} If $x \sim y$ for some $x, y \in X$, then it
            follows that $y \sim x$.
      \item \textit{Transitivity.} If $x \sim y$ and $y \sim z$ for some
            $x, y, z \in X$, then it follows that $x \sim z$.
   \end{itemize}
   is said to be an \textit{equivalence relation}. If we have an equivalence
   relation, then for $x \in X$, we let $[x]$ denote the set of all elements of
   $X$ that are related to $x$; that is, $[x] = \{y \in X: y \sim x\}$. The set
   $[x]$ is also referred to as the \textit{equivalence class} of $x$.

   \section*{Exercise 1.} Prove that if $n$ is a natural number, then
   \begin{equation}
      1^2 + 2^2 + \cdots + n^2 = \D\frac{n(n+1)(2n+1)}{6}. \label{1_1}
   \end{equation}

   \textbf{Proof.} We shall proceed by induction on $n$. \\

   \textbf{Base Case.} $n = 1$. Since $1 = \D\frac{1(1 + 1)(2\cdot1+1)}{6}$, it 
   follows that \eqref{1_1} holds whenever $n$ is 1. \\

   \textbf{Inductive Hypothesis.} Suppose that \eqref{1_1} holds for some
   positive integer $k$. That is,
   \begin{equation}
      1^2 + 2^2 + \cdots + k^2 = \D\frac{k(k+1)(2k+1)}{6}. \label{1_2}
   \end{equation}

   To complete the proof, we must now show that \eqref{1_1} holds for $k + 1$.
   Thus
   \begin{align*}
      1^2 + 2^2 + \cdots + k^2 + (k+1)^2 &= \frac{k(k+1)(2k+1)}{6} + (k+1)^2
         &[\text{From }\eqref{1_2}]\\
         &= \frac{k(k+1)(2k+1) + 6(k+1)^2}{6} \\
         &= \frac{(k+1)[k(2k+1) + 6(k+1)]}{6} \\
         &= \frac{(k+1)(2k^2+7k+6)}{6} \\
         &= \frac{(k+1)(k+2)(2k+3)}{6} \\
         &= \frac{(k+1)((k+1)+1)(2(k+1)+1)}{6},
   \end{align*}
   so that \eqref{1_1} holds for $k + 1$; thus it follows by Mathematical
   Induction that \eqref{1_1} holds for all natural numbers. \qed

   \section*{\textcolor{purple}{Functions.}} Let $f : X \rightarrow Y$ be a 
   function. Then $f$ is
   \begin{itemize}
      \item \textit{injective (or one-to-one)} if $f(x_1) = f(x_2)$, with
            $x_1, x_2 \in X$, then $x_1 = x_2$.
      \item \textit{surjective (or onto)} if for every $y \in Y$, there exists
            an $x \in X$ such that $f(x) = y$.
      \item \textit{well defined} if $x_1 = x_2$, with $x_1, x_2 \in X$, then
            $f(x_1) = f(x_2)$.
   \end{itemize}

   A function $h : S \rightarrow T$ is called an \textit{identity} on $S$ if
   $h(s) = s$ for all $s \in S$. Now consider the functions
   $f : X \rightarrow Y$ and $g : Y \rightarrow X$. The function $g$ is an
   inverse function of $f$ if $f \circ g$ is an identity on $Y$ and if
   $g \circ f$ is an identity on $X$.

   \section*{\textcolor{purple}{Integers mod $n$.}} Let $n$ be a positive
   integer. Define a relation on $\Z$ as follows:
   $$x \sim y \text{ if and only if there exists }
     k \in \Z \text{ such that } x = y + kn.$$

   Now we shall show that this is an equivalence relation on $\Z$.

   \textbf{Proof.} Let $x$, $y$, and $z$ be integers. We have
   \begin{itemize}
      \item \textit{Reflexivity.} Clearly $x \sim x$ since $x = x + 0\cdot n$;
            so this relation is reflexive.
      \item \textit{Symmetry.} Suppose that $x \sim y$. Then it follows by
            definition that $x = y + kn$ for some integer $k$. That is,
            $y = x + (-k)n$, so that $y \sim x$. Thus this relation is
            symmetric.
      \item \textit{Transitivity.} Suppose that $x \sim y$ and $y \sim z$. By
            definition, we have that $x = y + k_1n$ and $y = z + k_2n$ for some
            integers $k_1$ and $k_2$. Thus
            $x = (z + k_2n) + k_1n = z + (k_1 + k_2)n$, so that $x \sim z$. We
            have thus shown that this relation is transitive.
   \end{itemize}

   Since the relation is reflexive, symmetric, and transitive, it follows by
   definition that it is an equivalence relation. \qed \\

   To emphasize the dependence of this relation on $n$, we 
   usually write $x \equiv y$ (mod $n$) instead of $x \sim y$. Also the 
   equivalence class of an integer $y$ is denoted $[y]_n$. \\

   \textbf{Example.} $[2]_{11} = \{\ldots, 2, 13, 24, 35, \ldots\}$.

   \section*{Exercise 2.} Proof that addition mod $n$ is well-defined; i.e.,
   $$[a]_n + [b]_n = [a + b]_n.$$

   \textbf{Proof.} Let $x \in [a]_n$ and $y \in [b]_n$. Then
   $x \equiv a$ (mod $n$) and $y \equiv b$ (mod $n$). By definition,
   $n \mid x - a$ and $n \mid y - b$. So, by definition, $x - a = np$ and
   $y - b = nq$ for some integers $p$ and $q$. Then we have:
   \begin{align*}
      (x - a) + (y - b) &= np + nq &\Rightarrow \\
      (x + y) - (a + b) &= n(p + q) &\Rightarrow,
   \end{align*}
   so $n \mid (x + y) - (a + b)$. Therefore, by definition,
   $x + y \equiv a + b$ (mod $n$), as desired. \qed

   \section*{Chapter 3.}

   \section*{\textcolor{purple}{Groups.}} A group is a set $G$ together with a
   binary operation $*$ that satisfies the following:
   \begin{enumerate}
      \item \textit{Associativity.} For all $x, y, z \in G$, we have that
            $x * (y * z) = (x * y) * z$.
      \item \textit{Identity.} There exists an element $e$ in $G$ such that
            $e * x = x = x * e$ for all $x \in G$.
      \item \textit{Inverses.} For all $g \in G$, there exists $g^{-1} \in G$
            such that $g * g^{-1} = e = g^{-1} * g$.
   \end{enumerate}

   \section*{\textcolor{purple}{Subset and Subgroups.}} Let $(G, *)$ be a
   group and let $H$ be a subset of $G$. Then $H$ is called a subgroup of $G$
   if:
   \begin{enumerate}
      \item \textit{Closure.} For all $h_1, h_2 \in H$, we have that
            $h_1 * h_2 \in H$.
      \item \textit{Identity.} $e \in H$, where $e$ is the identity of $G$.
      \item \textit{Inverses.} For all $h \in H$, there exists $h^{-1} \in H$
            such that $h * h^{-1} = e = h^{-1} * h$.
   \end{enumerate}

   or, equivalently, if
   \begin{enumerate}
      \item $H$ is not empty, and
      \item $H$ is closed under $*$; that is $h_1 * h_2 \in H$ for 
            all $h_1, h_2 \in H$.
   \end{enumerate}

   \section*{\textcolor{purple}{Some Notations.}}

   \begin{align*}
      M_n(\R)&:=\text{the set of all $n \times n$ matrices with real entries} \\
      GL_n(\R)&:=\text{the set of all $n \times n$ invertible matrices with real entries} \\
      SL_n(\R)&:=\text{the set of all $n \times n$ matrices with real entries and determinant = 1} \\
   \end{align*}

   \textit{When are the three sets defined above groups?}
   $$
      \begin{tabular}{@{}|c|c|c|c|@{}} \hline         
             & \multicolumn{2}{c|}{Binary Operation} &  \\ \hline
         Set & Addition & Multiplication & Is abelian? \\ \hline
         $M_n(\R)$ & Yes & No & Yes \\ \hline
         $GL_n(\R)$ & No & Yes & No \\ \hline
        $SL_n(\R)$ & No & Yes & No \\ \hline
      \end{tabular}
   $$

   \section*{\textcolor{purple}{Theorem 3.4.}} Let $G$ be a group. If
   $a, b \in G$, then $(ab)^{-1} = b^{-1}a^{-1}$. \\

   \textbf{Proof.} We have
   $(ab)(b^{-1}a^{-1}) = a(bb^{-1})a^{-1} = aea^{-1} = aa^{-1} = e$. By
   uniqueness, $(ab)^{-1} =\nobreak b^{-1}a^{-1}$. \qed
   
\end{document}
