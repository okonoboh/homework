\documentclass[9pt]{article}

\usepackage{amssymb}
\usepackage{amsmath}
\usepackage{amsfonts}
\usepackage{comment}
\usepackage{fancyhdr}
\usepackage{mathrsfs}
\usepackage{enumitem}
\usepackage{bm}


\usepackage{tikz}

\voffset = -50pt
%\textheight = 700pt
\addtolength{\textwidth}{60pt}
\addtolength{\evensidemargin}{-30pt}
\addtolength{\oddsidemargin}{-30pt}
%\setlength{\headheight}{44pt}

\pagestyle{fancy}
\fancyhf{} % clear all fields
\fancyhead[R]{%
  \scshape
  \begin{tabular}[t]{@{}r@{}}
  MATH 590, Fall 2015\\Section 1 (9529)\\
  HW \#01, DUE: 2015, August 31
  \end{tabular}}
\fancyhead[L]{%
  \scshape
  \begin{tabular}[t]{@{}r@{}}
  JOSEPH OKONOBOH\\Mathematics\\Cal State Long Beach
  \end{tabular}}
\fancyfoot[C]{\thepage}

\newcommand{\qed}{\hfill \ensuremath{\Box}}


\newcommand*\circled[1]{\tikz[baseline=(char.base)]{
            \node[shape=circle,draw,inner sep=2pt] (char) {#1};}}

\newcommand{\Z}{\mathbb{Z}}
\newcommand{\I}{\mathbb{I}}
\newcommand{\F}{\mathbb{F}}
\newcommand{\Q}{\mathbb{Q}}
\newcommand{\R}{\mathbb{R}}
\newcommand{\C}{\mathbb{C}}
\newcommand{\D}{\displaystyle}
%\setcounter{section}{-1}

\begin{document}
\begin{enumerate}
%%%%%%%%%%%%%%%%%%%%%%%%%%%%%%%%%%%%%%%01%%%%%%%%%%%%%%%%%%%%%%%%%%%%%%%%%%%%%%%
   \item Consider the veracity or falsehood of each of the following statements.
         For bonus, argue for those that you believe are true while providing a
         counterexample for those that you believe are false. $F$ is a field.

         \begin{enumerate}[label=\protect\circled{\arabic*}]
            \item It is possible for $a + a = 0$ for every $a \in F$ and $F$
                  be infinite.
            \item $|\{a \in \F : a^5 = 1\}| \le 5$.
            \item For any set of real numbers, there is a smallest field of real
                  numbers containing them.
            \item A field with 8 elements contains a field with 4 elements.
            \item Any field with 4 elements is isomorphic to
                  $$
                     \left\{\left(\begin{tabular}{@{}cc@{}}
                        1 & 1 \\
                        1 & 0
                     \end{tabular}\right), \left(\begin{tabular}{@{}cc@{}}
                        1 & 1 \\
                        1 & 0
                     \end{tabular}\right), \textbf{0}, \boldsymbol{I}\right\}
                  $$
                  where these matrices are seen over $\Z_2$.
         \end{enumerate}
      
      \textbf{Solution.}

      \begin{enumerate}[label=\protect\circled{\arabic*}]
         \item True.
   
               \textbf{Example.} Let $F$ be the field of fractions of the
               integral domain $\Z/2\Z[x]$. Clearly $1 \in F$ and $1 + 1 = 0$
               because 1 is also in $\Z/2\Z$. Thus
               $$a + a = a \cdot (1 + 1) = a \cdot 0 = 0$$
               for all $a \in F$. Finally we observe that $F$ is infinite since
               it contains a polynomial of degree $n$ for each positive integer
               $n$.
         \item True.

               \textbf{Proof.} Let $S$ be a nonempty set of real numbers and let
               $\mathcal{F}$ be the set of subfields of $\R$ that contain $S$.
               The set $\mathcal{F}$ is not empty because $\R \in \mathcal{F}$. 
               Now define
               $$K := \bigcap_{F \in \mathcal{F}}F.$$
               Notice that $K$ is minimal, in that if some field $K'$ contains
               $S$, then $K'$ must also contain $K$. Thus it suffices to show
               that $K$ is a subfield of $\R$. All members of $\mathcal{F}$
               contain 0 and 1 because they are fields; thus $0, 1 \in K$. Now
               let $a, b \in K$. Then $a$ and $b$ are members of all the fields
               in $\mathcal{F}$, so that $a - b$ are also members of these
               fields. Thus $a - b \in K$. That is $(K, +)$ is a subgroup of
               $(\R, +)$. Now let $c$ and $d$ be nonzero members of $K$ (there
               exists at least one, namely the multiplicative identity). We can
               similarly argue as we did before that $cd^{-1}$ is a member of
               all the fields in $\mathcal{F}$, so that $cd^{-1} \in K$. That
               is, $(K^\times, \cdot)$ is a subgroup of $(\R^\times, \cdot)$.
               Now we conclude that $K$ is a subfield of $\R$. \qed
         \item True.
         \item False.

               \textbf{Proof.} Let $H^\times$ denote the nonzero elements of a 
               field $H$. Suppose $|F| = 8$ and suppose to the contrary that $F$ 
               contains a subfield $K$ of order 4. By definition of a field,
               $(F^\times, \cdot)$ is a multiplicative group. Similarly
               $(K^\times, \cdot)$ is also a multiplicative group. Since every
               element in $K^\times$ has a multiplicative inverse and since
               $K \subseteq F$, it follows that $K^\times \subseteq F^\times$,
               so that $K^\times \le F^\times$. But $|K^\times| = 3 \nmid 7 =
               |F^\times|$, a violation of Lagrange's Theorem. So we conclude
               that a field with 8 elements cannot contain a subfield of 4
               elements. \qed
         \item True.
      \end{enumerate}
%%%%%%%%%%%%%%%%%%%%%%%%%%%%%%%%%%%%%%%02%%%%%%%%%%%%%%%%%%%%%%%%%%%%%%%%%%%%%%%
   \item Let $F$ be a field. Let $S = \{s_1 = 1, s_2, \ldots, s_k\}$ be a finite
         multiplicative subgroup of $F$, which has $k$ elements, $k \ge 2$. Let
         $\sigma = s_1 + s_2 + \cdots + s_k$ and
         $\pi = s_1 \times s_2 \times \cdots \times s_k$.

         \begin{enumerate}[start=0, label=\protect\circled{\arabic*}]
            \item Does $\sigma \in S$? Does $\pi \in S$?
            \item Compute $s_2\sigma$ and conclude something about $\sigma$.
            \item Suppose $k$ is odd. Conclude something about $\pi$.
            \item Suppose $k$ is even. Conclude something about $\pi$.
            \item Let $F = \C$. Exhibit an example for \circled{2}.
            \item Let $F = \C$. Exhibit an example for \circled{3} with $k > 2$.
         \end{enumerate}
      
      \textbf{Solution.}

      \begin{enumerate}[start=0, label=\protect\circled{\arabic*}]
         \item No, $\sigma$ is not in $S$ because $\sigma = 0$ (See \circled{1})
               and 0 has no multiplicative inverse. Yes, $\pi \in S$ because
               $S$ is a multiplicative group and thus closed under
               multiplication.
      \end{enumerate}
%%%%%%%%%%%%%%%%%%%%%%%%%%%%%%%%%%%%%%%03%%%%%%%%%%%%%%%%%%%%%%%%%%%%%%%%%%%%%%%
   \item On Weighing.

         \begin{enumerate}[label=\protect\circled{\arabic*}]
            \item Construct a matrix with 2 rows, and as many different columns
                  as possible, that satisfies \textbf{all 4} of the following
                  conditions:
                  \begin{itemize}
                     \item No column is all 0's
                     \item Every entry is either 0, 1 or $-1$.
                     \item No two columns are negatives of each other.
                     \item Every row has as many 1's as $-1$'s.
                  \end{itemize}
            \item You have a balance by which you can only compare weights. You
                  have a collection of balls (at least 3) one of which is
                  either lighter or heavier than the others. You want to find
                  out which ball is that odd one. What is the maximum number of
                  balls you could that with in just two weighings?
            \item Construct a matrix with 3 rows, and as many different columns
                  as possible, that satisfies the same \textbf{4} conditions
                  above.
            \item How many ball can you handle if you are now allowed three
                  weighings in the balance to find the odd ball? Describe the
                  weighings.
            \item How many balls could you handle in 4 weighings?
         \end{enumerate}
      
      \textbf{Solution.}
\end{enumerate}
\end{document}
