\documentclass[9pt]{article}

\usepackage{amssymb}
\usepackage{amsmath, array}
\usepackage{amsfonts}
\usepackage{comment}
\usepackage{fancyhdr}
\usepackage{mathrsfs}
\usepackage{enumitem}
\usepackage{bm}


\usepackage{tikz}

\voffset = -50pt
%\textheight = 700pt
\addtolength{\textwidth}{60pt}
\addtolength{\evensidemargin}{-30pt}
\addtolength{\oddsidemargin}{-30pt}
%\setlength{\headheight}{44pt}

\pagestyle{fancy}
\fancyhf{} % clear all fields
\fancyhead[R]{%
  \scshape
  \begin{tabular}[t]{@{}r@{}}
  MATH 590, Fall 2015\\Section 1 (9529)\\
  HW \#01, DUE: 2015, August 31
  \end{tabular}}
\fancyhead[L]{%
  \scshape
  \begin{tabular}[t]{@{}r@{}}
  JOSEPH OKONOBOH\\Mathematics\\Cal State Long Beach
  \end{tabular}}
\fancyfoot[C]{\thepage}

\newcommand{\qed}{\hfill \ensuremath{\Box}}


\newcommand*\circled[1]{\tikz[baseline=(char.base)]{
            \node[shape=circle,draw,inner sep=2pt] (char) {#1};}}

\newcommand{\Z}{\mathbb{Z}}
\newcommand{\I}{\mathbb{I}}
\newcommand{\F}{\mathbb{F}}
\newcommand{\Q}{\mathbb{Q}}
\newcommand{\R}{\mathbb{R}}
\newcommand{\C}{\mathbb{C}}
\newcommand{\D}{\displaystyle}
%\setcounter{section}{-1}

\begin{document}
\begin{enumerate}
%%%%%%%%%%%%%%%%%%%%%%%%%%%%%%%%%%%%%%%01%%%%%%%%%%%%%%%%%%%%%%%%%%%%%%%%%%%%%%%
   \item Consider the veracity or falsehood of each of the following statements.
         For bonus, argue for those that you believe are true while providing a
         counterexample for those that you believe are false. $F$ is a field.

         \begin{enumerate}[label=\protect\circled{\arabic*}]
            \item It is possible for $a + a = 0$ for every $a \in F$ and $F$
                  be infinite.
            \item $|\{a \in \F : a^5 = 1\}| \le 5$.
            \item For any set of real numbers, there is a smallest field of real
                  numbers containing them.
            \item A field with 8 elements contains a field with 4 elements.
            \item Any field with 4 elements is isomorphic to
                  $$
                     \left\{\left(\begin{tabular}{@{}cc@{}}
                        0 & 1 \\
                        1 & 1
                     \end{tabular}\right), \left(\begin{tabular}{@{}cc@{}}
                        1 & 1 \\
                        1 & 0
                     \end{tabular}\right), \textbf{0}, \boldsymbol{I}\right\}
                  $$
                  where these matrices are seen over $\Z_2$.
         \end{enumerate}
      
      \textbf{Solution.} Let $H^\times$ denote the nonzero elements of a field
      $H$. 

      \begin{enumerate}[label=\protect\circled{\arabic*}]
         \item True.
   
               \textbf{Example.} Let $F$ be the field of fractions of the
               integral domain $\Z_2[x]$. Clearly $1 \in F$ and $1 + 1 = 0$
               because 1 is also in $\Z_2$. Thus
               $$a + a = a \cdot (1 + 1) = a \cdot 0 = 0$$
               for all $a \in F$. Finally we observe that $F$ is infinite since
               it contains a polynomial of degree $n$ for each positive integer
               $n$.
         \item True.

               The statement $|\{a \in F : a^5 = 1\}| \le 5$ is logically
               equivalent to the statement: every monic monomial of degree 5 in
               $F[x]$ (where $F$ is a field) has at most 5 roots. Instead of 
               proving this specific statement, we shall prove a general one. \\

               \textbf{Lemma 1.1} \textit{The number of roots of a polynomial
               of degree n, with coefficients in a field $F$, is at most n.} \\

               \textbf{Proof.} Let $F$ be a field. We shall proceed by induction 
               on $n$. It is clear that every polynomial of degree 0 (resp. 1)
               has 0 solutions (resp. 1 solution); so our statement is true
               whenever $n$ is 0 or 1. So assume that it is also true for some
               positive integer $k$. Now suppose that $p$ is a polynomial of
               degree $k + 1$, with $p \in F[x]$. If $p$ has no roots, then we
               are done, so we can suppose that $p$ has at least one root, say
               $a$. We can thus use the Division Algorithm to write
               $$p(x) = (x - a)h(x)$$
               where $h$ is a polynomial in $F[x]$ of degree $k$. If $b$ is a
               root of $p$ then we must have that $0 = p(b) = (b - a)h(b)$. We
               know that $F[x]$ is an integral domain since $F$ is an integral
               domain; that is, $b - a = 0$ or $h(b) = 0$. This says that a root
               of $p$ must either be a root of the polynomial $x - a$ or the
               polynomial $h$. But our inductive hypothesis says that $h$ must
               have at most $k$ roots, and since we know that $x - a$ has
               exactly one root, it follows that $p$ must have at most $k + 1$
               roots. Thus Lemma 1.1 follows by Mathematical Induction. \qed \\

               Problem 1.3 follows from Lemma 1.1.
         \item True.

               \textbf{Proof.} Let $S$ be a nonempty set of real numbers and let
               $\mathcal{F}$ be the set of subfields of $\R$ that contain $S$.
               The set $\mathcal{F}$ is not empty because $\R \in \mathcal{F}$. 
               Now define
               $$K := \bigcap_{F \in \mathcal{F}}F.$$
               Notice that $K$ is minimal, in that if some field $K'$ contains
               $S$, then $K'$ must also contain $K$. Thus it suffices to show
               that $K$ is a subfield of $\R$. All members of $\mathcal{F}$
               contain 0 and 1 because they are fields; thus $0, 1 \in K$. Now
               let $a, b \in K$. Then $a$ and $b$ are members of all the fields
               in $\mathcal{F}$, so that $a - b$ are also members of these
               fields. Thus $a - b \in K$. That is $(K, +)$ is a subgroup of
               $(\R, +)$. Now let $c$ and $d$ be nonzero members of $K$ (there
               exists at least one, namely the multiplicative identity). By
               considering the multiplicative structure of the fields in
               $\mathcal{F}$, we can similarly argue as we did for the additive
               case that $cd^{-1}$ is a member of all the fields in
               $\mathcal{F}$, so that $cd^{-1} \in K$. That is,
               $(K^\times, \cdot)$ is a subgroup of $(\R^\times, \cdot)$. So
               we conclude that $K$ is the minimal subfield of $\R$ that
               contains $S$. \qed
         \item False.

               \textbf{Proof.} Suppose $|F| = 8$ and suppose to the contrary
               that $F$ contains a subfield $K$ of order 4. By definition of a
               field, $(F^\times, \cdot)$ is a multiplicative group. Similarly
               $(K^\times, \cdot)$ is also a multiplicative group. Since every
               element in $K^\times$ has a multiplicative inverse and since
               $K \subseteq F$, it follows that $K^\times \subseteq F^\times$,
               so that $K^\times \le F^\times$. But $|K^\times| = 3 \nmid 7 =
               |F^\times|$, a violation of Lagrange's Theorem. So we conclude
               that a field with 8 elements cannot contain a subfield of 4
               elements. \qed
         \item True.
         
               \textbf{Proof.} Let $K = \{a_0, a_1, a_2, a_3\}$ be an arbitrary
               field where $a_0$ is the additive identity and $a_1$ is the
               multiplicative identity. By definition of a field, we know that
               $\{a_1, a_2, a_3\}$ is a multiplicative group. Thus it follows by
               Lagrange's Theorem that (multiplicative order)
               $|a_2| = |a_3| = 3$. That is,  each of $a_2$ and $a_3$ generates
               the group $\{a_1, a_2, a_3\}$. Hence, since
               $${a_2}^1 = a_2 \text{ and } {a_2}^3 = a_1$$
               it follows that ${a_2}^2 = a_3$. Similarly, ${a_3}^2 = a_2$.
               Now multiply the equality ${a_2}^2 = a_3$ by $a_2$ to conclude
               that $a_2$ and $a_3$ are multiplicative inverses of each other.
               Thus we can fill out the multiplication table for $K$ as follows:
               $$
                  \begin{tabular}{@{}|>{$}c<{$}|>{$}c<{$}|>{$}c<{$}|>{$}c<{$}|
                     >{$}c<{$}|@{}} \hline
                     \cdot & a_0 & a_1 & a_2 & a_3 \\ \hline
                     a_0 & a_0 & a_0 & a_0 & a_0 \\ \hline
                     a_1 & a_0 & a_1 & a_2 & a_3 \\ \hline
                     a_2 & a_0 & a_2 & a_3 & a_1 \\ \hline
                     a_3 & a_0 & a_3 & a_1 & a_2 \\ \hline
                  \end{tabular}
               $$
               
               Now we shall the investigate the additive structure of $K$. By
               Lagrange's Theorem, the additive order of 1 is either 2 or 4, but
               since $K$ is a field, we know that the characteristic of $K$ must
               be prime; thus, the additive order of 1 is 2. Now we must have
               that $a + a = a \cdot (1 + 1) = a \cdot 0 = 0$ for all $a \in K$.
               So we can partially fill out a Cayley table for $(K, +)$ as
               follows:
               $$
                  \begin{tabular}{@{}|>{$}c<{$}|>{$}c<{$}|>{$}c<{$}|>{$}c<{$}|
                     >{$}c<{$}|@{}} \hline
                     + & a_0 & a_1 & a_2 & a_3 \\ \hline
                     a_0 & a_0 & a_1 & a_2 & a_3 \\ \hline
                     a_1 & a_1 & a_0 &  &  \\ \hline
                     a_2 & a_2 &  & a_0 &  \\ \hline
                     a_3 & a_3 &  &  & a_0 \\ \hline
                  \end{tabular}
               $$
               By using the cancellation laws of groups, we know that every row
               and column of a Cayley table is a permutation of the group
               elements. Hence $a_1 + a_2 = a_2$ or $a_1 + a_2 = a_3$. The
               former implies that $a_1 = a_0$, a contradiction. That is,
               $a_1 + a_2 = a_3$ and $a_1 + a_3 = a_2$. We can use similar
               arguments to fill out the rest of the table to get:
               $$
                  \begin{tabular}{@{}|>{$}c<{$}|>{$}c<{$}|>{$}c<{$}|>{$}c<{$}|
                     >{$}c<{$}|@{}} \hline
                     + & a_0 & a_1 & a_2 & a_3 \\ \hline
                     a_0 & a_0 & a_1 & a_2 & a_3 \\ \hline
                     a_1 & a_1 & a_0 & a_3 & a_2  \\ \hline
                     a_2 & a_2 & a_3 & a_0 & a_1 \\ \hline
                     a_3 & a_3 & a_2 & a_1 & a_0 \\ \hline
                  \end{tabular}
               $$
               
               Finally let $b_0 = \textbf{0}$, $b_1 = \boldsymbol{I}$, $b_2 =
               \left(\begin{tabular}{@{}cc@{}}
                  0 & 1 \\
                  1 & 1
               \end{tabular}\right)$, and $b_3 =
               \left(\begin{tabular}{@{}cc@{}}
                  1 & 1 \\
                  1 & 0
               \end{tabular}\right)$, where these matrices are over $\Z_2$. We
               want to show $\{b_0, b_1, b_2, b_3\} \cong
               \{a_0, a_1, a_2, a_3\}$. The additive and multiplicative
               structures of $\{b_0, b_1, b_2, b_3\}$ can be seen in the tables
               below:
               $$
                  \begin{tabular}{@{}|>{$}c<{$}|>{$}c<{$}|>{$}c<{$}|>{$}c<{$}|
                     >{$}c<{$}|@{}} \hline
                     \cdot & b_0 & b_1 & b_2 & b_3 \\ \hline
                     b_0 & b_0 & b_0 & b_0 & b_0 \\ \hline
                     b_1 & b_0 & b_1 & b_2 & b_3 \\ \hline
                     b_2 & b_0 & b_2 & b_3 & b_1 \\ \hline
                     b_3 & b_0 & b_3 & b_1 & b_2 \\ \hline
                  \end{tabular} \quad
                  \begin{tabular}{@{}|>{$}c<{$}|>{$}c<{$}|>{$}c<{$}|>{$}c<{$}|
                     >{$}c<{$}|@{}} \hline
                     + & b_0 & b_1 & b_2 & b_3 \\ \hline
                     b_0 & b_0 & b_1 & b_2 & b_3 \\ \hline
                     b_1 & b_1 & b_0 & b_3 & b_2  \\ \hline
                     b_2 & b_2 & b_3 & b_0 & b_1 \\ \hline
                     b_3 & b_3 & b_2 & b_1 & b_0 \\ \hline
                  \end{tabular}
               $$
               
               Notice that for $0 \le i \le 3$, if we replace $a_i$ by $b_i$ in
               the tables for the field $K$, then the resulting tables are 
               exactly what we have above. This not only tells us that
               $\{b_0, b_1, b_2, b_3\}$ is a field, but also suggests how to
               construct the desired isomorphism between these two fields. That
               is, we simply define a map that takes $a_i$ to $b_i$, and we are
               done. \qed               
      \end{enumerate}
%%%%%%%%%%%%%%%%%%%%%%%%%%%%%%%%%%%%%%%02%%%%%%%%%%%%%%%%%%%%%%%%%%%%%%%%%%%%%%%
   \item Let $F$ be a field. Let $S = \{s_1 = 1, s_2, \ldots, s_k\}$ be a finite
         multiplicative subgroup of $F$, which has $k$ elements, $k \ge 2$. Let
         $\sigma = s_1 + s_2 + \cdots + s_k$ and
         $\pi = s_1 \times s_2 \times \cdots \times s_k$.

         \begin{enumerate}[start=0, label=\protect\circled{\arabic*}]
            \item Does $\sigma \in S$? Does $\pi \in S$?
            \item Compute $s_2\sigma$ and conclude something about $\sigma$.
            \item Suppose $k$ is odd. Conclude something about $\pi$.
            \item Suppose $k$ is even. Conclude something about $\pi$.
            \item Let $F = \C$. Exhibit an example for \circled{2}.
            \item Let $F = \C$. Exhibit an example for \circled{3} with $k > 2$.
         \end{enumerate}
      
      \textbf{Solution.}

      \begin{enumerate}[start=0, label=\protect\circled{\arabic*}]
         \item No, $\sigma$ is not in $S$ because $\sigma = 0$ (See \circled{1})
               and 0 has no multiplicative inverse. Yes, $\pi \in S$ because
               $S$ is a multiplicative group and thus closed under
               multiplication.
      \end{enumerate}
%%%%%%%%%%%%%%%%%%%%%%%%%%%%%%%%%%%%%%%03%%%%%%%%%%%%%%%%%%%%%%%%%%%%%%%%%%%%%%%
   \item On Weighing.

         \begin{enumerate}[label=\protect\circled{\arabic*}]
            \item Construct a matrix with 2 rows, and as many different columns
                  as possible, that satisfies \textbf{all 4} of the following
                  conditions:
                  \begin{itemize}
                     \item No column is all 0's
                     \item Every entry is either 0, 1 or $-1$.
                     \item No two columns are negatives of each other.
                     \item Every row has as many 1's as $-1$'s.
                  \end{itemize}
            \item You have a balance by which you can only compare weights. You
                  have a collection of balls (at least 3) one of which is
                  either lighter or heavier than the others. You want to find
                  out which ball is that odd one. What is the maximum number of
                  balls you could that with in just two weighings?
            \item Construct a matrix with 3 rows, and as many different columns
                  as possible, that satisfies the same \textbf{4} conditions
                  above.
            \item How many ball can you handle if you are now allowed three
                  weighings in the balance to find the odd ball? Describe the
                  weighings.
            \item How many balls could you handle in 4 weighings?
         \end{enumerate}
      
      \textbf{Solution.}
\end{enumerate}
\end{document}
