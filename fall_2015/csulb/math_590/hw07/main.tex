\documentclass[9pt]{article}

\usepackage{amssymb}
\usepackage{amsmath, array}
\usepackage{amsfonts}
\usepackage{comment}
\usepackage{fancyhdr}
\usepackage{mathrsfs}
\usepackage{enumitem}
\usepackage{bm}
\usepackage{tikz}

\voffset = -50pt
%\textheight = 700pt
\addtolength{\textwidth}{60pt}
\addtolength{\evensidemargin}{-30pt}
\addtolength{\oddsidemargin}{-30pt}
%\setlength{\headheight}{44pt}

\pagestyle{fancy}
\fancyhf{} % clear all fields
\fancyhead[R]{%
  \scshape
  \begin{tabular}[t]{@{}r@{}}
  MATH 590, Fall 2015\\Section 1 (9529)\\
  HW \#06, DUE: 2015, October 19
  \end{tabular}}
\fancyhead[L]{%
  \scshape
  \begin{tabular}[t]{@{}r@{}}
  JOSEPH OKONOBOH\\Mathematics\\Cal State Long Beach
  \end{tabular}}
\fancyfoot[C]{\thepage}

\newcommand{\qed}{\hfill \ensuremath{\Box}}

\newcommand{\col}[2]{\left(\begin{tabular}{@{}c@{}}
   $#1$ \\
   $#2$  
 \end{tabular}\right)}

\newcommand{\li}[3]{\ell\left(\col{#1}{#2}, #3\right)}

\newcommand*\circled[1]{\tikz[baseline=(char.base)]{
            \node[shape=circle,draw,inner sep=2pt] (char) {#1};}}


\everymath{\displaystyle}
\newcommand{\Z}{\mathbb{Z}}
\newcommand{\I}{\mathbb{I}}
\newcommand{\F}{\mathbb{F}}
\newcommand{\Q}{\mathbb{Q}}
\newcommand{\R}{\mathbb{R}}
\newcommand{\C}{\mathbb{C}}
\renewcommand{\S}{\mathbb{S}}
\newcommand{\N}{\mathbb{N}}
\newcommand{\D}{\displaystyle}
%\setcounter{section}{-1}

\begin{document}
\begin{enumerate}[label=\protect\circled{\arabic*}]
%%%%%%%%%%%%%%%%%%%%%%%%%%%%%%%%%%%%%%%01%%%%%%%%%%%%%%%%%%%%%%%%%%%%%%%%%%%%%%%
   \item Consider the veracity or falsehood of each of the following statements.
         For bonus, argue for those that you believe are true while providing a
         counterexample for those that you believe are false. $G$, $K$ are
         groups, $\alpha : G \rightarrow K$ is a homomorphism.

         \begin{enumerate}[label=\protect\circled{\arabic*}]
            \item ker($\alpha$) =
                  $\{g \in G : \alpha(g) = 1\} \trianglelefteq G$.
            \item Im($\alpha$) = $\{\alpha(g) : g \in G\} \trianglelefteq K$.
            \item If $H \trianglelefteq G$ then for any $g_1$, $g_2 \in G$,
                  $(Hg_1)(Hg_2) = H(g_1g_2)$.
            \item If $\alpha$ is surjective and $H \trianglelefteq G$, then
                  $\alpha(H) \trianglelefteq K$.
            \item If $N \trianglelefteq K$, then
                  $\alpha^{-1}(N) = \{g \in G : \alpha(g) \in N\}
                   \trianglelefteq G$.
         \end{enumerate}

      \textbf{Answer.}

      \begin{enumerate}[label=\protect\circled{\arabic*}]
         \item 
      \end{enumerate}
%%%%%%%%%%%%%%%%%%%%%%%%%%%%%%%%%%%%%%%02%%%%%%%%%%%%%%%%%%%%%%%%%%%%%%%%%%%%%%%
   \item Let $H$ be a subgroup of $G$. The normalizer of $H$ in $G$, is
         defined by $N_G(H) = \{g \in G : Hg = gH\}$.

         \begin{enumerate}[label=\protect\circled{\arabic*}]
            \item $H \le N_g(H) \le G$.
            \item Prove that $N_G(H)$ is the largest subgroup of $G$ in which
                  $H$ is normal.
            \item Show that if $|G| = 2|H|$, then $N_G(H) = G$.
            \item[\text{Bonus.}] Give an example where $|G| = 3|H|$ and
                  $N_G(H) = H$.
         \end{enumerate}
%%%%%%%%%%%%%%%%%%%%%%%%%%%%%%%%%%%%%%%03%%%%%%%%%%%%%%%%%%%%%%%%%%%%%%%%%%%%%%%
   \item Let $V$ be the code generated by the matrix
         $\left(\begin{tabular}{@{}ccccccc@{}}
            1 & 0 & 0 & 0 & 1 & 1 & 1 \\
            0 & 1 & 1 & 0 & 0 & 1 & 1 \\
            0 & 0 & 1 & 1 & 1 & 0 & 1
         \end{tabular}\right)$. Use the MacWilliams equations to find the weight
         enumerators for $V$ and $V^\perp$.
\end{enumerate}
\end{document}
