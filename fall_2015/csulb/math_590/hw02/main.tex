\documentclass[9pt]{article}

\usepackage{amssymb}
\usepackage{amsmath, array}
\usepackage{amsfonts}
\usepackage{comment}
\usepackage{fancyhdr}
\usepackage{mathrsfs}
\usepackage{enumitem}
\usepackage{bm}


\usepackage{tikz}

\voffset = -50pt
%\textheight = 700pt
\addtolength{\textwidth}{60pt}
\addtolength{\evensidemargin}{-30pt}
\addtolength{\oddsidemargin}{-30pt}
%\setlength{\headheight}{44pt}

\pagestyle{fancy}
\fancyhf{} % clear all fields
\fancyhead[R]{%
  \scshape
  \begin{tabular}[t]{@{}r@{}}
  MATH 590, Fall 2015\\Section 1 (9529)\\
  HW \#02, DUE: 2015, September 09
  \end{tabular}}
\fancyhead[L]{%
  \scshape
  \begin{tabular}[t]{@{}r@{}}
  JOSEPH OKONOBOH\\Mathematics\\Cal State Long Beach
  \end{tabular}}
\fancyfoot[C]{\thepage}

\newcommand{\qed}{\hfill \ensuremath{\Box}}


\newcommand*\circled[1]{\tikz[baseline=(char.base)]{
            \node[shape=circle,draw,inner sep=2pt] (char) {#1};}}

\newcommand{\Z}{\mathbb{Z}}
\newcommand{\I}{\mathbb{I}}
\newcommand{\F}{\mathbb{F}}
\newcommand{\Q}{\mathbb{Q}}
\newcommand{\R}{\mathbb{R}}
\newcommand{\C}{\mathbb{C}}
\renewcommand{\S}{\mathbb{S}}
\newcommand{\N}{\mathbb{N}}
\newcommand{\D}{\displaystyle}
%\setcounter{section}{-1}

\begin{document}
\begin{enumerate}
%%%%%%%%%%%%%%%%%%%%%%%%%%%%%%%%%%%%%%%01%%%%%%%%%%%%%%%%%%%%%%%%%%%%%%%%%%%%%%%
   \item Consider the veracity or falsehood of each of the following statements.
         For bonus, argue for those that you believe are true while providing a
         counterexample for those that you believe are false. Here $A$ and $B$
         are adjacency matrices of graphs.

         \begin{enumerate}[label=\protect\circled{\arabic*}]
            \item $AB$ is a symmetric matrix.
            \item $A$ is regular if and only if $AJ = kJ$ for some constant $k$.
            \item $A^2 + I$ is invertible.
            \item $AJ = JA$ implies $A$ is regular.
            \item If $A$ is connected and $AJ = 3J$, then 3 only occurs once as
                  an eigenvalue of $A$.
         \end{enumerate}
%%%%%%%%%%%%%%%%%%%%%%%%%%%%%%%%%%%%%%%02%%%%%%%%%%%%%%%%%%%%%%%%%%%%%%%%%%%%%%%
   \item \textbf{On Graphs \& Their Automorphisms.} Let $G = (V, E)$ be a graph.
         Thus $V$ is the (finite) set of vertices and $E$ is the set of edges,
         and we are assuming the relation is symmetric and irreflexive: if
         $x \sim y$ then $y \sim x$, and $x \not\sim x$. A permutation or
         bijection $\alpha : V \rightarrow V$ is said to be an automorphism of
         $G$ if whenever $x \sim y$ then $\alpha(x) \sim \alpha(y)$. Let
         Aut($G$) denote the set of automorphisms.

         \begin{enumerate}[label=\protect\circled{\arabic*}]
            \item Prove Aut($G$) is a subgroup of the group of permutations of
                  $V$.
            \item Show that for any two vertices, $x, y \in V$, if there exists
                  $\alpha \in \text{Aut}(G)$ such that $\alpha(x) = y$, then
                  they have the same degree---namely the number of edges coming
                  out of each is the same.
         \end{enumerate}
%%%%%%%%%%%%%%%%%%%%%%%%%%%%%%%%%%%%%%%03%%%%%%%%%%%%%%%%%%%%%%%%%%%%%%%%%%%%%%%
   \item We are going to describe a graph with 50 vertices. The vertices $V$ are
         simply the following $(a \quad b)$ and $\left(\begin{tabular}{@{}c@{}}
            $a$ \\
            $b$  
         \end{tabular}\right)$ where $a, b \in \Z_5$. Recall that the squares
         $\S$ in $\Z_5$ are $\{1, 4\}$ and the non-squares $\N$ are $\{2, 3\}$.
         Observe $-\S = \S$ and $-\N = \N$. We now describe the edges:

         $(a \quad b) \sim (c \quad d)$ if and only if $a = c$ and
         $b - d \in \S$. Note this is symmetric. $\left(\begin{tabular}{@{}c@{}}
            $a$ \\
            $b$  
         \end{tabular}\right) \sim \left(\begin{tabular}{@{}c@{}}
            $c$ \\
            $d$  
         \end{tabular}\right)$ if and only if $a = c$ and $b - d \in \N$. Note
         this is symmetric. $(a \quad b) \sim \left(\begin{tabular}{@{}c@{}}
            $c$ \\
            $d$  
         \end{tabular}\right)$ if and only if $\left(\begin{tabular}{@{}c@{}}
            $c$ \\
            $d$  
         \end{tabular}\right) \sim (a \quad b)$ if and only if $b = ca + d$.
         Since $0 \notin \S \cup \N$, the relation is irreflexive so we have a
         graph $G$. Let $A$ be its $50 \times 50$ adjacency matrix. Do the
         following:

         \begin{enumerate}[start=0, label=\protect\circled{\arabic*}]
            \item Find the vertices connected to $(0 \quad 0)$.
                  \textbf{Hint.} Do it by cases.
            \item Find the vertices connected to $\left(\begin{tabular}{@{}c@{}}
                     0 \\
                     0  
                  \end{tabular}\right)$.
            \item Show the mapping $\alpha_x : V \rightarrow V$ defined by
                  $(a \quad b) \mapsto (a \quad b + x)$ and
                  $\left(\begin{tabular}{@{}c@{}}
                     $c$ \\
                     $d$
                  \end{tabular}\right) \mapsto \left(\begin{tabular}{@{}c@{}}
                     $c$ \\
                     $d + x$  
                  \end{tabular}\right)$ is an automorphism of the graph.
            \item Let $m \neq 0 \in \Z_5$. Show that the mapping
                  $\beta_m(a \quad b) = (ma \quad b)$ and
                  $\beta_m\left(\begin{tabular}{@{}c@{}}
                     $c$ \\
                     $d$  
                  \end{tabular}\right) = \left(\begin{tabular}{@{}c@{}}
                     $m^{-1}c$ \\
                     $d$  
                  \end{tabular}\right)$ is an automorphism of the graph.
            \item Show that $AJ = kJ$ for some $k$ and find the $k$.
            \item Show that if $(a \quad b) \sim (c \quad d)$, or
                  $\left(\begin{tabular}{@{}c@{}}
                     $a$ \\
                     $b$  
                  \end{tabular}\right) \sim \left(\begin{tabular}{@{}c@{}}
                     $c$ \\
                     $d$  
                  \end{tabular}\right)$, or $(a \quad b) \sim
                  \left(\begin{tabular}{@{}c@{}}
                     $c$ \\
                     $d$  
                  \end{tabular}\right)$, then there is no vertex joined to both.
                  \textbf{Hint.} Do cases.
            \item Show that if $(a \quad b) \not\sim (c \quad d)$, or
                  $\left(\begin{tabular}{@{}c@{}}
                     $a$ \\
                     $b$  
                  \end{tabular}\right) \not\sim \left(\begin{tabular}{@{}c@{}}
                     $c$ \\
                     $d$  
                  \end{tabular}\right)$, or $(a \quad b) \not\sim
                  \left(\begin{tabular}{@{}c@{}}
                     $c$ \\
                     $d$  
                  \end{tabular}\right)$, then there is exactly one vertex joined
                  to both.
            \item Show there exist integers $m, n, l$ such that   
                  $A^2 = mA + nI + lJ$ and find them.
            \item Find the spectrum of the graph. \textbf{Hint.} Use the
                  spectrum of $J$ and the fact that the spectrum of a polynomial
                  is the polynomial of the spectrum, and use the trace.
         \end{enumerate}
%%%%%%%%%%%%%%%%%%%%%%%%%%%%%%%%%%%%%%Bonus%%%%%%%%%%%%%%%%%%%%%%%%%%%%%%%%%%%%% 
   \item[\textbf{Bonus.}]  Let $G$ be a connected graph. If for any two vertices
                           $x, y \in V$, there exists $\alpha \in \text{Aut}(G)$
                           such that $\alpha(x) = y$, then we say the group of
                           automorphisms is \textbf{vertex transitive}.

                           \begin{enumerate}[label=\protect\circled{\arabic*}]   
                              \item Prove that if $\text{Aut}(G)$ is vertex
                                    transitive, then the graph is regular,
                                    namely the number of edges leaving a vertex
                                    is the same for any two vertices. Assume
                                    vertex transitivity. Then $\text{Aut}(G)$ is
                                    \textbf{edge transitive} when for any pairs
                                    of points, $x_1 \sim y_1$ and $x_2 \sim y_2$
                                    there exists $\alpha \in \text{Aut}(G)$ such
                                    that $\alpha(x_1) = x_2$ and
                                    $\alpha(x_2) = y_2$.
                              \item Prove that if $\text{Aut}(G)$ is
                                    \textbf{edge transitive}, and if
                                    $x_1 \not\sim y_1$ and $x_2 \not\sim y_2$,
                                    there exists $\alpha \in \text{Aut}(G)$ such
                                    that $\alpha(x_1) = x_2$ and
                                    $\alpha(x_2) = y_2$.
                              \item Prove that if $\text{Aut}(G)$ is edge
                                    transitive, and $A$ is the adjacency matrix
                                    of the graph, then there exist integers
                                    $a, b, c$ such that $A^2 = aA + bI +cJ$.
                              \item Decide if the graph in the previous problem
                                    has a vertex transitive group or an edge
                                    transitive group.
                           \end{enumerate}
\end{enumerate}
\end{document}
