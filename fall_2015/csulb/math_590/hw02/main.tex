\documentclass[9pt]{article}

\usepackage{amssymb}
\usepackage{amsmath, array}
\usepackage{amsfonts}
\usepackage{comment}
\usepackage{fancyhdr}
\usepackage{mathrsfs}
\usepackage{enumitem}
\usepackage{bm}


\usepackage{tikz}

\voffset = -50pt
%\textheight = 700pt
\addtolength{\textwidth}{60pt}
\addtolength{\evensidemargin}{-30pt}
\addtolength{\oddsidemargin}{-30pt}
%\setlength{\headheight}{44pt}

\pagestyle{fancy}
\fancyhf{} % clear all fields
\fancyhead[R]{%
  \scshape
  \begin{tabular}[t]{@{}r@{}}
  MATH 590, Fall 2015\\Section 1 (9529)\\
  HW \#02, DUE: 2015, September 09
  \end{tabular}}
\fancyhead[L]{%
  \scshape
  \begin{tabular}[t]{@{}r@{}}
  JOSEPH OKONOBOH\\Mathematics\\Cal State Long Beach
  \end{tabular}}
\fancyfoot[C]{\thepage}

\newcommand{\qed}{\hfill \ensuremath{\Box}}

\newcommand{\col}[2]{\left(\begin{tabular}{@{}c@{}}
   $#1$ \\
   $#2$  
 \end{tabular}\right)}

\newcommand*\circled[1]{\tikz[baseline=(char.base)]{
            \node[shape=circle,draw,inner sep=2pt] (char) {#1};}}

\newcommand{\Z}{\mathbb{Z}}
\newcommand{\I}{\mathbb{I}}
\newcommand{\F}{\mathbb{F}}
\newcommand{\Q}{\mathbb{Q}}
\newcommand{\R}{\mathbb{R}}
\newcommand{\C}{\mathbb{C}}
\renewcommand{\S}{\mathbb{S}}
\newcommand{\N}{\mathbb{N}}
\newcommand{\D}{\displaystyle}
%\setcounter{section}{-1}

\begin{document}
\begin{enumerate}
%%%%%%%%%%%%%%%%%%%%%%%%%%%%%%%%%%%%%%%01%%%%%%%%%%%%%%%%%%%%%%%%%%%%%%%%%%%%%%%
   \item Consider the veracity or falsehood of each of the following statements.
         For bonus, argue for those that you believe are true while providing a
         counterexample for those that you believe are false. Here $A$ and $B$
         are adjacency matrices of graphs.

         \begin{enumerate}[label=\protect\circled{\arabic*}]
            \item $AB$ is a symmetric matrix.
            \item $A$ is regular if and only if $AJ = kJ$ for some constant $k$.
            \item $A^2 + I$ is invertible.
            \item $AJ = JA$ implies $A$ is regular.
            \item If $A$ is connected and $AJ = 3J$, then 3 only occurs once as
                  an eigenvalue of $A$.
         \end{enumerate}
         
      \textbf{Solution.}

      \begin{enumerate}[label=\protect\circled{\arabic*}]
         \item False.
         
               \textbf{Counterexample.} Let
               $$A = \left(\begin{tabular}{@{}ccc@{}}
                  0 & 0 & 1 \\
                  1 & 0 & 0 \\
                  0 & 0 & 0
               \end{tabular}\right) \text{ and }
               B = \left(\begin{tabular}{@{}ccc@{}}
                  0 & 1 & 1 \\
                  1 & 0 & 1 \\
                  1 & 1 & 0
               \end{tabular}\right).$$
               It follows that $AB = \left(\begin{tabular}{@{}ccc@{}}
                  1 & 0 & 1 \\
                  0 & 1 & 1 \\
                  0 & 0 & 0
               \end{tabular}\right)$, a non-symmetric matrix.
         \item True.

               $(\Rightarrow)$ Suppose $A$ is regular. Then it follows by 
               definition that every vertex in the graph that $A$ represents has 
               the same degree; that is, there exists a nonnegative integer $k$  
               such that the number of 1s in every row of $A$ is $k$. Thus
               $AJ = kJ$.

               $(\Leftarrow)$ Conversely suppose that $AJ = kJ$ for nonnegative
               integer $k$. Notice that the $(i,j)$ entry in the product $AJ$ is 
               the sum of the entries of row $i$. Since $AJ = kJ$, it follows
               that every $(i, j)$ entry is $k$, so that the sum of every row is
               $k$. Thus the number of 1s in every row of $A$ must be the same,
               so that $A$ is regular. Thus we conclude that $A$ is regular if 
               and only if $AJ = kJ$ for some constant $k$. \qed
         \item True.
         
               \textbf{Proof.} Let $x$ be an eigenvalue of $A$. Since $A$ is
               symmetric, we know from Linear Algebra that $x$ must be a real
               number. Now since the spectrum of a polynomial is the polynomial
               of the spectrum, it follows that $x^2 + 1$ must be an eigenvalue
               of $A^2 + I$. But since $x$ is real, the number $x^2 + 1$ can
               never be 0. Thus $A^2 + I$ has no zero eigenvalues; that is,
               $A^2 + I$ has a nonzero determinant, so that it is invertible.
               \mbox{ } \qed
         \item True.

               \textbf{Proof.} Suppose that $AJ = JA$. We previously noted that
               the $(i, j)$ entry of the product $AJ$ is the sum of the entries
               of row $i$, while the $(i, j)$ entry of the product $JA$ is the 
               sum of the entries of column $j$. Since $AJ = JA$, it follows
               that the sum of the entries in each row and each column must all
               be equal. Particularly, the sum of the entries in each row must
               be equal, so that the number of 1s in each row of $A$ is the
               same. Hence, we conclude that $A$ is regular. \qed
      \end{enumerate}
      
      
%%%%%%%%%%%%%%%%%%%%%%%%%%%%%%%%%%%%%%%02%%%%%%%%%%%%%%%%%%%%%%%%%%%%%%%%%%%%%%%
   \item \textbf{On Graphs \& Their Automorphisms.} Let $G = (V, E)$ be a graph.
         Thus $V$ is the (finite) set of vertices and $E$ is the set of edges,
         and we are assuming the relation is symmetric and irreflexive: if
         $x \sim y$ then $y \sim x$, and $x \not\sim x$. A permutation or
         bijection $\alpha : V \rightarrow V$ is said to be an automorphism of
         $G$ if whenever $x \sim y$ then $\alpha(x) \sim \alpha(y)$. Let
         Aut($G$) denote the set of automorphisms.

         \begin{enumerate}[label=\protect\circled{\arabic*}]
            \item Prove Aut($G$) is a subgroup of the group of permutations of
                  $V$.
            \item Show that for any two vertices, $x, y \in V$, if there exists
                  $\alpha \in \text{Aut}(G)$ such that $\alpha(x) = y$, then
                  they have the same degree---namely the number of edges coming
                  out of each is the same.
         \end{enumerate}
         
      \textbf{Proof.}

      \begin{enumerate}[label=\protect\circled{\arabic*}]
         \item The set $\text{Aut}(G)$ is not empty since it contains the
               identity permutation---the permutation that fixes all vertices of
               $V$. Associativity is already taken care of because the group of
               permutations is associative under function composition; the
               identity permutation serves as the identity for Aut($G$), and
               since Aut($G$) is finite, it is also closed under taking
               inverses. So we only need to show closure under function
               composition. To that end, let
               $\beta, \gamma \in \text{Aut}(G)$. Suppose we have $x \sim y$ for
               some $x, y \in V$. Since $\gamma$ is a member of Aut($G$), it
               follows that $\gamma(x) \sim \gamma(y)$. Similarly, since
               $\gamma(x)$ and $\gamma(y)$ are members of $V$, it follows that
               $\beta(\gamma(x)) \sim \beta(\gamma(y))$, so that
               $(\beta\circ\gamma)(x) \sim (\beta\circ\gamma)(y)$; that is,
               $\beta\circ\gamma \in \text{Aut}(G)$, and so we conclude that
               $\text{Aut}(G) \le S_V$.
         \item Let $S_v = \{u \in V: u \sim v\}$, where $v \in V$. It is clear
               that for $v \in V$, we have $\text{degree($v$)} = |S_v|$. Now
               suppose that there exist $x, y \in V$ and
               $\alpha \in \text{Aut}(G)$ such that $\alpha(x) = y$. We want to
               show that degree($x$) = degree($y$); that is, $|S_x| = |S_y|$.
               Assume that $S_x$ is not empty. So let
               $a \in S_x$, so that $x \sim a$. It follows that
               $y = \alpha(x) \sim \alpha(a)$; that is, $\alpha(a) \in S_y$.
               Since $\alpha$ is injective, it must therefore map all elements
               of $S_x$ to unique elements in $S_y$; thus $|S_x| \le |S_y|$.
               Now let $b \in S_y$, so that $y \sim b$; recall that Aut($G$) is
               a group, so we must have $\alpha^{-1} \in \text{Aut}(G)$. Thus
               since $y \sim b$, it follows that
               $x = \alpha^{-1}(y) \sim \alpha^{-1}(b)$, so that
               $\alpha^{-1}(b) \in S_x$. Similarly, we conclude that
               $\alpha^{-1}$ maps elements of $S_y$ to unique elements in $S_x$
               because it is injective. Thus $|S_y| \le |S_x|$ and we conclude
               that $|S_x| = |S_y|$, so that degree($x$) = degree($y$). Now if
               $S_x$ is empty, then our arguments above show us that $S_y$ must
               also be empty (since the preimage of the elements of $S_y$ under
               $\alpha$ are precisely the elements of $S_x$), and the proof is
               complete.\qed
      \end{enumerate}
      
      
%%%%%%%%%%%%%%%%%%%%%%%%%%%%%%%%%%%%%%%03%%%%%%%%%%%%%%%%%%%%%%%%%%%%%%%%%%%%%%%
   \item We are going to describe a graph with 50 vertices. The vertices $V$ are
         simply the following $(a \quad b)$ and $\col{a}{b}$ where
         $a, b \in \Z_5$. Recall that the squares $\S$ in $\Z_5$ are $\{1, 4\}$
         and the non-squares $\N$ are $\{2, 3\}$. Observe $-\S = \S$ and
         $-\N = \N$. We now describe the edges:

         $(a \quad b) \sim (c \quad d)$ if and only if $a = c$ and
         $b - d \in \S$. Note this is symmetric. $\col{a}{b} \sim \col{c}{d}$
         if and only if $a = c$ and $b - d \in \N$. Note this is symmetric.
         $(a \quad b) \sim \col{c}{d}$ if and only if
         $\col{c}{d} \sim (a \quad b)$ if and only if $b = ca + d$. Since
         $0 \notin \S \cup \N$, the relation is irreflexive so we have a
         graph $G$. Let $A$ be its $50 \times 50$ adjacency matrix. Do the
         following:

         \begin{enumerate}[start=0, label=\protect\circled{\arabic*}]
            \item Find the vertices connected to $(0 \quad 0)$.
                  \textbf{Hint.} Do it by cases.
            \item Find the vertices connected to $\col{0}{0}$.
            \item Show the mapping $\alpha_x : V \rightarrow V$ defined by
                  $(a \quad b) \mapsto (a \quad b + x)$ and
                  $\col{c}{d} \mapsto \col{c}{d+x}$ is an automorphism of the
                  graph.
            \item Let $m \neq 0 \in \Z_5$. Show that the mapping
                  $\beta_m(a \quad b) = (ma \quad b)$ and
                  $\col{c}{d} = \col{m^{-1}c}{d}$ is an automorphism of the
                  graph.
            \item Show that $AJ = kJ$ for some $k$ and find the $k$.
            \item Show that if $(a \quad b) \sim (c \quad d)$, or
                  $\col{a}{b} \sim \col{c}{d}$, or
                  $(a \quad b) \sim \col{c}{d}$, then there is no vertex joined
                  to both. \textbf{Hint.} Do cases.
            \item Show that if $(a \quad b) \not\sim (c \quad d)$, or
                  $\col{a}{b} \not\sim \col{c}{d}$, or
                  $(a \quad b) \not\sim\col{c}{d}$, then there is exactly one
                  vertex joined to both.
            \item Show there exist integers $m, n, l$ such that   
                  $A^2 = mA + nI + lJ$ and find them.
            \item Find the spectrum of the graph. \textbf{Hint.} Use the
                  spectrum of $J$ and the fact that the spectrum of a polynomial
                  is the polynomial of the spectrum, and use the trace.
         \end{enumerate}
         
      \textbf{Solution.}

      \begin{enumerate}[start=0, label=\protect\circled{\arabic*}]
         \item Suppose that $(0 \quad 0) \sim (a \quad b)$ and
               $(0 \quad 0)\sim \col{c}{d}$. It follows that $a = 0$ and
               $b - 0 = b \in \S$ and $0 = 0 + d$. Thus the vertices
               connected to $(0 \quad 0)$ are:
               $$(0 \quad 1), (0 \quad 4), \col{0}{0}, \col{1}{0}, \col{2}{0},
               \col{3}{0}, \text{ and }\col{4}{0}.$$
         \item Now suppose that $\col{0}{0} \sim (a \quad b)$ and 
               $\col{0}{0} \sim \col{c}{d}$. Then it follows that
               $b = 0 \cdot a + 0 = 0$, $c = 0$, and $d - 0 = d \in \N$. Thus
               the vertices connected to $(0 \quad 0)$ are:
               $$(0 \quad 0), (1 \quad 0), (2 \quad 0), (3 \quad 0),
                 (4 \quad 0), \col{0}{2}, \text{ and } \col{0}{3}.$$
         \item The map $\alpha_{-x}$ is a two sided inverse of $\alpha_x$;
               so $\alpha_x$ is bijective and thus a permutation of $V$.
               
               \textbf{Case I}. $(a \quad b) \sim (c \quad d)$; that is, $a = c$
               and $b - d \in \S$. It follows that
               \begin{align*}
                  \alpha_x((a \quad b)) &= (a \quad b+x) \\
                                        &= (c \quad b+x) &[\text{Since }a = c]\\
                                        &\sim (c \quad d+x) &[b+x - (d+x)=b-d \in \S]\\ 
                                        &= \alpha_x((c \quad d)),
               \end{align*}
               so that $ \alpha_x((a \quad b)) \sim  \alpha_x((c \quad d)).$
               
               \textbf{Case II}. $\col{a}{b} \sim \col{c}{d}$. The proof is
               similar to Case I (however, replace $\S$ with $\N$).
               
               \textbf{Case III}. $(a \quad b) \sim \col{c}{d}$.
      \end{enumerate}
      
      
%%%%%%%%%%%%%%%%%%%%%%%%%%%%%%%%%%%%%%Bonus%%%%%%%%%%%%%%%%%%%%%%%%%%%%%%%%%%%%% 
   \item[\textbf{Bonus.}]  Let $G$ be a connected graph. If for any two vertices
                           $x, y \in V$, there exists $\alpha \in \text{Aut}(G)$
                           such that $\alpha(x) = y$, then we say the group of
                           automorphisms is \textbf{vertex transitive}.

                           \begin{enumerate}[label=\protect\circled{\arabic*}]   
                              \item Prove that if $\text{Aut}(G)$ is vertex
                                    transitive, then the graph is regular,
                                    namely the number of edges leaving a vertex
                                    is the same for any two vertices. Assume
                                    vertex transitivity. Then $\text{Aut}(G)$ is
                                    \textbf{edge transitive} when for any pairs
                                    of points, $x_1 \sim y_1$ and $x_2 \sim y_2$
                                    there exists $\alpha \in \text{Aut}(G)$ such
                                    that $\alpha(x_1) = x_2$ and
                                    $\alpha(x_2) = y_2$.
                              \item Prove that if $\text{Aut}(G)$ is
                                    \textbf{edge transitive}, and if
                                    $x_1 \not\sim y_1$ and $x_2 \not\sim y_2$,
                                    there exists $\alpha \in \text{Aut}(G)$ such
                                    that $\alpha(x_1) = x_2$ and
                                    $\alpha(x_2) = y_2$.
                              \item Prove that if $\text{Aut}(G)$ is edge
                                    transitive, and $A$ is the adjacency matrix
                                    of the graph, then there exist integers
                                    $a, b, c$ such that $A^2 = aA + bI +cJ$.
                              \item Decide if the graph in the previous problem
                                    has a vertex transitive group or an edge
                                    transitive group.
                           \end{enumerate}
\end{enumerate}
\end{document}
