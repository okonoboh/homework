\documentclass[9pt]{article}

\usepackage{amssymb}
\usepackage{amsmath, array}
\usepackage{amsfonts}
\usepackage{comment}
\usepackage{fancyhdr}
\usepackage{mathrsfs}
\usepackage{enumitem}
\usepackage{bm}
\usepackage{tikz}

\voffset = -50pt
%\textheight = 700pt
\addtolength{\textwidth}{60pt}
\addtolength{\evensidemargin}{-30pt}
\addtolength{\oddsidemargin}{-30pt}
%\setlength{\headheight}{44pt}

\pagestyle{fancy}
\fancyhf{} % clear all fields
\fancyhead[R]{%
  \scshape
  \begin{tabular}[t]{@{}r@{}}
  MATH 590, Fall 2015\\Section 1 (9529)\\
  HW \#08, DUE: 2015, October 26
  \end{tabular}}
\fancyhead[L]{%
  \scshape
  \begin{tabular}[t]{@{}r@{}}
  JOSEPH OKONOBOH\\Mathematics\\Cal State Long Beach
  \end{tabular}}
\fancyfoot[C]{\thepage}

\newcommand{\qed}{\hfill \ensuremath{\Box}}

\newcommand{\col}[2]{\left(\begin{tabular}{@{}c@{}}
   $#1$ \\
   $#2$  
 \end{tabular}\right)}

\newcommand{\li}[3]{\ell\left(\col{#1}{#2}, #3\right)}

\newcommand*\circled[1]{\tikz[baseline=(char.base)]{
            \node[shape=circle,draw,inner sep=2pt] (char) {#1};}}


\everymath{\displaystyle}
\newcommand{\Z}{\mathbb{Z}}
\newcommand{\I}{\mathbb{I}}
\newcommand{\F}{\mathbb{F}}
\newcommand{\Q}{\mathbb{Q}}
\newcommand{\R}{\mathbb{R}}
\newcommand{\C}{\mathbb{C}}
\renewcommand{\S}{\mathbb{S}}
\newcommand{\N}{\mathbb{N}}
\newcommand{\D}{\displaystyle}
%\setcounter{section}{-1}

\begin{document}
\begin{enumerate}[label=\protect\circled{\arabic*}]
%%%%%%%%%%%%%%%%%%%%%%%%%%%%%%%%%%%%%%%01%%%%%%%%%%%%%%%%%%%%%%%%%%%%%%%%%%%%%%%
   \item Consider the veracity or falsehood of each of the following statements.
         For bonus, argue for those that you believe are true while providing a
         counterexample for those that you believe are false.
         \begin{enumerate}[label=\protect\circled{\arabic*}]
            \item If $H \trianglelefteq K$ and $K \trianglelefteq G$, then
                  $H \trianglelefteq G$.
            \item If $H \le K$ and $K \le G$, then $H \le G$.
            \item Every group of order 91 is cyclic.
            \item There exists an automorphism of the Petersen graph of order 6.
            \item There exists an automorphism of the Petersen graph of order 4.
         \end{enumerate}

      \textbf{Answer.}

      \begin{enumerate}[label=\protect\circled{\arabic*}]
         \item TODO
      \end{enumerate}
%%%%%%%%%%%%%%%%%%%%%%%%%%%%%%%%%%%%%%%02%%%%%%%%%%%%%%%%%%%%%%%%%%%%%%%%%%%%%%%
   \item Let $V = \{\{x, y\} : x, y \in \{1, 2, 3, 4, 5\}\}$.

         \begin{enumerate}[label=\protect\circled{\arabic*}]
            \item What is $|V|$?
         \end{enumerate}
    
         For $S, T \in V$, define $S \sim T$ if and only if
         $S \cap T = \emptyset$. This is a graph $\mathcal{G}$ on $V$.
         \begin{enumerate}[label=\protect\circled{\arabic*}, start=2]
            \item What is the degree of $\{1, 2\}$?
            \item Show that any element of $\sigma \in S_5$ can be viewed as an
                  automorphism of $\mathcal{G}$.
            \item Find 5 vertices that form a pentagon.
            \item Show this graph is the Petersen graph.
         \end{enumerate}
         
      \textbf{Solution.}
      
      \begin{enumerate}[label=\protect\circled{\arabic*}]
         \item TODO
      \end{enumerate}
%%%%%%%%%%%%%%%%%%%%%%%%%%%%%%%%%%%%%%%03%%%%%%%%%%%%%%%%%%%%%%%%%%%%%%%%%%%%%%%
   \item Let $G$ act on set $X$.
         \begin{enumerate}[label=\protect\circled{\arabic*}]
            \item Suppose $O_x = O_y$. Show that $G_x$ and $G_y$ are conjugate
                  subgroups.
            \item Suppose $G$ acts transitively on $X$. Suppose $G_x$ acts
                  transitively on $X\backslash\{x\}$. Then show that $G$ is 
                  doubly transitive on $X$ which means that for any
                  $y_1 \neq z_1$ and $y_2 \neq z_2$, there exists $g \in G$ such 
                  that $gy_1 = y_2$ and $gz_1 = z_2$.
            \item Suppose $G$ is doubly transitive on $X$ and suppose that
                  $G_{xy} = \{g : gx = x, gy = y\}$ acts transitively on
                  $X\backslash\{x, y\}$. Show that $G$ is triply transitive: 
                  which means that for any three (distinct) elements
                  $z_1, u_1, v_1 \in X$ and any three (distinct) elements
                  $z_2, u_2, v_2 \in X$, there exists $g \in G$ such that
                  $gz_1 = z_2$, $gu_1 = u_2$, and $gv_1 = v_2$.
         \end{enumerate}
\end{enumerate}
\end{document}
