\documentclass[9pt]{article}

\usepackage{amssymb}
\usepackage{amsmath, array}
\usepackage{amsfonts}
\usepackage{comment}
\usepackage{fancyhdr}
\usepackage{mathrsfs}
\usepackage{enumitem}
\usepackage{bm}
\usepackage{tikz}

\voffset = -50pt
%\textheight = 700pt
\addtolength{\textwidth}{60pt}
\addtolength{\evensidemargin}{-30pt}
\addtolength{\oddsidemargin}{-30pt}
%\setlength{\headheight}{44pt}

\pagestyle{fancy}
\fancyhf{} % clear all fields
\fancyhead[R]{%
  \scshape
  \begin{tabular}[t]{@{}r@{}}
  MATH 590, Fall 2015\\Section 1 (9529)\\
  HW \#08, DUE: 2015, October 26
  \end{tabular}}
\fancyhead[L]{%
  \scshape
  \begin{tabular}[t]{@{}r@{}}
  JOSEPH OKONOBOH\\Mathematics\\Cal State Long Beach
  \end{tabular}}
\fancyfoot[C]{\thepage}

\newcommand{\qed}{\hfill \ensuremath{\Box}}

\newcommand{\col}[2]{\left(\begin{tabular}{@{}c@{}}
   $#1$ \\
   $#2$  
 \end{tabular}\right)}

\newcommand{\li}[3]{\ell\left(\col{#1}{#2}, #3\right)}

\newcommand*\circled[1]{\tikz[baseline=(char.base)]{
            \node[shape=circle,draw,inner sep=2pt] (char) {#1};}}


\everymath{\displaystyle}
\newcommand{\Z}{\mathbb{Z}}
\newcommand{\I}{\mathbb{I}}
\newcommand{\F}{\mathbb{F}}
\newcommand{\Q}{\mathbb{Q}}
\newcommand{\R}{\mathbb{R}}
\newcommand{\C}{\mathbb{C}}
\renewcommand{\S}{\mathbb{S}}
\newcommand{\N}{\mathbb{N}}
\newcommand{\D}{\displaystyle}
%\setcounter{section}{-1}

\begin{document}
\begin{enumerate}[label=\protect\circled{\arabic*}]
%%%%%%%%%%%%%%%%%%%%%%%%%%%%%%%%%%%%%%%01%%%%%%%%%%%%%%%%%%%%%%%%%%%%%%%%%%%%%%%
   \item Consider the veracity or falsehood of each of the following statements.
         For bonus, argue for those that you believe are true while providing a
         counterexample for those that you believe are false.
         \begin{enumerate}[label=\protect\circled{\arabic*}]
            \item If $H \trianglelefteq K$ and $K \trianglelefteq G$, then
                  $H \trianglelefteq G$.
            \item If $H \le K$ and $K \le G$, then $H \le G$.
            \item Every group of order 91 is cyclic.
            \item There exists an automorphism of the Petersen graph of order 6.
            \item There exists an automorphism of the Petersen graph of order 4.
         \end{enumerate}

      \textbf{Answer.}

      \begin{enumerate}[label=\protect\circled{\arabic*}]
         \item False.
         
               \textbf{Counterexample.} Let
               $G = D_8 = \{1, r, r^2, r^3, s, sr, sr^2, sr^3\}$
               (the group of isometries of a square), $K = \{1, s, r^2, sr^2\}$,
               and $H = \{1, s\}$. We have that $H \le K$ and $K \le G$.
               Moreover, since $|K| = 2|H|$ and $|G| = 2|K|$, it follows by
               homework 7 Problem 2.3 that $H \trianglelefteq K$ and
               $K \trianglelefteq G$. Observe that
               $rHr^{-1} = \{1, sr^2\} \neq H$, so that $r \notin N_H(G)$; that
               is, $N_H(G) \neq G$, and it follows that $H$ is not normal in
               $G$.
         \item True.
         
               Suppose that $H \le K$ and $K \le G$. It follows immediately that
               $H \le G$ because $H \subseteq G$ and $H$ is closed under
               multiplication and taking inverses.
         \item True.
         \item True. (See Problem 2.3) We can view every $\sigma \in S_5$ as an
               automorphism of the Petersen graph. Let
               $\sigma =(1\;2\;3)(4\;5)$ so that $|\sigma| = 6$.
         \item True. Similarly let $\sigma = (1\;2\;3\;4) \in S_5$, so that
               $|\sigma| = 4$.
      \end{enumerate}
%%%%%%%%%%%%%%%%%%%%%%%%%%%%%%%%%%%%%%%02%%%%%%%%%%%%%%%%%%%%%%%%%%%%%%%%%%%%%%%
   \item Let $V = \{\{x, y\} : x, y \in \{1, 2, 3, 4, 5\}\}$.

         \begin{enumerate}[label=\protect\circled{\arabic*}]
            \item What is $|V|$?
         \end{enumerate}
    
         For $S, T \in V$, define $S \sim T$ if and only if
         $S \cap T = \emptyset$. This is a graph $\mathcal{G}$ on $V$.
         \begin{enumerate}[label=\protect\circled{\arabic*}, start=2]
            \item What is the degree of $\{1, 2\}$?
            \item Show that any element of $\sigma \in S_5$ can be viewed as an
                  automorphism of $\mathcal{G}$.
            \item Find 5 vertices that form a pentagon.
            \item Show this graph is the Petersen graph.
         \end{enumerate}
         
      \textbf{Solution.}
      
      \begin{enumerate}[label=\protect\circled{\arabic*}]
         \item $|V| = \binom{5}{2} = 10$.
         \item The degree of \{1, 2\} is 3 because it is connected to exactly
               three other vertices: \{3, 4\}, \{3, 5\}, and \{4, 5\}.
         \item Let $I_5 = \{1, 2, 3, 4, 5\}$, $\sigma \in S_5$ and consider the
               map $g : S_5 \times V \rightarrow V$, defined
               by $(\sigma, \{h, k\}) \mapsto \{\sigma(h), \sigma(k)\}$. We
               shall write $\sigma\{h, k\}$ instead of $g(\sigma, \{h, k\})$.
               Now we want to show that $S_5$ acts on $V$. So let
               $\sigma, \beta \in S_5$ and $\{h, k\} \in V$. Thus
               $$(\sigma \circ \beta)\{h, k\} = \{(\sigma \circ \beta)h,
                 (\sigma \circ \beta)k\} = \{\sigma(\beta(h)),
                 \sigma(\beta(k))\} = \sigma\{\beta(h), \beta(k)\}.$$
               Finally we have that
               $1\{h, k\} = \{1h, 1k\} = \{h, k\}$. Thus the identity in $S_5$
               fixes every vertex in $V$. We have shown that $S_5$ acts on $V$.
               By our discussion of group actions, we know that every element
               of $S_5$ induces a permutation of $V$. For $\sigma \in S_5$, we
               shall identity this permutation as $f_\sigma$, where
               $f_\sigma : V \rightarrow V$, defined by
               $\{h, k\} \mapsto g(\sigma, \{h, k\})$. To complete the proof, we
               want to now show that $f_\sigma$ is an automorphism; that is, we
               must now show that $f_\sigma$ maps every pair of connected
               vertices to another pair of  connected vertices. To that end, let
               $\{a, b\}$ and $\{c, d\}$ be arbitrary connected vertices in
               $V$; that is,
               $$a, b, c, d \in I_5, a \neq b, c \neq d,
               \text{ and } \{a, b\} \cap \{c, d\} = \emptyset.$$
               Suppose to the contrary that $f_\sigma(\{a, b\})$ and
               $f_\sigma(\{c, d\})$ are not connected; thus, it follows by
               definition that
               $$\{\sigma(a), \sigma(b)\} \cap \{\sigma(c), \sigma(d)\} \neq
                 \emptyset.$$               
               So assume without loss of generality that
               $\sigma(a) = \sigma(c)$; thus $a = c$ because $\sigma$ is
               injective. This is a contradiction since we assumed that
               $\{a, b\} \cap \{c, d\} = \emptyset$, so we conclude that
               $f_\sigma(\{a, b\}) \cap f_\sigma(\{c, d\}) = \emptyset$; that
               is, $f_\sigma(\{a, b\})$ is connected to $f_\sigma(\{c, d\})$,
               and the proof is complete. \qed
         \item Since
               $$\{1, 2\} \sim \{4, 5\} \sim \{1, 3\}
                  \sim \{2, 5\} \sim \{3, 4\} \sim \{1, 2\},$$
               it follows that the vertices above form a pentagon.
         \item The graph below represents $\mathcal{G}$. Relabelling the
               graph shows us that it is the Petersen graph. \\ \\ \\ \\ \\ \\ \\ \\ \\ \\
      \end{enumerate}
%%%%%%%%%%%%%%%%%%%%%%%%%%%%%%%%%%%%%%%03%%%%%%%%%%%%%%%%%%%%%%%%%%%%%%%%%%%%%%%
   \item Let $G$ act on set $X$.
         \begin{enumerate}[label=\protect\circled{\arabic*}]
            \item Suppose $O_x = O_y$. Show that $G_x$ and $G_y$ are conjugate
                  subgroups.
            \item Suppose $G$ acts transitively on $X$. Suppose $G_x$ acts
                  transitively on $X\backslash\{x\}$. Then show that $G$ is 
                  doubly transitive on $X$ which means that for any
                  $y_1 \neq z_1$ and $y_2 \neq z_2$, there exists $g \in G$ such 
                  that $gy_1 = y_2$ and $gz_1 = z_2$.
            \item Suppose $G$ is doubly transitive on $X$ and suppose that
                  $G_{xy} = \{g : gx = x, gy = y\}$ acts transitively on
                  $X\backslash\{x, y\}$. Show that $G$ is triply transitive: 
                  which means that for any three (distinct) elements
                  $z_1, u_1, v_1 \in X$ and any three (distinct) elements
                  $z_2, u_2, v_2 \in X$, there exists $g \in G$ such that
                  $gz_1 = z_2$, $gu_1 = u_2$, and $gv_1 = v_2$.
         \end{enumerate}
         
      \textbf{Solution.}      
      
      \begin{enumerate}[label=\protect\circled{\arabic*}]
         \item Suppose $O_x = O_y$. Particularly, we have that $x$ and $y$ are
               in the same orbit; thus $gx = y$ for some $g \in G$. Now we
               claim that $gG_xg^{-1} = G_y$;
               
               $(\subseteq):$ Let $a \in gG_xg^{-1}$. Then $a = ghg^{-1}$ for
               some $h \in G_x$. It follows that
               $$(ghg^{-1})y = (gh)(g^{-1}y) = (gh)(x) = g(h(x)) = gx = y.$$
               That is, $ghg^{-1} \in G_y$, and we conclude that
               $gG_xg^{-1} \subseteq G_y$.
               
               $(\supseteq):$ Let $b \in G_y$. Now we have that
               $$(g^{-1}bg)x = (g^{-1}b)(gx) = (g^{-1}b)y = g^{-1}(by) =
                 g^{-1}y = x,$$
               so that $g^{-1}bg \in G_x$. Since
               $b = g(g^{-1}bg)g^{-1} \in gG_xg^{-1}$, it follows that
               $G_y \subseteq  gG_xg^{-1}$; thus $gG_xg^{-1} = G_y$, and we
               conclude that $G_x$ and $G_y$ are conjugates.
         \item Suppose $G$ acts transitively on $X$; also suppose that $G_i$
               acts on $X\backslash\{i\}$ for all $i \in X$. Consider
               $y_1, z_1, y_2, z_2 \in X$, with $y_1 \neq z_1$ and
               $y_2 \neq z_2$. Since $G$ acts transitively on $X$, there exists
               $g \in G$ such that $gy_1 =  y_2$. If $gz_1 = y_2$, then we would
               have $gy_1 = gz_1$, so that $y_1 = z_1$, a contradiction; thus
               $gz_1 \neq y_2$, and we have that
               $gz_1, z_2 \in X\backslash\{y_2\}$, so that there exists
               $h \in G_{y_2}$ such that $h(gz_1) = z_2$. Hence
               $(hg)y_1 = h(gy_1) = h(y_2) = y_2$ and
               $(hg)z_1 = h(gz_1) = z_2$. Thus $G$ is doubly transitive on $X$.
         \item Suppose $G$ acts doubly transitively on $X$ and suppose that
               $G_{xy} = \{g : gx = x, gy = y\}$. Let $x_1$, $y_1$, and $z_1$ be
               different points in $X$ and $x_2$, $y_2$, and $z_2$ be
               different points in $X$. Since $G$ is doubly transitive on $X$
               there exists $g \in G$ such that $gx_1 = x_2$ and $gy_1 = y_2$.
               Now $gz_1 = x_2$ implies that $x_1 = z_1$ and $gz_1 = y_2$
               implies $z_1 = y_1$, both contradictions; thus $gz_1 \neq x_2$
               and $gz_1 \neq y_2$; that is, $gz_1, z_2 \in
               X\backslash\{x_2, y_2\}$, so that there exists
               $h \in X\backslash\{x_2, y_2\}$ such that $h(gz_1) = z_2$.
               Now we have that $(hg)x_1 = hx_2 = x_2$, $(hg)y_1 = hy_2 = y_2$,
               and $(hg)z_1 = h(gz_1) = z_2$, so that $G$ acts triply
               transitive on $X$.
               
      \end{enumerate}
\end{enumerate}
\end{document}
