\documentclass[9pt]{article}

\usepackage{amssymb}
\usepackage{amsmath, array}
\usepackage{amsfonts}
\usepackage{comment}
\usepackage{fancyhdr}
\usepackage{mathrsfs}
\usepackage{enumitem}
\usepackage{bm}
\usepackage{tikz}

\voffset = -50pt
%\textheight = 700pt
\addtolength{\textwidth}{60pt}
\addtolength{\evensidemargin}{-30pt}
\addtolength{\oddsidemargin}{-30pt}
%\setlength{\headheight}{44pt}

\pagestyle{fancy}
\fancyhf{} % clear all fields
\fancyhead[R]{%
  \scshape
  \begin{tabular}[t]{@{}r@{}}
  MATH 590, Fall 2015\\Section 1 (9529)\\
  HW \#06, DUE: 2015, October 12
  \end{tabular}}
\fancyhead[L]{%
  \scshape
  \begin{tabular}[t]{@{}r@{}}
  JOSEPH OKONOBOH\\Mathematics\\Cal State Long Beach
  \end{tabular}}
\fancyfoot[C]{\thepage}

\newcommand{\qed}{\hfill \ensuremath{\Box}}

\newcommand{\col}[2]{\left(\begin{tabular}{@{}c@{}}
   $#1$ \\
   $#2$  
 \end{tabular}\right)}

\newcommand{\li}[3]{\ell\left(\col{#1}{#2}, #3\right)}

\newcommand*\circled[1]{\tikz[baseline=(char.base)]{
            \node[shape=circle,draw,inner sep=2pt] (char) {#1};}}


\everymath{\displaystyle}
\newcommand{\Z}{\mathbb{Z}}
\newcommand{\I}{\mathbb{I}}
\newcommand{\F}{\mathbb{F}}
\newcommand{\Q}{\mathbb{Q}}
\newcommand{\R}{\mathbb{R}}
\newcommand{\C}{\mathbb{C}}
\renewcommand{\S}{\mathbb{S}}
\newcommand{\N}{\mathbb{N}}
\newcommand{\D}{\displaystyle}
%\setcounter{section}{-1}

\begin{document}
\begin{enumerate}[label=\protect\circled{\arabic*}]
%%%%%%%%%%%%%%%%%%%%%%%%%%%%%%%%%%%%%%%01%%%%%%%%%%%%%%%%%%%%%%%%%%%%%%%%%%%%%%%
   \item Consider the veracity or falsehood of each of the following statements.
         For bonus, argue for those that you believe are true while providing a
         counterexample for those that you believe are false.

         \begin{enumerate}[label=\protect\circled{\arabic*}]
            \item A matrix cannot be both a generating matrix for a code and a
                  parity-check matrix for the same code.
            \item If vectors in $\Z^n$ are independent over $\Q$, and none of
                  them mod out to 0 when viewed mod 2, then the vectors mod 2
                  are also independent.
            \item If $A$ is an invertible matrix in $\Z_p$, then $A$ is
                  invertible when viewed as in $\Q$.
            \item If $A$ is an invertible matrix in $\Z$, then $A$ is invertible
                  in $\Z_p$ for any $p$.
            \item The code in Quiz 5 is a self-dual code.
         \end{enumerate}

      \textbf{Answer.}

      \begin{enumerate}[label=\protect\circled{\arabic*}]
         \item False.

               \textbf{Counterexample.} Consider the code   
               $$V = \{(0 \quad 0), (1 \quad 1)\} \le \Z_2^2.$$
               The matrix $M = \left(\begin{tabular}{@{}cc@{}}
                  1 & 1
               \end{tabular}\right)$ is both a generating matrix and
               parity-check for $V$.
         \item False.

               \textbf{Counterexample.} Let $\textbf{a}, \textbf{b} \in \Z^n$
               such that
               $$\textbf{a} = (3 \quad 1 \quad 1 \quad \cdots \quad 1)
                 \text{ and }
                 \textbf{b} = (1 \quad 1 \quad 1 \quad \cdots \quad 1).$$
               The vectors \textbf{a} and \textbf{b} are linearly independent
               in $\Z_n$; however, observe that both vectors are equal when
               viewed mod 2; that is, they are not independent when viewed mod
               2.
         \item True. 

               \textbf{Proof.} Suppose that $A$ is an invertible matrix in
               $\Z_p$, for some $p \in \N$. Let $\det(A) \equiv k$ (mod $p$), 
               where $0 \le k \le p -1$. Since $A$ is invertible, it follows
               that $0 < k \le p - 1$, so that if $A$ is viewed in $\Q$, we 
               would have $\det(A) = k + pn$ for some integer $n$. If
               $k + pn = 0$, then that would imply that $k \equiv 0$ (mod $p$), 
               a contradiction; thus $k + pn \neq 0$, and it follows that
               $\det(A) \neq 0$ in $\Q$, so that $A$ is invertible in $\Q$. \qed
         \item True.

               \textbf{Proof.} Suppose $A$ is an invertible matrix with entries
               in $\Z$. We know that $\det(A)\det(A^{-1}) = 1$, so that
               $\det(A) = \pm 1$. If viewed in $\Z_p$, it follows that
               $\det(A) \equiv 1$ or $p - 1$ (mod $p$). Since $1$ and $p - 1$
               are nonzero in $\Z_p$, it follows that $A$ is nonsingular in
               $\Z_p$. \qed
         \item False. Let $V$ be the given code. If $V$ were self-dual, then we
               would have $V = V^{\perp}$; however,
               $(1 \quad 0 \quad 0 \quad 0 \quad 1 \quad 1)$ and
               $(0 \quad 1 \quad 0 \quad 1 \quad 0 \quad 1)$ are nonorthogonal
               vectors in $V$; that is, they are not in $V^{\perp}$ and it
               follows that $V \neq V^{\perp}$, so that $V$ is not self-dual.
      \end{enumerate}
%%%%%%%%%%%%%%%%%%%%%%%%%%%%%%%%%%%%%%%02%%%%%%%%%%%%%%%%%%%%%%%%%%%%%%%%%%%%%%%
   \item Recall that in the exam we investigated graphs whose adjacency matrix
         $A$ satisfied $A^2 + A = (k - 1)I + J$ and was regular of degree $k$.
         It was shown that $\sqrt{4k-3} = d$ could only be 3, 5, 15. Find the
         size and the spectrum when $d = 15$.

      \textbf{Solution.} We obtain $k = 57$ from the equation
      $\sqrt{4k-3} = 15$, so that $A$ is a $3250 \times 3250$ matrix. Since the
      spectrum of $J$ is $3250^{(1)}, 0^{(3249)}$, it follows that the spectrum 
      of $A^2 + A$ is $3306^{(1)}, 56^{(3249)}$.
      That is, one of the eigenvalue of $A$ must satisfy $\mu^2+\mu=3306$. We 
      know that $AJ = 57J$, so this eigenvalue must be 57. The other eigenvalues 
      must satisfy $\lambda^2+\lambda = 56$; thus $\lambda = -8$ or
      $\lambda = 7$. Let $m$ and $n$ be the multiplicities of $-8$ and $7$ 
      respectively. It follows that
      $$m + n = 3249 \text{ and } 57 - 8m + 7n = 0.$$
      Solving these equations will yield $m = 1520$ and $n = 1729$. Thus the 
      spectrum of $A$ is $57^{(1)}, -8^{(1520)}, 7^{(1729)}$.
%%%%%%%%%%%%%%%%%%%%%%%%%%%%%%%%%%%%%%%03%%%%%%%%%%%%%%%%%%%%%%%%%%%%%%%%%%%%%%%
   \item Define a family of codes of length $2^k$ (over $\Z_2$) as follows.
         $V_k$ is defined recursively as follows:

         $V_1 = \Z_2^2$ and if $V_n$ is defined, then $V_{n+1} = \{(\textbf{u} 
          \quad\textbf{u}), (\textbf{u}\quad\textbf{u}+\textbf{1}) :
          \textbf{u} \in V_n\}$.

         \begin{enumerate}[label=\protect\circled{\arabic*}]
            \item Give the elements of $V_2$ and $V_3$.
            \item Prove that $V_k$ is in fact linear.
            \item Find the dimension of $V_k$.
            \item Find the weight enumerator of $V_k$.
         \end{enumerate}

      \textbf{Solution.}

      \begin{enumerate}[label=\protect\circled{\arabic*}]
         \item Given that $V_1 = \{(0 \quad 0), (0 \quad 1), (1 \quad 0),
               (1 \quad 1)\}$, and using the recursive definition, we have that
               \begin{equation*}
                  \begin{split}
                     V_2 = \{&(0 \quad 0 \quad 0 \quad 0),
                           (0 \quad 0 \quad 1 \quad 1),
                           (0 \quad 1 \quad 0 \quad 1),
                           (0 \quad 1 \quad 1 \quad 0), \\
                           &(1 \quad 0 \quad 1 \quad 0),
                            (1 \quad 0 \quad 0 \quad 1),
                            (1 \quad 1 \quad 1 \quad 1),
                            (1 \quad 1 \quad 0 \quad 0)\}
                  \end{split}
               \end{equation*}
               and
               \begin{equation*}
                  \begin{split}
                     V_3 = \{&(0 \quad 0 \quad 0 \quad 0 \quad
                               0 \quad 0 \quad 0 \quad 0), (0 \quad 0 \quad
                               0 \quad 0 \quad 1 \quad 1 \quad 1 \quad 1) \\
                             &(0 \quad 0 \quad 1 \quad 1 \quad
                               0 \quad 0 \quad 1 \quad 1), (0 \quad 0 \quad
                               1 \quad 1 \quad 1 \quad 1 \quad 0 \quad 0) \\
                             &(0 \quad 1 \quad 0 \quad 1 \quad
                               0 \quad 1 \quad 0 \quad 1), (0 \quad 1 \quad
                               0 \quad 1 \quad 1 \quad 0 \quad 1 \quad 0) \\
                             &(0 \quad 1 \quad 1 \quad 0 \quad
                               0 \quad 1 \quad 1 \quad 0), (0 \quad 1 \quad
                               1 \quad 0 \quad 1 \quad 0 \quad 0 \quad 1) \\
                             &(1 \quad 0 \quad 1 \quad 0 \quad
                               1 \quad 0 \quad 1 \quad 0), (1 \quad 0 \quad
                               1 \quad 0 \quad 0 \quad 1 \quad 0 \quad 1) \\
                             &(1 \quad 0 \quad 0 \quad 1 \quad
                               1 \quad 0 \quad 0 \quad 1), (1 \quad 0 \quad
                               0 \quad 1 \quad 0 \quad 1 \quad 1 \quad 0) \\
                             &(1 \quad 1 \quad 1 \quad 1 \quad
                               1 \quad 1 \quad 1 \quad 1), (1 \quad 1 \quad
                               1 \quad 1 \quad 0 \quad 0 \quad 0 \quad 0) \\
                             &(1 \quad 1 \quad 0 \quad 0 \quad
                               1 \quad 1 \quad 0 \quad 0), (1 \quad 1 \quad
                               0 \quad 0 \quad 0 \quad 0 \quad 1 \quad 1)\}
                  \end{split}
               \end{equation*}
         \item We shall proceed by induction on $k$ to show that
               \begin{equation} \label{3_1}
                  V_k \le \Z_2^{2^k}.
               \end{equation}
               By definition, we have that $V_1 = \Z_2^2$, so that \eqref{3_1} 
               holds whenever $k$ is 1. Now suppose that \eqref{3_1} holds for
               some positive integer $j$. To complete the proof, we must now
               show that $V_{j+1}$ is a linear code; that is, we must show that
               $V_{j+1} \le \Z_2^{2^{j+1}}$. Since
               $(\underbrace{0 \quad 0 \cdots 0}_{2^j \text{ times }}) \in V_j$, 
               it follows by recursive definition that
               $$(\underbrace{0 \quad 0 \cdots 0}_{2^j \text{ times }}
                  \underbrace{0 \quad 0 \cdots 0}_{2^j \text{ times }}) = 
                  (\underbrace{0 \quad 0 \cdots 0}_{2^{j+1} \text{ times }})
                  \in V_{j+1},$$
               so that $V_{j+1}$ contains the \textbf{0} vector. Now $V_{j+1}$
               is trivially closed under scalar multiplication since
               $\Z_2 = \{0, 1\}$. All that remains is to show closure under
               addition, so let $\textbf{a}, \textbf{b} \in V_{j+1}$. That is,
               there exist $\textbf{u}, \textbf{v} \in V_j$ such that
               $$\textbf{a} = (\textbf{u} \quad \textbf{u}) \text{ or }
                 \textbf{a} = (\textbf{u} \quad \textbf{u} + \textbf{1})$$
               and
               $$\textbf{b} = (\textbf{v} \quad \textbf{v}) \text{ or }
                 \textbf{b} = (\textbf{v} \quad \textbf{v} + \textbf{1}).$$

               It follows that
               $$\textbf{a} + \textbf{b} = (\textbf{u}+\textbf{u} \quad
                 \textbf{v} + \textbf{v}) \text{ or } (\textbf{u}+\textbf{v} 
                 \quad \textbf{u} + \textbf{v} + \textbf{1}) \text{ or }
                 (\textbf{u}+\textbf{v} 
                 \quad \textbf{u} + \textbf{v} + \textbf{1} + \textbf{1}).$$
                 
               But \textbf{u} + \textbf{v} + \textbf{1} + \textbf{1} =
               \textbf{u} + \textbf{v}, so that
               $$\textbf{a} + \textbf{b} = (\textbf{u}+\textbf{u} \quad
                 \textbf{v} + \textbf{v}) \text{ or } (\textbf{u}+\textbf{v} 
                 \quad \textbf{u} + \textbf{v} + \textbf{1}).$$

               Since $V_j$ is a vector space, we have that
               $\textbf{u} + \textbf{v} \in V_j$; thus, by the recursive
               definition, it follows that both
               $(\textbf{u}+\textbf{v} \quad \textbf{u}+\textbf{v})$ and
               $(\textbf{u}+\textbf{v} \quad \textbf{u}+\textbf{v} +\textbf{1})$ 
               are in $V_{j+1}$, so that $\textbf{a} + \textbf{b} \in V_{j+1}$;
               that is, $V_{j+1}$ is closed under addition, and we conclude that
               $V_{j+1} \le \Z_2^{2^{j+1}}$. This says that \eqref{3_1} also
               holds for $j + 1$, so that, by induction, it holds for all each 
               positive integer $k$. We have thus shown that if $k \in \N$, then
               $V_k$ is a linear code.
         \item We have $|V_k| = 2^{k+1} = |\Z_2|^{k+1}$; thus
               $\dim(V_k) = k + 1$.
         \item $w_{V_k}(x) = 1 + (2^{k+1} - 2)x^{2^{k-1}} + x^{2^k}$.
      \end{enumerate}
\end{enumerate}
\end{document}
