\documentclass[9pt]{article}

\usepackage{amssymb}
\usepackage{amsmath, array}
\usepackage{amsfonts}
\usepackage{comment}
\usepackage{fancyhdr}
\usepackage{mathrsfs}
\usepackage{enumitem}
\usepackage{bm}
\usepackage{tikz}

\voffset = -50pt
%\textheight = 700pt
\addtolength{\textwidth}{60pt}
\addtolength{\evensidemargin}{-30pt}
\addtolength{\oddsidemargin}{-30pt}
%\setlength{\headheight}{44pt}

\pagestyle{fancy}
\fancyhf{} % clear all fields
\fancyhead[R]{%
  \scshape
  \begin{tabular}[t]{@{}r@{}}
  MATH 590, Fall 2015\\Section 1 (9529)\\
  HW \#04, DUE: 2015, September 21
  \end{tabular}}
\fancyhead[L]{%
  \scshape
  \begin{tabular}[t]{@{}r@{}}
  JOSEPH OKONOBOH\\Mathematics\\Cal State Long Beach
  \end{tabular}}
\fancyfoot[C]{\thepage}

\newcommand{\qed}{\hfill \ensuremath{\Box}}

\newcommand{\col}[2]{\left(\begin{tabular}{@{}c@{}}
   $#1$ \\
   $#2$  
 \end{tabular}\right)}

\newcommand{\li}[3]{\ell\left(\col{#1}{#2}, #3\right)}

\newcommand*\circled[1]{\tikz[baseline=(char.base)]{
            \node[shape=circle,draw,inner sep=2pt] (char) {#1};}}

\everymath{\displaystyle}
\newcommand{\Z}{\mathbb{Z}}
\newcommand{\I}{\mathbb{I}}
\newcommand{\F}{\mathbb{F}}
\newcommand{\Q}{\mathbb{Q}}
\newcommand{\R}{\mathbb{R}}
\newcommand{\C}{\mathbb{C}}
\renewcommand{\S}{\mathbb{S}}
\newcommand{\N}{\mathbb{N}}
\newcommand{\D}{\displaystyle}
%\setcounter{section}{-1}

\begin{document}
\begin{enumerate}[label=\protect\circled{\arabic*}]
%%%%%%%%%%%%%%%%%%%%%%%%%%%%%%%%%%%%%%%01%%%%%%%%%%%%%%%%%%%%%%%%%%%%%%%%%%%%%%%
   \item Consider the veracity or falsehood of each of the following statements.
         For bonus, argue for those that you believe are true while providing a
         counterexample for those that you believe are false. Work either
         column. These all concern the affine plane (see below) built using a
         field $\F$ with $q$ elements.

         \begin{enumerate}[label=\protect\circled{\arabic*}]
            \item There are $q^2$ points.
            \item There are $q^2 + 1$ lines.
            \item Every line has $q$ points.
            \item Every point is in $q$ lines.
            \item No two lines contain all points.
         \end{enumerate}

      \textbf{Answer.}

      \begin{enumerate}[label=\protect\circled{\arabic*}]
         \item True. There are $q$ choices for each component of a point, for a
               total of $q \cdot q = q^2$ choices.
         \item False. If $\F = \Z_2$, so that $q = 2$, then there are 6, not
               $2^2 + 1 = 5$, lines. See Exercise 3.4 for the lines.
         \item True. If $(c \quad d)$ is a point on the line $\li{1}{x}{k}$,
               where $x, k \in \F$, then it follows that $c + dx = k$, so that
               $c = k - dx$. There are $q$ choices for $d$, and once $d$ is
               chosen, the value of $c$ is automatically determined, so that 
               there are $q$ choices for $(c \quad d)$. Similarly if
               $(c' \quad d')$ is a point on the line $\li{0}{1}{k'}$,
               where $k' \in \F$, then it follows that $d' = k'$, so that there
               are $q$ choices for $(c' \quad d')$ since there are no
               restrictions on $c'$.
         \item False. If $\F = \Z_2$, so that $q = 2$, then we see observe from 
               Exercise 3.4 that the point $(0 \quad 0)$ is on
               three---not 2---lines.
         \item False. If $\F = \Z_2$, then the lines $\li{1}{0}{0}$ and
               $\li{1}{0}{1}$ contain all four points.
      \end{enumerate}   
%%%%%%%%%%%%%%%%%%%%%%%%%%%%%%%%%%%%%%%02%%%%%%%%%%%%%%%%%%%%%%%%%%%%%%%%%%%%%%%
   \item In the last homework you built an extension of the Hamming code using
         columns of size 4 over $\Z_2$. Give the weight enumerator of the code
         that is generated by that matrix---namely it is the row space of the
         matrix---not the null space.

         \textbf{Bonus.} Find the weight enumerator of the Hamming code that is
         the null space of the matrix.
%%%%%%%%%%%%%%%%%%%%%%%%%%%%%%%%%%%%%%%03%%%%%%%%%%%%%%%%%%%%%%%%%%%%%%%%%%%%%%%
   \item \textbf{The Affine Plane.} Let $\textbf{P} = \F^2 =
         \{(a \quad b) : a, b \in \F\}$ be the points in a geometry, where $\F$
         is a field with $q$ elements. Let
         $S = \left\{\col{1}{x}, \col{0}{1} : x \in \F\right\}$ be the set of
         slopes. For any $\textbf{u} \in S$ and any $k \in \F$ define the line
         $\ell(\textbf{u}, k) = \{(a \quad b) : (a \quad b)\textbf{u} = k\}$,
         which is a set of points. Then the set of lines
         $\textbf{L} = \{\ell(\textbf{u}, k) : \textbf{u} \in S, k \in \F\}$.

         \begin{enumerate}[label=\protect\circled{\arabic*}]
            \item Show that every two points are on a unique line.
            \item Show that if $p \in \textbf{P}$ and $\ell \in \textbf{L}$ and
                  $p \notin \ell$, then there exists a unique $\ell_1$ such that
                  $p \in \ell_1$ and $\ell$ and $\ell_1$ are parallel (they do
                  not intersect).
            \item Show there exist 4 points no three of which are collinear (on
                  a line together). \\

                  The three properties above define an \textbf{Affine Plane.}
            \item Give the lines and points when $\F = \Z_2$. \\
            
                  \textbf{Bonus.} Give the lines and points when $\F = GF(4)$.
            \item Use an affine plane to describe how you would solve the
                  following problem that was asked to a member of the department
                  from a professor teaching in the humanities: \\
      
                  \textbf{I currently have 47 students in my class. I am 
                  teaching in a room with cafe tables of 7 per table. If I 
                  wanted to have each student get a chance to meet every other
                  student in the class, how many days would it take and how 
                  would I figure out membership in the groups? I might end up
                  with 52, likely not more. I could fit 8 around a table if need
                  be.}
         \end{enumerate}

      \textbf{Solution.}

      \begin{enumerate}[label=\protect\circled{\arabic*}]
         \item \textbf{Proof.} Let $(a \quad b)$ and $(c \quad d)$ be different 
               points in $\textbf{P}$. We have the following cases:

               \textbf{Case 1.} $b = d$. That is, $(c \quad d) = (c \quad b)$. 
               Since we assumed that points $(a \quad b)$ and $(c \quad d)$ are 
               different and since $b = d$, it must be the case that $a \neq c$. 
               So suppose that $(a \quad b)$ and $(c \quad b)$ are on a line
               $\li{0}{1}{k}$, where $k \in \F$. Then it follows that
               $$(a \quad b)\col{0}{1} = b = k = b = (c \quad b)\col{0}{1},$$
               so that the points $(a \quad b)$ and $(c \quad b)$ are both on 
               the line $\li{0}{1}{b}$. Now suppose to the contrary that, for
               some $x', k' \in \F$, some line $\li{1}{x'}{k'}$ contains
               $(a \quad b)$ and $(c \quad b)$. It follows by definition that
               $$(a \quad b)\col{1}{x'} = a + bx' = k' = c + bx' =
                 (c \quad d)\col{1}{x'},$$
               so that $a + bx' = c + bx'$, and thus, $a = c$, a contradiction. 
               Thus, the only line that contains $(a \quad b)$ and
               $(c \quad b)$ is $\li{0}{1}{b}$.

               \textbf{Case 2.} $b \neq d$. First we make the observation that
               points $(a \quad b)$ and $(c \quad d)$ cannot lie on a line
               $\li{0}{1}{k''}$ for some $k'' \in \F$, because that would imply
               that
               $$(a \quad b)\col{0}{1} = b = k'' = d = (c \quad d)\col{0}{1},$$
               a contradiction since $b \neq d$. Now suppose that, for some
               $x, k \in \F$, some line $\li{1}{x}{k}$ contains points
               $(a \quad b)$ and $(c \quad d)$. It follows by definition that
               $$(a \quad b)\col{1}{x} = a + bx = k = c + dx =
                 (c \quad d)\col{1}{x},$$
               so that $(b - d)x = c - a$. Now since $b \neq d$, it follows that
               $b - d \neq 0$, so that $(b-d)^{-1}$ exists. That is,
               $x = (c - a)(b - d)^{-1}$. Substitute this value of $x$ above to
               get $k = \frac{bc-ad}{b-d}$. So the only line that contains
               points $(a \quad b)$ and $(c \quad d)$ is
               $\li{1}{(c - a)(b - d)^{-1}}{\frac{bc-ad}{b-d}}$. \qed
         \item \textbf{Proof.} Let us make the following observation:
               If $\textbf{v}$ is a slope, then for $i, j \in \F$, with
               $i \neq j$, it follows that $\ell(\textbf{v}, i)$ and
               $\ell(\textbf{v}, j)$ are disjoint. This is so because if
               $(i_1 \quad i_2) \in \ell(\textbf{v}, i)$ and
               $(j_1 \quad j_2) \in \ell(\textbf{v}, j)$, then it follows that
               $$(i_1 \quad i_2)\textbf{v} = i \neq j =
                 (j_1 \quad j_2)\textbf{v}.$$

               Now let $(a \quad b)$ be a point in \textbf{P} and $\ell_1$ a
               line in \textbf{L}. Suppose that $(a \quad b) \notin \ell_1$. 
               Then we have the following cases:

               \textbf{Case 1.} $\ell_1 = \li{0}{1}{k}$, where $k \in \F$ and
               $k \neq b$. The point $(a \quad b)$ does not lie on $\ell_1$
               because $k \neq b$. Let $\ell_2 = \li{0}{1}{b}$ and observe that
               $(a \quad b) \in \ell_2$. Since $k \neq b$ and since $\ell_1$ and
               $\ell_2$ have the same slope, it follows that 
               $\ell_1 \cap \ell_2 = \emptyset$, so that $\ell_1$ and
               $\ell_2$ are parallel.

               \textbf{Case 2.} $\ell_1 = \li{1}{x}{r}$, where $x, r \in \F$ 
               such that $a + bx \neq r$. Let $\ell_2 = \li{1}{x}{a+bx}$. The
               point $(a \quad b)$ does not lie on $\ell_1$ because
               $a + bx \neq r$, but it lies on $\ell_2$. Since $\ell_1$ and
               $\ell_2$ have the same slope and since $a + bx \neq r$, it
               follows that $\ell_1$ and $\ell_2$ are parallel.
         \item This is trivially true if $|\F| = 2$ since each line will contain
               exactly two points. So suppose $q = |\F| \ge 3$. The number of 
               lines is $|S| \cdot |\F| = (q+1)q = q^2 + q$.
      \end{enumerate}
%%%%%%%%%%%%%%%%%%%%%%%%%%%%%%%%%%%%%%Bonus%%%%%%%%%%%%%%%%%%%%%%%%%%%%%%%%%%%%% 
   \item[\textbf{Bonus.}]  \textbf{The Projective Plane.} Let \textbf{P} be the
                           collection of one dimensional subspaces of $\F^3$
                           where $\F$ is a field with $q$ elements. Let
                           $\textbf{L}$ be the collection of two dimensional
                           subspaces of $\F^3$. So
                           $\textbf{P} = \{[z] : z \in \F^3, z \neq 0\}$,
                           $\textbf{L} = \{[z, w] : z, w \in \F^3,
                            \text{independent}\}$. We refer to \textbf{P} as 
                           the points in the geometry and \textbf{L} as the
                           lines, and of course we say point $p$
                           \textbf{is on line} $\ell$ if $p \le \ell$.

                           \begin{enumerate}[label=\protect\circled{\arabic*}]   
                              \item Show that every two points are on a unique
                                    line. 
                              \item Show that every two lines intersect in a
                                    unique point.
                              \item Show there exist 4 points no three of which
                                    are collinear (on a line together). \\

                                    The three properties above define an
                                    \textbf{Projective Plane.}
                              \item Show there exist 4 lines no three of which
                                    intersect at a point.
                              \item Give the lines and points when $\F = \Z_2$.
                              \item Give the lines and points when $\F = GF(4)$.
                           \end{enumerate}
\end{enumerate}
\end{document}
