\documentclass[9pt]{article}

\usepackage{amssymb}
\usepackage{amsmath, array}
\usepackage{amsfonts}
\usepackage{comment}
\usepackage{fancyhdr}
\usepackage{mathrsfs}
\usepackage{enumitem}
\usepackage{bm}
\usepackage{tikz}

\voffset = -50pt
%\textheight = 700pt
\addtolength{\textwidth}{60pt}
\addtolength{\evensidemargin}{-30pt}
\addtolength{\oddsidemargin}{-30pt}
%\setlength{\headheight}{44pt}

\pagestyle{fancy}
\fancyhf{} % clear all fields
\fancyhead[R]{%
  \scshape
  \begin{tabular}[t]{@{}r@{}}
  MATH 590, Fall 2015\\Section 1 (9529)\\
  HW \#04, DUE: 2015, September 21
  \end{tabular}}
\fancyhead[L]{%
  \scshape
  \begin{tabular}[t]{@{}r@{}}
  JOSEPH OKONOBOH\\Mathematics\\Cal State Long Beach
  \end{tabular}}
\fancyfoot[C]{\thepage}

\newcommand{\qed}{\hfill \ensuremath{\Box}}

\newcommand{\col}[2]{\left(\begin{tabular}{@{}c@{}}
   $#1$ \\
   $#2$  
 \end{tabular}\right)}

\newcommand{\li}[3]{\ell\left(\col{#1}{#2}, #3\right)}

\newcommand*\circled[1]{\tikz[baseline=(char.base)]{
            \node[shape=circle,draw,inner sep=2pt] (char) {#1};}}

\everymath{\displaystyle}
\newcommand{\Z}{\mathbb{Z}}
\newcommand{\I}{\mathbb{I}}
\newcommand{\F}{\mathbb{F}}
\newcommand{\Q}{\mathbb{Q}}
\newcommand{\R}{\mathbb{R}}
\newcommand{\C}{\mathbb{C}}
\renewcommand{\S}{\mathbb{S}}
\newcommand{\N}{\mathbb{N}}
\newcommand{\D}{\displaystyle}
%\setcounter{section}{-1}

\begin{document}
\begin{enumerate}[label=\protect\circled{\arabic*}]
%%%%%%%%%%%%%%%%%%%%%%%%%%%%%%%%%%%%%%%01%%%%%%%%%%%%%%%%%%%%%%%%%%%%%%%%%%%%%%%
   \item Consider the veracity or falsehood of each of the following statements.
         For bonus, argue for those that you believe are true while providing a
         counterexample for those that you believe are false. Work either
         column. These all concern the affine plane (see below) built using a
         field $\F$ with $q$ elements.

         \begin{enumerate}[label=\protect\circled{\arabic*}]
            \item There are $q^2$ points.
            \item There are $q^2 + 1$ lines.
            \item Every line has $q$ points.
            \item Every point is in $q$ lines.
            \item No two lines contain all points.
         \end{enumerate}

      \textbf{Answer.}

      \begin{enumerate}[label=\protect\circled{\arabic*}]
         \item True. There are $q$ choices for each component of a point, for a
               total of $q \cdot q = q^2$ choices.
         \item False. If $\F = \Z_2$, so that $q = 2$, then there are 6, not
               $2^2 + 1 = 5$, lines. See Exercise 3.4 for the lines.
         \item True. If $(c \quad d)$ is a point on the line $\li{1}{x}{k}$,
               where $x, k \in \F$, then it follows that $c + dx = k$, so that
               $c = k - dx$. There are $q$ choices for $d$, and once $d$ is
               chosen, the value of $c$ is automatically determined, so that 
               there are $q$ choices for $(c \quad d)$. Similarly if
               $(c' \quad d')$ is a point on the line $\li{0}{1}{k'}$,
               where $k' \in \F$, then it follows that $d' = k'$, so that there
               are $q$ choices for $(c' \quad d')$ since there are no
               restrictions on $c'$.
         \item False. If $\F = \Z_2$, so that $q = 2$, then we see observe from 
               Exercise 3.4 that the point $(0 \quad 0)$ is on
               three---not 2---lines.
         \item False. If $\F = \Z_2$, then the lines $\li{1}{0}{0}$ and
               $\li{1}{0}{1}$ contain all four points.
      \end{enumerate}   
%%%%%%%%%%%%%%%%%%%%%%%%%%%%%%%%%%%%%%%02%%%%%%%%%%%%%%%%%%%%%%%%%%%%%%%%%%%%%%%
   \item In the last homework you built an extension of the Hamming code using
         columns of size 4 over $\Z_2$. Give the weight enumerator of the code
         that is generated by that matrix---namely it is the row space of the
         matrix---not the null space.

         \textbf{Bonus.} Find the weight enumerator of the Hamming code that is
         the null space of the matrix.
         
      \textbf{Solution.} Recall that the matrix is given by
      $$A = \left(
           \begin{tabular}{@{}ccccccccccccccc@{}}
              0 & 0 & 0 & 0 & 0 & 0 & 0 & 1 & 1 & 1 & 1 & 1 & 1 & 1 & 1 \\
              0 & 0 & 0 & 1 & 1 & 1 & 1 & 0 & 0 & 0 & 0 & 1 & 1 & 1 & 1 \\
              0 & 1 & 1 & 0 & 0 & 1 & 1 & 0 & 0 & 1 & 1 & 0 & 0 & 1 & 1 \\
              1 & 0 & 1 & 0 & 1 & 0 & 1 & 0 & 1 & 0 & 1 & 0 & 1 & 0 & 1 \\
            \end{tabular}\right).
      $$
      The rank of $A$ is 4, so its rows are linearly independent. Thus there are
      16 codes in the rowspace of $A$ and they are
      $$
         \begin{tabular}{@{}|c|c|c|@{}} \hline
            & \textbf{Code} & \textbf{Weight} \\ \hline
          1. & \textbf{000000000000000} & 0 \\ \hline
          2. & \textbf{101010101010101} & 8 \\ \hline
          3. & \textbf{011001100110011} & 8 \\ \hline
          4. & \textbf{110011001100110} & 8 \\ \hline
          5. & \textbf{000111100001111} & 8 \\ \hline         
          6. & \textbf{101101001011010} & 8 \\ \hline
          7. & \textbf{011110000111100} & 8 \\ \hline
          8. & \textbf{110100101101001} & 8 \\ \hline
          9. & \textbf{000000011111111} & 8 \\ \hline
         10. & \textbf{101010110101010} & 8 \\ \hline
         11. & \textbf{011001111001100} & 8 \\ \hline
         12. & \textbf{110011010011001} & 8 \\ \hline
         13. & \textbf{000111111110000} & 8 \\ \hline
         14. & \textbf{101101010100101} & 8 \\ \hline
         15. & \textbf{011110011000011} & 8 \\ \hline
         16. & \textbf{110100110010110} & 8 \\ \hline
         \end{tabular}
      $$
      Let $V$ be the codes of the row space of $A$; it follows from the table
      above that
      $$w_V(x) = 1 + 15x^8.$$
      
      \textbf{Bonus.} By the MacWilliams' Identity, it follows that the weight
      enumerator of the Hamming Code of the null space of $A$ is
      \begin{align*}
         w_{V^\perp}(x) &= \frac{1}{2^4}(1+x)^{15}
            w_V\left(\frac{1-x}{1+x}\right) \\
            &= \frac{1}{16}(1+x)^{15}
               \left(1 + 15\frac{(1-x)^8}{(1+x)^8}\right) \\
            &= \frac{1}{16}((1+x)^{15} + 15(1-x^2)^7(1-x))
      \end{align*}
%%%%%%%%%%%%%%%%%%%%%%%%%%%%%%%%%%%%%%%03%%%%%%%%%%%%%%%%%%%%%%%%%%%%%%%%%%%%%%%
   \item \textbf{The Affine Plane.} Let $\textbf{P} = \F^2 =
         \{(a \quad b) : a, b \in \F\}$ be the points in a geometry, where $\F$
         is a field with $q$ elements. Let
         $S = \left\{\col{1}{x}, \col{0}{1} : x \in \F\right\}$ be the set of
         slopes. For any $\textbf{u} \in S$ and any $k \in \F$ define the line
         $\ell(\textbf{u}, k) = \{(a \quad b) : (a \quad b)\textbf{u} = k\}$,
         which is a set of points. Then the set of lines
         $\textbf{L} = \{\ell(\textbf{u}, k) : \textbf{u} \in S, k \in \F\}$.

         \begin{enumerate}[label=\protect\circled{\arabic*}]
            \item Show that every two points are on a unique line.
            \item Show that if $p \in \textbf{P}$ and $\ell \in \textbf{L}$ and
                  $p \notin \ell$, then there exists a unique $\ell_1$ such that
                  $p \in \ell_1$ and $\ell$ and $\ell_1$ are parallel (they do
                  not intersect).
            \item Show there exist 4 points no three of which are collinear (on
                  a line together). \\

                  The three properties above define an \textbf{Affine Plane.}
            \item Give the lines and points when $\F = \Z_2$. \\
            
                  \textbf{Bonus.} Give the lines and points when $\F = GF(4)$.
            \item Use an affine plane to describe how you would solve the
                  following problem that was asked to a member of the department
                  from a professor teaching in the humanities: \\
      
                  \textbf{I currently have 47 students in my class. I am 
                  teaching in a room with cafe tables of 7 per table. If I 
                  wanted to have each student get a chance to meet every other
                  student in the class, how many days would it take and how 
                  would I figure out membership in the groups? I might end up
                  with 52, likely not more. I could fit 8 around a table if need
                  be.}
         \end{enumerate}

      \textbf{Solution.}

      \begin{enumerate}[label=\protect\circled{\arabic*}]
         \item \textbf{Proof.} Let $(a \quad b)$ and $(c \quad d)$ be different 
               points in $\textbf{P}$. We have the following cases:

               \textbf{Case 1.} $b = d$. That is, $(c \quad d) = (c \quad b)$. 
               Since we assumed that points $(a \quad b)$ and $(c \quad d)$ are 
               different and since $b = d$, it must be the case that $a \neq c$. 
               So suppose that $(a \quad b)$ and $(c \quad b)$ are on a line
               $\li{0}{1}{k}$, where $k \in \F$. Then it follows that
               $$(a \quad b)\col{0}{1} = b = k = b = (c \quad b)\col{0}{1},$$
               so that the points $(a \quad b)$ and $(c \quad b)$ are both on 
               the line $\li{0}{1}{b}$. Now suppose to the contrary that, for
               some $x', k' \in \F$, some line $\li{1}{x'}{k'}$ contains
               $(a \quad b)$ and $(c \quad b)$. It follows by definition that
               $$(a \quad b)\col{1}{x'} = a + bx' = k' = c + bx' =
                 (c \quad d)\col{1}{x'},$$
               so that $a + bx' = c + bx'$, and thus, $a = c$, a contradiction. 
               Thus, the only line that contains $(a \quad b)$ and
               $(c \quad b)$ is $\li{0}{1}{b}$.

               \textbf{Case 2.} $b \neq d$. First we make the observation that
               points $(a \quad b)$ and $(c \quad d)$ cannot lie on a line
               $\li{0}{1}{k''}$ for some $k'' \in \F$, because that would imply
               that
               $$(a \quad b)\col{0}{1} = b = k'' = d = (c \quad d)\col{0}{1},$$
               a contradiction since $b \neq d$. Now suppose that, for some
               $x, k \in \F$, some line $\li{1}{x}{k}$ contains points
               $(a \quad b)$ and $(c \quad d)$. It follows by definition that
               $$(a \quad b)\col{1}{x} = a + bx = k = c + dx =
                 (c \quad d)\col{1}{x},$$
               so that $(b - d)x = c - a$. Now since $b \neq d$, it follows that
               $b - d \neq 0$, so that $(b-d)^{-1}$ exists. That is,
               $x = (c - a)(b - d)^{-1}$. Substitute this value of $x$ above to
               get $k = \frac{bc-ad}{b-d}$. So the only line that contains
               points $(a \quad b)$ and $(c \quad d)$ is
               $\li{1}{(c - a)(b - d)^{-1}}{\frac{bc-ad}{b-d}}$. Note that the
               uniqueness of the line followed from the uniqueness of $x$,
               which, in turn, followed from the uniqueness of inverses. \qed
         \item \textbf{Proof.} Let us make the following observation:
               If $\textbf{v}$ is a slope, then for $i, j \in \F$, with
               $i \neq j$, it follows that $\ell(\textbf{v}, i)$ and
               $\ell(\textbf{v}, j)$ are disjoint. This is so because if
               $(i_1 \quad i_2) \in \ell(\textbf{v}, i)$ and
               $(j_1 \quad j_2) \in \ell(\textbf{v}, j)$, then it follows that
               $$(i_1 \quad i_2)\textbf{v} = i \neq j =
                 (j_1 \quad j_2)\textbf{v}.$$

               Now let $(a \quad b)$ be a point in \textbf{P} and $\ell_1$ a
               line in \textbf{L}. Suppose that $(a \quad b) \notin \ell_1$. 
               Then we have the following cases:

               \textbf{Case 1.} $\ell_1 = \li{0}{1}{k}$, where $k \in \F$ and
               $k \neq b$. The point $(a \quad b)$ does not lie on $\ell_1$
               because $k \neq b$. Let $\ell_2 = \li{0}{1}{b}$ and observe that
               $(a \quad b) \in \ell_2$. Since $k \neq b$ and since $\ell_1$ and
               $\ell_2$ have the same slope, it follows that 
               $\ell_1 \cap \ell_2 = \emptyset$, so that $\ell_1$ and
               $\ell_2$ are parallel.

               \textbf{Case 2.} $\ell_1 = \li{1}{x}{r}$, where $x, r \in \F$ 
               such that $a + bx \neq r$. Let $\ell_2 = \li{1}{x}{a+bx}$. The
               point $(a \quad b)$ does not lie on $\ell_1$ because
               $a + bx \neq r$, but it lies on $\ell_2$. Since $\ell_1$ and
               $\ell_2$ have the same slope and since $a + bx \neq r$, it
               follows that $\ell_1$ and $\ell_2$ are parallel.
         \item Consider the points $p_1 = (0 \quad 0)$, $p_2 = (0 \quad 1)$
               and $p_3 = (1 \quad 0)$, and $p_4 = (1 \quad 1)$.
               These points exist in \textbf{P} for every $\F$. Points
               $p_1$, $p_2$, $p_3$ are not collinear because they can never lie
               on a line $\li{1}{x}{k}$ because that will imply that
               $$0 = x = 1 = k,$$
               a contradiction, and, they can never lie on a line
               $\li{0}{1}{k'}$ because that would imply that
               $$0 = 1 = 0 = k',$$
               another contradiction. Argue similarly to conclude that
               any three points from $p_1$, $p_2$, $p_3$, and $p_4$ cannot be
               collinear.
         \item Let $\F = \Z_2$. Then the points are
               $$(0 \quad 0), (0 \quad 1), (1 \quad 0),
                 \text{ and } (1 \quad 1),$$
               the slopes are
               $$\textbf{u}_1 = \col{1}{0}, \textbf{u}_2 = \col{1}{1},
                 \text{ and } \textbf{u}_3 = \col{0}{1},$$
               and the lines are
               \begin{align*}
                  \ell(\textbf{u}_1, 0) &= \{(0 \quad 0), (0 \quad 1)\} \\
                  \ell(\textbf{u}_1, 1) &= \{(1 \quad 0), (1 \quad 1)\} \\
                  \ell(\textbf{u}_2, 0) &= \{(0 \quad 0), (1 \quad 1)\} \\
                  \ell(\textbf{u}_2, 1) &= \{(0 \quad 1), (1 \quad 0)\} \\
                  \ell(\textbf{u}_3, 0) &= \{(0 \quad 0), (1 \quad 0)\} \\
                  \ell(\textbf{u}_3, 1) &= \{(0 \quad 1), (1 \quad 1)\}.
               \end{align*}
         \item[\textbf{Bonus.}] Let $\F = GF(4) = \{0, 1, a, b\}$, where the addition and
               multiplication operations are defined below:
               $$
                  \begin{tabular}{@{}|>{$}c<{$}|>{$}c<{$}|>{$}c<{$}|>{$}c<{$}|
                     >{$}c<{$}|@{}} \hline
                     + & 0 & 1 & a & b \\ \hline
                     0 & 0 & 1 & a & b \\ \hline
                     1 & 1 & 0 & b & a  \\ \hline
                     a & a & b & 0 & 1 \\ \hline
                     b & b & a & 1 & 0 \\ \hline
                  \end{tabular} \quad
                  \begin{tabular}{@{}|>{$}c<{$}|>{$}c<{$}|>{$}c<{$}|>{$}c<{$}|
                     >{$}c<{$}|@{}} \hline
                     \cdot & 0 & 1 & a & b \\ \hline
                     0 & 0 & 0 & 0 & 0 \\ \hline
                     1 & 0 & 1 & a & b \\ \hline
                     a & 0 & a & b & 1 \\ \hline
                     b & 0 & b & 1 & a \\ \hline
                  \end{tabular}
               $$
               Then the points are
               $$
               \begin{tabular}{@{}cccc@{}}
                  $(0\quad 0),$ & $(0\quad 1),$ & $(0\quad a),$&$(0\quad b),$ \\
                  $(1\quad 0),$ & $(1\quad 1),$ & $(1\quad a),$&$(1\quad b),$ \\
                  $(a\quad 0),$ & $(a\quad 1),$ & $(a\quad a),$&$(a\quad b),$ \\
                  $(b\quad 0),$ & $(b\quad 1),$ & $(b\quad a),$&$(b\quad b),$
               \end{tabular}
               $$
               
               the slopes are
               $$\textbf{u}_1 = \col{1}{0}, \textbf{u}_2 = \col{1}{1},
                 \textbf{u}_3 = \col{1}{a}, \textbf{u}_4 = \col{1}{b}
                 \text{ and } \textbf{u}_5 = \col{0}{1},$$
               and the lines are
               \begin{align*}
                  \ell(\textbf{u}_1, 0) &=
                     \{(0 \quad 0), (0 \quad 1), (0 \quad a), (0 \quad b)\} \\
                  \ell(\textbf{u}_1, 1) &=
                     \{(1 \quad 0), (1 \quad 1), (1 \quad a), (1 \quad b)\} \\
                  \ell(\textbf{u}_1, a) &=
                     \{(a \quad 0), (a \quad 1), (a \quad a), (a \quad b)\} \\
                  \ell(\textbf{u}_1, b) &=
                     \{(b \quad 0), (b \quad 1), (b \quad a), (b \quad b)\} \\
                  \ell(\textbf{u}_2, 0) &=
                     \{(0 \quad 0), (1 \quad 1), (a \quad a), (b \quad b)\} \\
                  \ell(\textbf{u}_2, 1) &=
                     \{(0 \quad 1), (1 \quad 0), (a \quad b), (b \quad a)\} \\
                  \ell(\textbf{u}_2, a) &=
                     \{(0 \quad a), (1 \quad b), (a \quad 0), (b \quad 1)\} \\
                  \ell(\textbf{u}_2, b) &=
                     \{(0 \quad b), (1 \quad a), (a \quad 1), (b \quad 0)\} \\
                  \ell(\textbf{u}_3, 0) &=
                     \{(0 \quad 0), (1 \quad b), (a \quad 1), (b \quad a)\} \\
                  \ell(\textbf{u}_3, 1) &=
                     \{(0 \quad b), (1 \quad 0), (a \quad a), (b \quad 1)\} \\
                  \ell(\textbf{u}_3, a) &=
                     \{(0 \quad 1), (1 \quad a), (a \quad 0), (b \quad b)\} \\
                  \ell(\textbf{u}_3, b) &=
                     \{(0 \quad a), (1 \quad 1), (a \quad b), (b \quad 0)\} \\
                  \ell(\textbf{u}_4, 0) &=
                     \{(0 \quad 0), (1 \quad a), (a \quad b), (b \quad 1)\} \\
                  \ell(\textbf{u}_4, 1) &=
                     \{(0 \quad a), (1 \quad 0), (a \quad 1), (b \quad b)\} \\
                  \ell(\textbf{u}_4, a) &=
                     \{(0 \quad b), (1 \quad 1), (a \quad 0), (b \quad a)\} \\
                  \ell(\textbf{u}_4, b) &=
                     \{(0 \quad 1), (1 \quad b), (a \quad a), (b \quad 0)\} \\
                  \ell(\textbf{u}_5, 0) &=
                     \{(0 \quad 0), (1 \quad 0), (a \quad 0), (b \quad 0)\} \\
                  \ell(\textbf{u}_5, 1) &=
                     \{(0 \quad 1), (1 \quad 1), (a \quad 1), (b \quad 1)\} \\
                  \ell(\textbf{u}_5, a) &=
                     \{(0 \quad a), (1 \quad a), (a \quad a), (b \quad a)\} \\
                  \ell(\textbf{u}_5, b) &=
                     \{(0 \quad b), (1 \quad b), (a \quad b), (b \quad b)\}.
               \end{align*}
         \item We shall be working in the affine plane $\Z_7$, and we shall
               view every student as a point in the plane. Let $u_1$, $u_2$,
               $\ldots$, $u_8$ be the slopes in the plane. For each slope and
               each $x \in \Z_7$, there are seven points. Also recall
               $\ell(u_i, k_1) \neq \ell(u_i, k_2)$ if $k_1 \neq k_2$. So
               different lines with the same slopes partition the points.
               Since there are 8 slopes, all the
               students can meet one another if we divide them into groups of
               7, for 8 consecutive days, wherein each group corresponds to a
               line. For day $j$, the groups will consist of the following
               lines:
               $$\ell(u_j, 0), \ell(u_j, 1), \ell(u_j, 2), \ldots,
                 \ell(u_j, 6),$$
               where $1 \le j \le 8$.
               Note that since we have more points than students, there are
               days wherein each student will meet 4 or 5 (instead of 6) people
               in a group. Empty chairs may serve as the extraneous two points.
      \end{enumerate}
%%%%%%%%%%%%%%%%%%%%%%%%%%%%%%%%%%%%%%Bonus%%%%%%%%%%%%%%%%%%%%%%%%%%%%%%%%%%%%% 
   \item[\textbf{Bonus.}]  \textbf{The Projective Plane.} Let \textbf{P} be the
                           collection of one dimensional subspaces of $\F^3$
                           where $\F$ is a field with $q$ elements. Let
                           $\textbf{L}$ be the collection of two dimensional
                           subspaces of $\F^3$. So
                           $\textbf{P} = \{[z] : z \in \F^3, z \neq 0\}$,
                           $\textbf{L} = \{[z, w] : z, w \in \F^3,
                            \text{independent}\}$. We refer to \textbf{P} as 
                           the points in the geometry and \textbf{L} as the
                           lines, and of course we say point $p$
                           \textbf{is on line} $\ell$ if $p \le \ell$.

                           \begin{enumerate}[label=\protect\circled{\arabic*}]   
                              \item Show that every two points are on a unique
                                    line. 
                              \item Show that every two lines intersect in a
                                    unique point.
                              \item Show there exist 4 points no three of which
                                    are collinear (on a line together). \\

                                    The three properties above define an
                                    \textbf{Projective Plane.}
                              \item Show there exist 4 lines no three of which
                                    intersect at a point.
                              \item Give the lines and points when $\F = \Z_2$.
                              \item Give the lines and points when $\F = GF(4)$.
                           \end{enumerate}
                           
      \textbf{Solution.}
      
      \begin{enumerate}[label=\protect\circled{\arabic*}]
         \item Let $[p]$ and $[r]$ be different points in \textbf{P} ($p \neq 0$
               and $r \neq 0$ because $\dim [p] = \dim [r] = 1$). Since $[p]$
               and $[r]$ are different points, then they must be different one
               dimensional subspaces of $\F^3$. If $p$ is a multiple of $r$,
               then $r$ must also be a multiple of $p$, so that
               $[p] \subseteq [r]$ and $[r] \subseteq [p]$, and thus
               $[p] = [r]$, contradicting our assumption that these two points
               are different. Thus $p$ is not a multiple of $r$; that is,
               $p$ and $r$ are independent, so that $[p, r] \in \textbf{L}$.
               Since $p = 1 \cdot p + 0 \cdot r$ and
               $r = 0 \cdot p + 1 \cdot r$, it follows that both $[p]$ and
               $[r]$ are on the line $[p, r]$. Now suppose that $[p]$ and $[r]$
               lie on some other line $[x, y]$. Then it follows by closure
               under addition and scalar multiplication that
               $[p, r] \le [x, y]$. Note that $\dim [p, r] = \dim [x, y] = 2$.
               That is, a basis for $[p , r]$ is also a basis for $[x, y]$, so
               that $[p, r] = [x, y]$; thus, $[p, r]$ is the unique line that
               contains the points $[p]$ and $[r]$.
         \item Let $[x, y]$ and $[z, w]$ be two different lines in \textbf{L},
               and let $W = [x, y] \cap [z, w]$. To show that the lines $[x, y]$ and $[z, w]$ intersect at a
               unique point, it suffices to show that $\dim W = 1$. Since
               $W \le [x, y]$, it follows by Exercise 1.3 (Homework 3) that
               $\dim W \le \dim [x, y] = 2$. Suppose first that $\dim W = 2$.
               But since $W \le [x, y]$ and $\dim W = \dim [x, y]$, we conclude
               that $W = [x, y]$. Similarly, $W \le [z, w]$ and
               $\dim [z, w] = 2$, so that $W = [x, y] = [z, w]$, contradicting
               our assumption that $[x, y]$ and $[z, w]$ are two different
               lines. Now suppose that $\dim W = 0$. This implies that the lines
               $[x, y]$ and $[z, w]$ do not intersect. Recall that $x$ and $y$
               are linearly independent and $z$ and $w$ are linearly
               independent. Since $[x, y]$ and $[z, w]$ do not intersect, it
               follows that $\F^3$ has four different linearly independent
               vectors, namely $x$, $y$, $z$, and $w$, contradicting the fact
               that it can have at most 3, since $\dim \F^3 = 3$. Hence
               $\dim W > 0$, so that $\dim W = 1$, as desired. \qed
         \item Consider the points
               $$p_1 = [(0, 0, 1)], p_2 = [(0, 1, 1)], p_3 = [(1, 0, 0)],
                 \text{ and } [(1, 1, 0)].$$
               We shall be using the fact that two points are on a unique line
               and we shall also be referencing the lines in part \circled{5}.
               Points $p_1$ and $p_2$ are on Line 1 while $p_2$ and $p_3$ are
               on Line 6, so that $p_1$, $p_2$, and $p_3$ are not collinear.
               Points $p_2$ and $p_4$ are on Line 7, so that $p_1$, $p_2$, and
               $p_4$ are not collinear. Points $p_1$ and $p_3$ are on Line 2,
               while $p_3$ and $p_4$ are on Line 4, so that $p_1$, $p_3$, and
               $p_4$ are not collinear. Finally, points $p_2$, $p_3$, and $p_4$
               are not collinear because $p_2$ and $p_3$ are on Line 6 while
               $p_3$ and $p_4$ are on Line 4. Thus, no three points amongst
               $p_1$, $p_2$, $p_3$, and $p_4$ are collinear.
         \item Consider Line 1, Line 2, Line 4, and Line 7 from \circled{5}. Now
               if three lines, say $\ell_1$, $\ell_2$, and $\ell_3$, intersect
               at a point, then since two lines intersect at a unique point, it
               must be the case that $\ell_1 \cap \ell_2 = \ell_2 \cap \ell_3$.
               Now
               $$\text{Line 1 } \cap \text{ Line 2 } = [(0, 0, 1)] \neq
                 [(1, 0, 0)] = \text{Line 2 } \cap \text{ Line 4},$$
               so that Lines 1, 2, and 4 do not intersect at a point. Also since               
               $$\text{Line 1 } \cap \text{ Line 2 } = [(0, 0, 1)] \neq
                 [(1, 0, 1)] = \text{Line 2 } \cap \text{ Line 7},$$
               it follows that Lines 1, 2, and 7 do not intersect at a point.
               Similarly       
               $$\text{Line 1 } \cap \text{ Line 4 } = [(0, 1, 0)] \neq
                 [(1, 1, 0)] = \text{Line 4 } \cap \text{ Line 7},$$
               so that Lines 1, 4, and 7 do not intersect at a point. Finally we
               have that     
               $$\text{Line 2 } \cap \text{ Line 4 } = [(1, 0, 0)] \neq
                 [(1, 1, 0)] = \text{Line 4 } \cap \text{ Line 7},$$
               so that Lines 2, 4, and 7 do not intersect at a point. Thus we
               have shown that no three lines from Lines 1, 2, 4, and 7
               intersect at a point.
         \item Let $\F = \Z_2$. Then the points are:
               $$
               \begin{tabular}{@{}|c|c|@{}} \hline
                     & \textbf{Points.} \\ \hline
                  1. & {[(0, 0, 1)]} \\ \hline
                  2. & {[(0, 1, 1)]} \\ \hline
                  3. & {[(1, 0, 0)]} \\ \hline
                  4. & {[(0, 1, 0)]} \\ \hline
                  5. & {[(1, 0, 1)]} \\ \hline
                  6. & {[(1, 1, 0)]} \\ \hline
                  7. & {[(1, 1, 1)]} \\ \hline
               \end{tabular}
               $$
               and the lines are            
               $$
               \begin{tabular}{@{}|c|c|@{}} \hline
                     & \textbf{Lines.} \\ \hline
                  1. & {[(0, 0, 1), (0, 1, 0)]} \\ \hline
                  2. & {[(0, 0, 1), (1, 0, 0)]} \\ \hline
                  3. & {[(0, 0, 1), (1, 1, 0)]} \\ \hline
                  4. & {[(0, 1, 0), (1, 0, 0)]} \\ \hline
                  5. & {[(0, 1, 0), (1, 0, 1)]} \\ \hline
                  6. & {[(0, 1, 1), (1, 0, 0)]} \\ \hline
                  7. & {[(0, 1, 1), (1, 0, 1)]} \\ \hline
               \end{tabular}
               $$
         \item Let $\F = GF(4) = \{0, 1, a, b\}$ where addition and
               multiplication are defined as in the Bonus in Exercise 3. Then
               the points are:
               $$
               \begin{tabular}{@{}|c|c|@{}} \hline
                     & \textbf{Points.} \\ \hline
                   1. & {[(0, 0, 1)]} \\ \hline
                   2. & {[(0, 1, 0)]} \\ \hline
                   3. & {[(0, 1, 1)]} \\ \hline
                   4. & {[(0, 1, $a$)]} \\ \hline
                   5. & {[(0, 1, $b$)]} \\ \hline
                   6. & {[(1, 0, 0)]} \\ \hline
                   7. & {[(1, 0, 1)]} \\ \hline
                   8. & {[(1, 0, $a$)]} \\ \hline
                   9. & {[(1, 0, $b$)]} \\ \hline
                  10. & {[(1, 1, 0)]} \\ \hline
                  11. & {[(1, 1, 1)]} \\ \hline
                  12. & {[(1, 1, $a$)]} \\ \hline
                  13. & {[(1, 1, $b$)]} \\ \hline
                  14. & {[(1, $a$, 0)]} \\ \hline
                  15. & {[(1, $a$, 1)]} \\ \hline
                  16. & {[(1, $a$, $a$)]} \\ \hline
                  17. & {[(1, $a$, $b$)]} \\ \hline
                  18. & {[(1, $b$, 0)]} \\ \hline
                  19. & {[(1, $b$, 1)]} \\ \hline
                  20. & {[(1, $b$, $a$)]} \\ \hline
                  21. & {[(1, $b$, $b$)]} \\ \hline
               \end{tabular}
               $$
               and the lines are           
               $$
               \begin{tabular}{@{}|c|c|@{}} \hline
                     & \textbf{Lines.} \\ \hline
                   1. & {[(1, 0, 0), (0, 1, 0)]} \\ \hline
                   2. & {[(1, 0, 0), (0, 0, 1)]} \\ \hline
                   3. & {[(1, 0, 0), (0, 1, 1)]} \\ \hline
                   4. & {[(0, $a$, 1), (1, 0, 0)]} \\ \hline
                   5. & {[(1, 0, 0), (0, $b$, 1)]} \\ \hline
                   6. & {[(0, 1, 0), (0, 0, 1)]} \\ \hline
                   7. & {[(0, 1, 0), (1, 0, 1)]} \\ \hline
                   8. & {[(0, 1, 0), ($a$, 0, 1)]} \\ \hline
                   9. & {[(0, 1, 0), ($b$, 0, 1)]} \\ \hline
                  10. & {[(1, 1, 0), (0, 0, 1)]} \\ \hline
                  11. & {[(1, 1, 0), (1, 0, 1)]} \\ \hline
                  12. & {[(1, 1, 0), ($a$, 0, 1)]} \\ \hline
                  13. & {[(1, 1, 0), ($b$, 0, 1)]} \\ \hline
                  14. & {[($a$, 1, 0), (0, 0, 1)]} \\ \hline
                  15. & {[($b$, 1, 0), (1, 0, 1)]} \\ \hline
                  16. & {[($a$, 1, 0), ($a$, 0, 1)]} \\ \hline
                  17. & {[($a$, 1, 0), ($b$, 0, 1)]} \\ \hline
                  18. & {[($b$, 1, 0), (0, 0, 1)]} \\ \hline
                  19. & {[($a$, 1, 0), (1, 0, 1)]} \\ \hline
                  20. & {[($b$, 1, 0), ($a$, 0, 1)]} \\ \hline
                  21. & {[($b$, 1, 0), ($b$, 0, 1)]} \\ \hline
               \end{tabular}
               $$
      \end{enumerate}
\end{enumerate}
\end{document}
