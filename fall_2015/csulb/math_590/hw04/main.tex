\documentclass[9pt]{article}

\usepackage{amssymb}
\usepackage{amsmath, array}
\usepackage{amsfonts}
\usepackage{comment}
\usepackage{fancyhdr}
\usepackage{mathrsfs}
\usepackage{enumitem}
\usepackage{bm}
\usepackage{tikz}

\voffset = -50pt
%\textheight = 700pt
\addtolength{\textwidth}{60pt}
\addtolength{\evensidemargin}{-30pt}
\addtolength{\oddsidemargin}{-30pt}
%\setlength{\headheight}{44pt}

\pagestyle{fancy}
\fancyhf{} % clear all fields
\fancyhead[R]{%
  \scshape
  \begin{tabular}[t]{@{}r@{}}
  MATH 590, Fall 2015\\Section 1 (9529)\\
  HW \#04, DUE: 2015, September 21
  \end{tabular}}
\fancyhead[L]{%
  \scshape
  \begin{tabular}[t]{@{}r@{}}
  JOSEPH OKONOBOH\\Mathematics\\Cal State Long Beach
  \end{tabular}}
\fancyfoot[C]{\thepage}

\newcommand{\qed}{\hfill \ensuremath{\Box}}

\newcommand{\col}[2]{\left(\begin{tabular}{@{}c@{}}
   $#1$ \\
   $#2$  
 \end{tabular}\right)}

\newcommand*\circled[1]{\tikz[baseline=(char.base)]{
            \node[shape=circle,draw,inner sep=2pt] (char) {#1};}}

\newcommand{\Z}{\mathbb{Z}}
\newcommand{\I}{\mathbb{I}}
\newcommand{\F}{\mathbb{F}}
\newcommand{\Q}{\mathbb{Q}}
\newcommand{\R}{\mathbb{R}}
\newcommand{\C}{\mathbb{C}}
\renewcommand{\S}{\mathbb{S}}
\newcommand{\N}{\mathbb{N}}
\newcommand{\D}{\displaystyle}
%\setcounter{section}{-1}

\begin{document}
\begin{enumerate}[label=\protect\circled{\arabic*}]
%%%%%%%%%%%%%%%%%%%%%%%%%%%%%%%%%%%%%%%01%%%%%%%%%%%%%%%%%%%%%%%%%%%%%%%%%%%%%%%
   \item Consider the veracity or falsehood of each of the following statements.
         For bonus, argue for those that you believe are true while providing a
         counterexample for those that you believe are false. Work either
         column. These all concern the affine plane (see below) built using a
         field $\F$ with q elements.

         \begin{enumerate}[label=\protect\circled{\arabic*}]
            \item There are $q^2$ points.
            \item There are $q^2 + 1$ lines.
            \item Every line has $q$ points.
            \item Every point is in $q$ lines.
            \item No two lines contain all points.
         \end{enumerate}      
%%%%%%%%%%%%%%%%%%%%%%%%%%%%%%%%%%%%%%%02%%%%%%%%%%%%%%%%%%%%%%%%%%%%%%%%%%%%%%%
   \item In the last homework you built an extension of the Hamming code using
         columns of size 4 over $\Z_2$. Give the weight enumerator of the code
         that is generated by that matrix---namely it is the row space of the
         matrix---not the null space.

         \textbf{Bonus.} Find the weight enumerator of the Hamming code that is
         the null space of the matrix.
%%%%%%%%%%%%%%%%%%%%%%%%%%%%%%%%%%%%%%%03%%%%%%%%%%%%%%%%%%%%%%%%%%%%%%%%%%%%%%%
   \item \textbf{The Affine Plane.} Let $\textbf{P} = \F^2 =
         \{(a \quad b) : a, b \in \F\}$ be the points in a geometry, where $\F$
         is a field with $q$ elements. Let
         $S = \left\{\col{1}{x}, \col{0}{1} : x \in \F\right\}$ be the set of
         slopes. For any $\textbf{u} \in S$ and any $k \in \F$ define the line
         $\ell(\textbf{u}, k) = \{(a \quad b) : (a \quad b)\textbf{u} = k\}$,
         which is a set of points. Then the set of lines
         $\textbf{L} = \{\ell(\textbf{u}, k) : \textbf{u} \in S, k \in \F\}$.

         \begin{enumerate}[label=\protect\circled{\arabic*}]
            \item Show that every two points are on a unique line.
            \item Show that if $p \in \textbf{P}$ and $\ell \in \textbf{L}$ and
                  $p \notin \ell$, then there exists a unique $\ell_1$ such that
                  $p \in \ell_1$ and $\ell$ and $\ell_1$ are parallel (they do
                  not intersect).
            \item Show there exist 4 points no three of which are collinear (on
                  a line together). \\

                  The three properties above define an \textbf{Affine Plane.}
            \item Give the lines and points when $\F = \Z_2$. \\
            
                  \textbf{Bonus.} Give the lines and points when $\F = GF(4)$.
            \item Use an affine plane to describe how you would solve the
                  following problem that was asked to a member of the department
                  from a professor teaching in the humanities: \\
      
                  \textbf{I currently have 47 students in my class. I am 
                  teaching in a room with cafe tables of 7 per table. If I 
                  wanted to have each student get a chance to meet every other
                  student in the class, how many days would it take and how 
                  would I figure out membership in the groups? I might end up
                  with 52, likely not more. I could fit 8 around a table if need
                  be.}
         \end{enumerate}
%%%%%%%%%%%%%%%%%%%%%%%%%%%%%%%%%%%%%%Bonus%%%%%%%%%%%%%%%%%%%%%%%%%%%%%%%%%%%%% 
   \item[\textbf{Bonus.}]  \textbf{The Projective Plane.} Let \textbf{P} be the
                           collection of one dimensional subspaces of $\F^3$
                           where $\F$ is a field with $q$ elements. Let
                           $\textbf{L}$ be the collection of two dimensional
                           subspaces of $\F^3$. So
                           $\textbf{P} = \{[z] : z \in \F^3, z \neq 0\}$,
                           $\textbf{L} = \{[z, w] : z, w \in \F^3,
                            \text{independent}\}$. We refer to \textbf{P} as 
                           the points in the geometry and \textbf{L} as the
                           lines, and of course we say point $p$
                           \textbf{is on line} $\ell$ if $p \le \ell$.

                           \begin{enumerate}[label=\protect\circled{\arabic*}]   
                              \item Show that every two points are on a unique
                                    line. 
                              \item Show that every two lines intersect in a
                                    unique point.
                              \item Show there exist 4 points no three of which
                                    are collinear (on a line together). \\

                                    The three properties above define an
                                    \textbf{Projective Plane.}
                              \item Show there exist 4 lines no three of which
                                    intersect at a point.
                              \item Give the lines and points when $\F = \Z_2$.
                              \item Give the lines and points when $\F = GF(4)$.
                           \end{enumerate}
\end{enumerate}
\end{document}
