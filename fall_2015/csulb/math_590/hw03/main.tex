\documentclass[9pt]{article}

\usepackage{amssymb}
\usepackage{amsmath, array}
\usepackage{amsfonts}
\usepackage{comment}
\usepackage{fancyhdr}
\usepackage{mathrsfs}
\usepackage{enumitem}
\usepackage{bm}


\usepackage{tikz}

\voffset = -50pt
%\textheight = 700pt
\addtolength{\textwidth}{60pt}
\addtolength{\evensidemargin}{-30pt}
\addtolength{\oddsidemargin}{-30pt}
%\setlength{\headheight}{44pt}

\pagestyle{fancy}
\fancyhf{} % clear all fields
\fancyhead[R]{%
  \scshape
  \begin{tabular}[t]{@{}r@{}}
  MATH 590, Fall 2015\\Section 1 (9529)\\
  HW \#03, DUE: 2015, September 14
  \end{tabular}}
\fancyhead[L]{%
  \scshape
  \begin{tabular}[t]{@{}r@{}}
  JOSEPH OKONOBOH\\Mathematics\\Cal State Long Beach
  \end{tabular}}
\fancyfoot[C]{\thepage}

\newcommand{\qed}{\hfill \ensuremath{\Box}}

\newcommand{\col}[2]{\left(\begin{tabular}{@{}c@{}}
   $#1$ \\
   $#2$  
 \end{tabular}\right)}

\newcommand*\circled[1]{\tikz[baseline=(char.base)]{
            \node[shape=circle,draw,inner sep=2pt] (char) {#1};}}

\newcommand{\Z}{\mathbb{Z}}
\newcommand{\I}{\mathbb{I}}
\newcommand{\F}{\mathbb{F}}
\newcommand{\Q}{\mathbb{Q}}
\newcommand{\R}{\mathbb{R}}
\newcommand{\C}{\mathbb{C}}
\renewcommand{\S}{\mathbb{S}}
\newcommand{\N}{\mathbb{N}}
\newcommand{\D}{\displaystyle}
%\setcounter{section}{-1}

\begin{document}
\begin{enumerate}
%%%%%%%%%%%%%%%%%%%%%%%%%%%%%%%%%%%%%%%01%%%%%%%%%%%%%%%%%%%%%%%%%%%%%%%%%%%%%%%
   \item Consider the veracity or falsehood of each of the following statements.
         For bonus, argue for those that you believe are true while providing a
         counterexample for those that you believe are false. Let $V$, $W$ be
         subspaces of $F^n$ where $|F| = q$.

         \begin{enumerate}[label=\protect\circled{\arabic*}]
            \item The number of subspaces of $F^n$ of dimension $n-1$ is
                  $1+q+q^2+\cdots+q^{n-1}$.
            \item $V+W = \{v+w : v \in V, w \in W\}$ is the smallest subspace
                  containing both $V$ and $W$.
            \item If $V \le W$, then $\dim V \le \dim W$.
            \item If $\dim V = 2$, then $V$ has $q^2 - q$ bases.
            \item $\dim(V+W) = \dim V + \dim W - \dim(V \cap W)$.
         \end{enumerate}
%%%%%%%%%%%%%%%%%%%%%%%%%%%%%%%%%%%%%%%02%%%%%%%%%%%%%%%%%%%%%%%%%%%%%%%%%%%%%%%
   \item Suppose your alphabet consists of four symbols: 0, 1, $\alpha$, and
         $\beta$.

         \begin{enumerate}[label=\protect\circled{\arabic*}]
            \item How many words of length $n$ can be made with these four
                  letters?
            \item How many words are there in a ball of radius 2?
            \item Suppose we wanted a code with at least one hundred words that
                  will be able to correct two errors. What is the minimum length
                  of such a code?
         \end{enumerate}
%%%%%%%%%%%%%%%%%%%%%%%%%%%%%%%%%%%%%%%03%%%%%%%%%%%%%%%%%%%%%%%%%%%%%%%%%%%%%%%
   \item The parity check matrix for the Hamming Code was built by using all
         nonzero columns of size 3 over $\Z_2$ obtaining the $3 \times 7$ matrix
         of rank 4.

         \begin{enumerate}[label=\protect\circled{\arabic*}]
            \item Emulate the construction by using all the nonzero columns of
                  size 4. Using this new matrix as a parity check matrix for a
                  code, what is the length of the code and what is its
                  dimension.
            \item Decide if this is also a perfect code?
            \item Assume the probability of sending a correct bit is $p$. What
                  is the probability of correct decoding in this code?
            \item Would you say it is superior to the Hamming Code? Why or why
                  not?
         \end{enumerate}
%%%%%%%%%%%%%%%%%%%%%%%%%%%%%%%%%%%%%%Bonus%%%%%%%%%%%%%%%%%%%%%%%%%%%%%%%%%%%%%
\end{enumerate}
\end{document}
