\documentclass[9pt]{article}

\usepackage{amssymb}
\usepackage{amsmath, array}
\usepackage{amsfonts}
\usepackage{comment}
\usepackage{fancyhdr}
\usepackage{mathrsfs}
\usepackage{enumitem}
\usepackage{bm}


\usepackage{tikz}

\voffset = -50pt
%\textheight = 700pt
\addtolength{\textwidth}{60pt}
\addtolength{\evensidemargin}{-30pt}
\addtolength{\oddsidemargin}{-30pt}
%\setlength{\headheight}{44pt}

\pagestyle{fancy}
\fancyhf{} % clear all fields
\fancyhead[R]{%
  \scshape
  \begin{tabular}[t]{@{}r@{}}
  MATH 590, Fall 2015\\Section 1 (9529)\\
  HW \#03, DUE: 2015, September 14
  \end{tabular}}
\fancyhead[L]{%
  \scshape
  \begin{tabular}[t]{@{}r@{}}
  JOSEPH OKONOBOH\\Mathematics\\Cal State Long Beach
  \end{tabular}}
\fancyfoot[C]{\thepage}

\newcommand{\qed}{\hfill \ensuremath{\Box}}

\newcommand{\col}[2]{\left(\begin{tabular}{@{}c@{}}
   $#1$ \\
   $#2$  
 \end{tabular}\right)}

\newcommand*\circled[1]{\tikz[baseline=(char.base)]{
            \node[shape=circle,draw,inner sep=2pt] (char) {#1};}}

\newcommand{\Z}{\mathbb{Z}}
\newcommand{\I}{\mathbb{I}}
\newcommand{\F}{\mathbb{F}}
\newcommand{\Q}{\mathbb{Q}}
\newcommand{\R}{\mathbb{R}}
\newcommand{\C}{\mathbb{C}}
\renewcommand{\S}{\mathbb{S}}
\newcommand{\N}{\mathbb{N}}
\newcommand{\D}{\displaystyle}
%\setcounter{section}{-1}

\begin{document}
\begin{enumerate}
%%%%%%%%%%%%%%%%%%%%%%%%%%%%%%%%%%%%%%%01%%%%%%%%%%%%%%%%%%%%%%%%%%%%%%%%%%%%%%%
   \item Consider the veracity or falsehood of each of the following statements.
         For bonus, argue for those that you believe are true while providing a
         counterexample for those that you believe are false. Let $V$, $W$ be
         subspaces of $F^n$ where $|F| = q$.

         \begin{enumerate}[label=\protect\circled{\arabic*}]
            \item The number of subspaces of $F^n$ of dimension $n-1$ is
                  $1+q+q^2+\cdots+q^{n-1}$.
            \item $V+W = \{v+w : v \in V, w \in W\}$ is the smallest subspace
                  containing both $V$ and $W$.
            \item If $V \le W$, then $\dim V \le \dim W$.
            \item If $\dim V = 2$, then $V$ has $q^2 - q$ bases.
            \item $\dim(V+W) = \dim V + \dim W - \dim(V \cap W)$.
         \end{enumerate}
         
      \textbf{Solution.}

      \begin{enumerate}[label=\protect\circled{\arabic*}]
            \item True.
            
                  \textbf{Proof.} Suppose $W$ is a subspace of $F^n$ such that
                  $\dim W = n-1$. Consider the orthogonal complement of $W$,
                  $W^{\perp}$. It follows from our discussion in class that
                  $\dim W^{\perp} = n - (n - 1) = 1$. Recall from Linear
                  Algebra that $F^n = W \oplus W^{\perp}$; that is, every vector
                  in $F^n$ can be written as a unique sum of a vector in $W$ and
                  a vector in $W^{\perp}$. So there is a one to one
                  correspondence between the subspaces of dimension $n-1$ and
                  the subspaces of dimension 1. So we count the latter since it
                  is easier. To form a vector space of dimension 1, we choose
                  a nonzero vector (to form a singleton basis) from all possible
                  choices---$q^n$ total vectors; but we cannot choose the zero
                  vector, so we have a total choice of $q^n - 1$. Now notice
                  that we have over counted since all the $q^n-1$ vectors are a
                  multiple of exactly $q-1$ other vectors. Thus, we have
                  $\D\frac{q^n-1}{q-1}$ choices. And since
                  $$\frac{q^n-1}{q-1} = 1+q+q^2+\cdots+q^{n-1},$$
                  the proof is done. \qed
            \item True.
            
                  \textbf{Proof.} Suppose a vector space $X$ contains $V$ and
                  $W$. Let $u \in V + W$. Then it follows that there exist
                  $v \in V$ and $w \in W$ such that $u = v + w$; particularly,
                  we have that $v, w \in X$. By closure, it follows that
                  $u = v + w \in X$. That is, $X$ contains $V + W$. Since $X$
                  was arbitrary, it follows that $V + W$ is contained in every
                  vector space that contains $V$ and $W$. To complete the proof, we need to show that $V+W$ is indeed a
                  subspace. Let $a_1, a_2 \in V + W$ and $\alpha \in F$. It
                  follows by definition that $a_1 = v_1 + w_1$ and
                  $a_2 = v_2 + w_2$ for some $v_1, v_2 \in V$ and
                  $w_1, w_2 \in W$. By commutativity of addition, we
                  have that
                  $$a_1 + a_2 = (v_1 + w_1) + (v_2 + w_2) = (v_1 + v_2) +
                  (w_1 + w_2),$$
                  so that $a_1 + a_2 \in V + W$ since, by closure, we have that
                  $v_1 + v_2 \in V$ and $w_1 + w_2 \in W$. Finally, we have
                  that $\alpha a_1 = \alpha(v_1 + w_1) = \alpha v_1 +
                  \alpha w_1 \in V$, since $V$ is closed under addition and
                  scalar multiplication. Hence
                  $\alpha a_1 = \alpha a_1 + 0 \in V + W$; that is, $V + W$ is
                  closed under addition and scalar multiplication, so that it
                  is a subspace. \qed
            \item True.
            
                  \textbf{Proof.} Suppose $V \le W$. Let $\dim V = n$ and
                  $\dim W = m$. Suppose to the contrary that $n > m$. We know
                  that $V$ has $n$ independent vectors. Since $V \subseteq W$,
                  these $n$ independent vectors must also be in $W$, so that $W$
                  has $n$ linearly independent vectors, a contradiction since
                  its dimension, $m$, wa assumed to be less than $n$. Thus
                  $\dim V \le \dim W$. \qed
            \item False.
            
                  \textbf{Counterexample.} Let $F ={\Z_2}$ and $n = 2$. The
                   hypothesis says that we would have $2^2 - 2 = 2$ bases, but
                   we have the following three bases:
                   $$\{(1, 1), (0, 1)\}, \{(1, 1), (1, 0)\}, \text{ and }
                     \{(1, 0), (0, 1)\}$$
            \item True.
            
                  \textbf{Proof.} Let $V$ and $W$ be vector spaces. If either
                  one is contained in the other, the proof is trivial, so assume
                  otherwise. Now let $\dim V \cap W = n$. Since we are assuming
                  that $V \not\subseteq W$ and $W \not \subseteq V$, and since
                  the intersection of both vector spaces is contained in both
                  vector spaces, it follows that $n < \dim V$ and $n < \dim W$.
                  Let $\{u_1, \ldots, u_n\}$ be a basis for $V \cap W$. Extend
                  this to $\{u_1, \ldots, u_n, v_1, \ldots v_{r}\}$, a basis for
                  $V$, and $\{u_1, \ldots, u_n, w_1, \ldots, w_m\}$, a basis for
                  $W$. That is, $\dim V = n + r$ and $\dim W = n + m$. Now we
                  want to show that $\dim(V+W) = n + m + r$. To that end we
                  claim that $X = \{u_1, \ldots, u_n, v_1, \ldots, v_r, w_1,
                  \ldots, w_m\}$ is a basis for $V+W$. First, we want to show
                  that $X$ spans $V+W$, so let $v + w \in V + W$. Since $v$ can
                  be written as a linear combination of
                  $u_1, \ldots, u_n, v_1, \ldots v_{r}$ and since $w$ can be
                  written as a linear combination of
                  $u_1, \ldots, u_n, w_1, \ldots, w_m$, it follows that $v+w$
                  can be written as a linear combination of the vectors in $X$.
                  That is, $X$ spans $V+W$. To show linear independence,
                  consider the linear equation:
                  \begin{equation} \label{1_1}
                     a_1u_1 + \cdots + a_nu_n + b_1v_1 + \cdots + b_rv_r +
                        c_1w_1 + \cdots + c_mw_m = 0.
                  \end{equation}
                  From \eqref{1_1}, we have that
                  $$b_1v_1 + \cdots + b_rv_r = -a_1u_1 - \cdots - a_nu_n - 
                       c_1w_1 - \cdots - c_mw_m,$$
                  so that $b_1v_1 + \cdots + b_rv_r \in W$ and
                  $$c_1w_1 + \cdots + c_mw_m = -a_1u_1 - \cdots - a_nu_n -
                       b_1v_1 - \cdots - b_rv_r,$$
                  so that $c_1w_1 + \cdots + c_mw_m \in V$. Clearly
                  $b_1v_1 + \cdots + b_rv_r \in V$ and
                  $c_1w_1 + \cdots + c_mw_m \in W$. Thus we have that
                  $$b_1v_1 + \cdots + b_rv_r, c_1w_1 + \cdots + c_mw_m \in
                   V + W.$$
                  So we can write $b_1v_1 + \cdots + b_rv_r + c_1w_1 + \cdots +
                  c_mw_m$ as a linear combination of $u_1, \ldots, u_n$, so that
                  the sum in \eqref{1_1} can be written as a linear combination
                  of $u_1, \ldots, u_n$. Use the linear independence of 
                  $u_1, \ldots, u_n$ to conclude that
                  $$a_1 = \cdots = a_n = 0.$$
                  Thus \eqref{1_1} reduces to
                  \begin{equation} \label{1_2}
                     b_1v_1 + \cdots + b_rv_r = -c_1w_1 - \cdots - c_mw_m.
                  \end{equation}
                  Equation \eqref{1_2} tells us that
                  $b_1v_1 + \cdots + b_rv_r \in W$ and
                  $-c_1w_1 - \cdots - c_mw_m \in V$. That is,
                  $b_1v_1 + \cdots + b_rv_r$ and $-c_1w_1 - \cdots - c_mw_m$ are
                  both in $V + W$, so that
                  $$b_1v_1 + \cdots + b_rv_r = d_1u_1 + \cdots + d_nu_n$$
                  $$-c_1w_1 - \cdots - c_mw_m = e_1u_1 + \cdots + e_nu_n.$$
                  Subtracting the right-hand side of the two equations above
                  from the left-hand side, and using the fact that the vectors
                  $v_1, \ldots, v_r, u_1, \ldots, u_n$ form a basis for $V$
                  and that $\{w_1, \ldots, w_m, u_1, \ldots, u_n\}$ is a basis
                  for $W$, we conclude that
                  $$b_1 = \cdots b_r = c_1 = \cdots = c_m = 0$$
                  so that the vectors in $X$ are linearly independent and thus
                  form a basis for $V + W$. Now we have that
                  \begin{align*}
                     \dim(V+W) &= n + r + m \\
                               &= n + r + n + m - n \\
                               &= \dim V + \dim W - \dim(V \cap W),
                  \end{align*}
                  as desired. \qed
                  
                  
                  
       \end{enumerate}
      
      
%%%%%%%%%%%%%%%%%%%%%%%%%%%%%%%%%%%%%%%02%%%%%%%%%%%%%%%%%%%%%%%%%%%%%%%%%%%%%%%
   \item Suppose your alphabet consists of four symbols: 0, 1, $\alpha$, and
         $\beta$.

         \begin{enumerate}[label=\protect\circled{\arabic*}]
            \item How many words of length $n$ can be made with these four
                  letters?
            \item How many words are there in a ball of radius 2?
            \item Suppose we wanted a code with at least one hundred words that
                  will be able to correct two errors. What is the minimum length
                  of such a code?
         \end{enumerate}
         
      \textbf{Solution.}

      \begin{enumerate}[label=\protect\circled{\arabic*}]
         \item $4^n$.
         \item Let $w$ be a word. Then
               \begin{align*}
                  B_2(w) &= S_0(w) + S_1(w) + S_2(w) \\
                     &= 1 + 3n + 3^2 \cdot \binom{n}{2} \\
                     &= \frac{9n^2-3n+2}{2}.
               \end{align*}
         \item To be able to correct two errors, we must find $n$ such that
               $$100\left(\frac{9n^2-3n+2}{2}\right) \le 4^n.$$
               A few computations for $n$ will show us that the minimum $n$ 
               that works is 8.
      \end{enumerate}
      
      
%%%%%%%%%%%%%%%%%%%%%%%%%%%%%%%%%%%%%%%03%%%%%%%%%%%%%%%%%%%%%%%%%%%%%%%%%%%%%%%
   \item The parity check matrix for the Hamming Code was built by using all
         nonzero columns of size 3 over $\Z_2$ obtaining the $3 \times 7$ matrix
         of rank 4.

         \begin{enumerate}[label=\protect\circled{\arabic*}]
            \item Emulate the construction by using all the nonzero columns of
                  size 4. Using this new matrix as a parity check matrix for a
                  code, what is the length of the code and what is its
                  dimension.
            \item Decide if this is also a perfect code?
            \item Assume the probability of sending a correct bit is $p$. What
                  is the probability of correct decoding in this code?
            \item Would you say it is superior to the Hamming Code? Why or why
                  not?
         \end{enumerate}
         
      \textbf{Solution.}

      \begin{enumerate}[label=\protect\circled{\arabic*}]
         \item The new matrix is         
               $$\left(
               \begin{tabular}{@{}ccccccccccccccc@{}}
                  0 & 0 & 0 & 0 & 0 & 0 & 0 & 1 & 1 & 1 & 1 & 1 & 1 & 1 & 1 \\
                  0 & 0 & 0 & 1 & 1 & 1 & 1 & 0 & 0 & 0 & 0 & 1 & 1 & 1 & 1 \\
                  0 & 1 & 1 & 0 & 0 & 1 & 1 & 0 & 0 & 1 & 1 & 0 & 0 & 1 & 1 \\
                  1 & 0 & 1 & 0 & 1 & 0 & 1 & 0 & 1 & 0 & 1 & 0 & 1 & 0 & 1 \\
               \end{tabular}\right).$$               
               Thus the length of the code is 15 and the number of meaningful
               bits is 11 since the dimension of the nullspace of the matrix is
               11.
         \item We have $k = 11$ and $n = 15$. Using the sphere-packing bound
               inequality, we have $2^k(1+n) = 32768 = 2^n$, so that the code is
               perfect.
         \item
               $$
               \begin{tabular}{@{}|c|c|c|@{}} \hline
                  \# of errors & \# of words & \# Probability of word \\ \hline
                  0 & 1 & $p^{15}$ \\ \hline
                  1 & 15 & $p^{14}(1-p)$ \\ \hline
               \end{tabular}
               $$
               Thus the probability of correct decoding is
               $$p^{15} + 15p^{14}(1-p).$$
         \item This code is inferior to the Hamming code because for
               $0 < p < 1$, (where $p$ is the probability of decoding a
               message), we have $p^7 + 7p^6(1-p) > p^{15} + 15p^{14}(1-p)$, so
               that the probability of the Hamming code is much larger than that
               of this new code. The expression $p^7 + 7p^6(1-p)$ is the
               probability that the Hamming code correctly decodes.
      \end{enumerate}
      
%%%%%%%%%%%%%%%%%%%%%%%%%%%%%%%%%%%%%%Bonus%%%%%%%%%%%%%%%%%%%%%%%%%%%%%%%%%%%%%
\end{enumerate}
\end{document}
