\documentclass[9pt]{article}

\usepackage{amssymb}
\usepackage{amsmath, array}
\usepackage{amsfonts}
\usepackage{comment}
\usepackage{fancyhdr}
\usepackage{mathrsfs}
\usepackage{enumitem}
\usepackage{bm}


\usepackage{tikz}

\voffset = -50pt
%\textheight = 700pt
\addtolength{\textwidth}{60pt}
\addtolength{\evensidemargin}{-30pt}
\addtolength{\oddsidemargin}{-30pt}
%\setlength{\headheight}{44pt}

\pagestyle{fancy}
\fancyhf{} % clear all fields
\fancyhead[R]{%
  \scshape
  \begin{tabular}[t]{@{}r@{}}
  CECS 328, Fall 2015\\Section 1 (1273/3120)\\
  HW \#01, DUE: 2015, August 27
  \end{tabular}}
\fancyhead[L]{%
  \scshape
  \begin{tabular}[t]{@{}r@{}}
  JOSEPH OKONOBOH\\Computer Science\\Cal State Long Beach
  \end{tabular}}
\fancyfoot[C]{\thepage}

\newcommand{\qed}{\hfill \ensuremath{\Box}}


\newcommand*\circled[1]{\tikz[baseline=(char.base)]{
            \node[shape=circle,draw,inner sep=2pt] (char) {#1};}}

\newcommand{\Z}{\mathbb{Z}}
\newcommand{\I}{\mathbb{I}}
\newcommand{\F}{\mathbb{F}}
\newcommand{\Q}{\mathbb{Q}}
\newcommand{\R}{\mathbb{R}}
\newcommand{\C}{\mathbb{C}}
\newcommand{\D}{\displaystyle}
%\setcounter{section}{-1}

\begin{document}
\begin{enumerate}
%%%%%%%%%%%%%%%%%%%%%%%%%%%%%%%%%%%%%%%01%%%%%%%%%%%%%%%%%%%%%%%%%%%%%%%%%%%%%%%
   \item Compute the exact values for

         \begin{enumerate}
            \item $\D\sum_{i=-1}^43$.
            \item $\D\sum_{i=1}^n3$.
            \item $\D\sum_{i=1}^n (i^3 + 2i^2 - i + 1)$.
            \item $\D\sum_{i=1}^n \left(2^i - 4i + \frac{i}{5}\right)$.
            \item $\D\sum_{j=1}^k\sum_{i=1}^j (i - j^2 - 2)$.
            \item $\D\sum_{j=1}^m\sum_{k=1}^j (3C + k - 3j + i)$.
            \item $\D\sum_{l=1}^n\sum_{j=1}^k\sum_{i=1}^j (i - 4)$.
         \end{enumerate}
      
      \textbf{Solution.} 

      \begin{enumerate}
         \item $\D\sum_{i=-1}^43 = 3 \cdot 6 = 18$.
         \item $\D\sum_{i=1}^n3 = 3n$.
         \item \begin{align*}
                  \sum_{i=1}^n (i^3 + 2i^2 - i + 1) &= \sum_{i=1}^n i^3 +
                     2\sum_{i=1}^ni^2 - \sum_{i=1}^ni + \sum_{i=1}^n1 \\
                  &= \left(\frac{n(n+1)}{2}\right)^2 + \frac{n(n+1)(2n+1)}{3} -
                     \frac{n(n+1)}{2} + n \\
                  &= \frac{n(3n^3 + 14n^2 + 9n + 10)}{12}.
               \end{align*}
         \item $\D\sum_{i=1}^n \left(2^i - 4i + \frac{i}{5}\right) = 2^{n}-1
                -4\frac{n(n+1)}{2} + \frac{1}{5}\frac{n(n+1)}{2}$.
         \item \begin{align*}
                  \sum_{j=1}^k\sum_{i=1}^j (i - j^2 - 2) &=
                     \sum_{j=1}^k\left(\sum_{i=1}^ji - \sum_{i=1}^jj^2 -
                     \sum_{i=1}^j2\right) \\
                     &= \sum_{j=1}^k\left(\frac{j(j+1)}{2} - j^3 -2j\right) \\
                     &= \sum_{j=1}^k\left(-j^3 + \frac{1}{2}j^2 -
                           \frac{3}{2}j \right) \\
                     &= -\sum_{j=1}^kj^3 + \frac{1}{2}\sum_{j=1}^kj^2 -
                           \frac{3}{2}\sum_{j=1}^kj \\
                     &= -\left(\frac{k(k+1)}{2}\right)^2 +
                           \frac{k(k+1)(2k+1)}{12} - \frac{3k(k+1)}{4} \\
                     &= -\frac{k(k + 1)(3k^2 + k + 8)}{12}
               \end{align*}
         \item \begin{align*}
                  \sum_{j=1}^m\sum_{k=1}^j (3C + k - 3j + i) &=
                     \sum_{j=1}^m\left(\sum_{k=1}^j3C + \sum_{k=1}^jk -
                     \sum_{k=1}^j3j + \sum_{k=1}^ji\right) \\
                     &= \sum_{j=1}^m\left(3Cj + \frac{j(j+1)}{2} - 3j^2 +
                           ji\right) \\
                     &= \sum_{j=1}^m\left(-\frac{5}{2}j^2 +
                        \frac{6C+2i+1}{2}j\right) \\
                     &= -\frac{5}{2}\sum_{j=1}^mj^2 +
                        \frac{6C+2i+1}{2}\sum_{j=1}^mj \\
                     &= -\frac{5}{12}m(m+1)(2m+1) + \frac{1}{4}m(m+1)(6C+2i+1)\\
                     &= \frac{1}{12} m(m+1)[3(6C + 2i + 1)-5(2m+1)].
               \end{align*}
         \item \begin{align*}
                  \sum_{l=1}^n\sum_{j=1}^k\sum_{i=1}^j (i - 4) &=
                     \sum_{l=1}^n\sum_{j=1}^k\left(\sum_{i=1}^ji -
                     \sum_{i=1}^j4\right) \\
                     &= \sum_{l=1}^n\sum_{j=1}^k\left(\frac{j(j+1)}{2} -
                        4j\right) \\
                     &= \sum_{l=1}^n\sum_{j=1}^k\left(\frac{1}{2}j^2 -
                           \frac{7}{2}j\right) \\
                     &= \sum_{l=1}^n\left(\frac{1}{2}\sum_{j=1}^kj^2 -
                           \frac{7}{2}\sum_{j=1}^kj\right) \\
                     &= \sum_{l=1}^n\left(\frac{1}{12}k(k+1)(2k+1) -
                           \frac{7}{4}k(k+1)\right) \\
                     &= \frac{1}{6} nk(k+1)(k-10).
               \end{align*}
      \end{enumerate}
%%%%%%%%%%%%%%%%%%%%%%%%%%%%%%%%%%%%%%%02%%%%%%%%%%%%%%%%%%%%%%%%%%%%%%%%%%%%%%%
   \item Compute the derivative of

         \begin{enumerate}
            \item $-5x^3 + 2x - 1$.
            \item $3x^4 - 2\sqrt{x} + x^{1/2} - 6x^{-2/3} - 5$.
            \item $x\sqrt{x} + \sqrt{\sqrt{x}}$.
            \item $\log(x) - x^2\ln(x) + \ln(x^4)$.
            \item $\ln^3(x\sqrt{2x - 3}) + \sqrt{\ln(x^2)}$.
            \item $\D\frac{\sqrt{x+5} - \ln(x)}{(x - 1)^3}$.
            \item $\D\frac{\sqrt{x\sqrt{x}\ln^2(3x - 7)}}{\sqrt[8]{5x}}$.
            \item $-\tan(x^4 - 6) + x\sin(x) -
                    \D\frac{\log(x)\cot(x)}{x+2} - 467k$.
         \end{enumerate}
      
      \textbf{Solution.}

      \begin{enumerate}
         \item $\D\frac{d}{dx} (-5x^3 + 2x - 1) = -15x^2 + 2$.
         \item $\D\frac{d}{dx} (3x^4 - 2\sqrt{x} + x^{1/2} - 6x^{-2/3} - 5) = 
                12x^3 - \frac{1}{2}x^{-1/2} + 4x^{-5/3}$.
         \item $\D\frac{d}{dx} \left(x\sqrt{x} + \sqrt{\sqrt{x}}\right) =
                \frac{3}{2}\sqrt{x} + \frac{1}{4}x^{-3/4}$.
         \item $\D\frac{d}{dx} (\log(x) - x^2\ln(x) + \ln(x^4)) = 
                \frac{1}{x\ln(2)} - 2x\ln(x) - x + \frac{4}{x}$.
         \item $\D\frac{d}{dx} (\ln^3(x\sqrt{2x - 3}) + \sqrt{\ln(x^2)}) = 
                \frac{9(x-1)\ln^2(x\sqrt{2x-3})}{x(2x-3)} +
                \frac{1}{x\sqrt{\ln(x^2)}}$.
         \item $\D\frac{d}{dx}\left(\frac{\sqrt{x+5} -
                   \ln(x)}{(x - 1)^3}\right) = \frac{(x-1)
                   \left[\D\frac{x-2\sqrt{x+5}}{2x\sqrt{x+5}}\right]
                   -3\sqrt{x+5}+3\ln(x)}{(x-1)^4}$.
         \item \begin{align*}
                  \frac{d}{dx}\left(\frac{\sqrt{x\sqrt{x}\ln^2(3x - 7)}}
                     {\sqrt[8]{5x}}\right)
                     &= \frac{1}{5^{1/8}}\left(\frac{d}{dx}
                        x^{5/8}\ln(3x-7)\right) \\
                     & = \frac{1}{\sqrt[8]{5x^3}}\left(\frac{5}{8}
                         \ln(3x-1) + \frac{3x}{3x-1}\right).
               \end{align*}
         \item Let $f(x) = -\tan(x^4 - 6) + x\sin(x) -
                    \D\frac{\log(x)\cot(x)}{x+2} - 467k$. It follows that
               \begin{align*}
                  \frac{df}{dx} &= -4x^3\sec^2(x^4-6) + \sin(x) + x\cos(x) \\
                     &- \frac{(x+2)\left[\D\frac{\cot(x)}{x\ln(2)}-\csc^2(x)\log(x)\right] - \log(x)\cot(x)}{(x+2)^2}
               \end{align*}
      \end{enumerate}
%%%%%%%%%%%%%%%%%%%%%%%%%%%%%%%%%%%%%%%03%%%%%%%%%%%%%%%%%%%%%%%%%%%%%%%%%%%%%%%
   \item Determine the limit of

         \begin{enumerate}
            \item $\D\lim_{x \rightarrow \infty} \frac{3x + 2}{-5x - 6}$.
            \item $\D\lim_{x \rightarrow \infty} \ln(x)$.
            \item $\D\lim_{x \rightarrow \infty} \left(\frac{1}{x} + 3\right)$.
            \item $\D\lim_{x \rightarrow \infty}
                     \frac{3x\log(x)+ 2}{\sqrt{x^3} + 7x}$.
            \item $\D\lim_{x \rightarrow \infty}
                     \frac{\sqrt{x\sqrt{x}}}{\sqrt{\D\frac{2}{x}} +
                     \log(x^3 - 4\sqrt{x})}$.
            \item $\D\lim_{x \rightarrow 0}
                   \frac{x^3 + x - \sqrt{3x}}{\sqrt{x}}$.
            \item $\D\lim_{x \rightarrow 0}
                   \frac{x^3 + x - \sqrt{3x}}{5x^{2.25}\sqrt{\sqrt{x}}}$.
            \item $\D\lim_{x \rightarrow 0}
                   \frac{\D\frac{1}{x^3}}{\D\frac{\sqrt{x}}{x^4}}$.
            \item $\D\lim_{x \rightarrow 0}
                   \frac{x^{0.1} - \sqrt{3}}{\sqrt{\sqrt{x}}}$.
            \item $\D\lim_{x \rightarrow \infty}\frac{x^x}{2^x}$.
            \item $\D\lim_{x \rightarrow \infty}\frac{x^x}{x2^x}$.
            \item $\D\lim_{x \rightarrow \infty}\frac{x^{1/x}}{x^a}$.
            \item $\D\lim_{x \rightarrow \infty}
                     \frac{\log(x^{\log(x)})}{x^{1/5}}$.
            \item $\D\lim_{x \rightarrow \infty}
                     \frac{\sqrt{2}^{\log^4x^3}}{\log(2x+7)}$.
            \item $\D\lim_{x \rightarrow \infty}\frac{2x^3 + \log(x) - 10x}
                     {\D\frac{x^4}{\ln(x)}}$.
            \item $\D\lim_{x \rightarrow \infty}
                     \frac{\D\frac{x+1}{3x^{\ln(x)}}}{2x^2}$.
            \item $\D\lim_{x \rightarrow \infty}\frac{\sin(x)}{\ln(x) + 2}$.
            \item $\D\lim_{x \rightarrow \infty}\frac{\sqrt{2}^{\log(x^3)}}
                     {\log^{\ln(x)}(2x)}$.
         \end{enumerate}
      
      \textbf{Solution.}

      \begin{enumerate}
         \item $\D\lim_{x \rightarrow \infty} \frac{3x + 2}{-5x - 6} =
                -\frac{3}{5}$.
         \item $\D\lim_{x \rightarrow \infty} \ln(x) = \infty$.
         \item $\D\lim_{x \rightarrow \infty} \left(\frac{1}{x} + 3\right) = 3$.
         \item $\D\lim_{x \rightarrow \infty}
                  \frac{3x\log(x)+ 2}{\sqrt{x^3} + 7x} = 0$.
         \item $\D\lim_{x \rightarrow \infty}
                  \frac{\sqrt{x\sqrt{x}}}{\sqrt{\D\frac{2}{x}} +
                  \log(x^3 - 4\sqrt{x})} = \infty$.
         \item $\D\lim_{x \rightarrow 0}
                \frac{x^3 + x - \sqrt{3x}}{\sqrt{x}} = -\sqrt{3}$.
         \item $\D\lim_{x \rightarrow 0}
                \frac{x^3 + x - \sqrt{3x}}{5x^{2.25}\sqrt{\sqrt{x}}} = -\infty$.
         \item $\D\lim_{x \rightarrow 0}
                \frac{\D\frac{1}{x^3}}{\D\frac{\sqrt{x}}{x^4}} = 0$.
         \item $\D\lim_{x \rightarrow 0}
                \frac{x^{0.1} - \sqrt{3}}{\sqrt{\sqrt{x}}} = -\infty$.
         \item $\D\lim_{x \rightarrow \infty}\frac{x^x}{2^x} = \infty$.
         \item $\D\lim_{x \rightarrow \infty}\frac{x^x}{x2^x} = \infty$.
         \item \begin{equation*}
                  \D\lim_{x \rightarrow \infty}\frac{x^{1/x}}{x^a} = \left\{
                     \begin{array}{cl}
                        \infty & \text{if } a < 0,\\
                        1      & \text{if } a = 0,\\
                        0      & \text{if } a > 0.
                     \end{array} \right.
               \end{equation*}
         \item $\D\lim_{x \rightarrow \infty}
                  \frac{\log(x^{\log(x)})}{x^{1/5}} = 0$.
         \item $\D\lim_{x \rightarrow \infty}
                   \frac{\sqrt{2}^{\log^4x^3}}{\log(2x+7)} = \infty$.
         \item $\D\lim_{x \rightarrow \infty}\frac{2x^3 + \log(x) - 10x}
                  {\D\frac{x^4}{\ln(x)}} = 0$.
         \item $\D\lim_{x \rightarrow \infty}
                  \frac{\D\frac{x+1}{3x^{\ln(x)}}}{2x^2} = 0$.
         \item $\D\lim_{x \rightarrow \infty}\frac{\sin(x)}{\ln(x) + 2} = 0$.
         \item $\D\lim_{x \rightarrow \infty}\frac{\sqrt{2}^{\log(x^3)}}
                  {\log^{\ln(x)}(2x)} = 1$.
      \end{enumerate}
%%%%%%%%%%%%%%%%%%%%%%%%%%%%%%%%%%%%%%%04%%%%%%%%%%%%%%%%%%%%%%%%%%%%%%%%%%%%%%%
   \item Compute the exact values for

         \begin{enumerate}
            \item $\D\int_1^n(x + 1)\;dx$.
            \item $\D\int_1^n(2x + 10)\;dx$.
            \item $\D\int_1^n(x^4 + \sqrt{x})\;dx$.
            \item $\D\int_1^n\left(Cx^3 - \frac{1}{x^2}\right)\;dx$.
            \item $\D\int_1^n\left(x^4 - 3x^2 + \frac{1}{x}\right)\;dx$.
            \item $\D\int_1^n\left(\frac{3}{\sqrt{x}} +\ln(x) + e^x\right)\;dx$.
            \item $\D\int_1^nxe^x\;dx$.
            \item $\D\int_1^nx\ln(x)\;dx$.
            \item $\D\int_1^n\ln(x)\;dx$.
            \item $\D\int_1^n\sin(x)\;dx$.
            \item $\D\int_1^nx\sin(x)\;dx$.
         \end{enumerate}
      
      \textbf{Solution.}

      \begin{enumerate}
         \item $\D\int_1^n(x + 1)\;dx = \frac{1}{2}n^2 + n - \frac{3}{2}$.
         \item $\D\int_1^n(2x + 10)\;dx = n^2 + 10n - 11$.
         \item $\D\int_1^n(x^4 + \sqrt{x})\;dx = \frac{1}{5}n^5 +
                   \frac{2}{3}n^{3/2} - \frac{13}{15}$.
         \item $\D\int_1^n\left(Cx^3 - \frac{1}{x^2}\right)\;dx =
                   \frac{C}{4}n^4 + \frac{1}{n} - \frac{C+4}{4}$.
         \item $\D\int_1^n\left(x^4 - 3x^2 + \frac{1}{x}\right)\;dx =
                   \frac{1}{5}n^5 - n^3 + \ln(n) - \frac{4}{5}$.
         \item $\D\int_1^n\left(\frac{3}{\sqrt{x}} +\ln(x) + e^x\right)\;dx = 
                   6\sqrt{n} + n \ln(n) - n + e^n - 5 - e$.
         \item $\D\int_1^nxe^x\;dx = ne^n - e^n$.
         \item $\D\int_1^nx\ln(x)\;dx = \frac{1}{2}n^2\ln(n) -
                   \frac{1}{4}n^2 + \frac{1}{4}$.         
         \item $\D\int_1^n\ln(x)\;dx = n\ln(n) - n + 1$.
         \item $\D\int_1^n\sin(x)\;dx = -\cos(n) + \cos(1)$.
         \item $\D\int_1^nx\sin(x)\;dx = -n\cos(n) + \sin(n) + \cos(1) -
                   \sin(1)$.
      \end{enumerate}
%%%%%%%%%%%%%%%%%%%%%%%%%%%%%%%%%%%%%%%05%%%%%%%%%%%%%%%%%%%%%%%%%%%%%%%%%%%%%%%
   \item Use mathematical induction to prove that
         \begin{equation} \label{5_1}
            1 + 2 + \cdots + n = \frac{n(n+1)}{2}.
         \end{equation}

      \textbf{Proof.} 

      \textbf{Base case.} $n = 1$. Since $1 = \D\frac{1(1 + 1)}{2}$, it is clear 
      that \eqref{5_1} holds when $n$ equals 1.

      \textbf{Inductive hypothesis.} Now assume that \eqref{5_1} holds for some 
      positive integer $k$. That is, assume that
      $$1 + 2 + \cdots + k = \frac{k(k+1)}{2}.$$

      Now we must show that \eqref{5_1} also holds for $(k+1)$. So it follows
      that
      \begin{align*}
         1 + 2 + \cdots + k + (k+1) &= (1 + 2 + \cdots + k) + (k + 1) \\
            &= \frac{k(k+1)}{2} + (k+1) &[\text{Inductive hypothesis}] \\
            &= \frac{k(k+1)+2(k+1)}{2} \\
            &= \frac{(k+1)(k+2)}{2} \\
            &= \frac{(k+1)((k+1)+1)}{2},
      \end{align*}
      so that \eqref{5_1} holds for $(k+1)$. Thus it follows by mathematical 
      induction that \eqref{5_1} holds for each positive integer $n$. \qed
%%%%%%%%%%%%%%%%%%%%%%%%%%%%%%%%%%%%%%%06%%%%%%%%%%%%%%%%%%%%%%%%%%%%%%%%%%%%%%%
   \item Use mathematical induction to prove that
         \begin{equation} \label{6_1}
            1^2 + 2^2 + \cdots + n^2 = \frac{n(n+1)(2n+1)}{6}.
         \end{equation}

      \textbf{Proof.} 

      \textbf{Base case.} \eqref{6_1} holds whenever $n$ equals 1 since
      $1^2 = 1 = \D\frac{1(1+1)(2\cdot1+1)}{6}$.

      \textbf{Inductive hypothesis.} Now assume that
      $$1^2 + 2^2 + \cdots + j^2 = \frac{j(j+1)(2j+1)}{6}$$
      for some positive integer $j$. Now it suffices to show that \eqref{6_1} 
      also holds for $(j+1)$. Hence
      \begin{align*}
         1^2 + 2^2 + \cdots + j^2 + (j+1)^2 &=
            (1^2 + 2^2 + \cdots + j^2) + (j + 1)^2 \\
            &= \frac{j(j+1)(2j+1)}{6} + (j + 1)^2
                  &[\text{Inductive hypothesis}] \\
            &= \frac{j(j+1)(2j+1) + 6(j+1)^2}{6} \\
            &= \frac{(j+1)(j(2j+1)+6(j+1))}{6} \\
            &= \frac{(j+1)(j+2)(2j+3)}{6} \\
            &= \frac{(j+1)((j+1)+1)(2(j+1)+1)}{6}
      \end{align*}
      so that \eqref{6_1} holds for $(j+1)$. Thus it follows by mathematical 
      induction that \eqref{6_1} holds for each positive integer $n$. \qed
\item[Extra Credit 1.] Prove that 
   $$\log_bx^a = a \log_bx.$$
   
   \textbf{Proof.} Let $y =\log_bx^a$. Then it follows that $b^y = x^a$. That
   is, $b^{y/a} = x$. Now $b^{y/a} = x$ implies that $\log_bx = y/a$. Multiply
   $\log_bx = y/a$ by $a$ to get $a\log_bx = y$. Since $y = \log_bx^a$, we have
   that $a\log_bx = \log_bx^a$. \qed
\item[Extra Credit 2.] Prove that 
   $$y^{\log_bx} =x^{\log_yb}.$$
   
   \textbf{Proof.} Let $t = y^{\log_bx}$. Then it follows that
   $\log_yt = \log_bx$. So
   \begin{align*}
      \log_bx = \log_yt = \frac{\log_xt}{\log_xy}.
   \end{align*}
   Thus
   $\log_bx \cdot \log_xy = \log_xt$. But $\log_bx \cdot \log_xy = \log_by$. So
   $\log_by = \log_bx\cdot\log_xy = \log_xt$. Since 
   $$\log_by = \log_xt$$
   we have $t = x^{\log_yb}$, so that $y^{\log_bx} = x^{\log_yb}$. \qed
\item[Extra Problems]
   $\D\sum_{i=2}^ni = -1 + \frac{n(n+1)}{2}$. \\
   $\D\sum_{i=-1}^n(i^2-i+5) = 1 + \frac{n(n+1)(2n+1)}{6} + 1 - \frac{n(n+1)}{2} + 5(n+2)$.
   
   $\D\sum_{k=0}^{n+1}2i = 2i(n+2)$.
      
\end{enumerate}
\end{document}
