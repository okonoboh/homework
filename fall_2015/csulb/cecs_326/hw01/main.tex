\documentclass[9pt]{article}

\usepackage{amssymb}
\usepackage{amsmath, array}
\usepackage{amsfonts}
\usepackage{comment}
\usepackage{fancyhdr}
\usepackage{mathrsfs}
\usepackage{enumitem}
\usepackage{bm}


\usepackage{tikz}

\voffset = -50pt
%\textheight = 700pt
\addtolength{\textwidth}{60pt}
\addtolength{\evensidemargin}{-30pt}
\addtolength{\oddsidemargin}{-30pt}
%\setlength{\headheight}{44pt}

\pagestyle{fancy}
\fancyhf{} % clear all fields
\fancyhead[R]{%
  \scshape
  \begin{tabular}[t]{@{}r@{}}
  CECS 326, Fall 2015\\Section 1 (10772/10773)\\
  HW \#01, DUE: 2015, September 27
  \end{tabular}}
\fancyhead[L]{%
  \scshape
  \begin{tabular}[t]{@{}r@{}}
  JOSEPH OKONOBOH\\Computer Science\\Cal State Long Beach
  \end{tabular}}
\fancyfoot[C]{\thepage}

\newcommand{\qed}{\hfill \ensuremath{\Box}}


\newcommand*\circled[1]{\tikz[baseline=(char.base)]{
            \node[shape=circle,draw,inner sep=2pt] (char) {#1};}}

\newcommand{\Z}{\mathbb{Z}}
\newcommand{\I}{\mathbb{I}}
\newcommand{\F}{\mathbb{F}}
\newcommand{\Q}{\mathbb{Q}}
\newcommand{\R}{\mathbb{R}}
\newcommand{\C}{\mathbb{C}}
\newcommand{\D}{\displaystyle}
%\setcounter{section}{-1}

\begin{document}
\begin{enumerate}
%%%%%%%%%%%%%%%%%%%%%%%%%%%%%%%%%%%%%%%01%%%%%%%%%%%%%%%%%%%%%%%%%%%%%%%%%%%%%%%
   \item Can the sleeping thread print its periodic messages while the main
         thread is waiting for keyboard input? Explain your answer.
         
      \textbf{Answer.} Yes. The sleeping thread represents a different unit of
      execution from the main thread; since threads run concurrently, the
      sleeping thread will echo some message to the standard output
      periodically, so long as the user has not pressed the Enter key.
%%%%%%%%%%%%%%%%%%%%%%%%%%%%%%%%%%%%%%%02%%%%%%%%%%%%%%%%%%%%%%%%%%%%%%%%%%%%%%%
   \item Can the main thread read input, kill the sleeping thread, and print a
         message while the sleeping thread is in the early part of one of its
         five-second sleeps? Explain your answer.
   
      \textbf{Answer.} Yes. The main thread also executes independently from the
      sleeping thread, so the user can still interact with the program while the
      sleeping thread is in the early part of any of its five-second sleeps.
   \item \textbf{Code.}
   
         \begin{verbatim}
#include <pthread.h>
#include <unistd.h>
#include <stdio.h>
#include <stdlib.h>

#define NUMCHAR   1024

static void* child_infinite(void* unused) {
   while(1) {
      sleep(5);
      printf("I am stuck in an infinite loop.\n");
   }

   return NULL;
}

static void eat_input() {
   char unused_string[NUMCHAR];

   fgets(unused_string, NUMCHAR >> 1, stdin);
}

int main(int argc, char* argv[]) {
   pthread_t child_thread;
   int code;

   code = pthread_create(&child_thread, NULL, child_infinite, NULL);

   if(code) {
      fprintf(stderr, "pthread_create failed with code %d.\n", code);
      exit(EXIT_FAILURE);
   }

   printf("Please enter a string: ");
   eat_input();

   code = pthread_cancel(child_thread);

   if(code) {
      fprintf(stderr, "pthread_cancel failed with code %d.\n", code);
      exit(EXIT_FAILURE);
   }

   printf("Main thread: I have terminated the useless child thread.\n");

   return EXIT_SUCCESS;
}
         \end{verbatim}
\end{enumerate}
\end{document}
