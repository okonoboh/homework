\documentclass[9pt]{article}

\usepackage{amssymb}
\usepackage{amsmath, array}
\usepackage{amsfonts}
\usepackage{comment}
\usepackage{fancyhdr}
\usepackage{mathrsfs}
\usepackage{enumitem}
\usepackage{bm}

\voffset = -50pt
%\textheight = 700pt
\addtolength{\textwidth}{60pt}
\addtolength{\evensidemargin}{-30pt}
\addtolength{\oddsidemargin}{-30pt}
%\setlength{\headheight}{44pt}

\pagestyle{fancy}
\fancyhf{} % clear all fields
\fancyhead[R]{%
  \scshape
  \begin{tabular}[t]{@{}r@{}}
  CECS 326, Fall 2015\\Section 1 (10772/10773)\\
  HW \#02, DUE: 2015, September 28
  \end{tabular}}
\fancyhead[L]{%
  \scshape
  \begin{tabular}[t]{@{}r@{}}
  JOSEPH OKONOBOH\\Computer Science\\Cal State Long Beach
  \end{tabular}}
\fancyfoot[C]{\thepage}

\newcommand{\qed}{\hfill \ensuremath{\Box}}


\newcommand*\circled[1]{\tikz[baseline=(char.base)]{
            \node[shape=circle,draw,inner sep=2pt] (char) {#1};}}

\newcommand{\Z}{\mathbb{Z}}
\newcommand{\I}{\mathbb{I}}
\newcommand{\F}{\mathbb{F}}
\newcommand{\Q}{\mathbb{Q}}
\newcommand{\R}{\mathbb{R}}
\newcommand{\C}{\mathbb{C}}
\newcommand{\D}{\displaystyle}
%\setcounter{section}{-1}
\begin{document}
\begin{enumerate}
%%%%%%%%%%%%%%%%%%%%%%%%%%%%%%%%%%%%%%%01%%%%%%%%%%%%%%%%%%%%%%%%%%%%%%%%%%%%%%%
   \item \textbf{Purpose of the OS.} The Operating System is a family of
         system software that coordinates the use of hardware and software
         resources. Operating Systems are typically found in video game systems,
         cars, embedded systems, computers, smartphones, etc. Without an
         Operating System, every programmer will have to write low level codes
         to access devices (such as hard drives and printers), to write 
         characters to the graphics device (screen), to perform multithreading,
         to perform networking, and so on; this is practically impossible because
         different devices will have to programmed differently, and since one
         cannot be certain as to what kind of devices the users will have, the
         programmer's program will then be extremely limited to a particular set
         of devices. Most importantly, every system will only have the
         capability of running exactly one program if no OS is available. This
         will severely waste system resources since most applications on
         general purpose computer are usually not executing. Simply put, an
         Operating System is akin to a traffic coordinator at a very busy
         intersection, informing each vehicle about when to move and when to
         halt. In general purpose computer systems, Operating Systems are
         responsible for the Abstraction, Sharing, and Isolation of Resources.
         These concepts are explained below.
   \item \textbf{Resource Abstraction.} Everyday, billions of people interact
         with some form of technology; some use the microwave, some drive cars,
         some use their cell phones, some use elevators, and a variety of other
         technologies. What is typically common in all these scenarios is that
         the users have no knowledge about the underlying workings of these
         systems. In fact, if a person can operate one elevator, then that same
         person can---in almost all cases---operate another elevator. Indeed,
         one does not need to understand the intimate details of a computer in
         order to avail oneself of its services. This is so because the
         operating system abstracts(or, more accurately, hides) these details
         from the users. So we can define Resource Abstraction as the process
         by which an operating system hides the inner workings of its parts, so
         that the application programmer (or other user) can use the computer
         relatively easily without bothering about low-level details, such as
         writing to and reading from specific hardware, polling devices, etc.
   \item \textbf{Resource Sharing.} Resource Sharing is simply the mechanism
         wherein multiple programs (processes or their threads) share the
         resources of a computer. The most common type of sharing occurs in
         general purpose computers, where the programs share the processor. This
         is known as time-multiplexed sharing: every process is given a quantum
         of time to execute and then control in transferred to other processes
         in a roundrobin fashion. In fact, in most general purpose computers,
         the executing programs themselves share the main memory, a process
         known as space-multiplexing. If this were
         not the case, then a user would have to wait for one program to
         completely finish its task before another program can use the resources
         of the computer. Thus, if a file is being downloaded from the Internet,
         the user would have to wait for the file to be completely downloaded
         before he or she can startup another program. The operating system
         coordinates the sharing of the resources of the computer. It tries to
         do so in an equitable manner; that is, on average, every user program
         gets an approximately equal amount of time in a particular time period.
   \item \textbf{Resource Isolation.} Imagine that you parked your car at a
         particular spot in a parking lot. Except for extradordinary
         circumstances, no other driver would be able to park his or her car in
         your spot. We can view the parking spots as resources(which are
         themselves part of a larger resource, the parking lot) and each car as
         a process, so that no two different processes can interfere with each
         other. This property is also desired for executing processes. Each
         process has its allocated space in main memory---both for data and
         program---and the integrity of the system will be compromised if a
         process can illegally access memory locations outside of its boundary.
         With Resource Isolation, each program is limited to whatever boundary
         that the OS has imposed on it; that is, a program cannot access other
         resources that have been allocated to other programs. Of course, the
         last statement is not entirely true because malicious programs could
         take advantage of bugs in a program to gain access to other programs.
         Nevertheless, there are precautions that programmers can take to make
         sure that their programs are less susceptible to these types of errors.
\end{enumerate}

\begin{thebibliography}{9}

\bibitem{lamport94}
  Gary Nutt,
  \emph{Operating Systems},
  Pearson,
  3rd edition,
  2003.

\end{thebibliography}
%This is a citation. \autocite{Saussure1995}. Another citation \autocite{Kavanaugh1976}

\end{document}
