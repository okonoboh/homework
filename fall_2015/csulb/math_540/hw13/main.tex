\documentclass[9pt]{article}

\usepackage{amssymb}
\usepackage{amsmath}
\usepackage{amsfonts}
\usepackage{comment}
\usepackage{fancyhdr}
\usepackage{mathrsfs}
\usepackage{enumitem}
%\usepackage[retainorgcmds]{IEEEtrantools}

\everymath{\displaystyle}

\usepackage{tikz}

\voffset = -50pt
%\textheight = 700pt
\addtolength{\textwidth}{60pt}
\addtolength{\evensidemargin}{-30pt}
\addtolength{\oddsidemargin}{-30pt}
%\setlength{\headheight}{44pt}

\pagestyle{fancy}
\fancyhf{} % clear all fields
\fancyhead[R]{%
  \scshape
  \begin{tabular}[t]{@{}r@{}}
  MATH 590, Fall 2015 \\
  Project, DUE: 2015, December 09
  \end{tabular}}
\fancyhead[L]{%
  \scshape
  \begin{tabular}[t]{@{}r@{}}
  JOSEPH$^2$\\Mathematics\\Cal State Long Beach
  \end{tabular}}
\fancyfoot[C]{\thepage}

\newcommand{\qed}{\hfill \ensuremath{\Box}}


\newcommand*\circled[1]{\tikz[baseline=(char.base)]{
            \node[shape=circle,draw,inner sep=2pt] (char) {#1};}}


\newcommand{\cyc}[1]{\langle #1 \rangle}
\newcommand{\Z}{\mathbb{Z}}
\newcommand{\C}{\mathbb{C}}
\newcommand{\F}{\mathbb{F}}
\newcommand{\M}{\mathbb{M}}
\newcommand{\R}{\mathbb{R}}
\newcommand{\Q}{\mathbb{Q}}
\everymath{\displaystyle}
\newcommand{\CYC}[1]{\left\langle #1 \right\rangle}
%\setcounter{section}{-1}

\begin{document}
\noindent
\textbf{Notations and Definitions.}
\begin{enumerate}
   \item Groups are written additively with identity 0.
   \item For a positive integer $p$ and a set $S$,
         $$S^p = \underbrace{S \times \cdots \times S}_{p \text{ times}} =
           \text{the set of all $p$ tuples with entries in }S$$
         and
         $$\Z_p = \Z/p\Z.$$
   \item An abelian group $L$ is said to be finitely generated if there exist a
         positive integer $p$ and a vector
         $\vec{l} = (l_1, \ldots, l_p) \in L^p$ such that
         \begin{align*}
            L &= \{\vec{z}^T \cdot \vec{l} : \vec{z} \in \Z^p\} \\
              &= \{z_1l_1 + \cdots + z_pl_p : \vec{z} \in \Z^p, \text{ where }\vec{z}^T = (z_1, \ldots, z_p)\}.
         \end{align*}
         The vector $\vec{l}$ is called a generating vector for $L$ and the
         elements $l_1, \ldots, l_p$ are generators of $L$.
   \item Let $A$ be a finitely generated abelian group with a generating vector
         $\vec{a} = (a_1, \ldots, a_q)$. Define
         $$\alpha_{_{A, \vec{a}}} : \Z^q \rightarrow A$$
         by $\vec{z}^T = (z_1, \ldots, z_q) \mapsto z_1a_1 + \cdots + z_qa_q = \vec{z}^T \cdot \vec{a}$.
\end{enumerate}
\textbf{First Isomorphism Theorem.} \textit{If $\pi : A \rightarrow B$ is a
homomorphism of groups, then it follows that}
$$A/\text{kernel}(\pi)  \cong \pi(A),$$
where $\text{kernel}(\pi) = \{l \in A : \pi(l) = 0\} \le A$. \\

\noindent
\textbf{Theorem 1.} \textit{If $H \le \Z^p$, then $H$ is finitely generated,
with at most $p$ generators.} \\

\noindent
\textbf{The Smith Normal Form.} Let $R$ be a Euclidean Domain and
$A \in M_{m \times n}(R)$. Then there exist $P \in GL_m(R)$ and $Q \in GL_n(R)$
such that $PAQ = D \oplus 0$ where $D = $ diag$(d_1, \ldots, d_{\rho(A)})$ and
$d_1 \mid d_2 \cdots \mid d_{\rho(A)-1} \mid d_{\rho(A)}$, $\rho(A) =$
rank of $A$. \\

\noindent
\textbf{The Fundamental Theorem of Finitely Generated Abelian Group.} Let $G$ be
a finitely generated abelian group with a generating vector
$\vec{b} = (b_1, \ldots, b_n) \in G^n$. Then 
$$G \cong \Z_{d_1} \times \cdots \times \Z_{d_r} \times \Z^{n-r},$$
where $0 \le r \le n$, $d_i > 0$ and $d_1 \mid d_2 \mid \cdots \mid d_{r-1}
\mid d_r$. \\

\noindent
\textbf{Proof.} We shall break this proof down into the following steps.

\begin{enumerate}
   \item Show that $\alpha_{_{G, \vec{b}}}$ is a surjective homomorphism, so that
         $\Z^n/K \cong G$ by the First Isomorphism Theorem, where
         $K = $ kernel($\alpha_{_{G, \vec{b}}}$).
   \item Use Theorem 1 to get a generating vector
         $(\vec{k_1}, \ldots, \vec{k_m})$ for $K$, $m \le n$.
   \item Using the Smith Normal Form, show that
         $$K \cong d_1\Z \times \cdots \times d_r\Z$$
         for some positive integers $d_i$, where
         $d_j \mid d_{j + 1}$, $1 \le j \le r - 1$ and $0 \le r \le n$.
   \item Now show that
         $$\Z^n/(d_1\Z \times \cdots \times d_r\Z) \cong \Z_{d_1} \times \cdots
           \times \Z_{d_r} \times \Z^{n-r}.$$
         Then conclude that $G \cong \Z_{d_1} \times \cdots
           \times \Z_{d_r} \times \Z^{n-r}$.         
\end{enumerate}
\end{document}
