\documentclass[9pt]{article}

\usepackage{amssymb}
\usepackage{amsmath}
\usepackage{amsfonts}
\usepackage{comment}
\usepackage{fancyhdr}
\usepackage{mathrsfs}
\usepackage{enumitem}
%\usepackage[retainorgcmds]{IEEEtrantools}

\everymath{\displaystyle}

\usepackage{tikz}

\voffset = -50pt
%\textheight = 700pt
\addtolength{\textwidth}{60pt}
\addtolength{\evensidemargin}{-30pt}
\addtolength{\oddsidemargin}{-30pt}
%\setlength{\headheight}{44pt}

\pagestyle{fancy}
\fancyhf{} % clear all fields
\fancyhead[R]{%
  \scshape
  \begin{tabular}[t]{@{}r@{}}
  MATH 540, Fall 2015\\Section 1 (9932)\\
  HW \#10, DUE: 2015, November 12
  \end{tabular}}
\fancyhead[L]{%
  \scshape
  \begin{tabular}[t]{@{}r@{}}
  JOSEPH OKONOBOH\\Mathematics\\Cal State Long Beach
  \end{tabular}}
\fancyfoot[C]{\thepage}

\newcommand{\qed}{\hfill \ensuremath{\Box}}


\newcommand*\circled[1]{\tikz[baseline=(char.base)]{
            \node[shape=circle,draw,inner sep=2pt] (char) {#1};}}


\newcommand{\cyc}[1]{\langle #1 \rangle}
\newcommand{\Z}{\mathbb{Z}}
\newcommand{\C}{\mathbb{C}}
\newcommand{\F}{\mathbb{F}}
\newcommand{\M}{\mathbb{M}}
\newcommand{\R}{\mathbb{R}}
\newcommand{\Q}{\mathbb{Q}}
\everymath{\displaystyle}
\newcommand{\CYC}[1]{\left\langle #1 \right\rangle}
%\setcounter{section}{-1}

\begin{document}
\begin{enumerate}
%%%%%%%%%%%%%%%%%%%%%%%%%%%%%%%%%%%%%4.5.08%%%%%%%%%%%%%%%%%%%%%%%%%%%%%%%%%%%%%
   \item[4.5.8]   Exhibit two distinct Sylow 2-subgroups of $S_5$ and an element
                  of $S_5$ that conjugates one into the other.
%%%%%%%%%%%%%%%%%%%%%%%%%%%%%%%%%%%%%4.5.13%%%%%%%%%%%%%%%%%%%%%%%%%%%%%%%%%%%%%
   \item[4.5.13]  Prove that a group of order 56 has a normal Sylow
                  $p$-subgroup for some prime $p$ dividing its order.
%%%%%%%%%%%%%%%%%%%%%%%%%%%%%%%%%%%%%4.5.14%%%%%%%%%%%%%%%%%%%%%%%%%%%%%%%%%%%%%
   \item[4.5.14]  Prove that a group of order 312 has a normal Sylow
                  $p$-subgroup for some prime $p$ dividing its order.
%%%%%%%%%%%%%%%%%%%%%%%%%%%%%%%%%%%%%4.5.15%%%%%%%%%%%%%%%%%%%%%%%%%%%%%%%%%%%%%
   \item[4.5.15]  Prove that a group of order 351 has a normal Sylow
                  $p$-subgroup for some prime $p$ dividing its order.
%%%%%%%%%%%%%%%%%%%%%%%%%%%%%%%%%%%%%4.5.16%%%%%%%%%%%%%%%%%%%%%%%%%%%%%%%%%%%%%
   \item[4.5.16]  Let $|G| = pqr$, where $p$, $q$, and $r$ are primes with
                  $p < q < r$. Prove that $G$ has a normal Sylow subgroup for
                  either $p$, $q$, or $r$.
%%%%%%%%%%%%%%%%%%%%%%%%%%%%%%%%%%%%%4.5.29%%%%%%%%%%%%%%%%%%%%%%%%%%%%%%%%%%%%%
   \item[4.5.29]  If $G$ is a non-abelian simple group of order $< 100$, prove
                  that $G \cong A_5$. [Eliminate all orders but 60]
%%%%%%%%%%%%%%%%%%%%%%%%%%%%%%%%%%%%%7.1.03%%%%%%%%%%%%%%%%%%%%%%%%%%%%%%%%%%%%%
   \item[7.1.3]   Let $R$ be a ring with identity and let $S$ be a subring of
                  $R$ containing the identity. Prove that if $u$ is a unit in
                  $S$ then $u$ is a unit in $R$. Show by example that the
                  converse is false.
%%%%%%%%%%%%%%%%%%%%%%%%%%%%%%%%%%%%%7.1.09%%%%%%%%%%%%%%%%%%%%%%%%%%%%%%%%%%%%%
   \item[7.1.9]   For a fixed element $a \in R$ define
                  $C(a) = \{r \in R : ra = ar\}$. Prove that $C(a)$ is a subring
                  of $R$ containing $a$. Prove that the center of $R$ is the
                  intersection of the subrings $C(a)$ over all $a \in R$.
%%%%%%%%%%%%%%%%%%%%%%%%%%%%%%%%%%%%%7.1.12%%%%%%%%%%%%%%%%%%%%%%%%%%%%%%%%%%%%%
   \item[7.1.12]  Prove that any subring of a field which contains the identity
                  is an integral domain.
\end{enumerate}
\end{document}
