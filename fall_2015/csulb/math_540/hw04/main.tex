\documentclass[9pt]{article}

\usepackage{amssymb}
\usepackage{amsmath}
\usepackage{amsfonts}
\usepackage{comment}
\usepackage{fancyhdr}
\usepackage{mathrsfs}
\usepackage{enumitem}
%\usepackage[retainorgcmds]{IEEEtrantools}

\everymath{\displaystyle}

\usepackage{tikz}

\voffset = -50pt
%\textheight = 700pt
\addtolength{\textwidth}{60pt}
\addtolength{\evensidemargin}{-30pt}
\addtolength{\oddsidemargin}{-30pt}
%\setlength{\headheight}{44pt}

\pagestyle{fancy}
\fancyhf{} % clear all fields
\fancyhead[R]{%
  \scshape
  \begin{tabular}[t]{@{}r@{}}
  MATH 540, Fall 2015\\Section 1 (9932)\\
  HW \#4, DUE: 2015, September 24
  \end{tabular}}
\fancyhead[L]{%
  \scshape
  \begin{tabular}[t]{@{}r@{}}
  JOSEPH OKONOBOH\\Mathematics\\Cal State Long Beach
  \end{tabular}}
\fancyfoot[C]{\thepage}

\newcommand{\qed}{\hfill \ensuremath{\Box}}


\newcommand*\circled[1]{\tikz[baseline=(char.base)]{
            \node[shape=circle,draw,inner sep=2pt] (char) {#1};}}


\newcommand{\cyc}[1]{\langle #1 \rangle}
\newcommand{\Z}{\mathbb{Z}}
\newcommand{\I}{\mathbb{I}}
\newcommand{\F}{\mathbb{F}}
\newcommand{\M}{\mathbb{M}}
\newcommand{\R}{\mathbb{R}}
\newcommand{\Q}{\mathbb{Q}}
\everymath{\displaystyle}
\newcommand{\CYC}[1]{\left\langle #1 \right\rangle}
%\setcounter{section}{-1}

\begin{document}
\begin{enumerate}
%%%%%%%%%%%%%%%%%%%%%%%%%%%%%%%%%%%%%2.3.12%%%%%%%%%%%%%%%%%%%%%%%%%%%%%%%%%%%%%
   \item[2.3.12]  Prove that the following groups are \textit{not} cyclic:
                  \begin{enumerate}
                     \item $Z_2 \times Z_2$
                     \item $Z_2 \times \Z$
                     \item $\Z \times \Z$.
                  \end{enumerate}
      
      \textbf{Proof.}
      \begin{enumerate}
         \item Let $x$ be the element of order 2 in $Z_2$. Observe that
               $\cyc{(1, x)}$ and $\cyc{(x, x)}$ are different subgroups
               (since $(x\;x) \neq (1, x)$) of order 2 in $Z_2 \times Z_2$;
               thus, since $Z_2 \times Z_2$ is finite and since it has two
               different subgroups of order 2, it follows by Theorem 7 (3) that
               $Z_2 \times Z_2$ is not cyclic.
         \item Let $Z_2 = \cyc{x}$. Observe that $Z_2 \times \Z$ is not finite,
               so, for it to be cyclic, Theorem 4 (2) says it must be isomorphic
               to $\Z$. But this is impossible because $Z_2 \times \Z$ has two
               elements of finite order(namely $(1, 0)$ and $(x, 0)$) while
               $\Z$ has exactly 1 element of finite order, namely the identity.
               Hence $Z_2 \times \Z$ is not cyclic.
         \item Suppose to the contrary that $\Z \times \Z$ is cyclic. Then there
               exist nonzero integers $a$ and $b$ such that
               $$\Z \times \Z = \cyc{(a,b)} = \{(na, nb) : n \in \Z\}.$$
               The integers $a$ and $b$ must be nonzero, for otherwise, at least
               one of the components of every element generated by $(a, b)$
               will always be zero. Since $(0, 1) \in \cyc{(a, b)}$, it follows
               that there exists an integer $m$ such that
               $(ma, mb) = (0, 1)$. That is, $ma = 0$ and $mb = 1$. Since
               $ma = 0$, we must have $m = 0$ or $a = 0$. Now $m \neq 0$
               because $mb = 1$, so it must necessarily be the case that
               $a = 0$, contradicting our assumption that $a$ is nonzero. Thus,
               $\Z \times \Z$ is not cyclic.
      \end{enumerate} \qed
%%%%%%%%%%%%%%%%%%%%%%%%%%%%%%%%%%%%%2.3.19%%%%%%%%%%%%%%%%%%%%%%%%%%%%%%%%%%%%%
   \item[2.3.19]  Show that if $H$ is any group and $h$ is an element of $H$,
                  then there is a unique homomorphism from $\Z$ to $H$ such that
                  $1 \mapsto h$.
                  
      \textbf{Proof.} Let $H$ be a group and let $h \in H$. First we shall show
      that there exists a homomorphism from $\Z$ to $H$ such that $1 \mapsto h$.
      So consider the map $\alpha : \Z \rightarrow H$ defined by
      $n \mapsto h^n$. Indeed $\alpha(1) = h^1 = h$ and
      $$\alpha(x+y) = h^{x+y} = h^xh^y = \alpha(x)\alpha(y) \text{ for all }
        x, y \in \Z^+,$$
      so that $\alpha$ is a homomorphism. To show uniqueness, suppose that
      $\alpha' : \Z \rightarrow H$ is an homomorphism such that
      $\alpha'(1) = h$. Then according to Exercise 1.6.1, we have that
      $\alpha'(n) = \alpha'(n\cdot1) = \alpha'(1)^n = h^n$ for all $n \in \Z$;
      that is, $\alpha' = \alpha$, as desired. \qed
%%%%%%%%%%%%%%%%%%%%%%%%%%%%%%%%%%%%%2.3.23%%%%%%%%%%%%%%%%%%%%%%%%%%%%%%%%%%%%%
   \item[2.3.23]  Show that $(\Z/2^n\Z)^\times$ is not cyclic for any $n \ge 3$.
                  [Find two distinct subgroups of order 2.]

      \textbf{Proof.} Let $n \ge 3$ be an integer. First we note that
      $(\Z/2^n\Z)^\times$ is finite, so to show that it is not cyclic, it
      suffices to exhibit two different subgroups of order 2. To that end,
      consider $2^{n-1}-1, 2^n-1 \in \Z/2^n\Z$. Since $2^{n-1}-1$ and $2^n-1$
      are both odd, it follows that both of them are relatively prime to $2^n$,
      so that they are both in $(\Z/2^n\Z)^\times$. Now note that
      $2^{n-1}-1 \neq 1$ and $2^n-1 \neq 1$ because
      $1 < 2^{n-1}-1 < 2^n-1 < 2^n$. That is, the multiplicative order of each
      of $2^{n-1}-1$ and $2^n-1$ is at least 2. Now we have that
      $$(2^{n-1}-1)(2^{n-1}-1) = 2^n2^{n-2}-2^n + 1 \equiv 1\text{ (mod }2^n)$$
      and
      $$(2^n-1)(2^n-1) = (-1)(-1) = 1 \equiv 1\text{ (mod }2^n).$$
      Thus both $2^{n-1}-1$ and $2^n-1$ have order 2 in $(\Z/2^n\Z)^\times$.
      However, since these two elements are not equal, it follows that
      $(\Z/2^n\Z)^\times$ has two distinct subgroups of order 2, namely
      $$\cyc{(2^{n-1}-1)} \text{ and } \cyc{(2^n-1)},$$
      so, according to Theorem 7 (3), $(\Z/2^n\Z)^\times$ is not cyclic. \qed
%%%%%%%%%%%%%%%%%%%%%%%%%%%%%%%%%%%%%2.4.5%%%%%%%%%%%%%%%%%%%%%%%%%%%%%%%%%%%%%%
   \item[2.4.5]   Prove that the subgroup generated by any two distinct elements
                  of order 2 in $S_3$ is all of $S_3$.

      \textbf{Proof.} The elements of order 2 in $S_3$ are
      $$(1\;2), (1\;3), \text{ and } (2\;3).$$
      First we shall investigate the group generated by (1 2) and (2 3). A few
      computations will show us that
      \begin{align*}
         (1\;2)^0 &= (1) \\
         (1\;2)^1 &= (1\;2) \\
         (2\;3)^1 &= (2\;3) \\
         (1\;2)(2\;3) &= (1\;2\;3) \\
         (2\;3)(1\;2) &= (1\;3\;2) \\
         (1\;2)(2\;3)(1\;2) &= (1\;3).
      \end{align*}
      That is,
      $$S_3 = \{(1), (1\;2), (1\;3), (1\;2\;3), (1\;3\;2), (1\;3)\} \subseteq
        \cyc{(1\;2), (1\;3)},$$
      and since $\cyc{(1\;2), (1\;3)} \subseteq S_3$, we conclude that
      $\cyc{(1\;2), (1\;3)} = S_3$. Similarly, we have that
      \begin{align*}
         (1\;2)^0 &= (1) \\
         (1\;2)^1 &= (1\;2) \\
         (1\;3)^1 &= (1\;3) \\
         (1\;2)(1\;3) &= (1\;3\;2) \\
         (1\;3)(1\;2) &= (1\;2\;3) \\
         (1\;2)(1\;3)(1\;2) &= (2\;3).
      \end{align*}
      That is, $S_3 \subseteq \cyc{(1\;2), (1\;3)}$, so that
      $\cyc{(1\;2)(1\;3)} = S_3$. Finally,
       \begin{align*}
         (1\;3)^0 &= (1) \\
         (1\;3)^1 &= (1\;3) \\
         (2\;3)^1 &= (2\;3) \\
         (1\;3)(2\;3) &= (1\;3\;2) \\
         (2\;3)(1\;3) &= (1\;2\;3) \\
         (1\;3)(2\;3)(1\;3) &= (1\;2),
      \end{align*}
      so that $S_3 \subseteq \cyc{(1\;3)(2\;3)}$ and we conclude that
      $\cyc{(2\;3)(1\;3)} = S_3$. Hence every two distinct elements of order 2
      in $S_3$ generate $S_3$. \qed
   \item[2.4.14]  A group $H$ is called \textit{finitely generated} if there is
                  a finite set $A$ such that $H = \cyc{A}$.
                  \begin{enumerate}
                     \item Prove that every finite group is finitely generated.
                     \item Prove that $\Z$ is finitely generated.
                     \item Prove that every finitely generated subgroup of the
                           additive group $\Q$ is cyclic. [If $H$ is a finitely
                           generated subgroup of $\Q$, show that
                           $H \le \CYC{\frac{1}{k}}$, where $k$ is the product
                           of all the denominators which appear in a set of
                           generators for $H$.]
                     \item Prove that $\Q$ is not finitely generated.
                  \end{enumerate}

      \textbf{Proof.}

      \begin{enumerate}
         \item Let $H$ be a finite group. We showed in class that $\cyc{H} = H$,
               so that $H$ is finitely generated.
         \item The group of integers $\Z$ is finitely generated because it is
               generated by the finite set $\{1\}$.
         \item Let $H$ be a finitely generated subgroup of $\Q$. We can assume
               that $H$ is nontrivial, so that its set of generators is
               nonempty. Suppose $H$ has $n \in \Z^+$ generators. Then it
               follows that there exist rational numbers
               $$\frac{a_1}{b_1}, \frac{a_2}{b_2}, \cdots,\frac{a_n}{b_n}$$
               such that $H = \CYC{\frac{a_1}{b_1}, \frac{a_2}{b_2}, \cdots,
               \frac{a_n}{b_n}}$. Let us now consider the cyclic group
               $\CYC{\frac{1}{k}}$, where $k = b_1b_2\cdots b_n$. Since
               $\CYC{\frac{1}{k}}$ is an additive group, it follows that
               $\frac{m}{k} \in \CYC{\frac{1}{k}}$ for every integer $m$. To
               show that $H \le \CYC{\frac{1}{k}}$, it suffices to show that
               $\frac{a_i}{b_i} \in \CYC{\frac{1}{k}}$, for every
               $1 \le i \le n$. So let $\frac{a_j}{b_j}$ be one of the
               generators for some $1 \le j \le n$. Notice that $\frac{k}{b_j}$
               is an integer. Thus $\frac{1}{b_j}  =
               \left(\frac{k}{b_j}\right)\frac{1}{k} \in \CYC{\frac{1}{k}}$, so
               that
               $$\frac{a_j}{b_j} = a_j \left(\frac{1}{b_j}\right) \in
               \CYC{\frac{1}{k}};$$
               that is, $H \le \CYC{\frac{1}{k}}$, as desired. Since $H$ is a
               subgroup of the cyclic group $\CYC{\frac{1}{k}}$, it follows by
               Theorem 7 (1) that $H$ is cyclic. \qed
         \item If $\Q$ were finitely generated, then, according to (c) above,
               $\Q$ would be cyclic, a contradiction, since we showed in
               Exercise 1.6.6 that $\Z$ and $\Q$ are not isomorphic, so that
               $\Q$ is not cyclic.
      \end{enumerate} \qed
%%%%%%%%%%%%%%%%%%%%%%%%%%%%%%%%%%%%%2.4.15%%%%%%%%%%%%%%%%%%%%%%%%%%%%%%%%%%%%%
   \item[2.4.15]  Exhibit a proper subgroup of $\Q$ which is not cyclic.

      \textbf{Solution.} Let $H = \left\{\frac{1}{2^i} : i \text{ is a 
      nonnegative integer}\right\}$ be a subset of the additive group $\Q$.
      We thus have that
      \begin{align*}
         \cyc{H} &= \left\{\frac{a_1}{2^{\varepsilon_1}} +
            \frac{a_2}{2^{\varepsilon_2}} + \cdots +
            \frac{a_n}{2^{\varepsilon_n}} : a_i, \varepsilon_i, n \in \Z, 
            \varepsilon_i \ge 0, n \ge 1 \right\} \\
                 &= \left\{\frac{a}{2^i} : a, i \in \Z, i \ge 0\right\}.
      \end{align*}
      Notice that the only prime factor that the denominator of each rational
      number(in its lowest term) in $\cyc{H}$ can have is 2; thus
      $1/7 \notin \cyc{H}$, so that $\cyc{H} \lneq \Q$. Now suppose to the
      contrary that $\cyc{H}$ is cyclic; thus it follows by definition that
      $\cyc{H}$ is generated by some rational number $c/d$, where $c, d \in \Z$
      and $d \neq 0$. Let $k$ be the maximum nonnegative integer such that
      $2^k \mid d$. Since $c/2^{k+1} \in \cyc{H}$, it follows that there exists
      some integer $m$ such that $mc/d = c/2^{k+1}$. That is $m = d / 2^{k+1}$,
      so that $2^{k+1} \mid d$, contradicting the maximality of $k$. Thus
      $\cyc{H}$ is a proper non cyclic subgroup of $\Q$.
%%%%%%%%%%%%%%%%%%%%%%%%%%%%%%%%%%%%%2.5.5%%%%%%%%%%%%%%%%%%%%%%%%%%%%%%%%%%%%%%
   \item[2.5.5]   Use the given lattice to find all elements $x \in D_{16}$
                  such that $D_{16} = \cyc{x, s}$ (there are 8 such elements
                  $x$).
                  
      \textbf{Solution.} By observing the given lattice of $D_{16}$, we find
      that      
      $$x \in \{r, r^3, r^5, r^7, sr^3, sr^7, sr^5, sr\}.$$
\end{enumerate}
\end{document}
