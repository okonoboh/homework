\documentclass[9pt]{article}

\usepackage{amssymb}
\usepackage{amsmath}
\usepackage{amsfonts}
\usepackage{comment}
\usepackage{fancyhdr}
\usepackage{mathrsfs}
\usepackage{enumitem}
%\usepackage[retainorgcmds]{IEEEtrantools}

\everymath{\displaystyle}

\usepackage{tikz}

\voffset = -50pt
%\textheight = 700pt
\addtolength{\textwidth}{60pt}
\addtolength{\evensidemargin}{-30pt}
\addtolength{\oddsidemargin}{-30pt}
%\setlength{\headheight}{44pt}

\pagestyle{fancy}
\fancyhf{} % clear all fields
\fancyhead[R]{%
  \scshape
  \begin{tabular}[t]{@{}r@{}}
  MATH 540, Fall 2015\\Section 1 (9932)\\
  HW \#4, DUE: 2015, September 24
  \end{tabular}}
\fancyhead[L]{%
  \scshape
  \begin{tabular}[t]{@{}r@{}}
  JOSEPH OKONOBOH\\Mathematics\\Cal State Long Beach
  \end{tabular}}
\fancyfoot[C]{\thepage}

\newcommand{\qed}{\hfill \ensuremath{\Box}}


\newcommand*\circled[1]{\tikz[baseline=(char.base)]{
            \node[shape=circle,draw,inner sep=2pt] (char) {#1};}}


\newcommand{\cyc}[1]{\langle #1 \rangle}
\newcommand{\Z}{\mathbb{Z}}
\newcommand{\I}{\mathbb{I}}
\newcommand{\F}{\mathbb{F}}
\newcommand{\M}{\mathbb{M}}
\newcommand{\R}{\mathbb{R}}
\newcommand{\Q}{\mathbb{Q}}
\newcommand{\D}{\displaystyle}
\newcommand{\CYC}[1]{\left\langle #1 \right\rangle}
%\setcounter{section}{-1}

\begin{document}
\begin{enumerate}
%%%%%%%%%%%%%%%%%%%%%%%%%%%%%%%%%%%%%2.3.12%%%%%%%%%%%%%%%%%%%%%%%%%%%%%%%%%%%%%
   \item[2.3.12]  Prove that the following groups are \textit{not} cyclic:
                  \begin{enumerate}
                     \item $Z_2 \times Z_2$
                     \item $Z_2 \times \Z$
                     \item $\Z \times \Z$.
                  \end{enumerate}
      
      \textbf{Proof.}
      \begin{enumerate}
         \item The order of $Z_2 \times Z_2$ is 4, but no element in this group
               has order 4; thus $Z_2 \times Z_2$ is not cyclic.
         \item Let $Z_2 = \cyc{x}$. Observe that $Z_2 \times \Z$ is not finite,
               so in order for it to be cyclic it must be isomorphic to $\Z$.
               But this is not the case since $Z_2 \times \Z$ has two elements
               of finite order(namely $(1, 0)$ and $(x, 0)$) while $\Z$ has
               exactly 1 element of finite order.
         \item Suppose to the contrary that $\Z \times \Z$ is cyclic. Then there
               exist nonzero integers $a$ and $b$ such that
               $$\Z \times \Z = \cyc{(a,b)} = \{(na, nb) : n \in \Z\}.$$
               Thus there exists an integer $m$ such that
               $(ma, mb) = (0, 1)$. That is, $ma = 0$ and $mb = 1$. Since
               $ma = 0$, we must have $m = 0$ or $a = 0$. If $m$ is 0, then
               $(ma, mb) = (0, 0) \neq (0, 1)$, a contradiction; thus we must
               have $a = 0$, contradicting our assumption that $a$ is nonzero.
               Thus $\Z \times \Z$ is not cyclic.
      \end{enumerate} \qed
%%%%%%%%%%%%%%%%%%%%%%%%%%%%%%%%%%%%%2.3.19%%%%%%%%%%%%%%%%%%%%%%%%%%%%%%%%%%%%%
   \item[2.3.19]  Show that if $H$ is any group and $h$ is an element of $H$,
                  then there is a unique homomorphism from $\Z$ to $H$ such that
                  $1 \mapsto h$.
                  
      \textbf{Proof.} Let $H$ be a group and let $h \in H$. First we shall show
      that there exists a homomorphism from $\Z$ to $H$ such that $1 \mapsto h$.
      So consider the map $\alpha : \Z \rightarrow H$ defined by
      $n \mapsto h^n$. Clearly $\alpha(1) = h$ and
      $$\alpha(x+y) = h^{x+y} = h^xh^y = \alpha(x)\alpha(y) \text{ for all }
        x, y \in \Z^+,$$
      so that $\alpha$ is a homomorphism. To show uniqueness, suppose that
      $\alpha' : \Z \rightarrow H$ is an homomorphism such that
      $\alpha'(1) = h$. Then according to Exercise 1.6.1, we have that
      $\alpha'(n) = \alpha'(n\cdot1) = \alpha'(1)^n = h^n$ for all $n \in \Z$;
      that is, $\alpha' = \alpha$, as desired. \qed
%%%%%%%%%%%%%%%%%%%%%%%%%%%%%%%%%%%%%2.3.23%%%%%%%%%%%%%%%%%%%%%%%%%%%%%%%%%%%%%
   \item[2.3.23]  Show that $(\Z/2^n\Z)^\times$ is not cyclic for any $n \ge 3$.
                  [Find two distinct subgroups of order 2.]

      \textbf{Proof.} Let $n \ge 3$ be an integer. Consider
      $2^{n-1}-1, 2^n-1 \in \Z/2^n\Z$. Note that $2^{n-1}-1 \neq 1$ and
      $2^n-1 \neq 1$ because $1 < 2^{n-1}-1 < 2^n-1 < 2^n$. Since $2^{n-1}-1$
      and $2^n-1$ are both odd, it follows that both of them are relatively
      prime to $2^n$, so that they are both in $(\Z/2^n\Z)^\times$. Now we have
      that
      $$(2^{n-1}-1)(2^{n-1}-1) = 2^n2^{n-2}-2^n + 1 \equiv 1\text{ (mod }2^n)$$
      and
      $$(2^n-1)(2^n-1) = 2^n2^n-2^n2 + 1 \equiv 1\text{ (mod }2^n).$$
      Thus both $2^{n-1}-1$ and $2^n-1$ have order 2 in $(\Z/2^n\Z)^\times$.
      However, since these two elements are not equal, it follows that
      $(\Z/2^n\Z)^\times$ has two distinct subgroups of order 2, so that
      $(\Z/2^n\Z)^\times$ is not cyclic. \qed
%%%%%%%%%%%%%%%%%%%%%%%%%%%%%%%%%%%%%2.4.5%%%%%%%%%%%%%%%%%%%%%%%%%%%%%%%%%%%%%%
   \item[2.4.5]   Prove that the subgroup generated by any two distinct elements
                  of order 2 in $S_3$ is all of $S_3$.

      \textbf{Proof.} We know from the discussion on Page 43 of the textbook
      that $D_6 \cong S_3$, so it suffices to show that any two distinct
      elements of order 2 in $D_6$ is all of $D_6$. The elements of order 2 in
      $D_6$ are: $s$, $sr$, and $sr^2$. Since $\cyc{r, s} = D_6$, it suffices to
      show that $r$ and $s$ are both elements of $\cyc{s, sr}$, $\cyc{s, sr^2}$,
      and $\cyc{sr, sr^2}$. Now we have $s(sr) = r$, so that
      $r, s \in \cyc{s, sr}$. Also $(s)(sr^2)(s)(sr^2) = r$, so that
      $r, s \in \cyc{s, sr^2}$. Finally we have $(sr)(sr^2) = r$ and
      $(sr)(sr^2)(sr) = s$, so that $r, s \in \cyc{sr, sr^2}$. We can thus
      conclude that
      $$\cyc{s, sr} = \cyc{s, sr^2} = \cyc{sr, sr^2} = D_6,$$
      as desired. \qed
   \item[2.4.14]  A group $H$ is called \textit{finitely generated} if there is
                  a finite set $A$ such that $H = \cyc{A}$.
                  \begin{enumerate}
                     \item Prove that every finite group is finitely generated.
                     \item Prove that $\Z$ is finitely generated.
                     \item Prove that every finitely generated subgroup of the
                           additive group $\Q$ is cyclic. [If $H$ is a finitely
                           generated subgroup of $\Q$, show that
                           $H \le \CYC{\D\frac{1}{k}}$, where $k$ is the product
                           of all the denominators which appear in a set of
                           generators for $H$.]
                     \item Prove that $\Q$ is not finitely generated.
                  \end{enumerate}

      \textbf{Proof.}

      \begin{enumerate}
         \item Let $H$ be a finite group. Then by Exercise 2.4.1, we have
               $\cyc{H} = H$, so that $H$ is finitely generated.
         \item The group of integers $\Z$ is finitely generated because
               $\Z = \cyc{1}$.
         \item Let $H$ be a finitely generated subgroup of $\Q$. We can assume
               that $H$ is nontrivial (since $H = \{1\}$ is clearly cyclic), so 
               that its set of generators is nonempty. Then it follows that
               $H = \CYC{\D\frac{a_1}{b_1}, \D\frac{a_2}{b_2}, \cdots,
               \D\frac{a_n}{b_n}}$, where $a_i/b_i \in \Q$. Let us now
               consider the cyclic group $\CYC{\D\frac{1}{k}}$, where
               $k = b_1b_2\cdots b_n$. Since $\CYC{\D\frac{1}{k}}$ is an
               additive group, it follows that $n/k \in \CYC{\D\frac{1}{k}}$
               for each integer $n$. To show that $H \le \CYC{\D\frac{1}{k}}$,
               it suffices to show that $a_i/b_i \in \CYC{\D\frac{1}{k}}$, for
               $1 \le i \le n$. So let $a_j/b_j$ be one of the generators for
               some $1 \le j \le n$. Notice that $k/b_j$ is an integer. Thus
               $(k/b_j)(1/k) = 1/b_j \in \CYC{\D\frac{1}{k}}$, so that
               $a_j/b_j = a_j (1/b_j) \in \CYC{\D\frac{1}{k}}$; that is,
               $H \le \CYC{\D\frac{1}{k}}$, as desired. By Theorem 7 (1), Page
               58, it follows that $H$ is cyclic.
         \item If $\Q$ were finitely generated, then, according to
               Exercise 2.4.14 (c) $\Q$ would be cyclic, a contradiction, since
               $\Q$ is not cyclic by our arguments in Exercise 2.3.15.
      \end{enumerate} \qed
%%%%%%%%%%%%%%%%%%%%%%%%%%%%%%%%%%%%%2.4.15%%%%%%%%%%%%%%%%%%%%%%%%%%%%%%%%%%%%%
   \item[2.4.15]  Exhibit a proper subgroup of $\Q$ which is not cyclic.

      \textbf{Solution.} Let $H = \left\{\D\frac{1}{2^i} : i \text{ is a 
      nonnegative integer}\right\}$ be a subset of the additive group $\Q$.
      We thus have that
      \begin{align*}
         \cyc{H} &= \left\{\frac{a_1}{2^{\varepsilon_1}} +
            \frac{a_2}{2^{\varepsilon_2}} + \cdots +
            \frac{a_n}{2^{\varepsilon_n}} : a_i, \varepsilon_i, n \in \Z, 
            \varepsilon_i \ge 0, n \ge 1 \right\} \\
                 &= \left\{\frac{a}{2^i} : a, i \in \Z, i \ge 0\right\}.
      \end{align*}
      Notice that the only prime factor that the denominator of each rational
      number(in its lowest term) in $\cyc{H}$ can have is 2; thus
      $1/7 \notin \cyc{H}$, so that $\cyc{H} \lneq Q$. Now suppose to the
      contrary that $\cyc{H}$ is cyclic; thus it follows by definition that
      $\cyc{H}$ is generated by some rational number $c/d$, where $c, d \in \Z$
      and $d \neq 0$. Let $k$ be the maximum nonnegative integer such that
      $2^k \mid d$. Since $c/2^{k+1} \in \cyc{H}$, it follows that there exists
      some integer $m$ such that $mc/d = c/2^{k+1}$. That is $m = d / 2^{k+1}$,
      so that $2^{k+1} \mid d$, contradicting the maximality of $k$. Thus
      $\cyc{H}$ is a proper non cyclic subgroup of $Q$.
%%%%%%%%%%%%%%%%%%%%%%%%%%%%%%%%%%%%%2.5.5%%%%%%%%%%%%%%%%%%%%%%%%%%%%%%%%%%%%%%
   \item[2.5.5]   Use the given lattice to find all elements $x \in D_{16}$
                  such that $D_{16} = \cyc{x, s}$ (there are 8 such elements
                  $x$).
                  
      \textbf{Solution.} By observing the given lattice of $D_{16}$, we find
      that      
      $$x \in \{r, r^3, r^5, r^7, sr^3, sr^7, sr^5, sr\}.$$
\end{enumerate}
\end{document}
