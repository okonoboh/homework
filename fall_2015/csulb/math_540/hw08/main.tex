\documentclass[9pt]{article}

\usepackage{amssymb}
\usepackage{amsmath}
\usepackage{amsfonts}
\usepackage{comment}
\usepackage{fancyhdr}
\usepackage{mathrsfs}
\usepackage{enumitem}
%\usepackage[retainorgcmds]{IEEEtrantools}

\everymath{\displaystyle}

\usepackage{tikz}

\voffset = -50pt
%\textheight = 700pt
\addtolength{\textwidth}{60pt}
\addtolength{\evensidemargin}{-30pt}
\addtolength{\oddsidemargin}{-30pt}
%\setlength{\headheight}{44pt}

\pagestyle{fancy}
\fancyhf{} % clear all fields
\fancyhead[R]{%
  \scshape
  \begin{tabular}[t]{@{}r@{}}
  MATH 540, Fall 2015\\Section 1 (9932)\\
  HW \#8, DUE: 2015, October 29
  \end{tabular}}
\fancyhead[L]{%
  \scshape
  \begin{tabular}[t]{@{}r@{}}
  JOSEPH OKONOBOH\\Mathematics\\Cal State Long Beach
  \end{tabular}}
\fancyfoot[C]{\thepage}

\newcommand{\qed}{\hfill \ensuremath{\Box}}


\newcommand*\circled[1]{\tikz[baseline=(char.base)]{
            \node[shape=circle,draw,inner sep=2pt] (char) {#1};}}


\newcommand{\cyc}[1]{\langle #1 \rangle}
\newcommand{\Z}{\mathbb{Z}}
\newcommand{\C}{\mathbb{C}}
\newcommand{\F}{\mathbb{F}}
\newcommand{\M}{\mathbb{M}}
\newcommand{\R}{\mathbb{R}}
\newcommand{\Q}{\mathbb{Q}}
\everymath{\displaystyle}
\newcommand{\CYC}[1]{\left\langle #1 \right\rangle}
%\setcounter{section}{-1}

\begin{document}
\begin{enumerate}
%%%%%%%%%%%%%%%%%%%%%%%%%%%%%%%%%%%%%4.1.02%%%%%%%%%%%%%%%%%%%%%%%%%%%%%%%%%%%%%
   \item[4.1.2]   Let $G$ be a \textit{permutation group} on the set $A$
                  (i.e., $G \le S_A$), let $\sigma \in G$ and let $a \in A$.
                  Prove that $\sigma G_a\sigma^{-1} = G_{\sigma(a)}$. Deduce
                  that if $G$ acts transitively on $A$ then
                  $$\bigcap_{\sigma \in G}\sigma G_a\sigma^{-1} = 1.$$
%%%%%%%%%%%%%%%%%%%%%%%%%%%%%%%%%%%%%4.1.03%%%%%%%%%%%%%%%%%%%%%%%%%%%%%%%%%%%%%
   \item[4.1.3]   Assume that $G$ is an abelian, transitive subgroup of $S_A$.
                  Show that $\sigma(a) \neq a$ for all $\sigma \in G - \{1\}$
                  and all $a \in A$. Deduce that $|G| = |A|$. [Use the preceding
                  exercise.]
%%%%%%%%%%%%%%%%%%%%%%%%%%%%%%%%%%%%%4.1.08%%%%%%%%%%%%%%%%%%%%%%%%%%%%%%%%%%%%%
   \item[4.1.8]   A transitive permutation group $G$ on a set $A$ is called
                  \textit{doubly transitive} if for any (hence all) $a \in A$
                  the subgroup $G_a$ is transitive on the set $A - \{a\}$.
                  \begin{enumerate}
                     \item Prove that $S_n$ is doubly transitive on
                           $\{1, 2, \ldots, n\}$ for all $n \ge 2$.
                     \item Prove that a doubly transitive group is primitive.
                           Deduce that $D_8$ is not doubly transitive in its
                           action on the 4 vertices of a square.
                  \end{enumerate}
%%%%%%%%%%%%%%%%%%%%%%%%%%%%%%%%%%%%%4.2.04%%%%%%%%%%%%%%%%%%%%%%%%%%%%%%%%%%%%%
   \item[4.2.4]   Use the left regular representation of $Q_8$ to produce two
                  elements of $S_8$ which generate a subgroup of $S_8$
                  isomorphic to the quaternion group $Q_8$.
%%%%%%%%%%%%%%%%%%%%%%%%%%%%%%%%%%%%%4.2.07%%%%%%%%%%%%%%%%%%%%%%%%%%%%%%%%%%%%%
   \item[4.2.7]   Let $Q_8$ be the quaternion group of order 8.
                  \begin{enumerate}
                     \item Prove that $Q_8$ is isomorphic to a subgroup of
                           $S_8$.
                     \item Prove that $Q_8$ is not isomorphic to a subgroup of
                           $S_n$ for any $n \le 7$. [If $Q_8$ acts on any set
                           $A$ of order $\le 7$ show that the stabilizer of any
                           point $a \in A$ must contain the subgroup
                           $\cyc{-1}$.]
                  \end{enumerate}
%%%%%%%%%%%%%%%%%%%%%%%%%%%%%%%%%%%%%4.2.10%%%%%%%%%%%%%%%%%%%%%%%%%%%%%%%%%%%%%
   \item[4.2.10]  Prove that every non-abelian group of order 6 has a nonnormal
                  subgroup of order 2. Use this to classify groups of order 6.
                  [Produce an injective homomorphism into $S_3$.]
\end{enumerate}
\end{document}
