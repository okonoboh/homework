\documentclass[9pt]{article}

\usepackage{amssymb}
\usepackage{amsmath}
\usepackage{amsfonts}
\usepackage{comment}
\usepackage{fancyhdr}
\usepackage{mathrsfs}
\usepackage{enumitem}
%\usepackage[retainorgcmds]{IEEEtrantools}

\everymath{\displaystyle}

\usepackage{tikz}

\voffset = -50pt
%\textheight = 700pt
\addtolength{\textwidth}{60pt}
\addtolength{\evensidemargin}{-30pt}
\addtolength{\oddsidemargin}{-30pt}
%\setlength{\headheight}{44pt}

\pagestyle{fancy}
\fancyhf{} % clear all fields
\fancyhead[R]{%
  \scshape
  \begin{tabular}[t]{@{}r@{}}
  MATH 540, Fall 2015\\Section 1 (9932)\\
  HW \#8, DUE: 2015, October 29
  \end{tabular}}
\fancyhead[L]{%
  \scshape
  \begin{tabular}[t]{@{}r@{}}
  JOSEPH OKONOBOH\\Mathematics\\Cal State Long Beach
  \end{tabular}}
\fancyfoot[C]{\thepage}

\newcommand{\qed}{\hfill \ensuremath{\Box}}


\newcommand*\circled[1]{\tikz[baseline=(char.base)]{
            \node[shape=circle,draw,inner sep=2pt] (char) {#1};}}


\newcommand{\cyc}[1]{\langle #1 \rangle}
\newcommand{\Z}{\mathbb{Z}}
\newcommand{\C}{\mathbb{C}}
\newcommand{\F}{\mathbb{F}}
\newcommand{\M}{\mathbb{M}}
\newcommand{\R}{\mathbb{R}}
\newcommand{\Q}{\mathbb{Q}}
\everymath{\displaystyle}
\newcommand{\CYC}[1]{\left\langle #1 \right\rangle}
%\setcounter{section}{-1}

\begin{document}
\begin{enumerate}
%%%%%%%%%%%%%%%%%%%%%%%%%%%%%%%%%%%%%4.1.02%%%%%%%%%%%%%%%%%%%%%%%%%%%%%%%%%%%%%
   \item[4.1.2]   Let $G$ be a \textit{permutation group} on the set $A$
                  (i.e., $G \le S_A$), let $\sigma \in G$ and let $a \in A$.
                  Prove that $\sigma G_a\sigma^{-1} = G_{\sigma(a)}$. Deduce
                  that if $G$ acts transitively on $A$ then
                  $$\bigcap_{\sigma \in G}\sigma G_a\sigma^{-1} = 1.$$
                  
      \textbf{Proof.} $(\subseteq):$ Let $\gamma \in \sigma G_a\sigma^{-1}$.
      Then $\gamma = \sigma \beta\sigma^{-1}$ for some $\beta \in G_a$. So we
      have that
      \begin{align*}
         \gamma(\sigma(a)) = (\sigma \beta\sigma^{-1})(\sigma(a)) &=
            (\sigma \beta)(\sigma^{-1}(\sigma(a))) \\
            &= (\sigma \beta)((\sigma^{-1}\sigma)(a)) \\
            &= (\sigma \beta)(a) \\
            &= \sigma(\beta(a)) \\
            &= \sigma(a),  & [\beta \in G_a]
      \end{align*}
      so that $\gamma \in G_{\sigma(a)}$, and thus,
      $\sigma G_a\sigma^{-1} \subseteq G_{\sigma(a)}$.
      
      $(\supseteq)$: Let $\alpha \in G_{\sigma(a)}$. Since
      \begin{align*}
         (\sigma^{-1}\alpha\sigma)(a) &= \sigma^{-1}((\alpha\sigma)(a)) \\
            &= \sigma^{-1}(\alpha(\sigma(a))) \\
            &= \sigma^{-1}(\sigma(a)) &[\alpha \in G_{\sigma(a)}] \\
            &= (\sigma^{-1}\sigma)(a) \\
            &= a,
      \end{align*}
      it follows that $\sigma^{-1}\alpha\sigma \in G_a$. Hence
      $$\alpha = \sigma(\sigma^{-1}\alpha\sigma)\sigma^{-1} \in \sigma
        G_a\sigma^{-1},$$
      so that $\sigma G_a\sigma^{-1} \supseteq G_{\sigma(a)}$, and we conclude
      that $\sigma G_a\sigma^{-1} = G_{\sigma(a)}$.

      Now suppose that $G$ acts transitively on $A$. This implies that if
      $b \in A$, then there is a permutation in $G$ that maps $a$ to $b$. Thus
      \begin{align*}
         \bigcap_{\sigma \in G}\sigma G_a\sigma^{-1} &=
            \bigcap_{\sigma \in G} G_{\sigma(a)} \\
            &= \bigcap_{b \in A} G_b \\
            &= \text{Kernel of the action of $G$ on $A$} \\
            &= 1.
      \end{align*}
      The last equality follows because the trivial permutation is the only
      permutation in $S_A$ that fixes all elements of $A$. \qed
%%%%%%%%%%%%%%%%%%%%%%%%%%%%%%%%%%%%%4.1.03%%%%%%%%%%%%%%%%%%%%%%%%%%%%%%%%%%%%%
   \item[4.1.3]   Assume that $G$ is an abelian, transitive subgroup of $S_A$.
                  Show that $\sigma(a) \neq a$ for all $\sigma \in G - \{1\}$
                  and all $a \in A$. Deduce that $|G| = |A|$. [Use the preceding
                  exercise.]
                  
      \textbf{Proof.} Suppose that $G$ is an abelian, transitive subgroup of
      $S_A$. Let $a \in A$. Now consider some $b \in A$. Since $G$ is transitive
      on $A$, it follows that there exists $\sigma \in G$ such that
      $\sigma(a) = b$. Using Exericse 4.1.2 and the assumption that $G$ is
      abelian we have
      $$G_b = G_{\sigma(a)} = \sigma G_a\sigma^{-1} =
        \sigma\sigma^{-1}G_a = G_a.$$
      That is, all elements in $A$ have the same stabilizer; thus, the kernel
      of the action of $G$ on $A$ is $G_a$; from Exercise 4.1.2, we have that
      $G_a = \{1\}$. Hence if a permutation stabilizes an element of $A$, then
      that permutation must be the identity. Equivalently, if
      $\alpha \in G - \{1\}$ and $c \in A$, then $\alpha(c) \neq c$. The orbit
      of $a$ is $A$ because the action is transitive; thus from Proposition 4.2,
      it follows that $|A| = |G : G_a| = |G|/|G_a| = |G|/1$, and we conclude
      that $|A| = |G|$. \qed
%%%%%%%%%%%%%%%%%%%%%%%%%%%%%%%%%%%%%4.1.08%%%%%%%%%%%%%%%%%%%%%%%%%%%%%%%%%%%%%
   \item[4.1.8]   A transitive permutation group $G$ on a set $A$ is called
                  \textit{doubly transitive} if for any (hence all) $a \in A$
                  the subgroup $G_a$ is transitive on the set $A - \{a\}$.
                  \begin{enumerate}
                     \item Prove that $S_n$ is doubly transitive on
                           $\{1, 2, \ldots, n\}$ for all $n \ge 2$.
                     \item Prove that a doubly transitive group is primitive.
                           Deduce that $D_8$ is not doubly transitive in its
                           action on the 4 vertices of a square.
                  \end{enumerate}
                  
      \textbf{Solution.}
      
      \begin{enumerate}
         \item Let $G = S_n$, $I_n = \{1, 2, \ldots, n\}$. If $n = 2$, then
               $G_1$ is transitive on $\{2\}$ because the identity in $G_1$ maps
               2 to 2; $G_2$ is similarly transitive on $\{1\}$. Now suppose
               that $n \ge 3$, so that $I_n$ has at least three elements. Let
               $a \in I_n$. Since $n \ge 3$, consider $c, d \in I_n - \{a\}$,
               where $c \neq d$. Consider the permutation
               $\sigma = (c \quad d) \in G$. We have that $\sigma(a) = a$, so
               that $\sigma \in G_a$ and $\sigma(c) = d$. If $c = d$, then the
               identity in $G_a$ maps $c$ to $d$. Thus $G_a$ is transitive on
               $I_n - \{a\}$, and we conclude that $S_n$ is doubly transitive on
               $I_n$, $n \ge 2$.
         \item Let $G$ be a doubly transitive group on a set $A$. If $|A| = 2$,
               then the only blocks are the trivial ones; so assume that $A$
               has more than 2 elements. Let $B$ be a nonempty proper subset of
               $A$ with more than 1 element. Let $b_1, b_2 \in B$,
               $b_1 \neq b_2$. Since $B \subsetneq A$, there exists $b_3 \in A$
               such that $b_3 \notin B$. Since $G$ is transitive on $A$, there
               exists $\alpha \in G$ such that $\alpha(b_1) = b_2$. If
               $\alpha(b_2) = b_2$, then we would have that
               $\alpha(b_2) = \alpha(b_1)$, so that $b_1 = b_2$ (since $\alpha$
               is injective), a contradiction. Thus
               $\alpha(b_2), b_3 \notin A - \{b_2\}$. Since $G$ is doubly
               transitive on $A$, there exists $\beta \in G_{b_2}$ such that
               $\beta(\alpha(b_2)) = b_3$. Now
               $$\beta(\alpha(b_1)) = \beta(b_2) = b_2.$$
               Thus $(\beta\alpha)(B) \neq B$ because
               $b_3 = (\beta\alpha)(b_2) \in (\beta\alpha)(B)$ but
               $b_3 \notin B$. Moreover, $b_2 \in B$ and
               $b_2 = (\beta\alpha)(b_1) \in (\beta\alpha)(B)$, so that
               $(\beta\alpha)(B) \cap B \neq \emptyset$. We have thus shown
               that $B$ is not a block of $A$, so that the only blocks of $A$
               are the trivial ones, and we conclude that $G$ is primitive. Let
               $D_8 = \cyc{r, s}$, where $r$ is a clockwise rotation by $90^\circ$
               through the origin of a square (labelled as on Page 24) and $s$
               is a reflection through vertex 1. Let $D_8$ act on the vertices
               of the square, $V = \{1, 2, 3, 4\}$. Let $B = \{1 , 3\}$. Then we
               have that
               \begin{align*}
                  1\{1, 3\} &= \{1, 3\} & s\{1, 3\} &= \{1, 3\} \\
                  r\{1, 3\} &= \{2, 4\} & sr\{1, 3\} &= \{2, 4\} \\
                  r^2\{1, 3\} &= \{1, 3\} & sr^2\{1, 3\} &= \{1, 3\} \\
                  r^3\{1, 3\} &= \{2, 4\} & sr^3\{1, 3\} &= \{2, 4\},
               \end{align*}
               so that $\sigma(B) = B$ or $\sigma(B) \cap B = \emptyset$. That
               is, $B$ is a nontrivial block of $V$. It follows by the preceding
               proof that $D_8$ is not doubly transitive. \qed               
      \end{enumerate}
%%%%%%%%%%%%%%%%%%%%%%%%%%%%%%%%%%%%%4.2.04%%%%%%%%%%%%%%%%%%%%%%%%%%%%%%%%%%%%%
   \item[4.2.4]   Use the left regular representation of $Q_8$ to produce two
                  elements of $S_8$ which generate a subgroup of $S_8$
                  isomorphic to the quaternion group $Q_8$.
                  
      \textbf{Proof.} Label the elements of $Q_8$, $1$, $-1$, $i$, $-i$, $j$,
      $-j$, $k$, $-k$, as 1, 2, 3, 4, 5, 6, 7, and 8 respectively. Let
      $\pi : Q_8 \rightarrow S_8$ be the homomorphism of the representation
      induced by the action of $Q_8$ on itself by left multiplication. It
      follows by Theorem 4.3(3) (in this case $H = \{1\}$) that the kernel of
      this action is trivial, so that $\pi$ is injective, and we have by the
      First Isomorphism Theorem that
      $Q_8/\{1\} \cong Q_8 \cong \pi(Q_8) \le S_8$. Recall that
      $Q_8 = \cyc{i, j}$, so it follows that $\pi(i)$ and $\pi(j)$ generate
      $\pi(Q_8)$, where
      $$\pi(i) = (1 \quad 3 \quad 2 \quad 4)(5 \quad 7 \quad 6 \quad 8)
        \text{ and }
        \pi(j) = (1 \quad 5 \quad 2 \quad 6)(3 \quad 8 \quad 4 \quad 7).$$
%%%%%%%%%%%%%%%%%%%%%%%%%%%%%%%%%%%%%4.2.07%%%%%%%%%%%%%%%%%%%%%%%%%%%%%%%%%%%%%
   \item[4.2.7]   Let $Q_8$ be the quaternion group of order 8.
                  \begin{enumerate}
                     \item Prove that $Q_8$ is isomorphic to a subgroup of
                           $S_8$.
                     \item Prove that $Q_8$ is not isomorphic to a subgroup of
                           $S_n$ for any $n \le 7$. [If $Q_8$ acts on any set
                           $A$ of order $\le 7$ show that the stabilizer of any
                           point $a \in A$ must contain the subgroup
                           $\cyc{-1}$.]
                  \end{enumerate}
                  
      \textbf{Solution.}
      
      \begin{enumerate}
         \item Since $Q_8$ is a group of order 8, it follows by Corollary 4.4
               that $Q_8$ isomorphic to a subgroup of $S_8$.
         \item Suppose to the contrary that $Q_8$ is isomorphic to a subgroup of
               $S_n$, where $n \ge 7$; that is, there exists an injective
               homomorphism
               $$\pi : Q_8 \rightarrow S_n.$$
               By Proposition 4.1, there exists an action of $Q_8$ on
               $A = \{1, 2, \ldots, n\}$, where the kernel of this action is the
               kernel of $\pi$. Since $\pi$ is injective, its action is trivial,
               so that the action of $Q_8$ on $A$ is also trivial. Let
               $a \in A$ and $O_a$ be the orbit of $a$. It follows by
               Proposition 4.2 that $|O_a| = |Q_8 : Q_{8_a}|$, so that
               $8 = |Q_8| = |O_a| \cdot |Q_{8_a}|$. Recall that $Q_{8_a}$ is a
               subgroup of $Q_8$, so we have by Lagrange's Theorem that
               $$|Q_{8_a}| \in \{1, 2, 4, 8\}.$$
               If $|Q_{8_a}| = 1$, then we would have that $|O_a| = 8$, a
               contradiction since $O_a \subseteq A$ and $|A| \le 7$. Thus
               $$|Q_{8_a}| \in \{2, 4, 8\},$$
               so that $2$ divides the order of $Q_{8_a}$, and it follows by
               Cauchy's Theorem that $Q_{8_a}$ has an element of order 2. But
               $-1$ is the only element of order 2 in $Q_8$; thus
               $-1 \in Q_{8_a}$. Since $a$ was arbitrary, it follows that
               $\cyc{-1} \subseteq Q_{8_a}$ for all $a \in A$. Thus
               $-1$ stabilizes every element of $A$, so that $-1$ is in the
               kernel of this action. That is, the kernel of this action is not
               trivial, a contradiction. So we conclude that $Q_8$ is not
               isomorphic to a subgroup of $S_n$, where $n \le 7$. \qed
      \end{enumerate}
%%%%%%%%%%%%%%%%%%%%%%%%%%%%%%%%%%%%%4.2.10%%%%%%%%%%%%%%%%%%%%%%%%%%%%%%%%%%%%%
   \item[4.2.10]  Prove that every non-abelian group of order 6 has a nonnormal
                  subgroup of order 2. Use this to classify groups of order 6.
                  [Produce an injective homomorphism into $S_3$.]
\end{enumerate}
\end{document}
