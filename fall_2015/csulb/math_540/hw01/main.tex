\documentclass[9pt]{article}

\usepackage{amssymb}
\usepackage{amsmath}
\usepackage{amsfonts}
\usepackage{comment}
\usepackage{fancyhdr}
\usepackage{mathrsfs}
\usepackage{enumitem}
%\usepackage[retainorgcmds]{IEEEtrantools}

\everymath{\displaystyle}

\usepackage{tikz}

\voffset = -50pt
%\textheight = 700pt
\addtolength{\textwidth}{60pt}
\addtolength{\evensidemargin}{-30pt}
\addtolength{\oddsidemargin}{-30pt}
%\setlength{\headheight}{44pt}

\pagestyle{fancy}
\fancyhf{} % clear all fields
\fancyhead[R]{%
  \scshape
  \begin{tabular}[t]{@{}r@{}}
  MATH 540, Fall 2015\\Section 1 (9932)\\
  HW \#1, DUE: 2015, September 03
  \end{tabular}}
\fancyhead[L]{%
  \scshape
  \begin{tabular}[t]{@{}r@{}}
  JOSEPH OKONOBOH\\Mathematics\\Cal State Long Beach
  \end{tabular}}
\fancyfoot[C]{\thepage}

\newcommand{\qed}{\hfill \ensuremath{\Box}}


\newcommand*\circled[1]{\tikz[baseline=(char.base)]{
            \node[shape=circle,draw,inner sep=2pt] (char) {#1};}}


\newcommand{\cyc}[1]{\langle #1 \rangle}
\newcommand{\Z}{\mathbb{Z}}
\newcommand{\I}{\mathbb{I}}
\newcommand{\F}{\mathbb{F}}
\newcommand{\M}{\mathbb{M}}
\newcommand{\R}{\mathbb{R}}
\newcommand{\Q}{\mathbb{Q}}
\newcommand{\D}{\displaystyle}
%\setcounter{section}{-1}

\begin{document}
\begin{enumerate}
%%%%%%%%%%%%%%%%%%%%%%%%%%%%%%%%%%%%%%L.1%%%%%%%%%%%%%%%%%%%%%%%%%%%%%%%%%%%%%%%
   \item[]        \textbf{Lemma 1.} \textit{If $g$ is an element of a group
                  such that $g^m = 1$, then $|g|$ must divide $m$.}

                  \textbf{Proof.} Let $g$ be an element of some group. Suppose
                  that there exists an integer $m$ such that $g^m = 1$. This
                  implies that $|g|$ is finite. So let $|g| = n \in \Z^+$. By
                  the Euclidean algorithm, there exist unique integers $q$ and
                  $r$ such that
                  $$m = qn + r, \text{ where } 0 \le r < n.$$
                  That is,
                  $$1 = g^m = g^{qn + r} = g^{qn}g^r = (g^n)^qg^r =1^qg^r=g^r.$$
                  If $0 < r < n$, then this will contradict the fact that the
                  order of $g$ is $n$, so we are forced to conclude that
                  $r = 0$. That is, $m = qn$, as desired. \qed
%%%%%%%%%%%%%%%%%%%%%%%%%%%%%%%%%%%%%%L.2%%%%%%%%%%%%%%%%%%%%%%%%%%%%%%%%%%%%%%%
   \item[]        \textbf{Lemma 2.} \textit{Let $r$ and $s$ be the generators of
                  $D_{2n}$. Then the statement}
                  \begin{equation} \label{L1_1}
                     r^is = sr^{-i}
                  \end{equation}
                  \textit{holds for all positive integers $i$.}

                  \textbf{Proof.} Proceed by induction on $i$. The base
                  case($i = 1$) follows trivially from the presentation of
                  $D_{2n}$. Now assume that \eqref{L1_1} also holds for some
                  positive integer $k$. Then it follows that
                  \begin{align*}
                     r^{k+1}s &= (r^kr)s &[\text{Exercise 1.1.19(a)}] \\
                        &= r^k(rs) &[\text{Associativity}] \\
                        &= r^k(sr^{-1}) &[\text{Relation in presentation}] \\
                        &= (r^ks)r^{-1} &[\text{Associativity}] \\
                        &= (sr^{-k})r^{-1} &[\text{Inductive hypothesis}] \\
                        &= s(r^{-k}r^{-1}) &[\text{Associativity}] \\
                        &= sr^{-(k+1)} &[\text{Exercise 1.1.19(c)}], \\
                  \end{align*}
                  so that \eqref{L1_1} also holds for $k+1$. Thus it follows by
                  induction that it holds for each positive integer $i$. \qed

%%%%%%%%%%%%%%%%%%%%%%%%%%%%%%%%%%%%%%1.6%%%%%%%%%%%%%%%%%%%%%%%%%%%%%%%%%%%%%%%
   \item[1.1.6]   Determine which of the following sets are groups under
                  addition:
                  \begin{enumerate}
                     \item the set of rational numbers (including $0 = 0/1$) in
                           lowest terms whose denominators are odd.
                     \item the set of rational numbers (including $0 = 0/1$) in
                           lowest terms whose denominators are even.
                     \item the set of rational numbers of absolute value $< 1$.
                     \item the set of rational numbers of absolute value $\ge 1$
                           together with 0.
                     \item the set of rational numbers with denominators equal
                           to 1 or 2.
                     \item the set of rational numbers with denominators equal
                           to 1, 2, or 3.
                  \end{enumerate}

      \textbf{Solution.}

      \begin{enumerate}
         \item We claim that the set
               $$S = \left\{\frac{a}{b} \in \Q : a, b \in \Z, b \text{ is odd,} 
                        \text{ and } \gcd(a, b) = 1\right\},$$
               is a group under addition.

               \textbf{Proof.} First we want show that $S$ is closed under 
               addition. Notice that $S$ is nonempty since it contains 7/5, so
               let $r, s \in S$. By definition of $S$, we have that
               $r = \frac{a_1}{b_1}$ and $s = \frac{a_2}{b_2}$ for some
               integers $a_1$, $a_2$, and odd integers $b_1$ and $b_2$, such
               that $\gcd(a_1, b_1) = \gcd(a_2, b_2) = 1$. It follows that
               \begin{align*}
                  r + s &= \frac{a_1}{b_1} + \frac{a_2}{b_2} \\
                        &= \frac{a_1b_2 + a_2b_1}{b_1b_2}.
               \end{align*}

               Since $b_1$ and $b_2$ are both odd, it must necessarily be the 
               case that $b_1b_2$ is also odd. In order words, $b_1b_2$ contains 
               no factor of 2, so that if we reduce $r + s$ to its lowest term, 
               the denominator of this lowest term will still be odd. Hence
               $r + s \in S$, so that $S$ is closed under addition. To complete 
               the proof we must now show that $S$ satisfies the group axioms. 
               We observe that $0/1$ is the identity element in $S$ and that, 
               for all $s \in S$, we have $-s \in S$, so that every 
               element of $S$ has an inverse under addition. Since
               $S \subset \Q$, and since $\Q$ is associative under addition, 
               it follows that $S$ is also associative under addition. Thus $S$ 
               satisfies the group axioms, so that $(S, +)$ is a group. \qed
         \item The set
               $$S = \left\{\frac{a}{b} \in \Q : a, b \in \Z, b
                        \text{ is nonzero and even,} \text{ and }
                        \gcd(a, b) = 1\right\},$$
               is not a group under addition because it is not closed. Indeed,
               we have $\frac{3}{14} \in S$, but
               $\frac{3}{14} + \frac{3}{14} = \frac{3}{7} \notin S$ because 7 is
               not even.
         \item The set
               $$S = \left\{\frac{a}{b} \in \Q : a, b \in \Z \text{ with }
                  b \neq 0, \text{ and }
                  \left|\frac{a}{b}\right| < 1\right\},$$
               is not a group under addition because it is not closed. For 
               example, we have $\frac{9}{10} \in S$, but
               $\frac{9}{10} + \frac{9}{10} = \frac{18}{10} \notin S$
               because $\left|\frac{18}{10}\right| > 1$.
         \item The set
               $$S = \left\{\frac{a}{b} \in \Q : a = 0 \text{ or }
                     \left|\frac{a}{b}\right| \ge 1\right\},$$
               is not a group under addition because it is not closed. For
               example, we have $-\frac{11}{10}, \frac{10}{10} \in S$, but
               $-\frac{11}{10} + \frac{10}{10} = -\frac{1}{10} \notin S$ because
               $\left|-\frac{1}{10}\right| < 1$.
         \item We claim that the set
               $$S = \left\{\frac{a}{b} \in \Q : a \in \Z \text{ and }
                   b = 1 \text{ or } b = 2\right\},$$
               is a group under addition.

               \textbf{Proof.} The number $0/1 = 0$ is the identity for $S$
               because $0 \in S$ and $s + 0 = 0 + s = s$ for all $s \in S$. The
               set $S$ is associative under addition because $S \subset \Q$ and
               $\Q$ is associative under addition. Finally, if the denominator
               of a rational number $s$ is 1 or 2, then it trivially follows
               that $-s$ has the same property. Thus $-s \in S$, and we have
               that $S$ is closed under inverses. So to complete the proof, we
               need only show that $S$ is closed under addition. To that end,
               let $\frac{a_1}{b_1}, \frac{a_2}{b_2} \in S$. Assume without loss
               of generality that $b_1 = b_2 = 2$ (if either is 1, simply
               multiply the numerator and denominator by 2). So it follows that
               $$\frac{a_1}{2} + \frac{a_2}{2} = \frac{a_1+a_2}{2}.$$
               
               If $a_1 + a_2$ is odd, then the sum above is a member of $S$.
               Also, if $a_1 + a_2$ is even, then we can reduce the sum to
               $\frac{c_1}{1} \in S$, where $2c_1 = a_1 + a_2$. In either case,
               we have shown that $S$ is closed under addition. Thus we conclude
               that $(S, +)$ is a group. \qed
         \item The set
               $$S = \left\{\frac{a}{b} \in \Q : a \in \Z \text{ and }
                   b \in \{1, 2, 3\} \right\},$$
               is not a group under addition because it is not closed. Indeed,
               for $\frac{1}{2}, \frac{1}{3} \in S$, we have $\frac{1}{2} +
               \frac{1}{3} = \frac{5}{6} \notin S$.
      \end{enumerate}
%%%%%%%%%%%%%%%%%%%%%%%%%%%%%%%%%%%%%%1.19%%%%%%%%%%%%%%%%%%%%%%%%%%%%%%%%%%%%%%
   \item[1.1.19]  Let $x \in G$ and let $a, b \in \Z^+$.
                  \begin{enumerate}
                     \item Prove that $x^{a+b} = x^ax^b$ \text{ and }
                           $(x^a)^b = x^{ab}$.
                     \item Prove that $(x^a)^{-1} = x^{-a}$.
                     \item Establish part (a) for arbitrary integers $a$ and $b$
                           (positive, negative or zero).
                  \end{enumerate}
               
      \textbf{Proof.} Fix positive integers $a$ and $b$, a group $G$, and
      $x \in G$. We shall be making use of the following definitions: 
      $$x^0 = 1, x^1 = x \text{ and, for each $m \in \Z^+$, }
        x^{m+1} = x^mx^1,$$
      and
      $$\text{for each } p \in \Z^+, x^{-p} = (x^{-1})^p.$$
      
      \begin{enumerate}
         \item First, we shall prove that
               \begin{equation} \label{1_1_19_1}
                  x^{a+n} = x^ax^n 
               \end{equation}               
               holds for each positive integer $n$. So we shall proceed by
               induction. Statement \eqref{1_1_19_1} holds for the base
               case ($n = 1$) by definition. So, for our inductive hypothesis,
               assume that \eqref{1_1_19_1} holds for some positive integer $k$.
               Thus it follows that
               \begin{align*}
                  x^{a+(k+1)} &=x^{(a+k)+1}&[\text{Associativity of addition}]\\
                     &= x^{a+k}x^1  &[\text{Definition}] \\
                     &= (x^ax^k)x^1 &[\text{Inductive hypothesis}] \\
                     &= x^a(x^kx^1) &[\text{Associativity of group operation}]\\
                     &= x^ax^{k+1}.  &[\text{Definition}]
               \end{align*}
               That is, \eqref{1_1_19_1} holds for $k + 1$. Hence, by induction,
               \eqref{1_1_19_1} holds for each positive integer $n$.
               Particularly, it follows that that $x^{a+b} = x^ax^b$. \\
               \mbox{ } \qed \\
               
               Now we shall show by induction that
               \begin{equation} \label{1_1_19_2}
                  (x^a)^n = x^{an} 
               \end{equation}
               for each positive integer $n$. Statement \eqref{1_1_19_2} holds
               trivially for the base case ($n = 1$). For our inductive
               hypothesis, we shall now assume that \eqref{1_1_19_2} holds for
               some positive integer $k$. So
               \begin{align*}
                  (x^a)^{k+1} &= (x^a)^k(x^a)^1 &[\text{Definition}] \\
                     &= (x^a)^kx^a &[\text{Base case}] \\
                     &= x^{ak}x^a &[\text{Inductive hypothesis}] \\
                     &= x^{ak+a} &[1.1.19 (a)] \\
                     &= x^{a(k+1)}.
               \end{align*}
               That is, \eqref{1_1_19_2} holds for $k + 1$, and we conclude by
               induction that it also holds for each positive integer $n$.
               Particularly, we have that $(x^a)^b = x^{ab}$. \\
               \mbox{ } \qed
         \item For a positive integer $n$, consider the statement
               \begin{equation} \label{1_1_19_3}
                  x^{-n} = (x^n)^{-1}. 
               \end{equation}
               
               We shall use induction to prove \eqref{1_1_19_3}. Observe that
               \begin{align*}
                  x^{-1} &= (x)^{-1} \\
                         &= (x^1)^{-1}. &[x^1 = x \text{ by definition}]
               \end{align*}
               That is, \eqref{1_1_19_3} holds for the base case: $n = 1$. For
               the inductive hypothesis, assume that \eqref{1_1_19_3} holds for
               some positive integer $k$. It follows that
               \begin{align*}
                  x^{-(k+1)} &= (x^{-1})^{k+1} &[\text{By definition}] \\
                             &= (x^{-1})^{k}(x^{-1})^1 &[\text{By definition}]\\
                             &= (x^k)^{-1}(x^{-1})^1
                                 &[\text{Inductive hypothesis}] \\
                             &= (x^k)^{-1}(x^1)^{-1} &[\text{Base case}] \\
                             &= (x^1x^k)^{-1} &[(rs)^{-1} = s^{-1}r^{-1}] \\
                             &= (x^{1+k})^{-1} &[1.1.19 (a)] \\
                             &= (x^{k+1})^{-1}.
                                 &[\text{Commutativity of addition}]
               \end{align*}
               We have thus shown that \eqref{1_1_19_3} also holds for $k+1$;
               hence, by induction, it also holds for each positive integer $n$;
               particularly, we have that
               $$(x^a)^{-1} = x^{-a}.$$ \qed
         \item We now must establish 
               \begin{equation} \label{1_1_19_5}
                  x^{m+n} = x^mx^n
               \end{equation}
               for all integers $m$ and $n$. Statement \eqref{1_1_19_5} is
               trivial whenever $m$ or $n$ is 0, and we already showed that it
               also holds for positive $m$ and $n$. So consider the following
               cases:
               
               \textbf{Case 1.} $m$ and $n$ are both negative. Let $c = -m$ and
               $d = -n$; that is, $c$ and $d$ are both positive. Hence
               
               \begin{align*}
                  x^{m+n} &= x^{-(c+d)} \\
                     &= (x^{-1})^{c+d} &[\text{By definition}] \\
                     &= (x^{-1})^c(x^{-1})^d &[\text{1.1.19 (a)}] \\
                     &= x^{-c}x^{-d} &[\text{By definition}] \\
                     &= x^mx^n.
               \end{align*}
               
               \textbf{Case 2.} $m$ and $n$ have different signs. Assume without
               loss of generality that $m$ is positive and $n$ is negative.
               \begin{quote}
               \textbf{Case 2(i)} $m + n > 0$. Since $m+n$ and $-n$ are both
               positive, it follows by 1.1.19(a) that
               \begin{equation} \label{1_1_19_6}
                  x^{m+n}x^{-n} = x^{m+n+(-n)} = x^m.
               \end{equation}
               1.1.19(b) tells us that $x^{-n} = (x^n)^{-1}$. Thus we will
               multiply \eqref{1_1_19_6} on the right by $x^n$ to get
               $$x^{m+n} = x^mx^n.$$
               \end{quote}
               
               \begin{quote}               
               \textbf{Case 2(ii)} $m + n < 0$. Let $c = -n$ and $d = -m$, so
               that $c + d > 0$. Then it follows that
               \begin{align*}
                  x^{m+n} &= x^{-(c+d)} \\
                     &= (x^{-1})^{c+d} &[\text{By definition}] \\
                     &= (x^{-1})^c(x^{-1})^d &[\text{Case 2(i)}]\\
                     &= x^{-c}x^{-d} \\
                     &= x^nx^m = x^mx^n.
               \end{align*}
               \end{quote}
               
               \begin{quote}               
               \textbf{Case 2(iii)} $m + n = 0$. That is
               \begin{align*}
                  x^{m+n} &= x^0 \\
                     &= 1 &[\text{By definition}] \\
                     &= x^m(x^m)^{-1} \\
                     &= x^mx^{-m} &[\text{1.1.19(b)}] \\
                     &= x^mx^n.
               \end{align*}
               \end{quote}
         
               Finally we want to prove that
               \begin{equation} \label{1_1_19_4}
                  (x^m)^n = (x^{mn}).
               \end{equation}
               for arbitrary integers $m$ and $n$. Statement \eqref{1_1_19_4}
               trivially holds whenever $m$ or $n$ is 0. We have already shown
               that \eqref{1_1_19_4} holds whenever $m$ and $n$ are both
               positive. So we shall investigate the remaining cases.

               \textbf{Case 1.} \textit{$m$ is positive and $n$ is negative}. 
               Let $c = -n$, a positive integer. So
               \begin{align*}
                  (x^m)^n &= (x^m)^{-c} \\
                          &= [(x^m)^{-1}]^c &[\text{By definition}] \\
                          &= (x^{-m})^c &[\text{1.1.19 (b)}] \\
                          &= [(x^{-1})^m]^c &[\text{By definition}] \\
                          &= (x^{-1})^{mc} &[\text{1.1.19 (a)}] \\
                          &= x^{(-mc)} &[\text{By definition}] \\
                          &= x^{mn}.
               \end{align*}

               \textbf{Case 2.} \textit{$m$ and $n$ are both negative}. Let 
               $c = -m$, a positive integer. Thus
               \begin{align*}
                  (x^m)^n &= (x^{-c})^n \\
                          &= [(x^{-1})^c]^n &[\text{By definition}] \\
                          &= (x^{-1})^{cn} &[\text{Case 1}] \\
                          &= (x^{-1})^{-(mn)} \\
                          &= [(x^{-1})^{-1}]^{mn}. &[\text{Definition}] \\
                          &= x^{mn}. &[\text{Proposition 1 (3)}]
               \end{align*}

               \textbf{Case 3.} \textit{$m$ is negative and $n$ is positive}.
               Let $c = -m$, a positive integer. Thus
               \begin{align*}
                  (x^m)^n &= [x^{-c}]^n \\
                          &= [(x^{-1})^c]^n &[\text{By definition}] \\
                          &= (x^{-1})^{cn} &[\text{Case 1}] \\
                          &= x^{-cn} &[\text{By definition}] \\
                          &= x^{mn}.
               \end{align*}

               Combining these results with part (a), we can conclude that
               \eqref{1_1_19_4} and \eqref{1_1_19_5} hold for all integers $m$
               and $n$. \qed
      \end{enumerate}
%%%%%%%%%%%%%%%%%%%%%%%%%%%%%%%%%%%%%1.2.3%%%%%%%%%%%%%%%%%%%%%%%%%%%%%%%%%%%%%%
   \item[1.2.3]   Use the generators and relations above to show that every
                  element of $D_{2n}$ which is not a power of $r$ has order 2.
                  Deduce that $D_{2n}$ is generated by the two elements $s$ and
                  $sr$, both of which have order 2.
                  
      \textbf{Proof.} Let $x \in D_{2n}$. Now suppose that $x$ is not an
      integral power of $r$. That is, $x$ is not one of the rotations. Thus
      we can write $x = sr^i$, for some $0 \le i < n$. Notice that that
      $|x| \le 2$ because
      \begin{align*}
         x^2 &= (sr^isr^i) \\
             &= (sr^ir^{-i}s) \\
             &= s1s \\
             &= s^2 = 1.
      \end{align*}
      From our discussion in class, we know that $sr^i \neq 1$ because
      $s \neq r^j$ for every integer $j$; that is, $|x| = 1$, and we conclude
      that $|x| = 2$. Recall that $s$ and $r$ generate $D_{2n}$. So in order to
      show that $s$ and $sr$ also generate $D_{2n}$, it suffices to show that
      $s$ and $r$ are in the set generated by $s$ and $sr$. This follows
      immediately since
      $$s = s^1(sr)^0 \text{ and } r = s^1(sr)^1.$$ \qed
%%%%%%%%%%%%%%%%%%%%%%%%%%%%%%%%%%%%%1.2.4%%%%%%%%%%%%%%%%%%%%%%%%%%%%%%%%%%%%%%
   \item[1.2.4]   If $n = 2k$ is even and $n \ge 4$, show that $z = r^k$ is an
                  element of order 2 which commutes with all elements of
                  $D_{2n}$. Show also that $z$ is the only nonidentity element
                  of $D_{2n}$ which commutes with all elements of $D_{2n}$. [cf.
                  Exercise 33 of Section 1.]
                  
      \textbf{Proof.} Let $r$ and $s$ be the generators of the dihedral group,
      $D_{2n}$. Suppose that $n = 2k$ for some positive integer $k \ge 2$.
      Let $z = r^k$. Since $k < n$ and since $|r| = n$, it must be
      the case that $|z| \neq 1$, so that $|z| \ge 2$. Now we have that
      $z^2 = r^kr^k = r^{2k} = r^n = 1$; i.e, $|z| = 2$. In particular, since
      $r^kr^k = 1$, it follows that $r^{-k} = r^k$. Since $z$ is an integral
      power of $r$, it must also commute with all elements that are integral 
      powers of $r$. From our discussion in class, we know that an element of
      $D_{2n}$ is either an integral power of $r$ or $s$ times an integral power
      of $r$. So we now want to show that $z$ commutes with the latter type.
      Thus consider $sr^i$, with $0 \le i < n$. Then we have that
      \begin{align*}
         z(sr^i) &= (r^k)sr^i = (r^ks)r^i  \\
               &= (sr^{-k})r^i  &[\text{Lemma 2}] \\
               &= (sr^k)r^i     &[r^k = r^{-k}] \\
               &= s(r^kr^i) = s(r^ir^k) = (sr^i)r^k = (sr^i)z.
      \end{align*}
      That is, $z$ commutes with all elements of $D_{2n}$. Now we want to show
      that $z$ is the only nonidentity element of $D_{2n}$ that commutes with
      every element of $D_{2n}$. So let $y \in D_{2n}$ be such an element (we
      already showed that at least one---$r^k$---exists). As previously stated,
      $y$ must be either of two forms, which we will now investigate:
      
      \textbf{Case 1.} \textit{$y = r^i$, with $1 \le i < n$.} (Note that we
      left out the identity since it has order 1.) Now we must particularly have 
      that $y$ commutes with $s$. Using Lemma 2, we have that
      $rs = sr^i = r^is$, so that $r^{-i} = r^i$. Multiply the preceding 
      equality by $r^i$ to obtain $r^{2i} = 1$. It follows by Lemma 1 that
      $n = 2k \mid 2i$, or equivalently, $k \mid i$, so that $i$ is a multiple
      of $k$. Since $i$ is positive and less than $n$, and since $k = n/2$, it 
      follows that the only multiple of $k$ in the interval $[1, n)$ is $k$.
      Thus $k = i$, and we have that $y = r^i = r^k = z$.
      
      \textbf{Case 2.} \textit{$y = sr^i$, with $0 \le i < n$.} Particularly, we
      must have that $y$ commutes with $r$. Then we have that
      \begin{align*}
         (sr^i)r &= r(sr^i) &[yr = ry] \\
                 &= (rs)r^i &[\text{Associativity}] \\
                 &= (sr^{-1})r^i &[D_{2n}\text{ presentation}] \\
                 &= s(r^{-1}r^i) &[\text{Associativity}] \\
                 &= s(r^ir^{-1}) &[\text{Powers of $r$ commute}] \\
                 &= (sr^i)r^{-1}, &[\text{Associativity}]
      \end{align*}
      so that $r = r^{-1}$ by left cancellation; thus $r^2 = 1$, a
      contradiction. We can now conclude that $r^k$ is the only nonidentity
      element of $D_{2n}$ that commutes with every element of $D_{2n}$. \qed
%%%%%%%%%%%%%%%%%%%%%%%%%%%%%%%%%%%%%1.2.17%%%%%%%%%%%%%%%%%%%%%%%%%%%%%%%%%%%%%
   \item[1.2.17]  Let $X_{2n}$ be the group whose presentation
                                  is displayed in (1.2).
                  \begin{enumerate}
                     \item Show that if $n = 3k$, then $X_{2n}$ has order 6, and
                           it has the same generators and relations as $D_6$
                           when $x$ is replaced by $r$ and $y$ by $s$.
                     \item Show that if $(3, n) = 1$, then $x$ satisfies the
                           additional relation: $x = 1$. In this case deduce
                           that $X_{2n}$ has order 2. [Use the facts that
                           $x^n = 1$ and $x^3 = 1$.]
                  \end{enumerate}

      \textbf{Proof.} Let $X_{2n} = \cyc{x, y :  x^n = y^2 = 1, \;xy = yx^2}$ be
      a presentation.

      \begin{enumerate}
         \item Suppose $n = 3k$. From page 27 of the textbook, we have the
               following:
               \begin{align*}
                  x &= xy^2 = (xy)y = (yx^2)y = (yx)(xy) = (yx)(yx^2) \\
                    &= y(xy)x^2 = y(yx^2)x^2 = y^2x^4 = x^4,
               \end{align*}
               so that $x^3 = 1$, by cancellation. Now multiply the equality
               $x^3 = 1$ by $x^{-1}$ to get $x^2 = x^{-1}$, and notice that the
               the relation $1 = x^n = x^{3k}$ can be deduced from the relation 
               $x^3 = 1$ by taking the $k$-th power of the latter. Hence
               $$X_{2n} = \cyc{x, y :  x^3 = y^2 = 1, \;xy = yx^{-1}}.$$
               We know that
               $$D_6 = \cyc{r, s :  r^3 = s^2 = 1, \;rs = sr^{-1}}$$
               and $r^n = r^{3k} = (r^3)^k = 1^k = 1$. That is, if we replace
               $x$ by $r$ and $y$ by $s$, then we see that $X_{2n}$ has the same 
               generators and relations as $D_6$. Now suppose we have other
               hidden relations, or collapsing, in the presentation of $X_{2n}$,
               then, since $D_6$ and $X_{2n}$ satisfy the same relations, the 
               same must be true for $D_6$. However, we know this to be false
               for $D_6$, so that it is also false for $X_{2n}$, and we conclude
               that $|X_{2n}| = |D_6| = 6$.
         \item Suppose $(3, n) = 1$. It follows by the Bezout's identity that
               there exist integers $p$ and $q$ such
               that $1 = 3p + nq$. So
               $x^1 = x^{3p+nq} = (x^3)^p(x^n)^q = 1^p1^q = 1$. Thus
               $X_{2n} = \cyc{y : y^2 = 1}$, so that $|X_{2n}| = 2$. 
      \end{enumerate} \qed
%%%%%%%%%%%%%%%%%%%%%%%%%%%%%%%%%%%%%1.3.11%%%%%%%%%%%%%%%%%%%%%%%%%%%%%%%%%%%%%
   \item[1.3.11]  Let $\sigma$ be the $m$-cycle (1 2 $\ldots$ $m$). Show that
                  $\sigma^i$ is also an $m$-cycle if and only if $i$ is
                  relatively prime to $m$.
                  
      \textbf{Proof.} ($\Leftarrow$) Suppose that $\gcd(i, m) = 1$. By
      definition, the elements, in order (and with possible repetition), of the
      first cycle in the cycle decomposition of $\sigma^i$ are
      $$1 = \sigma^{0i}(1), \sigma^{1i}(1), \sigma^{2i}(1), \ldots,
        \sigma^{(m-1)i}(1).$$
      Now if $\sigma^i$ is not an $m$-cycle, then at least two of the $m$
      elements above will be equal. So it suffices to show that all the $m$
      elements above are unique. Using 1.3.10, we can rewrite these $m$ elements
      as
      $$1+0i, 1+1i, 1+2i, \ldots, 1+(m-1)i$$
      where the numbers above are taken mod $m$. Now suppose that
      $$1 + pi \equiv 1 + qi \mbox{ (mod $m$)}$$
      for some $p, q \in \{0, 1, \ldots, m-1\}$. It follows that
      $pi \equiv qi$ (mod $m$). Since $i$ and $m$ are relatively prime, it
      follows by the Bezout's identity that there exist some integers $a$ and
      $b$ such that $ai + bm = 1$. That is, $ai \equiv 1$ (mod $m$). So if
      we multiply $pi \equiv qi$ (mod $m$) by $a$, we shall get $p \equiv q$
      (mod $m$). Since $0 \le p < m$ and $0 \le q < m$, it follows that $p = q$.
      Thus, all the $m$ elements are unique. Now observe that $\sigma^i$ cannot
      have more than one cycle in its cycle decomposition since the first cycle
      has exhausted all the possible $m$ elements. Thus
      $\sigma^i = (1\;\;1+i\;\;1+2i\;\;\ldots\;\;1+(m-1)i)$, an $m$-cycle.
      
      ($\Rightarrow$) Next suppose that $\sigma^i$ is an $m$-cycle. By 1.3.10,
      it follows that the order of $\sigma^i$ is $m$. Suppose to the contrary
      that $\gcd(i, m) > 1$. Thus
      $$\frac{m}{\gcd(i, m)} < m.$$
      Note that, by definition, both
      $$\frac{m}{\gcd(i, m)} \text{ and } \frac{i}{\gcd(i, m)}$$
      are positive integers. So
      \begin{align*}
         (\sigma^i)^{\frac{m}{\gcd(i,m)}} = (\sigma)^{\frac{im}{\gcd(i,m)}} =
         (\sigma^m)^{\frac{i}{\gcd(i,m)}} = 1^{\frac{i}{\gcd(i,m)}} = 1,
      \end{align*}
      so that $|\sigma^i| \le \frac{m}{\gcd(i, m)} < m$, contradicting the fact
      that $|\sigma^i| = m$. That is, we must have that $\gcd(i, m) = 1$, and
      the proof is done. \qed
     
%%%%%%%%%%%%%%%%%%%%%%%%%%%%%%%%%%%%%1.3.15%%%%%%%%%%%%%%%%%%%%%%%%%%%%%%%%%%%%%
   \item[1.3.15]  Prove that the order of an element in $S_n$ equals the least
                  common multiple of the lengths of the cycles in its cycle
                  decomposition.
                  [\textbf{Hint.} Use Exercises 1.3.10 and 1.1.24.]

      \textbf{Proof.} Let $\sigma \in S_n$. Consider the cycle decomposition of
      $\sigma$, say $\sigma = \sigma_1\cdots\sigma_m$. Let
      $r = \text{lcm}(|\sigma_1|, \ldots, |\sigma_m|)$, so that
      $|\sigma_i| \mid r$ (i.e there exists an integer $k_i$ such that
      $r = k_i \cdot |\sigma_i|$) for all $1 \le i \le m$. We can assume that
      $r > 1$ since the proof is trivial otherwise. Now
      \begin{align*}
         \sigma^r &= (\sigma_1\cdots\sigma_m)^r \\
               &= {\sigma_1}^r\cdots{\sigma_m}^r &[\text{Exercise 1.1.24}] \\
               &= ({\sigma_1}^{|\sigma_1|})^{k_1}\cdots
                  ({\sigma_m}^{|\sigma_m|})^{k_m} \\
               &= 1^{k_1}\cdots1^{k_m} = 1, &[\text{Exercise 1.3.10}]
      \end{align*}
      so that $|\sigma| \le r$. Now consider any positive integer $s < r$. Since
      $s$ is less than the least common multiple of the orders of the cycles in
      the cycle decomposition of $\sigma$, it follows that there exists some
      $j$, with $1 \le j \le m$,  such that $|\sigma_j| \nmid s$ (otherwise,
      all the orders will divide $s$, and by definition of $r$, we will have
      $r \le s$, a contradiction). Now suppose to the contrary that
      ${\sigma_j}^s = 1$, then it follows by Lemma 1 that
      $|\sigma_j| \mid s$, a contradiction. Thus ${\sigma_j}^s \neq 1$. So there
      exist two non-equal elements $a_1, a_2 \in \sigma_j$ such that
      ${\sigma_j}^s(a_1) = a_2$. Indeed, since $a_1$ and $a_2$ are only
      contained in $\sigma_j$, it follows that $\sigma^s(a_1) = a_2$. This says
      that $\sigma^s \neq 1$. So no positive power of $\sigma$ less than $r$ is
      the identity. We thus conclude that $|\sigma| = r$, as desired. \qed
%%%%%%%%%%%%%%%%%%%%%%%%%%%%%%%%%%%%%1.4.2%%%%%%%%%%%%%%%%%%%%%%%%%%%%%%%%%%%%%%
   \item[1.4.2]   Write out all the elements of $GL_2(\F_2)$ and compute the
                  order of each element.

      \textbf{Solution.}
      \begin{center}
         \begin{tabular}{@{}|c|c|@{}} \hline
            Matrix in $GL_2(\F_2)$           & Order  \\ \hline            
            $\left(\begin{tabular}{@{}cc@{}}
               1 & 0 \\
               0 & 1
            \end{tabular}\right)$            & 1      \\ \hline           
            $\left(\begin{tabular}{@{}cc@{}}
               1 & 0 \\
               1 & 1
            \end{tabular}\right)$            & 2      \\ \hline           
            $\left(\begin{tabular}{@{}cc@{}}
               1 & 1 \\
               0 & 1
            \end{tabular}\right)$            & 2      \\ \hline           
            $\left(\begin{tabular}{@{}cc@{}}
               1 & 1 \\
               1 & 0
            \end{tabular}\right)$            & 3      \\ \hline           
            $\left(\begin{tabular}{@{}cc@{}}
               0 & 1 \\
               1 & 0
            \end{tabular}\right)$            & 2      \\ \hline           
            $\left(\begin{tabular}{@{}cc@{}}
               0 & 1 \\
               1 & 1
            \end{tabular}\right)$            & 3      \\ \hline
         \end{tabular}
      \end{center}
%%%%%%%%%%%%%%%%%%%%%%%%%%%%%%%%%%%%%1.4.10%%%%%%%%%%%%%%%%%%%%%%%%%%%%%%%%%%%%%
   \item[1.4.10]  Let $G = \left\{\left(\begin{tabular}{@{}cc@{}}
                     $a$ & $b$ \\
                      0  & $c$
                  \end{tabular}\right) : a, b, c \in \R, a \neq 0, c \neq 0
                  \right\}$.

                  \begin{enumerate}
                     \item Compute the product of
                           $\left(\begin{tabular}{@{}cc@{}}
                              $a_1$ & $b_1$ \\
                              0  & $c_1$
                           \end{tabular}\right)$ and
                           $\left(\begin{tabular}{@{}cc@{}}
                              $a_2$ & $b_2$ \\
                              0  & $c_2$
                           \end{tabular}\right)$ to show that $G$ is closed under
                           matrix multiplication.
                     \item Find the matrix inverse of
                           $\left(\begin{tabular}{@{}cc@{}}
                              $a$ & $b$ \\
                              0  & $c$
                           \end{tabular}\right)$ and deduce that $G$ is closed 
                           under inverses.
                     \item Deduce that $G$ is a subgroup of $GL_2(\R)$.
                     \item Prove that the set of elements of $G$ whose two
                           diagonal entries are equal is also a subgroup of
                           $GL_2(\R)$.
                  \end{enumerate}

      \textbf{Solution.}

      \begin{enumerate}
         \item Let
               $$A = \left(\begin{tabular}{@{}cc@{}}
                     $a_1$ & $b_1$ \\
                     0  & $c_1$
               \end{tabular}\right) \text{ and }
                 B = \left(\begin{tabular}{@{}cc@{}}
                     $a_2$ & $b_2$ \\
                     0  & $c_2$
               \end{tabular}\right)$$
               be matrices in $G$. It follows that
               $$AB = \left(\begin{tabular}{@{}cc@{}}
                  $a_1a_2$ & $a_1b_2+b_1c_2$ \\
                  0  & $c_1c_2$
               \end{tabular}\right).
               $$
               By membership of $A$ and $B$ in $G$, we know that $a_1$, $c_1$,
               $a_2$, and $c_2$ are all nonzero real numbers. Hence $a_1a_2$ and
               $c_1c_2$ must also be nonzero real numbers; thus, $AB \in G$, and
               $G$ is therefore closed under multiplication.
         \item Now we want to find the inverse of an arbitrary matrix
               $$A = \left(\begin{tabular}{@{}cc@{}}
                  $a$ & $b$ \\
                  0  & $c$
               \end{tabular}\right)$$
               in $G$. Since $a \neq 0$ and $c \neq 0$, it follows that
               $ac \neq 0$, so that
               $$A^{-1} = \D\frac{1}{ac}
               \left(\begin{tabular}{@{}cr@{}}
                  $c$ & $-b$ \\
                   0  & $a$
               \end{tabular}\right) \in G;$$
               i.e. $G$ is closed under inverses.
         \item Since $G$ is nonempty, associative under matrix multiplication,
               has the $2 \times 2$ identity, and has an inverse for each of its
               matrices, it follows by definition that $G$ is a subgroup of
               $GL_2(\R)$.
         \item Consider the following subset of $G$:
               $$G' = \left\{\left(\begin{tabular}{@{}cc@{}}
                     $a$ & 0 \\
                      0  & $a$
                  \end{tabular}\right) : a \text{ is a nonzero real number}
               \right\}.$$
               By observation, we see that $G'$ contains the $2 \times 2$
               identity. Also we know that $G'$ is associative under matrix
               multiplication. Thus, to show that $G' \le G$, it suffices to
               show that $G'$ is closed under multiplication and inverses. So
               let 
               $$C = \left(\begin{tabular}{@{}cc@{}}
                     $c$ & 0 \\
                     0  & $c$
               \end{tabular}\right) \text{ and }
                 D = \left(\begin{tabular}{@{}cc@{}}
                     $d$ & 0 \\
                     0  & $d$
               \end{tabular}\right)$$
               be matrices in $G'$. By membership in $G'$, $c$ and $d$ are
               nonzero, so that $cd$ is also nonzero. Thus $G$ is closed under
               multiplication and inverses because
               $$CD = \left(\begin{tabular}{@{}cc@{}}
                     $cd$ & 0 \\
                     0  & $cd$
               \end{tabular}\right) \in G'$$
               and
               $$C^{-1} = \left(\begin{tabular}{@{}cc@{}}
                  $1/c$ & 0 \\
                  0  & $1/c$
               \end{tabular}\right) \in G',$$
               and we conclude that $G' \le G$.
      \end{enumerate}
\end{enumerate}
\end{document}
