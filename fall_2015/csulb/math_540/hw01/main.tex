\documentclass[9pt]{article}

\usepackage{amssymb}
\usepackage{amsmath}
\usepackage{amsfonts}
\usepackage{comment}
\usepackage{fancyhdr}
\usepackage{mathrsfs}
\usepackage{enumitem}
%\usepackage[retainorgcmds]{IEEEtrantools}

\everymath{\displaystyle}

\usepackage{tikz}

\voffset = -50pt
%\textheight = 700pt
\addtolength{\textwidth}{60pt}
\addtolength{\evensidemargin}{-30pt}
\addtolength{\oddsidemargin}{-30pt}
%\setlength{\headheight}{44pt}

\pagestyle{fancy}
\fancyhf{} % clear all fields
\fancyhead[R]{%
  \scshape
  \begin{tabular}[t]{@{}r@{}}
  MATH 540, Fall 2015\\Section 1 (9932)\\
  HW \#1, DUE: 2015, September 03
  \end{tabular}}
\fancyhead[L]{%
  \scshape
  \begin{tabular}[t]{@{}r@{}}
  JOSEPH OKONOBOH\\Mathematics\\Cal State Long Beach
  \end{tabular}}
\fancyfoot[C]{\thepage}

\newcommand{\qed}{\hfill \ensuremath{\Box}}


\newcommand*\circled[1]{\tikz[baseline=(char.base)]{
            \node[shape=circle,draw,inner sep=2pt] (char) {#1};}}


\newcommand{\cyc}[1]{\langle #1 \rangle}
\newcommand{\Z}{\mathbb{Z}}
\newcommand{\I}{\mathbb{I}}
\newcommand{\F}{\mathbb{F}}
\newcommand{\M}{\mathbb{M}}
\newcommand{\R}{\mathbb{R}}
\newcommand{\Q}{\mathbb{Q}}
\newcommand{\D}{\displaystyle}
%\setcounter{section}{-1}

\begin{document}
\begin{enumerate}
%%%%%%%%%%%%%%%%%%%%%%%%%%%%%%%%%%%%%%1.6%%%%%%%%%%%%%%%%%%%%%%%%%%%%%%%%%%%%%%%
   \item[1.1.6]   Determine which of the following sets are groups under
                  addition:
                  \begin{enumerate}
                     \item the set of rational numbers (including $0 = 0/1$) in
                           lowest terms whose denominators are odd.
                     \item the set of rational numbers (including $0 = 0/1$) in
                           lowest terms whose denominators are even.
                     \item the set of rational numbers of absolute value $< 1$.
                     \item the set of rational numbers of absolute value $\ge 1$
                           together with 0.
                     \item the set of rational numbers with denominators equal
                           to 1 or 2.
                     \item the set of rational numbers with denominators equal
                           to 1, 2, or 3.
                  \end{enumerate}

      \textbf{Solution.}

      \begin{enumerate}
         \item We claim that the set
               $$S = \left\{\frac{a}{b} \in \Q : a, b \in \Z, b \text{ is odd,} 
                        \text{ and } \gcd(a, b) = 1\right\},$$
               is a group under addition.

               \textbf{Proof.} First we want show that $S$ is closed under 
               addition. Notice that $S$ is nonempty since it contains
               $\frac{7}{5}$, so let $r, s \in S$. By definition of $S$, we have
               that $r = \frac{a_1}{b_1}$ and $s = \frac{a_2}{b_2}$ for some
               integers $a_1$ with $a_2$, odd integers $b_1$ and $b_2$, and
               $\gcd(a_1, b_1) = \gcd(a_2, b_2) = 1$.
               It follows that
               \begin{align*}
                  r + s &= \frac{a_1}{b_1} + \frac{a_2}{b_2} \\
                        &= \frac{a_1b_2 + a_2b_1}{b_1b_2}.
               \end{align*}

               Since $b_1$ and $b_2$ are both odd, it must necessarily be the 
               case that $b_1b_2$ is also odd. In order words, $b_1b_2$ contains 
               no factor of 2, so that if we reduce $r + s$ to its lowest term, 
               the denominator of this lowest term will still be odd. Hence
               $r + s \in S$, so that $S$ is closed under addition. To complete 
               the proof we must now show that $S$ satisfies the group axioms. 
               We observe that $0/1$ is the identity element in $S$. Also, it is 
               clear that for all $s \in S$, we have $-s \in S$, so that every 
               element of $S$ has an inverse under addition. Since
               $S \subseteq \Q$, and since $\Q$ is associative under addition, 
               it follows that $S$ is also associative under addition. Thus $S$ 
               satisfies the group axioms, so that $(S, +)$ is a group. \qed
         \item The set
               $$S = \left\{\frac{a}{b} \in \Q : a, b \in \Z, b
                        \text{ is nonzero and even,} \text{ and }
                        \gcd(a, b) = 1\right\},$$
               is not a group under addition because it is not closed. Indeed,
               we have $\frac{3}{14} \in S$, but
               $\frac{3}{14} + \frac{3}{14} = \frac{3}{7} \notin S$.
         \item The set
               $$S = \left\{\frac{a}{b} \in \Q : a, b \in \Z \text{ with }
                  b \neq 0, \text{ and }
                  \left|\frac{a}{b}\right| < 1\right\},$$
               is not a group under addition because it is not closed. For 
               example, we have $\frac{9}{10} \in S$, but
               $\frac{9}{10} + \frac{9}{10} = \frac{18}{10} \notin S$
               because $\left|\frac{18}{10}\right| > 1$.
         \item The set
               $$S = \left\{\frac{a}{b} \in \Q : a = 0 \text{ or }
                     \left|\frac{a}{b}\right| \ge 1\right\},$$
               is not a group under addition because it is not closed. For
               example, we have $-\frac{11}{10}, \frac{10}{10} \in S$, but
               $-\frac{11}{10} + \frac{10}{10} = -\frac{1}{10} \notin S$ because
               $\left|-\frac{1}{10}\right| < 1$.
         \item We claim that the set
               $$S = \left\{\frac{a}{b} \in \Q : a \in \Z \text{ and }
                   b = 1 \text{ or } b = 2\right\},$$
               is a group under addition.

               \textbf{Proof.} It is clear that 0 is the identity for $S$ under
               addition, that $S$ is associative under addition (because
               $S \subset \Q$ and $\Q$ is associative under addition), and that
               the inverse of an element in $S$ is its additive inverse in $\Q$.
               So to complete the proof, we need only show that $S$ is closed
               under addition. To that end, let
               $\frac{a_1}{b_1}, \frac{a_2}{b_2} \in S$.
               
               \textbf{Case 1.} $b_1 = b_2 = 1$. Then
               $$\frac{a_1}{b_1} + \frac{a_2}{b_2} = \frac{a_1+a_2}{1} \in S.$$
               
               \textbf{Case 2.} $b_1 = 1$ and $b_2 = 2$. Then
               $$\frac{a_1}{b_1} + \frac{a_2}{b_2} = \frac{2a_1+a_2}{2} \in S.$$
               
               \textbf{Case 3.} $b_2 = 1$ and $b_1 = 2$. Use Case 2, with the
               roles of $b_2$ and $b_1$ interchanged.
               
               \textbf{Case 4.} $b_1 = b_2 = 2$. Then
               $$\frac{a_1}{b_1} + \frac{a_2}{b_2} = \frac{a_1+a_2}{2} \in S.$$
         \item The set
               $$S = \left\{\frac{a}{b} \in \Q : a \in \Z \text{ and }
                   b \in \{1, 2, 3\} \right\},$$
               is not a group under addition because it is not closed. Indeed,
               for $\frac{1}{2}, \frac{1}{3} \in S$, we have $\frac{1}{2} +
               \frac{1}{3} = \frac{5}{6} \notin S$.
      \end{enumerate}
%%%%%%%%%%%%%%%%%%%%%%%%%%%%%%%%%%%%%%1.19%%%%%%%%%%%%%%%%%%%%%%%%%%%%%%%%%%%%%%
   \item[1.1.19]  Let $x \in G$ and let $a, b \in \Z^+$.
                  \begin{enumerate}
                     \item Prove that $x^{a+b} = x^ax^b$.
                     \item Prove that $(x^a)^b = x^{ab}$.
                     \item Prove that $(x^a)^{-1} = x^{-a}$.
                     \item Establish part (a) for arbitrary integers $a$ and $b$
                           (positive, negative or zero).
                     \item Establish part (b) for arbitrary integers $a$ and $b$
                           (positive, negative or zero).
                  \end{enumerate}
               
      \textbf{Proof.} Fix positive integers $a$ and $b$. We shall be making use
      of the following definitions
      $$x^1 = x \text{ and, for each $m \in \Z^+$, } x^{m+1} = x^mx^1,$$
      and
      $$\text{for each } p \in \Z^+, x^{-p} = (x^{-1})^p.$$
      
      \begin{enumerate}
         \item Now we want to show that
               \begin{equation} \label{1_1_19_1}
                  x^{a+n} = x^ax^n 
               \end{equation}
               holds for each positive integer $n$. So we shall proceed by
               induction. Statement \eqref{1_1_19_1} holds for the base
               case ($n = 1$) by definition. So assume that \eqref{1_1_19_1}
               holds for some positive integer $k$. Thus it follows that
               \begin{align*}
                  x^{a+(k+1)} &=x^{(a+k)+1}&[\text{Associativity of addition}]\\
                     &= x^{a+k}x^1  &[\text{Definition}] \\
                     &= (x^ax^k)x^1 &[\text{Inductive hypothesis}] \\
                     &= x^a(x^kx^1) &[\text{Associativity of group operation}]\\
                     &= x^ax^{k+1}.  &[\text{Definition}]
               \end{align*}
               Hence \eqref{1_1_19_1} holds for each positive integer $n$.
               Particularly we have that $x^{a+b} = x^ax^b$. \\ \mbox{ } \qed
         \item Next, we shall show by induction that
               \begin{equation} \label{1_1_19_2}
                  (x^a)^n = x^{an} 
               \end{equation}
               for each positive integer $n$. Statement \eqref{1_1_19_2} holds
               trivially whenever $n = 1$, the base case. For our inductive
               hypothesis, we shall now assume that \eqref{1_1_19_2} holds for a
               positive integer $k$. So
               \begin{align*}
                  (x^a)^{k+1} &= (x^a)^k(x^a)^1 &[\text{Definition}] \\
                     &= (x^a)^kx^a &[\text{Base case}] \\
                     &= x^{ak}x^a &[\text{Inductive hypothesis}] \\
                     &= x^{ak+a} &[1.1.19 (a)] \\
                     &= x^{a(k+1)}.
               \end{align*}
               Hence \eqref{1_1_19_2} holds for each positive integer $n$.
               Particularly we have that $(x^a)^b = x^{ab}$. \\ \mbox{ } \qed
         \item Now consider the statement
               \begin{equation} \label{1_1_19_3}
                  x^{-n} = (x^n)^{-1}. 
               \end{equation}
               Observe that
               \begin{align*}
                  x^{-1} &= (x)^{-1} \\
                         &= (x^1)^{-1}. &[x^1 = x \text{ by definition}]
               \end{align*}
               That is, \eqref{1_1_19_3} holds for the base case, $n = 1$. For
               the inductive hypothesis, assume that \eqref{1_1_19_3} holds for
               some positive integer $k$. It follows that
               \begin{align*}
                  x^{-(k+1)} &= (x^{-1})^{k+1} &[\text{By definition}] \\
                             &= (x^{-1})^{k}(x^{-1})^1 &[\text{By definition}]\\
                             &= (x^k)^{-1}(x^{-1})^1
                                 &[\text{Inductive hypothesis}] \\
                             &= (x^k)^{-1}(x^1)^{-1} &[\text{Base case}] \\
                             &= (x^1x^k)^{-1} &[(rs)^{-1} = s^{-1}r^{-1}] \\
                             &= (x^{1+k})^{-1} &[1.1.19 (a)] \\
                             &= (x^{k+1})^{-1}.
                                 &[\text{Commutativity of addition}]
               \end{align*}
               We have thus shown that \eqref{1_1_19_3} for each positive
               integer $n$; particularly, we have that
               $$(x^a)^{-1} = x^{-a}.$$
         \item Now suppose that $a$ is an integer and $b$ is a positive integer.
               We shall induct on $b$ to show that
               \begin{equation}
                  x^{a+b} = x^ax^b. \label{1_1_19_11}
               \end{equation}
               By Lemma 1.1.1, \eqref{1_1_19_11} holds if $b$ equals 1. So assume
               that it also holds for some positive integer $k$. We now have
               that
               \begin{align*}
                  x^ax^{k+1} &= x^ax^kx^1 &[\text{Lemma 1.1.1}] \\
                             &= (x^ax^k)x^1 \\
                             &= x^{a+k}x^1 &[\text{Inductive hypothesis}] \\
                             &= x^{(a+k)+1} &[\text{Lemma 1.3.2}] \\
                             &= x^{a+(k+1)}, &[\text{Associativity of addition}]
               \end{align*}
               so that \eqref{1_1_19_11} holds for $k + 1$, and hence, by the 
               Principle of Mathematical Induction, it holds for each positive
               integer $n$. \\

               If $a$ is 0 or $b$ is 0, then Lemma 1.1.1 tells us that
               \eqref{1_1_19_11} holds, so the only remaining possibility is $a$ 
               and $b$ are negative.\footnote{If $a$ is positive and $b$ is
               negative, then interchange the roles of $a$ and $b$ in the 
               induction proof.} Now suppose that $a$ and $b$ are negative.
               Hence
               \begin{align*}
                  x^ax^b &= x^{-(-a)}x^{-(-b)} \\
                     &= (x^{-1})^{-a}(x^{-1})^{-b} &[\text{Definition}] \\
                     &= (x^{-1})^{(-a + (-b))} &[\text{Part (a)}] \\
                     &= x^{-(-a + (-b))} &[\text{Definition}] \\
                     &= x^{a+b}.
               \end{align*}

               Combining this result with part (a), we thus shown that
               \eqref{1_1_19_11} holds for all integers $a$ and $b$. \qed
         \item It is clear that part (b) holds if $a$ is 0 or $b$ is 0, so let
               us complete the proof for arbritrary integers $a$ and $b$.

               \textbf{Case 1:} \textit{$a$ is positive and $b$ is negative}. 
               Hence
               \begin{align*}
                  (x^a)^b &= (x^a)^{-(-b)} \\
                          &= [(x^a)^{-1}]^{-b} &[\text{Definition}] \\
                          &= (x^{-a})^{-b} &[\text{Part (c)}] \\
                          &= [(x^{-1})^a]^{-b} &[\text{Definition}] \\
                          &= (x^{-1})^{-ab} &[\text{Part (b)}] \\
                          &= x^{-(-ab)} &[\text{Definition}] \\
                          &= x^{ab}.
               \end{align*}

               \textbf{Case 2:} \textit{$a$ and $b$ are negative}. Thus
               \begin{align*}
                  (x^a)^b &= [x^{-(-a)}]^b \\
                          &= [(x^{-1})^{-a}]^b &[\text{Definition}] \\
                          &= (x^{-1})^{-ab} &[\text{Case 1}] \\
                          &= [(x^{-1})^{-1}]^{ab}. &[\text{Definition}] \\
                          &= x^{ab}. &[\text{Proposition 1 (3)}]
               \end{align*}

               \textbf{Case 3:} \textit{$a$ is negative and $b$ is positive}. 
               Thus
               \begin{align*}
                  (x^a)^b &= [x^{-(-a)}]^b \\
                          &= [(x^{-1})^{-a}]^b &[\text{Definition}] \\
                          &= (x^{-1})^{-ab} &[\text{Case 1}] \\
                          &= x^{-(-ab)} &[\text{Definition}] \\
                          &= x^{ab}.
               \end{align*}

               Combining these results with part (a), we can conclude that
               $(x^a)^b = x^{ab}$ holds for all integers $a$ and $b$ and
               $x \in G$. \qed
      \end{enumerate}
%%%%%%%%%%%%%%%%%%%%%%%%%%%%%%%%%%%%%1.2.3%%%%%%%%%%%%%%%%%%%%%%%%%%%%%%%%%%%%%%
   \item[1.2.3]   Use the generators and relations above to show that every
                  element of $D_{2n}$ which is not a power of $r$ has order 2.
                  Deduce that $D_{2n}$ is generated by the two elements $s$ and
                  $sr$, both of which have order 2.
                  
      \textbf{Proof.} Suppose that $x \in D_{2n}$ such that $x$ is not an
      integral power of $r$. Thus $x = sr^i$, with $0 \le i < n$ so that
      \begin{align*}
         x^2 &= (sr^isr^i) \\
             &= (sr^ir^{-i}s) &[\text{Lemma 1.2.1}] \\
             &= s1s \\
             &= s^2 = 1.
      \end{align*}
      That is, $|x| \le 2$. Now $sr^i \neq 1$ because $s \neq r^j$ for any
      integer $j$, so that we can conclude that $|x| = 2$. Recall: for
      $w \in D_{2n}$, there exist integers $x$ and $y$, such that $w = s^xr^y$.
      That is, $w = s^x(ssr)^y$. This says that $D_{2n}$ is generated by $s$ and
      $sr$. \qed
%%%%%%%%%%%%%%%%%%%%%%%%%%%%%%%%%%%%%1.2.4%%%%%%%%%%%%%%%%%%%%%%%%%%%%%%%%%%%%%%
   \item[1.2.4]   If $n = 2k$ is even and $n \ge 4$, show that $z = r^k$ is an
                  element of order 2 which commutes with all elements of
                  $D_{2n}$. Show also that $z$ is the only nonidentity element
                  of $D_{2n}$ which commutes with all elements of $D_{2n}$. [cf.
                  Exercise 33 of Section 1.]
                  
      \textbf{Proof.} Consider $D_{2n}$ where $n = 2k$ for some positive integer
      $k \ge 2$. Let $z = r^k$. Since $k < n$ and since $|r| = n$, it must be
      the case that $|z| \neq 1$, so that $|z| \ge 2$. Now we have that
      $z^2 = r^kr^k = r^{2k} = r^n = 1$; i.e, $|z| = 2$. In particular, since
      $r^kr^k = 1$, it follows that $r^{-k} = r^k$. It is clear that $z$
      commutes with all elements that are integral powers of $r$. So consider
      $sr^i$, with $0 \le i < n$. Then we have that
      \begin{align*}
         zsr^i &= r^ksr^i \\
               &= sr^{-k}r^i  &[\text{Lemma 1.2.1}] \\
               &= sr^kr^i     &[r^k = r^{-k}] \\
               &= sr^ir^k = sr^iz.
      \end{align*}
      That is $z$ commutes with all elements of $D_{2n}$. Now we want to show
      that $z$ is the only nonidentity element of $D_{2n}$ that commutes with
      every element of $D_{2n}$. So let $y \in D_{2n}$ be such an element (we
      already showed that at least one, $r^k$, exists).
      
      \textbf{Case 1:} \textit{$y = r^i$, with $1 \le i < n$.} Particularly, we
      must have that $y$ commutes with $s$. Using Lemma 1.2.1, we have that
      $r^{-i}s = sr^i = r^is$, so that $r^{-i} = r^i$. Then we must have that
      $r^{2i} = 1$, so that by Exercise 1.1.33, we have $i = k$.
      
      \textbf{Case 2:} \textit{$y = sr^i$, with $0 \le i < n$.} Particularly, we
      must have that $y$ commutes with $r$. Then we have that
      \begin{align*}
         (sr^i)r &= r(sr^i) &[yr = ry] \\
                 &= (rs)r^i &[\text{Associativity}] \\
                 &= (sr^{-1})r^i &[D_{2n}\text{ presentation}] \\
                 &= s(r^{-1}r^i) &[\text{Associativity}] \\
                 &= s(r^ir^{-1}) &[\text{Powers of $r$ commute}] \\
                 &= (sr^i)r^{-1}, &[\text{Associativity}]
      \end{align*}
      so that $r = r^{-1}$ by left cancellation; thus $r^2 = 1$, a
      contradiction. We can now conclude that $r^k$ is the only nonidentity
      element of $D_{2n}$ that commutes with every element of $D_{2n}$. \qed
%%%%%%%%%%%%%%%%%%%%%%%%%%%%%%%%%%%%%1.2.17%%%%%%%%%%%%%%%%%%%%%%%%%%%%%%%%%%%%%
   \item[1.2.17]  Let $X_{2n}$ be the group whose presentation
                                  is displayed in (1.2).
                  \begin{enumerate}
                     \item Show that if $n = 3k$, then $X_{2n}$ has order 6, and
                           it has the same generators and relations as $D_6$
                           when $x$ is replaced by $r$ and $y$ by $s$.
                     \item Show that if $(3, n) = 1$, then $x$ satisfies the
                           additional relation: $x = 1$. In this case deduce
                           that $X_{2n}$ has order 2. [Use the facts that
                           $x^n = 1$ and $x^3 = 1$.]
                  \end{enumerate}

      \textbf{Proof.} Given $X_{2n} = \cyc{x, y :  x^n = y^2 = 1, \;xy = yx^2}$.
      Also from the discussion on Page 27 of the textbook, we have $x^3 = 1$.

      \begin{enumerate}
         \item Let $n = 3k$. Notice that we can deduce the relation
               $x^n = x^{3k} = 1$ from the relation $x^3 = 1$. Thus
               $$X_{2n} = \cyc{x, y :  x^3 = y^2 = 1, \;xy = yx^2}.$$
               Since no other collapsing is possible it follows that
               $|X_{2n}| = |x| \cdot |y| = 3 \cdot 2 = 6$. Since $x^3 = 1$, it
               follows that $x^2 = x^{-1}$. Thus the relation $xy = yx^2$ is
               equivalent to $xy = yx^{-1}$. If we replace $x$ by $r$ and $y$ by
               $s$, then we see that $X_{2n}$ has the same generators and
               relations as $D_6$.
         \item Suppose $(3, n) = 1$. Thus there exist integers $p$ and $q$ such
               that $1 = 3p + nq$. So
               $x^1 = x^{3p+nq} = (x^3)^p(x^n)^q = 1^p1^q = 1$. Thus
               $X_{2n} = \cyc{y : y^2 = 1}$, so that $|X_{2n}| = 2$.
      \end{enumerate} \qed
%%%%%%%%%%%%%%%%%%%%%%%%%%%%%%%%%%%%%1.3.11%%%%%%%%%%%%%%%%%%%%%%%%%%%%%%%%%%%%%
   \item[1.3.11]  Let $\sigma$ be the $m$-cycle (1 2 $\ldots$ $m$). Show that
                  $\sigma^i$ is also an $m$-cycle if and only if $i$ is
                  relatively prime to $m$.
%%%%%%%%%%%%%%%%%%%%%%%%%%%%%%%%%%%%%1.3.15%%%%%%%%%%%%%%%%%%%%%%%%%%%%%%%%%%%%%
   \item[1.3.15]  Prove that the order of an element in $S_n$ equals the least
                  common multiple of the lengths of the cycles in its cycle
                  decomposition.
                  [\textbf{Hint.} Use Exercises 1.3.10 and 1.1.24.]

      \textbf{Proof.} Let $\sigma \in S_n$. Consider the cycle decomposition of
      $\sigma$, say $\sigma = \sigma_1\cdots\sigma_m$. Let
      $r = \text{lcm}(|\sigma_1|, \ldots, |\sigma_m|)$, so that
      $|\sigma_i| \mid r$ (i.e there exists an integer $k_i$ such that
      $r = k_i \cdot |\sigma_i|$) for all $1 \le i \le m$. We can assume that
      $r > 1$ since the proof is trivial otherwise. Now
      \begin{align*}
         \sigma^r &= (\sigma_1\cdots\sigma_m)^r \\
               &= {\sigma_1}^r\cdots{\sigma_m}^r &[\text{Exercise 1.1.24}] \\
               &= ({\sigma_1}^{|\sigma_1|})^{k_1}\cdots
                  ({\sigma_m}^{|\sigma_m|})^{k_m} \\
               &= 1^{k_1}\cdots1^{k_m} = 1, &[\text{Exercise 1.3.10}]
      \end{align*}
      so that $|\sigma| \le r$. Now consider any positive integer $s < r$. Since
      $s$ is less than the least common multiple of the lengths of the cycles in
      the cycle decomposition of $\sigma$, it follows that $|\sigma| \nmid j$
      for some $1 \le j \le m$, so that ${\sigma_j}^s \neq 1$. Thus
      $\sigma^s \neq 1$, and we can conclude that $|\sigma| = r$. \qed
%%%%%%%%%%%%%%%%%%%%%%%%%%%%%%%%%%%%%1.4.2%%%%%%%%%%%%%%%%%%%%%%%%%%%%%%%%%%%%%%
   \item[1.4.2]   Write out all the elements of $GL_2(\F_2)$ and compute the
                  order of each element.

      \textbf{Solution.}
      \begin{center}
         \begin{tabular}{@{}|c|c|@{}} \hline
            Matrix in $GL_2(\F_2)$           & Order  \\ \hline            
            $\left(\begin{tabular}{@{}cc@{}}
               1 & 0 \\
               0 & 1
            \end{tabular}\right)$            & 1      \\ \hline           
            $\left(\begin{tabular}{@{}cc@{}}
               1 & 0 \\
               1 & 1
            \end{tabular}\right)$            & 2      \\ \hline           
            $\left(\begin{tabular}{@{}cc@{}}
               1 & 1 \\
               0 & 1
            \end{tabular}\right)$            & 2      \\ \hline           
            $\left(\begin{tabular}{@{}cc@{}}
               1 & 1 \\
               1 & 0
            \end{tabular}\right)$            & 3      \\ \hline           
            $\left(\begin{tabular}{@{}cc@{}}
               0 & 1 \\
               1 & 0
            \end{tabular}\right)$            & 2      \\ \hline           
            $\left(\begin{tabular}{@{}cc@{}}
               0 & 1 \\
               1 & 1
            \end{tabular}\right)$            & 3      \\ \hline
         \end{tabular}
      \end{center}
%%%%%%%%%%%%%%%%%%%%%%%%%%%%%%%%%%%%%1.4.10%%%%%%%%%%%%%%%%%%%%%%%%%%%%%%%%%%%%%
   \item[1.4.10]  Let $G = \left\{\left(\begin{tabular}{@{}cc@{}}
                     $a$ & $b$ \\
                      0  & $c$
                  \end{tabular}\right) : a, b, c \in \R, a \neq 0, c \neq 0
                  \right\}$.

                  \begin{enumerate}
                     \item Compute the product of
                           $\left(\begin{tabular}{@{}cc@{}}
                              $a_1$ & $b_1$ \\
                              0  & $c_1$
                           \end{tabular}\right)$ and
                           $\left(\begin{tabular}{@{}cc@{}}
                              $a_2$ & $b_2$ \\
                              0  & $c_2$
                           \end{tabular}\right)$ to show that $G$ is closed under
                           matrix multiplication.
                     \item Find the matrix inverse of
                           $\left(\begin{tabular}{@{}cc@{}}
                              $a$ & $b$ \\
                              0  & $c$
                           \end{tabular}\right)$ and deduce that $G$ is closed 
                           under inverses.
                     \item Deduce that $G$ is a subgroup of $GL_2(\R)$.
                     \item Prove that the set of elements of $G$ whose two
                           diagonal entries are equal is also a subgroup of
                           $GL_2(\R)$.
                  \end{enumerate}

      \textbf{Solution.}

      \begin{enumerate}
         \item It follows immediately that $G$ is closed under matrix
               multiplication because
               $$\left(\begin{tabular}{@{}cc@{}}
                     $a_1$ & $b_1$ \\
                     0  & $c_1$
                  \end{tabular}\right)\left(\begin{tabular}{@{}cc@{}}
                     $a_2$ & $b_2$ \\
                     0  & $c_2$
                  \end{tabular}\right) = \left(\begin{tabular}{@{}cc@{}}
                     $a_1a_2$ & $a_1b_2+b_1c_2$ \\
                     0  & $c_1c_2$
                  \end{tabular}\right) \in G.$$
         \item Since $a \neq 0$ and $c \neq 0$, it follows that $ac \neq 0$
               so that $\left(\begin{tabular}{@{}cc@{}}
                  $a$ & $b$ \\
                   0  & $c$
               \end{tabular}\right)^{-1} = \D\frac{1}{ac}
               \left(\begin{tabular}{@{}cr@{}}
                  $c$ & $-b$ \\
                   0  & $a$
               \end{tabular}\right) \in G$; i.e. $G$ is closed under inverses.
         \item Since $G$ is nonempty, it follows by (a) and (b) that $G$ is a
               subgroup of $GL_2(\R)$.
         \item Proof is identical to what we did in (a)$-$(c).
      \end{enumerate}
\end{enumerate}
\end{document}
