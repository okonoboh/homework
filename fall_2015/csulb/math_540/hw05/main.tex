\documentclass[9pt]{article}

\usepackage{amssymb}
\usepackage{amsmath}
\usepackage{amsfonts}
\usepackage{comment}
\usepackage{fancyhdr}
\usepackage{mathrsfs}
\usepackage{enumitem}
%\usepackage[retainorgcmds]{IEEEtrantools}

\everymath{\displaystyle}

\usepackage{tikz}

\voffset = -50pt
%\textheight = 700pt
\addtolength{\textwidth}{60pt}
\addtolength{\evensidemargin}{-30pt}
\addtolength{\oddsidemargin}{-30pt}
%\setlength{\headheight}{44pt}

\pagestyle{fancy}
\fancyhf{} % clear all fields
\fancyhead[R]{%
  \scshape
  \begin{tabular}[t]{@{}r@{}}
  MATH 540, Fall 2015\\Section 1 (9932)\\
  HW \#5, DUE: 2015, October 01
  \end{tabular}}
\fancyhead[L]{%
  \scshape
  \begin{tabular}[t]{@{}r@{}}
  JOSEPH OKONOBOH\\Mathematics\\Cal State Long Beach
  \end{tabular}}
\fancyfoot[C]{\thepage}

\newcommand{\qed}{\hfill \ensuremath{\Box}}


\newcommand*\circled[1]{\tikz[baseline=(char.base)]{
            \node[shape=circle,draw,inner sep=2pt] (char) {#1};}}


\newcommand{\cyc}[1]{\langle #1 \rangle}
\newcommand{\Z}{\mathbb{Z}}
\newcommand{\C}{\mathbb{C}}
\newcommand{\F}{\mathbb{F}}
\newcommand{\M}{\mathbb{M}}
\newcommand{\R}{\mathbb{R}}
\newcommand{\Q}{\mathbb{Q}}
\everymath{\displaystyle}
\newcommand{\CYC}[1]{\left\langle #1 \right\rangle}
%\setcounter{section}{-1}

\begin{document}
\begin{enumerate}
   \item[]        \textbf{Lemma 1.} \textit{If $G/N$ is a quotient group, then
                  $(gN)^\alpha = g^\alpha N$ for all $\alpha \in \Z$}.
                  
      \textbf{Proof.} Let $G/N$ be a quotient group. First we shall use
      induction on $\alpha$ to show that: if $g \in G$, then
      \begin{equation} \label{1_1}
         (gN)^\alpha = g^\alpha N
      \end{equation}
      for all $\alpha \in \Z^+$. Let $g \in G$. For the base case, we have
      $\alpha = 1$, so it follows trivially that
      $$(gN)^1 = gN = g^1N.$$
      That is, \eqref{1_1} holds whenever $\alpha$ is 1. For the inductive
      hypothesis, assume that \eqref{1_1} holds for some positive integer $k$.
      Thus
      \begin{align*}
         (gN)^{k+1} &= (gN)^k(gN)^1 &[\text{Definition of }(gN)^{k+1}] \\
            &= (g^kN)(gN) &[\text{Inductive hypothesis}] \\
            &= g^kgN &[\text{Proposition 5 (1)}] \\
            &= g^{k+1}N, &[\text{Definition of }g^{k+1}]
      \end{align*}
      so that \eqref{1_1} also holds for $k + 1$. Hence, by mathematical
      induction, \eqref{1_1} holds for every positive integer $\alpha$. Now let
      $m$ be a negative integer, so that $m = -n$ for some positive integer $n$.
      Thus
      \begin{align*}
         (gN)^m = (gN)^{-n} &= ((gN)^{-1})^n &[\text{Definition of }(gN)^{-n}]\\
            &= (g^{-1}N)^n &[\text{Proposition 5 (2)}] \\
            &= (g^{-1})^nN &[\text{Replace $g$ with $g^{-1}$ in }\eqref{1_1}] \\
            &= g^{-n}N &[\text{Definition of }g^{-n}] \\
            &= g^mN,
      \end{align*}
      so that \eqref{1_1} also holds for every negative integer $\alpha$.
      Finally $(g^N)^0 = N = 1N = g^0N$. That is, \eqref{1_1} holds for every
      integer $\alpha$. \qed
   \item[]        \textbf{Lemma 2.} \textit{If $g$ and $a$ are elements of a
                  group $G$, then $(gag^{-1})^n = ga^ng^{-1}$, for all
                  $n \in \Z$.}
                  
                  \textbf{Proof.} Let $G$ be a group and $g, a \in G$. For
                  an integer $n$, consider the statement
                  \begin{equation} \label{1_2}
                     (gag^{-1})^n = ga^ng^{-1}.
                  \end{equation}
                  
                  First we shall show that \eqref{1_2} holds for every positive
                  integer $n$. Since
                  $$(gag^{-1})^1 = gag^{-1} = ga^1g^{-1},$$
                  it follows that \eqref{1_2} holds whenever $n$ is 1. So, for
                  the inductive hypothesis, assume that \eqref{1_2} holds for
                  some positive integer $j$. It follows that
                  \begin{align*}
                     (gag^{-1})^{j+1} &= (gag^{-1})^j(gag^{-1})^1
                           &[\text{Definition of }(gag^{-1})^{j+1}] \\
                        &= (ga^jg^{-1})(gag^{-1})
                           &[\text{Inductive hypothesis}] \\
                        &= ga^j(g^{-1}g)ag^{-1} \\
                        &= ga^jag^{-1} \\
                        &= ga^{j+1}g^{-1}. &[\text{Definition of }a^{j+1}]
                  \end{align*}
                  
                  That is, \eqref{1_2} holds for $j+1$, and so we conclude by
                  mathematical induction that it holds for every positive
                  integer $n$. Now let $r$ be a negative integer, so that
                  $r = -s$, for some positive integer $s$. Thus
                  \begin{align*}
                     (gag^{-1})^r = (gag^{-1})^{-s} &= ((gag^{-1})^{-1})^s
                        &[\text{Definition of }(gag^{-1})^{-s}] \\
                        &= (g^{-1})^{-1}a^{-1}g^{-1})^s \\
                        &= (ga^{-1}g^{-1})^s \\
                        &= g(a^{-1})^sg^{-1} &[\eqref{1_2}] \\
                        &= ga^{-s}g^{-1} &[\text{Definition of }a^{-s}] \\
                        &= ga^rg^{-1},
                  \end{align*}
                  so that \eqref{1_2} also holds for every negative integer $n$.
                  Finally
                  $$(gag^{-1})^0 = 1 = gg^{-1} = g1g^{-1} = ga^0g^{-1}.$$
                  Hence \eqref{1_2} holds for every integer $n$. \qed
%%%%%%%%%%%%%%%%%%%%%%%%%%%%%%%%%%%%%3.1.02%%%%%%%%%%%%%%%%%%%%%%%%%%%%%%%%%%%%%
   \item[3.1.2]   Let $\varphi : G \rightarrow H$ be a homomorphism of groups 
                  with kernel $K$ and let $a, b \in \varphi(G)$. Let $X \in G/K$ 
                  be the  fiber above $a$ and let $Y$ be the fiber above $b$, 
                  i.e., $X = \varphi^{-1}(a)$, $Y = \varphi^{-1}(b)$. Fix an 
                  element $u$ of $X$ (so $\varphi(u) = a$). Prove that if
                  $XY = Z$ in the quotient group $G/K$ and $w$ is any member of 
                  $Z$, then there is some $v \in Y$ such that $uv = w$.
                  [Show $u^{-1}w \in Y$.]
                  
      \textbf{Proof.} Suppose $XY = Z$ in the quotient group $G/K$. Then it
      follows by definition that $Z$ is the fiber over $ab$. Let $w \in Z$.
      That is, $\varphi(w) = ab = \varphi(u)b$. Pre-multiply the equality
      $\varphi(w) = \varphi(u)b$ by $\varphi(u)^{-1}$ to get
      $\varphi(u)^{-1}\varphi(w) = b$. Use Proposition 1 (2) and the fact that
      $\varphi$ is a homomorphism to arrive at $b = \varphi(u^{-1}w)$. The
      preceding equality tells us that $u^{-1}w$ is in the fiber of $b$, so that
      $u^{-1}w \in Y$. Let $v = u^{-1}w$ and conclude that $uv = w$, as desired.
      \qed
%%%%%%%%%%%%%%%%%%%%%%%%%%%%%%%%%%%%%3.1.09%%%%%%%%%%%%%%%%%%%%%%%%%%%%%%%%%%%%%
   \item[3.1.9]   Define $\varphi : \C^\times \rightarrow \R^\times$ by
                  $\varphi(a + bi) = a^2 + b^2$. Prove that $\varphi$ is a
                  homomorphism and find the image of $\varphi$. Describe the
                  kernel and the fibers of $\varphi$ geometrically (as subsets
                  of the plane).
                  
      \textbf{Proof.} Let $z_1$ and $z_2$ be arbitrary elements in $\C^\times$.
      So there exist real numbers $a_1$, $b_1$, $a_2$, and $b_2$ such that
      $z_1 = a_1 + b_1i$ and $z_2 = a_2 + b_2i$, where at least one of $a_1$ and
      $b_1$ is nonzero and at least one of $a_2$ and $b_2$ is nonzero. The map
      $\varphi$ is a homomorphism because
      \begin{align*}
         \varphi(z_1z_2) &= \varphi((a_1+b_1i)(a_2+b_2i)) \\
            &= \varphi(a_1a_2 - b_1b_2 + (a_1b_2 + a_2b_1)i) \\
            &= (a_1a_2 - b_1b_2)^2 + (a_1b_2 + a_2b_1)^2 \\
            &= (a_1a_2)^2 + (b_1b_2)^2 - 2a_1a_2b_1b_2 + (a_1b_2)^2 +
               (a_2b_1)^2 + 2a_1a_2b_1b_2 \\
            &= (a_1a_2)^2 + (b_1b_2)^2  + (a_1b_2)^2 + (a_2b_1)^2 \\
            &= ({a_1}^2 + {b_1}^2)({a_2}^2 + {b_2}^2) \\
            &= \varphi(z_1)\varphi(z_2).
      \end{align*}
      
      Observe that if $r$ is a nonpositive real number then $r$ cannot be in the
      image of $\varphi$; however if $r$ is a positive real number, then
      $\varphi(\sqrt{r}+0i) = r$, so that the image of $\varphi$ is the set of
      positive real numbers.
      The kernel of $\varphi$ is the set of complex numbers that lie on the
      circle of radius 1, centered at the origin. The fibers of $\varphi$ are
      circles centered at the orign; that is, if $X$ is a fiber of $\varphi$
      then there exists a positive real number $s$ such that $X$ is the set of
      all complex numbers on the circle of radius $\sqrt{s}$, centered at the
      origin. \qed
%%%%%%%%%%%%%%%%%%%%%%%%%%%%%%%%%%%%%3.1.16%%%%%%%%%%%%%%%%%%%%%%%%%%%%%%%%%%%%%
   \item[3.1.16]  Let $G$ be a group, let $N$ be a normal subgroup of $G$ and
                  let $\overline{G} = G/N$. Prove that if $G = \cyc{x, y}$ then
                  $\overline{G} = \cyc{\overline{x}, \overline{y}}$. Prove more
                  generally that if $G = \cyc{S}$ for any subset $S$ of $G$,
                  then $\overline{G} = \cyc{\overline{S}}$.
                  
      \textbf{Proof.} Suppose $G = \cyc{S}$, for some $S \subseteq G$ (for the
      particular part, where $G = \cyc{x, y}$, let $S = \{x, y\}$ and use the
      procedure below). First observe that
      $$\overline{S} = \{\overline{s} : s \in S\}.$$
      If $S$ is empty, then $\overline{S}$ must also be empty, so that
      $G = \cyc{S} = \{1\}$, and, thus, every normal subgroup of $G$ must also
      be the trivial group, so that
      $G/N = \{N\} = \{\{1\}\} =\cyc{\overline{S}}$. So assume that
      $S \neq \emptyset$. Since $\overline{G}$ is a group, and since
      $\overline{S} \subseteq \overline{G}$, it follows by closure that
      $\cyc{\overline{S}} \subseteq \overline{G}$. So to show that
      $\cyc{\overline{S}} = \overline{G}$, it suffices to show that
      $\overline{G} \subseteq \cyc{\overline{S}}$. To that end, let
      $\overline{g} \in \overline{G}$. Since $g \in G$ and since $G$ is
      generated by $S$, it follows that there exist a nonnegative integer
      $m$, $s_i \in S$, $\epsilon_i \in \{1, -1\}$, for each $1 \le i \le m$,
      such that
      $$g = {s_1}^{\epsilon_1}{s_2}^{\epsilon_2}\cdots{s_m}^{\epsilon_m}.$$
      Hence
      \begin{align*}
         \overline{g} &= gN \\
            &= ({s_1}^{\epsilon_1}{s_2}^{\epsilon_2}\cdots{s_m}^{\epsilon_m})N\\
            &= ({s_1}^{\epsilon_1}N)({s_2}^{\epsilon_2}N)\cdots
               ({s_n}^{\epsilon_n}N) &[\text{Proposition 5 (1)}] \\
            &= (s_1N)^{\epsilon_1}(s_2N)^{\epsilon_2}\cdots
               (s_nN)^{\epsilon_n} &[\text{Lemma 1}] \\
            &= \overline{s_1}^{\epsilon_1}\overline{s_2}^{\epsilon_2}\cdots
               \overline{s_n}^{\epsilon_n}.
      \end{align*}
      We observe that $\overline{s_1}$, $\overline{s_1}$, $\ldots$,
      $\overline{s_n}$ are elements of $\overline{S}$; that is, the equality
      $$g = \overline{s_1}^{\epsilon_1}\overline{s_2}^{\epsilon_2}\cdots
        \overline{s_n}^{\epsilon_n}$$
      implies that $\overline{g} \in \cyc{\overline{S}}$ so that
      $\overline{G} \subseteq \cyc{\overline{S}}$, and we conclude that
      $\overline{G} = \cyc{\overline{S}}$. \qed
      
%%%%%%%%%%%%%%%%%%%%%%%%%%%%%%%%%%%%%3.1.22%%%%%%%%%%%%%%%%%%%%%%%%%%%%%%%%%%%%%%
   \item[3.1.22]  \begin{enumerate}
                     \item Prove that if $H$ and $K$ are normal subgroups of a
                           group $G$ then their intersection $H \cap K$ is also
                           a normal subgroup of $G$.
                     \item Prove that the intersection of an arbitrary nonempty
                           collection of normal subgroups of a group is a normal
                           subgroup (do not assume the collection is countable).
                  \end{enumerate}
                  
      \textbf{Proof.}
      
      \begin{enumerate}
         \item Suppose $H$ and $K$ are normal subgroups of a group $G$. Let
               $I = \{1, 2\}$, $G_1 = H$, $G_2 = K$, and $S = G_1 \cap G_2$.
               Then use part (b) to conclude that $S$ is normal in $G$.
         \item Let $G$ be a group. Suppose $G_i \trianglelefteq G$, for all
               $i \in I$, where $I$ is some indexing set. Now let
               $S = \bigcap_{i\in I} G_i$. We showed in Exercise 2.1.10(b) that
               $S$ is a subgroup of $G$, so, in order to complete the proof, we
               must show that $S$ is normal in $G$. To that end, let $g \in G$
               and $s \in S$. By definition, we have that $s \in G_i$, for
               every $i \in I$. Since $G_i \trianglelefteq G$, it follows that
               $gsg^{-1} \in G_i$, for every $i \in I$. That is,
               $gsg^{-1} \in S$. Because $g$ and $s$ were arbitrary, we conclude
               that $gSg^{-1} \subseteq S$, for all $g \in G$, so that
               $S \trianglelefteq G$ by Theorem 6 (5).
               
      \end{enumerate} \qed
%%%%%%%%%%%%%%%%%%%%%%%%%%%%%%%%%%%%%3.1.24%%%%%%%%%%%%%%%%%%%%%%%%%%%%%%%%%%%%%
   \item[3.1.24]  Prove that if $N \trianglelefteq G$ and $H$ is any subgroup of
                  $G$ then $N \cap H \trianglelefteq H$.
                  
      \textbf{Proof.} Let $G$ be a group. Suppose that $H \le G$ and
      $N \trianglelefteq G$. By Exercise 2.1.10 (a), $N \cap H \le H$, so we
      now want to show that $N \cap H \trianglelefteq H$. Now let $h \in H$
      and $y \in N \cap H$. On the one hand, we have that $y \in H$, so it
      follows by closure of $H$ that $hyh^{-1} \in H$. On the other hand, we
      have $y \in N$, and $h, h^{-1} \in G$
      (because $h, h^{-1} \in H$ and $H \le G$), so that $hyh^{-1} \in N$
      because $N \trianglelefteq G$. Thus $hyh^{-1}$ is in both $H$ and $N$, so
      that $hyh^{-1} \in N \cap H$. That is,
      $h(N \cap H)h^{-1} \subseteq N \cap H$, for all $h \in H$, and we conclude
      by Theorem 6 (5)  that $N \cap H$ is a normal subgroup of $H$. \qed
%%%%%%%%%%%%%%%%%%%%%%%%%%%%%%%%%%%%%3.1.26%%%%%%%%%%%%%%%%%%%%%%%%%%%%%%%%%%%%%
   \item[3.1.26]  Let $a, b \in G$.
                  \begin{enumerate}
                     \item Prove that the conjugate of the product of $a$ and
                           $b$ is the product of the conjugate of $a$ and the
                           conjugate of $b$. Prove that the order of $a$ and the
                           order of any conjugate of $a$ are the same.
                     \item Prove that the conjugate of $a^{-1}$ is the inverse
                           of the conjugate of $a$.
                     \item Let $N = \cyc{S}$ for some subset $S$ of $G$. Prove
                           that $N \trianglelefteq G$ if $gSg^{-1} \subseteq N$
                           for all $g \in G$.
                     \item Deduce that if $N$ is the cyclic group $\cyc{x}$,
                           then $N$ is normal in $G$ if and only if for each
                           $g \in G$, $gxg^{-1} = x^k$ for some $k \in \Z$.
                     \item Let $n$ be a positive integer. Prove that the
                           subgroup $N$ of $G$ generated by all the elements of
                           $G$ of order $n$ is a normal subgroup of $G$.
                  \end{enumerate}
     
   \textbf{Proof.}
   
   \begin{enumerate}
      \item Let $g \in G$ and consider the conjugate of $ab$ by $g$, namely
            $g(ab)g^{-1}$. The conjugate of $a$ by $g$ is $gag^{-1}$ and the
            conjugate of $b$ by $g$ is $gbg^{-1}$. Since
            $$(gag^{-1})(gbg^{-1}) = ga(g^{-1}g)bg^{-1} = g(ab)g^{-1},$$
            it follows that conjugate of the product of $a$ and $b$ is the
            product of the conjugate of $a$ and the conjugate of $b$. Now we
            shall show that $|a| = |gag^{-1}|$. There are two possibilities:

            \textbf{Case 1.} $|a| = n$ for some positive integer $n$. Now we 
            have that
            \begin{align*}
               (gag^{-1})^n &= ga^ng^{-1} &[\text{Lemma 2}] \\
                  &= g1g^{-1} &[a^n = 1 \text{ since }|a| = n] \\
                  &= 1,
            \end{align*}
            so that $|gag^{-1}| = m \le n$, for some positive integer $m$. Using
            Lemma 2 and the fact that $|gag^{-1}| = m$, it follows that
            $$g1g^{-1} = 1 = (gag^{-1})^m = ga^mg^{-1},$$
            and we conclude by left and right cancellations that $a^m = 1$; the
            equality $a^m = 1$ implies that $n \le m$ because $m \in \Z^+$ and
            $|a| = n$. We have thus shown that $m \le n$ and $n \le m$, so that
            $m = n$; that is, $|a| = |gag^{-1}|$.

            \textbf{Case 2.} $|a| = \infty$. Suppose to the contrary that
            $|gag^{-1}| = k$, for some positive integer $k$. Using Lemma 2 and
            the fact that $|gag^{-1}| = k$, it follows that
            $$g1g^{-1} = 1 = (gag^{-1})^k = ga^kg^{-1},$$
            so that $a^k = 1$ by cancellation; that is, $|a| \le k$, a
            contradiction since we assumed that $a$ has infinite order. Thus
            $|a| = |gag^{-1}| = \infty$. \\

            Since $g$ was arbitrarily chosen, it follows that $a$ and its
            conjugates have the same order. \qed
      \item Consider the conjugate of $a^{-1}$ by some $h \in G$, namely
            $ha^{-1}h^{-1}$. Now we have that
            \begin{align*}
               (ha^{-1}h^{-1})(hah^{-1}) &= ha^{-1}(h^{-1}h)ah^{-1} \\
                  &= ha^{-1}ah^{-1} \\
                  &= hh^{-1} \\
                  &= 1.
            \end{align*}
            Post-multiply the equality $(ha^{-1}h^{-1})(hah^{-1}) = 1$ by
            $(hah^{-1})^{-1}$ to conclude that
            $$ha^{-1}h^{-1} = (hah^{-1})^{-1}.$$
            That is, the conjugate of $a^{-1}$ is the inverse of the conjugate 
            of $a$, which is what we wanted to show. \qed
      \item Let $N = \cyc{S}$ for some subset $S$ of $G$. Suppose that
            $gSg^{-1} \subseteq N$ for all $g \in G$. If $S$ is empty, then
            $N = \{1\}$, a normal subgroup of $G$, so we can further assume that
            $S$ is not empty. Let $h \in G$ and $n \in N$. By definition, it
            follows that
            $$n = {s_1}^{\epsilon_1}{s_2}^{\epsilon_2}\cdots{s_m}^{\epsilon_m}$$
            for some nonegative integer $m$, $s_i \in S$,
            $\epsilon_i \in \{-1, 1\}$, for all $1 \le i \le m$. Hence
            \begin{align*}
               hnh^{-1} &= h({s_1}^{\epsilon_1}{s_2}^{\epsilon_2}
                  \cdots{s_m}^{\epsilon_m})h^{-1} \\
                  &= (h{s_1}^{\epsilon_1}h^{-1})(h{s_2}^{\epsilon_2}h^{-1})
                  \cdots(h{s_m}^{\epsilon_m}h^{-1}) &[\text{Apply (a) $m$ times}]\\
                  &= (hs_1h^{-1})^{\epsilon_1}(hs_2h^{-1})^{\epsilon_2}
                  \cdots(hs_mh^{-1})^{\epsilon_m} &[\text{Lemma 2}]
            \end{align*}

            By our hypothesis, it follows that $hs_ih^{-1} \in N$, for each
            $1 \le i \le m$. Also, since $N$ is a group, it follows by closure
            that that $(hs_ih^{-1})^{\epsilon_i} \in N$, for each
            $1 \le i \le m$. Similarly, we use the closure property of $N$ to
            conclude that
            $$hnh^{-1} = (hs_1h^{-1})^{\epsilon_1}(hs_2h^{-1})^{\epsilon_2}
              \cdots(hs_mh^{-1})^{\epsilon_m} \in N;$$
            we just proved that $h \in G$ and $n \in N$ implies that
            $hnh^{-1} \in N$, so that $gNg^{-1} \subseteq N$, for all $g \in G$.
            Thus, $N \trianglelefteq G$ by Theorem 6 (5). \qed
      \item Fix $x \in G$ and let $N$ be the cyclic group generated by $x$.

            ($\Leftarrow$) Suppose for each $g \in G$, there exists an integer
            $k$ such that $gxg^{-1} = x^k$. That is, for each $g \in G$,
            $g\{x\}g^{-1} \subseteq \cyc{x} = N$. It follows by part (c) that
            $N \trianglelefteq G$.

            ($\Rightarrow$) Now suppose conversely that $N \trianglelefteq G$. It
            follows by Theorem 6 (5) that $gNg^{-1} \subseteq N$ for all
            $g \in G$. Particularly, since $x \in N$, it follows that
            $gxg^{-1} \in N$ for all $g \in G$, and since $N = \cyc{x}$, we
            have that, for each $g \in G$, $gxg^{-1} = x^k$, for some integer
            $k$, and the proof is done. \qed
      \item Let $n$ be a positive integer. Let $N = \cyc{S}$, where
            $$S = \{x \in G : |x| = n\}.$$
            If $G$ has no element of order $n$, then $S$ is empty, so that
            $N = \{1\}$, a normal subgroup of $G$. So assume that $G$ has at
            least one element of order $n$, so that $S \neq \emptyset$. Let
            $g \in G$ and $s \in S$. Since $s \in S$, it follows that $|s| = n$.
            By part (a), it follows that $|gsg^{-1}| = |s| = n$. That is,
            $gsg^{-1} \in S$. Since $g$ and $s$ were arbitrarily chosen, it
            follows that $gsg^{-1} \in S$ for all $g \in G$ and $s \in S$; that
            is, $gSg^{-1} \subseteq S \subseteq N$ for all $g \in G$, so we
            conclude by part (c) that $N \trianglelefteq G$. \qed
   \end{enumerate}
%%%%%%%%%%%%%%%%%%%%%%%%%%%%%%%%%%%%%3.1.36%%%%%%%%%%%%%%%%%%%%%%%%%%%%%%%%%%%%%
   \item[3.1.36]  Prove that if $G/Z(G)$ is cyclic then $G$ is abelian. [If
                  $G/Z(G)$ is cyclic with generator $xZ(G)$, show that every
                  element of $G$ can be written in the form $x^az$ for some
                  integer $a \in \Z$ and some element $z \in Z(G)$.]

      \textbf{Proof.} Let $G$ be a group and suppose that $G/Z(G)$ is cyclic.
      That is, there exists an element $gZ(G) \in G/Z(G)$ such that
      $G/Z(G) = \cyc{gZ(G)}$. Now let $x \in G$. Since $xZ(G)$ is an element of 
      $G/Z(G)$, it follows that there exists an integer $a$ such that
      $xZ(G) = (gZ(G))^a = g^aZ(G)$, where the last equality follows from Lemma
      1. Since the identity is in the center of $G$, we must have that
      $x = x \cdot 1 \in xZ(G)$, and by the equality $xZ(G) = g^aZ(G)$, it must
      be the case that $x \in g^aZ(G)$; that is, $x = g^az_1$ for some $z_1$ in
      $Z(G)$. Let us now choose another arbitrary element $y \in G$. Thus, by
      our preceding arguments, there exist an integer $b$ and $z_2 \in Z(G)$
      such that $y = g^bz_2$. Hence
      \begin{align*}
         xy &= (g^az_1)(g^bz_2) \\
            &= g^a(z_1g^b)z_2 \\ 
            &= g^a(g^bz_1)z_2 &[z_1 \in Z(G)] \\
            &= (g^ag^b)(z_1z_2) \\
            &= g^{a+b}z_2z_1 &[\text{Exercise 1.1.19(a) and }z_1 \in Z(G)] \\
            &= g^{b+a}z_2z_1 \\
            &= g^b(g^az_2)z_1 &[\text{Exercise 1.1.19(a)]} \\
            &= g^bz_2g^az_1 &[z_2 \in Z(G)] \\
            &= yx,
      \end{align*}
      so that $G$ is abelian. \qed
%%%%%%%%%%%%%%%%%%%%%%%%%%%%%%%%%%%%%3.1.41%%%%%%%%%%%%%%%%%%%%%%%%%%%%%%%%%%%%%
   \item[3.1.41]  Let $G$ be a group. Prove that
                  $N = \cyc{x^{-1}y^{-1}xy : x, y \in G}$ is a normal subgroup
                  of $G$ and $G/N$ is abelian ($N$ is called the
                  \textit{commutator subgroup} of $G$).

      \textbf{Proof.} Let $G$ be a group and consider the subgroup of $G$:
      $$N = \cyc{S}, \text{ where } S = \{x^{-1}y^{-1}xy : x, y \in G\}.$$
      The set $S$ is not empty since $1 = 1^{-1}\cdot1^{-1}\cdot1\cdot1$, so
      that $1 \in S$. Now let $s \in S$ and $g \in G$. That is,
      $s = r^{-1}t^{-1}rt$, for some $r, t \in G$. It follows that
      \begin{align*}
         gsg^{-1} &= gr^{-1}t^{-1}rtg^{-1} \\
            &= gr^{-1}(1)t^{-1}(1)r(1)tg^{-1} \\
            &= gr^{-1}(g^{-1}g)t^{-1}(g^{-1}g)r(g^{-1}g)tg^{-1} \\
            &= (gr^{-1}g^{-1})(gt^{-1}g^{-1})(grg^{-1})(gtg^{-1}) \\
            &= (grg^{-1})^{-1}(gtg^{-1})^{-1}(grg^{-1})(gtg^{-1})
      \end{align*}
      so that $gsg^{-1}$ can be written in the form $x^{-1}y^{-1}xy$, where
      $x = grg^{-1}$ and $y = gtg^{-1}$. This implies that $gsg^{-1} \in S$, so
      that, by the arbitrariness of $g$ and $s$, $gsg^{-1} \in S$, for all
      $g \in G$ and $s \in S$; we thus conclude that
      $gSg^{-1} \subseteq S \subseteq N$, for all $g \in G$, so that $N 
      \trianglelefteq G$ by Exercise 3.1.26(c). Finally let $xN$ and $yN$ be
      elements if $G/N$. By definition of $S$, it follows that
      $x^{-1}y^{-1}xy \in S$, so that $x^{-1}y^{-1}xy \in N$. That is,
      $x^{-1}y^{-1}xy = (yx)^{-1}(xy) \in N$. According to Proposition 4,
      $yxN = xyN$. But $yNxN = yxN = xyN = xNyN$, so that $G/N$ is abelian.
      \qed
\end{enumerate}
\end{document}
