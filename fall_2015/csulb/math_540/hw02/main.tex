\documentclass[9pt]{article}

\usepackage{amssymb}
\usepackage{amsmath}
\usepackage{amsfonts}
\usepackage{comment}
\usepackage{fancyhdr}
\usepackage{mathrsfs}
\usepackage{enumitem}
%\usepackage[retainorgcmds]{IEEEtrantools}

\everymath{\displaystyle}

\usepackage{tikz}

\voffset = -50pt
%\textheight = 700pt
\addtolength{\textwidth}{60pt}
\addtolength{\evensidemargin}{-30pt}
\addtolength{\oddsidemargin}{-30pt}
%\setlength{\headheight}{44pt}

\pagestyle{fancy}
\fancyhf{} % clear all fields
\fancyhead[R]{%
  \scshape
  \begin{tabular}[t]{@{}r@{}}
  MATH 540, Fall 2015\\Section 1 (9932)\\
  HW \#2, DUE: 2015, September 10
  \end{tabular}}
\fancyhead[L]{%
  \scshape
  \begin{tabular}[t]{@{}r@{}}
  JOSEPH OKONOBOH\\Mathematics\\Cal State Long Beach
  \end{tabular}}
\fancyfoot[C]{\thepage}

\newcommand{\qed}{\hfill \ensuremath{\Box}}


\newcommand*\circled[1]{\tikz[baseline=(char.base)]{
            \node[shape=circle,draw,inner sep=2pt] (char) {#1};}}


\newcommand{\cyc}[1]{\langle #1 \rangle}
\newcommand{\Z}{\mathbb{Z}}
\newcommand{\I}{\mathbb{I}}
\newcommand{\F}{\mathbb{F}}
\newcommand{\M}{\mathbb{M}}
\newcommand{\R}{\mathbb{R}}
\newcommand{\Q}{\mathbb{Q}}
\newcommand{\D}{\displaystyle}
%\setcounter{section}{-1}

\begin{document}
\begin{enumerate}
%%%%%%%%%%%%%%%%%%%%%%%%%%%%%%%%%%%%%1.6.1%%%%%%%%%%%%%%%%%%%%%%%%%%%%%%%%%%%%%%
   \item[1.6.1]   Let $\varphi : G \rightarrow H$ be a homomorphism.
                  \begin{enumerate}
                     \item Prove that $\varphi(x^n) = \varphi(x)^n$ for all
                           $n \in \Z^+$.
                     \item Do part (a) for $n = -1$ and deduce that
                           $\varphi(x^n) = \varphi(x)^n$ for all $n \in \Z$.
                  \end{enumerate}

      \textbf{Solution.}

      \begin{enumerate}
         \item \textbf{Proof.} We shall proceed by induction on $n$. It is clear
               that $\varphi(x^1) = \varphi(x)^1$. Now suppose that
               $\varphi(x^k) = \varphi(x)^k$ for some integer $k$. Thus
               \begin{align*}
                  \varphi(x^{k+1}) &= \varphi(x^kx) \\
                     &= \varphi(x^k)\varphi(x)
                        &[\varphi\text{ is a homomorphism}] \\
                     &= \varphi(x)^k\varphi(x) &[\text{Inductive hypothesis}] \\
                     &= \varphi(x)^{k+1},
               \end{align*}
               so that, by Mathematical Induction, $\varphi(x^n) = \varphi(x)^n$ 
               for all $n \in \Z^+$. \qed
         \item Since
               $$1 \cdot \varphi(1) = \varphi(1) = \varphi(1 \cdot 1) =
                 \varphi(1)\cdot\varphi(1),$$
               it follows by cancellation that $\varphi(1) = 1$. Thus
               $$\varphi(x)\varphi(x^{-1}) = \varphi(xx^{-1}) =\varphi(1) = 1,$$
               so that $\varphi(x^{-1}) = \varphi(x)^{-1}$. Now let $n$ be a
               positive integer. Then it follows that
               \begin{align*}
                  \varphi(x^{-n}) &= \varphi((x^{-1})^n) \\
                     &= \varphi(x^{-1})^n &[\text{1.6.1(a)}] \\
                     &= (\varphi(x)^{-1})^n \\
                     &= \varphi(x)^{-n}.
               \end{align*}
               Moreover $\varphi(x^0) = 1 = \varphi(x)^0$; thus we can conclude 
               that $\varphi(x^n) = \varphi(x)^n$ for all $n \in \Z$.
      \end{enumerate}
%%%%%%%%%%%%%%%%%%%%%%%%%%%%%%%%%%%%%1.6.3%%%%%%%%%%%%%%%%%%%%%%%%%%%%%%%%%%%%%%
   \item[1.6.3]   If $\varphi : G \rightarrow H$ is an isomorphism, prove that
                  $G$ is abelian if and only if $H$ is abelian. If
                  $\varphi : G \rightarrow H$ is a homomorphism, what additional
                  conditions on $\varphi$ (if any) are sufficient to ensure that
                  if $G$ is abelian, then so is $H$?

      \textbf{Proof.} Suppose $\varphi : G \rightarrow H$ is an isomorphism.

      ($\Rightarrow$) Assume $G$ is abelian. Consider $h_1$ and $h_2$ in $H$.
      Since $\varphi$ is surjective, there exist $g_1$, $g_2 \in G$ such that
      $\varphi(g_1) = h_1$ and $\varphi(g_2) = h_2$. Since $G$ is abelian, we
      have that
      $$h_1h_2 = \varphi(g_1)\varphi(g_2) = \varphi(g_1g_2) = \varphi(g_2g_1) = 
        \varphi(g_2)\varphi(g_1) = h_2h_1,$$
      so that $H$ is also abelian.

      ($\Rightarrow$) Now assume that $H$ is abelian. Consider $g_3$ and $g_4$ 
      in $G$. Since $H$ is abelian, we have that
      $$\varphi(g_3g_4) = \varphi(g_3)\varphi(g_4) = \varphi(g_4)\varphi(g_3) =
        \varphi(g_4g_3).$$
      We thus conclude that $g_3g_4 = g_4g_3$ since $\varphi$ is injective. That
      is, $G$ is abelian. \qed

      Finally suppose $\varphi$ is an homomorphism and $G$ is abelian. Looking
      at the first direction of our proof above, we see that restricting
      $\varphi$ to be surjective is sufficient to make $H$ abelian.
%%%%%%%%%%%%%%%%%%%%%%%%%%%%%%%%%%%%%1.6.6%%%%%%%%%%%%%%%%%%%%%%%%%%%%%%%%%%%%%%
   \item[1.6.6]   Prove that the additive groups of $\Z$ and $\Q$ are not
                  isomorphic.

      \textbf{Proof.} First we shall show that $(\Q, +)$ is not cyclic. So
      suppose that $(\Q, +) = \cyc{c}$. Thus there exists an integer $n$ such 
      that $nc = \D\frac{c}{2}$, so that $\D n = \frac{1}{2}$, a contradiction. 
      That is, $(\Q, +)$ is not cyclic. Now suppose to the contrary that
      $\varphi : \Z \rightarrow \Q$ is a homomorphism from $(\Z, +)$ to
      $(\Q, +)$. Let $q \in \Q$ and let $z$ be the unique preimage of $q$ under
      $\varphi$. Thus
      \begin{align*}
         q &= \varphi(z) \\
           &= \varphi(z \cdot 1) \\
           &= z \cdot \varphi(1),  &[1.6.1(a)]
      \end{align*}
      so that $\Q = \cyc{\varphi(1)}$, a contradiction. Thus $(\Z, +)$ and
      $(\Q, +)$ are not isomorphic. \qed
%%%%%%%%%%%%%%%%%%%%%%%%%%%%%%%%%%%%%1.6.14%%%%%%%%%%%%%%%%%%%%%%%%%%%%%%%%%%%%%
   \item[1.6.14]  Let $G$ and $H$ be groups and let $\varphi : G \rightarrow H$
                  be a homomorphism. Define the \text{kernel} of $\varphi$ to be
                  $\{g \in G : \varphi(g) = 1_H\}$. Prove that the kernel of
                  $\varphi$ is a subgroup of $G$. Prove that $\varphi$ is
                  injective if and only if the kernel of $\varphi$ is the
                  identity subgroup of $G$.

      \textbf{Proof 1.} Let ker($\varphi$) be the kernel of $\varphi$. The set
      ker($\varphi$) is not empty because $1_G \in \text{ker}(\varphi)$. Now let
      $x, y \in \text{ker}(\varphi)$. Then we have that
      $$\varphi(xy^{-1}) = \varphi(x)\varphi(y)^{-1} = 1_H{1_H}^{-1} = 1_H,$$
      so that $xy^{-1} \in \text{ker}(\varphi)$. That is ker($\varphi$) is a
      subgroup of $G$. \qed

      \textbf{Proof 2.} ($\Rightarrow$) Assume that $\varphi$ is injective. Let
      $x \in \text{ker}(\varphi)$. Thus we have that
      $\varphi(1_G) = \varphi(x) = 1_H$, so that $x = 1_G$ since $\varphi$ is
      one to one. That is ker($\varphi$) = $\{1_G\}$, the identity subgroup of
      $G$.

      ($\Leftarrow$) Assume that ker($\varphi$) = $\{1_G\}$. Suppose that
      $\varphi(g_1) = \varphi(g_2)$ for some $g_1, g_2 \in G$. Then it follows
      that $\varphi(g_1)\varphi(g_2)^{-1} = 1_H$, so that
      $\varphi(g_1{g_2}^{-1}) = 1_H$. Since ker($\varphi$) = $\{1_G\}$, we can
      conclude that $g_1{g_2}^{-1} = 1_G$, so that $g_1 = g_2$; i.e., $\varphi$
      is injective. \qed
%%%%%%%%%%%%%%%%%%%%%%%%%%%%%%%%%%%%%1.7.3%%%%%%%%%%%%%%%%%%%%%%%%%%%%%%%%%%%%%%
   \item[1.7.3]   Show that the additive group $\R$ acts on the $x, y$ plane
                  $\R \times \R$ by $r \cdot (x, y) = (x + ry, y)$.

      \textbf{Proof.} Let $r_1$, $r_2$, $r_3$, and $r_4$ be real numbers. Since
      $\R$ is closed under addition, we have that
      $r_1 \cdot (r_2, r_3) = (r_2 + r_1r_3, r_3) \in \R \times \R$. We also
      have that
      $$0 \cdot (r_1, r_2) = (r_1 + 0 \cdot r_2, r_2) = (r_1, r_2).$$
      Finally we have that
      \begin{align*}
         r_1 \cdot (r_2 \cdot (r_3, r_4)) &= r_1 \cdot (r_3 + r_2r_4, r_4) \\
            &= (r_3 + r_2r_4 + r_1r_4, r_4) \\
            &= (r_3 + (r_2 + r_1)r_4, r_4) \\
            &= (r_2 + r_1) \cdot (r_3, r_4), \\
      \end{align*}
      so that $\R$ acts on the cartesian plane by the given operation. \qed
%%%%%%%%%%%%%%%%%%%%%%%%%%%%%%%%%%%%%1.7.6%%%%%%%%%%%%%%%%%%%%%%%%%%%%%%%%%%%%%%
   \item[1.7.6]   Prove that a group $G$ acts faithfully on a set $A$ if and
                  only if the kernel of the action is the set consisting only of
                  the identity.

      \textbf{Proof.} Let $\varphi : G \rightarrow S_A$ be the corresponding 
      permutation representation.

      ($\Rightarrow$) Assume that $G$ acts faithfully on $A$. Let $g$ be an
      element of the kernel of this action. Then it follows that
      $\varphi(g) = 1$; but $\varphi$ is injective (since this action is
      faithul), so that $g = 1$, and it follows that the kernel of the action is
      trivial.

      ($\Leftarrow$) Conversely, suppose the kernel of this action is trivial.
      Then it follows by Exercise 1.7.5 that the kernel of $\varphi$ is also
      trivial. So suppose that $\varphi(g_2) = \varphi(g_3)$. Then we have
      that $\varphi(g_2{g_3}^{-1}) = 1$, and since the kernel of $\varphi$ is 
      trivial, it follows that $g_2{g_3}^{-1} = 1$. Thus $g_2 = g_3$ and we
      have that $\varphi$ is injective; that is, $G$ acts faithfully on $A$.
      \qed
%%%%%%%%%%%%%%%%%%%%%%%%%%%%%%%%%%%%%1.7.11%%%%%%%%%%%%%%%%%%%%%%%%%%%%%%%%%%%%%
   \item[1.7.11]  Write out the cycle decomposition of the eight permutations in
                  $S_4$ corresponding to the elements of $D_8$ given by the
                  action of $D_8$ on the vertices of a square (where the
                  vertices of the square are labelled as in Section 2).

      \textbf{Solution.}

      $$
         \begin{tabular}{@{}|c|c|@{}} \hline
            Element in $D_8$ & Corresponding element in $S_4$ \\ \hline         
            1      & (1) \\ \hline
            $r$    & (1 2 3 4) \\ \hline
            $r^2$  & (1 3)(2 4) \\ \hline
            $r^3$  & (1 4 3 2) \\ \hline        
            $s$    & (2 4) \\ \hline
            $sr$   & (1 4)(2 3) \\ \hline
            $sr^2$ & (1 3) \\ \hline
            $sr^3$ & (1 2)(3 4)\\ \hline
         \end{tabular}
      $$      
%%%%%%%%%%%%%%%%%%%%%%%%%%%%%%%%%%%%%1.7.18%%%%%%%%%%%%%%%%%%%%%%%%%%%%%%%%%%%%%
   \item[1.7.18]  Let $H$ be a group acting on a set $A$. Prove that the
                  relation $\sim$ on $A$ defined by
                  $$a \sim b \quad \text{if and only if} \quad
                    a = hb \quad \text{for some }h \in H$$
                  is an equivalence relation. (For each $x \in A$ the
                  equivalence class of $x$ under $\sim$ is called the
                  \textit{orbit} of $x$ under the action of $H$. The orbits
                  under the action of $H$ partition the set $A$.)

      \textbf{Proof.} Let $a, b, c \in H$.

      \textbf{Reflexivity.} Since $1a = a$, it follows that $a \sim a$, so that
      $\sim$ is reflexive on $A$.

      \textbf{Symmetry.} Suppose $a \sim b$. Then there exists $h \in H$ such
      that $a = hb$. Now $h^{-1}a = h^{-1}hb = 1b = b$, so that $b \sim a$; thus
      $\sim$ is symmetric on $A$.

      \textbf{Transitivity.} Suppose $a \sim b$ and $b \sim c$. Then there exist
      $h_1, h_2 \in H$ such that $a = h_1b$ and $b = h_2c$. Thus
      $a = h_1b = h_1(h_2c) = (h_1h_2)c$, so that $a \sim c$; thus $\sim$ is 
      transitive on $A$.

      We can thus conclude that $\sim$ is an equivalence relation on $A$. \qed
\end{enumerate}
\end{document}
