\documentclass[9pt]{article}

\usepackage{amssymb}
\usepackage{amsmath}
\usepackage{amsfonts}
\usepackage{comment}
\usepackage{fancyhdr}
\usepackage{mathrsfs}
\usepackage{enumitem}
%\usepackage[retainorgcmds]{IEEEtrantools}

\everymath{\displaystyle}

\usepackage{tikz}

\voffset = -50pt
%\textheight = 700pt
\addtolength{\textwidth}{60pt}
\addtolength{\evensidemargin}{-30pt}
\addtolength{\oddsidemargin}{-30pt}
%\setlength{\headheight}{44pt}

\pagestyle{fancy}
\fancyhf{} % clear all fields
\fancyhead[R]{%
  \scshape
  \begin{tabular}[t]{@{}r@{}}
  MATH 540, Fall 2015\\Section 1 (9932)\\
  HW \#2, DUE: 2015, September 10
  \end{tabular}}
\fancyhead[L]{%
  \scshape
  \begin{tabular}[t]{@{}r@{}}
  JOSEPH OKONOBOH\\Mathematics\\Cal State Long Beach
  \end{tabular}}
\fancyfoot[C]{\thepage}

\newcommand{\qed}{\hfill \ensuremath{\Box}}


\newcommand*\circled[1]{\tikz[baseline=(char.base)]{
            \node[shape=circle,draw,inner sep=2pt] (char) {#1};}}


\newcommand{\cyc}[1]{\langle #1 \rangle}
\newcommand{\Z}{\mathbb{Z}}
\newcommand{\I}{\mathbb{I}}
\newcommand{\F}{\mathbb{F}}
\newcommand{\M}{\mathbb{M}}
\newcommand{\R}{\mathbb{R}}
\newcommand{\Q}{\mathbb{Q}}
\newcommand{\D}{\displaystyle}
%\setcounter{section}{-1}

\begin{document}
\begin{enumerate}
%%%%%%%%%%%%%%%%%%%%%%%%%%%%%%%%%%%%%1.6.1%%%%%%%%%%%%%%%%%%%%%%%%%%%%%%%%%%%%%%
   \item[1.6.1]   Let $\varphi : G \rightarrow H$ be a homomorphism.
                  \begin{enumerate}
                     \item Prove that
                           \begin{equation} \label{1_6_1_0}
                              \varphi(x^n) = \varphi(x)^n
                           \end{equation}
                           for all $n \in \Z^+$.
                     \item Do part (a) for $n = -1$ and deduce that
                           $\varphi(x^n) = \varphi(x)^n$ for all $n \in \Z$.
                  \end{enumerate}

      \textbf{Solution.} Let $x \in G$.

      \begin{enumerate}
         \item \textbf{Proof.} We shall proceed by induction on $n$. It is
               immediate that statement \eqref{1_6_1_0} holds for the base
               case ($n = 1$) because
               $$\varphi(x^1) = \varphi(x) = \varphi(x)^1.$$
               For our inductive hypothesis, we shall now suppose that
               \eqref{1_6_1_0} holds for some positive integer $k$. Thus
               \begin{align*}
                  \varphi(x^{k+1}) &= \varphi(x^kx) &[\text{Exercise 1.1.19}] \\
                     &= \varphi(x^k)\varphi(x)
                        &[\varphi\text{ is a homomorphism}] \\
                     &= \varphi(x)^k\varphi(x) &[\text{Inductive hypothesis}] \\
                     &= \varphi(x)^{k+1} &[\text{Exercise 1.1.19}],
               \end{align*}
               so that \eqref{1_6_1_0} also holds for $k+1$; hence, we
               conclude by mathematical induction that \eqref{1_6_1_0} holds for
               all $n \in \Z^+$. \qed
         \item Since
               $$1_H \cdot \varphi(1_G) = \varphi(1_G) = \varphi(1_G \cdot 1_G)
                 = \varphi(1_G)\cdot\varphi(1_G),$$
               it follows by cancellation that $\varphi(1_G) = 1_H$. Thus
               $$\varphi(x)\varphi(x^{-1}) = \varphi(xx^{-1}) =\varphi(1_G) =
                  1_H,$$
               so that $\varphi(x^{-1}) = \varphi(x)^{-1}$. Now let $m$ be a
               negative integer, so that $m = -n$ for some positive integer $n$.
               Then it follows that
               \begin{align*}
                  \varphi(x^m) &= \varphi(x^{-n}) \\
                     &= \varphi((x^{-1})^n) &[\text{By Definition}] \\
                     &= \varphi(x^{-1})^n &[\text{1.6.1(a)}] \\
                     &= (\varphi(x)^{-1})^n &[\varphi(x^{-1}) = \varphi(x)^{-1}] \\
                     &= \varphi(x)^{-n}. &[\text{Exercise 1.1.19}] \\
                     &= \varphi(x)^m.
               \end{align*}
               Moreover $\varphi(x^0) = \varphi(1_G) = 1_H = \varphi(x)^0$; thus
               we conclude that $\varphi(x^n) = \varphi(x)^n$ for all
               $n \in \Z$.
      \end{enumerate}
%%%%%%%%%%%%%%%%%%%%%%%%%%%%%%%%%%%%%1.6.3%%%%%%%%%%%%%%%%%%%%%%%%%%%%%%%%%%%%%%
   \item[1.6.3]   If $\varphi : G \rightarrow H$ is an isomorphism, prove that
                  $G$ is abelian if and only if $H$ is abelian. If
                  $\varphi : G \rightarrow H$ is a homomorphism, what additional
                  conditions on $\varphi$ (if any) are sufficient to ensure that
                  if $G$ is abelian, then so is $H$?

      \textbf{Proof.} Suppose $\varphi : G \rightarrow H$ is an isomorphism.

      ($\Rightarrow$) Assume $G$ is abelian. Let $h_1, h_2 \in H$. Since
      $\varphi$ is an isomorphism, it must be surjective; thus, by definition,
      there exist $g_1$, $g_2 \in G$ such that
      $\varphi(g_1) = h_1$ and ${\varphi(g_2) = h_2}$. It follows that
      \begin{align*}
         h_1h_2 &= \varphi(g_1)\varphi(g_2) \\
            &= \varphi(g_1g_2) &[\varphi \text{ is a homomorphism}] \\
            &= \varphi(g_2g_1) &[G \text{ is abelian}] \\
            &= \varphi(g_2)\varphi(g_1) &[\varphi \text{ is a homomorphism}] \\
            &= h_2h_1,
      \end{align*}
      so that $H$ is also abelian.

      ($\Rightarrow$) Now assume that $H$ is abelian. Consider arbitrary 
      elements $g_3$ and $g_4$ in $G$. It follows that
      \begin{align*}
         \varphi(g_3g_4) &= \varphi(g_3)\varphi(g_4)
            &[\varphi \text{ is a homomorphism}] \\
            &= \varphi(g_4)\varphi(g_3) &[H \text{ is abelian}] \\
            &= \varphi(g_4g_3), &[\varphi \text{ is a homomorphism}]
      \end{align*}
      and, since $\varphi$ is injective, we conclude that $g_3g_4 = g_4g_3$.
      That is, $G$ is abelian. \qed

      Finally suppose $\varphi$ is an homomorphism and $G$ is abelian. Looking
      at the first direction of our proof above, we see that requiring
      $\varphi$ to be surjective is sufficient to make $H$ abelian.
%%%%%%%%%%%%%%%%%%%%%%%%%%%%%%%%%%%%%1.6.6%%%%%%%%%%%%%%%%%%%%%%%%%%%%%%%%%%%%%%
   \item[1.6.6]   Prove that the additive groups of $\Z$ and $\Q$ are not
                  isomorphic.

      \textbf{Proof.} Assume to the contrary that $\Z$ and $\Q$ are isomorphic.
      Then it follows by definition that there exists a bijective homomorphism
      $$\varphi : \Z \rightarrow \Q.$$
      Let $q \in \Q$. Since $\varphi$ is surjective, there exists an integer $z$
      such that $\varphi(z) = q$. Thus
      \begin{align*}
         q &= \varphi(z) \\
           &= \varphi(z \cdot 1) \\
           &= z \cdot \varphi(1).  &[1.6.1(a)]
      \end{align*}
      Since $q$ was arbitrary, we have just shown that every element in $\Q$ can
      be written as an integral multiple of $\varphi(1)$; that is, $\varphi(1)$
      generates $\Q$. Particularly, we must have that
      $$\frac{\varphi(1)}{2} \in \cyc{\varphi(1)} = \Q.$$
      So there exists an integer $n$ such that
      $$\frac{\varphi(1)}{2} = n \cdot \varphi(1).$$
      Observe that $\varphi(1) \neq 0$ since that would imply that
      $\cyc{\varphi(1)} = 0$. So divide the preceding equality by $\varphi(1)$
      to get $n = 1/2$, contradicting our assumption that $n$ was an integer.
      Therefore Thus $(\Z, +)$ and $(\Q, +)$ are not isomorphic. \qed
%%%%%%%%%%%%%%%%%%%%%%%%%%%%%%%%%%%%%1.6.14%%%%%%%%%%%%%%%%%%%%%%%%%%%%%%%%%%%%%
   \item[1.6.14]  Let $G$ and $H$ be groups and let $\varphi : G \rightarrow H$
                  be a homomorphism. Define the \text{kernel} of $\varphi$ to be
                  $\{g \in G : \varphi(g) = 1_H\}$. Prove that the kernel of
                  $\varphi$ is a subgroup of $G$. Prove that $\varphi$ is
                  injective if and only if the kernel of $\varphi$ is the
                  identity subgroup of $G$.

      \textbf{Proof 1.} Let ker($\varphi$) be the kernel of $\varphi$. In
      Exercise 1.6.1(b), we showed that ${\varphi(1_G) = 1_H}$; that is,
      $1_G  \in \text{ker}(\varphi)$. Since ker($\varphi$) is not empty, we let
      $x, y \in \text{ker}(\varphi)$. To show that ker($\varphi$) is a subgroup
      of $G$, it suffices to show that $xy$ and $x^{-1}$ are also members of
      ker($\varphi$). Thus
      \begin{align*}
         \varphi(xy) &= \varphi(x)\varphi(y)
            &[\varphi \text{ is a homomorphism}] \\
            &= 1_H1_H &[x, y \in \text{ker}(\varphi)] \\
            &= 1_H,
      \end{align*}
      so that $xy \in \ker(\varphi)$. Also, we have that      
      \begin{align*}
         \varphi(x^{-1}) &= \varphi(x)^{-1} &[\text{Exercise }1.6.1] \\
            &= {1_H}^{-1} &[x \in \text{ker}(\varphi)] \\
            &= 1_H,
      \end{align*}
      so that $x^{-1} \in \text{ker}(\varphi)$. We have just shown that
      ker($\varphi$) is closed under multiplication and inverses; thus
      ker($\varphi$) is a subgroup of $G$. \qed

      \textbf{Proof 2.} ($\Rightarrow$) Assume that $\varphi$ is injective. Let
      $g \in \text{ker}(\varphi)$. Thus we have that
      $$\varphi(1_G) = 1_H = \varphi(g),$$
      so that $\varphi(1_G) = \varphi(g)$. But since $\varphi$ is injective, it
      follows that $1_G = g$. This says that if an element of $G$ maps to the
      identity in $H$, then that element must necessarily be the identity in
      $G$. That is, ker($\varphi$) = $\{1_G\}$, the identity subgroup of $G$.

      ($\Leftarrow$) Conversely, assume that ker($\varphi$) = $\{1_G\}$. Suppose that
      $\varphi(g_1) = \varphi(g_2)$ for some $g_1, g_2 \in G$. Then it follows
      that
      \begin{align*}
         1_H &= \varphi(g_2)\varphi(g_2)^{-1} \\
             &= \varphi(g_1)\varphi(g_2)^{-1} &[\varphi(g_1) = \varphi(g_2)] \\
             &= \varphi(g_1)\varphi({g_2}^{-1}) &[\text{Exercise }1.6.1] \\
             &= \varphi(g_1{g_2}^{-1}), &[\varphi \text{ is a homomorphism}]
      \end{align*}
      so that $g_1{g_2}^{-1} \in \text{ker}(\varphi)$. But ker($\varphi$) is a
      singleton containing $1_G$; that is,
      $g_1{g_2}^{-1} = 1_G = g_2{g_2}^{-1}$, and $g_1 = g_2$, by cancellation.
      Thus, $\varphi$ is injective. \qed
%%%%%%%%%%%%%%%%%%%%%%%%%%%%%%%%%%%%%1.7.3%%%%%%%%%%%%%%%%%%%%%%%%%%%%%%%%%%%%%%
   \item[1.7.3]   Show that the additive group $\R$ acts on the $x, y$ plane
                  $\R \times \R$ by $r \cdot (x, y) = (x + ry, y)$.

      \textbf{Proof.} Let $r_1$, $r_2$, $r_3$, and $r_4$ be real numbers. First
      we want to show that the map is well defined; that is, we want to show
      that $r_1 \cdot (r_2, r_3)  \in \R \times \R$. Since $\R$ is closed under
      addition and multiplication, it follows that
      $r_1 \cdot (r_2, r_3) = (r_2 + r_1r_3, r_3) \in \R \times \R$. Now we
      want to show that the identity of $\R$---0---fixes every point on the
      $x, y$ plane. So we have that
      $$0 \cdot (r_1, r_2) = (r_1 + 0 \cdot r_2, r_2) = (r_1, r_2).$$
      Finally we have that
      \begin{align*}
         r_1 \cdot (r_2 \cdot (r_3, r_4)) &= r_1 \cdot (r_3 + r_2r_4, r_4) \\
            &= (r_3 + r_2r_4 + r_1r_4, r_4) \\
            &= (r_3 + (r_2 + r_1)r_4, r_4) \\
            &= (r_2 + r_1) \cdot (r_3, r_4). \\
      \end{align*}
      Hence $\R$ acts on the cartesian plane by the given operation. \qed
%%%%%%%%%%%%%%%%%%%%%%%%%%%%%%%%%%%%%1.7.6%%%%%%%%%%%%%%%%%%%%%%%%%%%%%%%%%%%%%%
   \item[1.7.6]   Prove that a group $G$ acts faithfully on a set $A$ if and
                  only if the kernel of the action is the set consisting only of
                  the identity.

      \textbf{Proof.} Let $G$ be a group acting on a set $A$.
      
      ($\Rightarrow$) Assume that $G$ acts faithfully on $A$. Let $g$ be an
      element of the kernel of this action (the kernel of the action is not
      empty since it contains $1_G$). Since $g$ is an element of the kernel of
      the action, it follows by definition that $g$ must fix all the elements of
      $A$. Since the group action is faithful, all the elements of $G$ must
      induce different permutations of $A$; but $1_G$ also fixes all the
      elements of $A$, so that $1_G$ and $g$ induce the same permutation of $A$.
      Thus $g = 1_G$, and it follows that the kernel of the action is a 
      singleton containing $1_G$.

      ($\Leftarrow$) Conversely, assume the kernel of this action is trivial.
      Suppose that $g, h \in G$ such that $ga = ha$ for all $a \in A$. To show
      that the group action is faithful, it suffices to show that $g = h$. Let
      $b \in A$. Using the axioms of a group action and the fact that $gb = hb$,
      it follows that
      $$b = 1_Gb = (g^{-1}g)b = g^{-1}(gb) = g^{-1}(hb) = (g^{-1}h)b.$$
      Since $b$ was arbitrary, it follows that $(g^{-1}h)c = c$ for all
      $c \in A$. That is, $g^{-1}h$ fixes every element of $A$, so that
      $g^{-1}h$ is in the kernel of the action. But the action is trivial, so
      we have that $g^{-1}h = 1_G$. (Left) Multiply the equality
      $g^{-1}h = 1_G$ by $g$ to conclude that $g = h$. That is, $G$ acts
      faithfully on $A$, as desired. \qed
      g
%%%%%%%%%%%%%%%%%%%%%%%%%%%%%%%%%%%%%1.7.11%%%%%%%%%%%%%%%%%%%%%%%%%%%%%%%%%%%%%
   \item[1.7.11]  Write out the cycle decomposition of the eight permutations in
                  $S_4$ corresponding to the elements of $D_8$ given by the
                  action of $D_8$ on the vertices of a square (where the
                  vertices of the square are labelled as in Section 2).

      \textbf{Solution.}

      $$
         \begin{tabular}{@{}|c|c|@{}} \hline
            Element in $D_8$ & Corresponding element in $S_4$ \\ \hline         
            1      & (1) \\ \hline
            $r$    & (1 2 3 4) \\ \hline
            $r^2$  & (1 3)(2 4) \\ \hline
            $r^3$  & (1 4 3 2) \\ \hline        
            $s$    & (2 4) \\ \hline
            $sr$   & (1 4)(2 3) \\ \hline
            $sr^2$ & (1 3) \\ \hline
            $sr^3$ & (1 2)(3 4)\\ \hline
         \end{tabular}
      $$      
%%%%%%%%%%%%%%%%%%%%%%%%%%%%%%%%%%%%%1.7.18%%%%%%%%%%%%%%%%%%%%%%%%%%%%%%%%%%%%%
   \item[1.7.18]  Let $H$ be a group acting on a set $A$. Prove that the
                  relation $\sim$ on $A$ defined by
                  $$a \sim b \quad \text{if and only if} \quad
                    a = hb \quad \text{for some }h \in H$$
                  is an equivalence relation. (For each $x \in A$ the
                  equivalence class of $x$ under $\sim$ is called the
                  \textit{orbit} of $x$ under the action of $H$. The orbits
                  under the action of $H$ partition the set $A$.)

      \textbf{Proof.} Let $a, b, c \in A$. To show that $\sim$ is an equivalence
      relation on $A$, we must show that $\sim$ is reflexive, symmetric, and
      transitive on $A$. So:

      \textbf{Reflexivity.} Since $1a = a$, it follows by definition that
      $a \sim a$, so that $\sim$ is reflexive on $A$.

      \textbf{Symmetry.} Suppose $a \sim b$. Then there exists $h \in H$ such
      that $a = hb$. Now $h^{-1}a = h^{-1}hb = 1b = b$, so that $b \sim a$; thus
      $\sim$ is symmetric on $A$.

      \textbf{Transitivity.} Suppose $a \sim b$ and $b \sim c$. Then there exist
      $h_1, h_2 \in H$ such that $a = h_1b$ and $b = h_2c$. Thus
      $a = h_1b = h_1(h_2c) = (h_1h_2)c$, so that $a \sim c$; thus $\sim$ is 
      transitive on $A$.

      We thus conclude that $\sim$ is an equivalence relation on $A$. \qed
\end{enumerate}
\end{document}
