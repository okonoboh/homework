\documentclass[9pt]{article}

\usepackage{amssymb}
\usepackage{amsmath}
\usepackage{amsfonts}
\usepackage{comment}
\usepackage{fancyhdr}
\usepackage{mathrsfs}
\usepackage{enumitem}
%\usepackage[retainorgcmds]{IEEEtrantools}

\everymath{\displaystyle}

\usepackage{tikz}

\voffset = -50pt
%\textheight = 700pt
\addtolength{\textwidth}{60pt}
\addtolength{\evensidemargin}{-30pt}
\addtolength{\oddsidemargin}{-30pt}
%\setlength{\headheight}{44pt}

\pagestyle{fancy}
\fancyhf{} % clear all fields
\fancyhead[R]{%
  \scshape
  \begin{tabular}[t]{@{}r@{}}
  MATH 540, Fall 2015\\Section 1 (9932)\\
  HW \#3, DUE: 2015, September 17
  \end{tabular}}
\fancyhead[L]{%
  \scshape
  \begin{tabular}[t]{@{}r@{}}
  JOSEPH OKONOBOH\\Mathematics\\Cal State Long Beach
  \end{tabular}}
\fancyfoot[C]{\thepage}

\newcommand{\qed}{\hfill \ensuremath{\Box}}


\newcommand*\circled[1]{\tikz[baseline=(char.base)]{
            \node[shape=circle,draw,inner sep=2pt] (char) {#1};}}


\newcommand{\cyc}[1]{\langle #1 \rangle}
\newcommand{\Z}{\mathbb{Z}}
\newcommand{\I}{\mathbb{I}}
\newcommand{\F}{\mathbb{F}}
\newcommand{\M}{\mathbb{M}}
\newcommand{\R}{\mathbb{R}}
\newcommand{\Q}{\mathbb{Q}}
\newcommand{\D}{\displaystyle}
%\setcounter{section}{-1}

\begin{document}
\begin{enumerate}
%%%%%%%%%%%%%%%%%%%%%%%%%%%%%%%%%%%%%2.1.1%%%%%%%%%%%%%%%%%%%%%%%%%%%%%%%%%%%%%%
   \item[2.1.1]   In each of (c) and (d) prove that the specified subset is a
                  subgroup of the given group:
                  \begin{enumerate}
                     \item[(c)]  for fixed $n \in \Z^+$ the set of rational 
                                 numbers(in lowest terms) whose denominators 
                                 divide $n$ (under addition)
                     \item[(d)]  for fixed $n \in \Z^+$ the set of rational 
                                 numbers whose denominators are relatively prime 
                                 to $n$ (under addition)
                  \end{enumerate}
                  
      \textbf{Proof.}
      
      \begin{enumerate}
         \item[(c)]
               Fix a positive integer $n$ and consider the set
               $$S = \left\{\frac{a}{b} \in
               \Q : \text{ if } \frac{a'}{b'} = \frac{a}{b},
               \text{ with }\gcd(a', b') = 1, \text{ then } b' \mid n\right\}.$$

               \textbf{Identity.} The set $S$ is not empty since it contains
               $0=\frac{0}{1}$, the identity element.
               
               \textbf{Closure \& Inverse.} Let $x, y \in S$. Then there exist
               integers $a$ and $c$ and nonzero integers $b$ and $d$ such that
               $x = \D\frac{a}{b}$ and $y = \D\frac{c}{d}$. Write these
               rationals in lowest terms, so that
               $\D\frac{a}{b} = \frac{a'}{b'}$ and
               $\D\frac{c}{d} = \frac{c'}{d'}$, for some integers $a'$ and $c'$
               and nonzero integers $b'$ and $d'$, where
               $\gcd(a', b') = \gcd(c', d') = 1$. Let $l$ be the lowest common
               multiple of $b'$ and $d'$.
               By membership of $x$ and $y$ in $S$, we have that $b' \mid n$
               and $d' \mid n$. Thus $l \mid n$. By definition, there exist
               integers $b''$ and $d''$ such that $l = b'b'' = d'd''$; that is,
               $$x - y = \frac{a'}{b'} - \frac{c'}{d'} =
                  \frac{a'b''}{b'b''} - \frac{c'd''}{d'd''} =
               \frac{a'b''-c'd''}{l}.$$
               Let $m = a'b''-c'd''$, and write $\D\frac{m'}{l'} = \frac{m}{l}$,
               for some integer $m'$ such that $\gcd(m', l') = 1$. Hence we
               have that
               $$l\frac{m'}{l'} =m $$
               and since $\gcd(m', l') = 1$, it must be the case that
               $l' \mid l$; since $l \mid n$, it follows by transitivity that
               $l' \mid n$,
               so that
               $$x-y = \frac{a}{b} - \frac{c}{d} = \frac{a'}{b'} -
               \frac{c'}{d'} = \frac{m}{l} =\frac{m'}{l'} \in S.$$
               Thus $S \le \Q$ by the Subgroup Criterion. \qed
         \item[(d)]
               Fix a positive integer $n$ and consider the set
               $$S = \left\{\frac{a}{b} \in
               \Q : \text{ if } \frac{a'}{b'} = \frac{a}{b},
               \text{ with }\gcd(a',b')=1, \text{ then } \gcd(b',n)=1\right\}.$$

               \textbf{Identity.} Since $0 = \frac{0}{1}$ and since
               $\gcd(1, n) = 1$, it follows that $S$ contains 0, the
               identity element 

               \textbf{Closure \& Inverse.} Let $x, y \in S$. Thus there exist
               integers $a$ and $c$ and nonzero integers $b$ and $d$, with
               $\gcd(a, b) = \gcd(c, d) = 1$, such that $x = \D\frac{a}{b}$
               and $y = \D\frac{c}{d}$. Let $l$ be the lowest common multiple of
               $b$ and $d$. By definition of $S$, we have that
               $\gcd(b,n) = \gcd(d ,n) = 1$. Observe that the a prime factor of
               $l$ is either a prime factor of $b$ or a prime factor of $d$. So
               since $b$ and $n$ have no prime factors in common and $d$ and
               $n$ have no prime factors in common, it follows that
               $\gcd(l,n)=1$. Now there exist integers $b'$ and $d'$ such that
               $bb' = dd' = l$. That is, 
               $$x - y = \D\frac{a}{b} - \frac{c}{d} =
                \D\frac{ab'}{bb'} - \frac{cd'}{dd'} = \frac{m}{l},$$ where
               $m = ab' - cd'$. So we now write $\frac{m}{l}$ in lowest terms;
               that is, there exists an integer $m'$ such that
               $\D\frac{m'}{l'} = \frac{m}{l}$ and $\gcd(m', l') = 1$. Since
               $l\frac{m'}{l'} =m $  and since $\gcd(m', l') = 1$, it follows
               that $l' \mid l$, and moreover since $\gcd(l,n)=1$, it follows that
               $\gcd(l',n)=1$, so that
               $$x - y = \frac{a}{b}-\frac{c}{d}=
                 \frac{m}{l} =\frac{m'}{l'} \in S.$$ Thus $S \le \Q$ by the
               subgroup criterion.
      \end{enumerate} \qed
%%%%%%%%%%%%%%%%%%%%%%%%%%%%%%%%%%%%%2.1.5%%%%%%%%%%%%%%%%%%%%%%%%%%%%%%%%%%%%%%
   \item[2.1.5]   Prove that $G$ cannot have a subgroup $H$ with $|H| = n - 1$,
                  where $n = |G| > 2$.

      \textbf{Proof.} Let $G$ be a finite group of order $n > 2$. Now suppose to
      the contrary that $H \le G$, such that $|H| = n - 1$. That is, $H$ contains
      all elements of $G$ but one. Let $y \in G$ be the element not in $H$. Note
      that $y$ cannot be the identity element because all groups must have an
      identity element. Since $|G| \ge 3$, there exists an $x \in H$ such that
      $x \neq 1$. Now consider the element
      $x^{-1}y$. If $x^{-1}y$ is not in $H$, then it must necessarily be the
      case that $x^{-1}y = y$, so that $x = 1$, contrary to our assumption. That
      is, $x^{-1}y \in H$; by closure of $H$, it follows that
      $y = x(x^{-1}y) \in H$, a contradiction. Thus $H$ cannot have size
      $n - 1$. \qed
%%%%%%%%%%%%%%%%%%%%%%%%%%%%%%%%%%%%%2.1.10%%%%%%%%%%%%%%%%%%%%%%%%%%%%%%%%%%%%%
   \item[2.1.10]  \begin{enumerate}
                     \item Prove that if $H$ and $K$ are subgroups of $G$ then
                           so is their intersection $H \cap K$.
                     \item Prove that the intersection of an arbitrary nonempty
                           collection of subgroups of $G$ is again a subgroup of
                           $G$ (do not assume the collection is countable).
                  \end{enumerate}

      \textbf{Proof.}

      \begin{enumerate}
         \item Let $G$ be a group such that $H \le G$ and $K \le G$. Use part
               part (b) with $I = \{1, 2\}$ and $I_1 = H$ and $I_2 = K$.
         \item Let $G$ be a group and $S = \D\bigcap_{i \in I}G_i$ be the
               intersection of an arbitrary collection of subgroups of $G$,
               where $G_i \le G$, for each $i \in I$, and $I$ is some indexing   
               set.
               
               \textbf{Identity.} Since 1 belongs to $G_i$, for all $i \in I$,
               it follows that $1 \in S$.
               
               \textbf{Closure \& Inverse.} Let $x, y \in S$. Then it follows
               that $x \in G_i$ and $y \in G_i$, for all $i \in I$. Now since,
               for every $i \in I$, $G_i$ is a group, it follows by closure
               under the operation and inverses that $G_i$ also contains
               $xy^{-1}$, so that $xy^{-1} \in S$. Thus we conclude by the
               Subgroup Criterion that $S \le G$.
      \end{enumerate} \qed
%%%%%%%%%%%%%%%%%%%%%%%%%%%%%%%%%%%%%2.2.3%%%%%%%%%%%%%%%%%%%%%%%%%%%%%%%%%%%%%%
   \item[2.2.3]   Prove that if $A$ and $B$ are subsets of $G$ with
                  $A \subseteq B$ then $C_G(B)$ is a subgroup of $C_G(A)$.

      \textbf{Proof.} Let $G$ be a group. Suppose that $A$ and $B$ are sets such
      that $A \subseteq B \subseteq G$, with $A \neq \emptyset$. By the
      proof on Page 49 of the textbook, it follows that $C_G(B)$ and $C_G(A)$
      are both subgroups of $G$. So to show that $C_G(B) \le C_G(A)$, it
      suffices to show that $C_G(B) \subseteq C_G(A)$. Let $h \in C_G(B)$ and
      let $r \in A$. Since $A$ is a subset of $B$, it must be the case that
      $r$ is also a member of $B$. That is, $hr = rh$, since $h$ commutes with
      all elements of $B$. Thus $h \in C_G(A)$, and we conclude that
      $C_G(B) \subseteq C_G(A)$, so that $C_G(B) \le C_G(A)$. \qed
%%%%%%%%%%%%%%%%%%%%%%%%%%%%%%%%%%%%%2.2.5%%%%%%%%%%%%%%%%%%%%%%%%%%%%%%%%%%%%%%
   \item[2.2.5]   In each parts (a) to (c) show that for the specified group $G$
                  and subgroup $A$ of $G$, $C_G(A) = A$ and $N_G(A) = G$.
                  \begin{enumerate}
                     \item[(a)]  $G = S_3$ and $A = $ \{1, (1 2 3), (1 3 2)\}.
                     \item[(c)]  $G = D_{10}$ and $A = \{1, r, r^2, r^3, r^4\}$.
                  \end{enumerate}
      
      \textbf{Solution.}
      
      \begin{enumerate}
         \item[(a)] Let $G = S_3$.
               The set $A$ is a cyclic subgroup of $G$ since $A$ is generated by
               (1 2 3). Thus $A$ is a subgroup of $C_G(A)$ because all powers of
               an element of a group commute. But
               $$(1\;2)(1\;2\;3) = (2\;3) \neq (1\;3) = (1\;2\;3)(1\;2),$$
               $$(1\;3)(1\;2\;3) = (1\;2) \neq (2\;3) = (1\;2\;3)(1\;3),$$
               and
               $$(2\;3)(1\;2\;3) = (1\;3) \neq (1\;2) = (1\;2\;3)(2\;3),$$
               so that the transpositions in $G$ do not commute with
               (1 2 3). Thus we have that $C_G(A) = A$. Since the centralizer of
               a set is contained in the normalizer of that same set, it follows
               that $A \le N_G(A)$ so that $|N_G(A)| \ge 3$.
               Now we have to check if the other elements of $G$ normalize $A$.
               Thus
               \begin{align*}
                  (1\;2)A(1\;2)^{-1} &= (1\;2)A(1\;2) \\
                     &= \{(1\;2)(1)(1\;2), (1\;2)(1\;2\;3)(1\;2),
                          (1\;2)(1\;3\;2)(1\;2)\} \\
                     &= \{(1), (1\;3\;2), (1\;2\;3)\} \\
                     &= A. \\ \\
                  (1\;3)A(1\;3)^{-1} &= (1\;3)A(1\;3) \\
                     &= \{(1\;3)(1)(1\;3), (1\;3)(1\;2\;3)(1\;3),
                          (1\;3)(1\;3\;2)(1\;3)\} \\
                     &= \{(1), (1\;3\;2), (1\;2\;3)\} \\
                     &= A. \\ \\
                  (2\;3)A(2\;3)^{-1} &= (2\;3)A(2\;3) \\
                     &= \{(2\;3)(1)(2\;3), (2\;3)(1\;2\;3)(2\;3),
                          (2\;3)(1\;3\;2)(2\;3)\} \\
                     &= \{(1), (1\;3\;2), (1\;2\;3)\} \\
                     &= A.
               \end{align*}
               
               This says that transpositions are also in $N_G(A)$ so that
               $N_G(A) = G$.
         \item[(c)] Let $G = D_{10}$.
               The set $A$ is the cyclic subgroup of rotations of $G$, so
               $A \le C_G(A)$ because the powers of an element of a group
               commute. Since
               $sr \neq sr^4 = rs$; $(sr)r \neq s = r(sr)$;
               $(sr^2)r \neq sr = r(sr^2)$; $(sr^3)r \neq sr^2 = r(sr^3)$;
               $(sr^4)r \neq sr^3 = r(sr^4)$, it follows that $s$, $sr$,
               $sr^2$, $sr^3$, and $sr^4$ do not commute with $r$, so that
               $C_G(A) = A$. Now since $C_G(A) \le N_G(A)$, it follows that
               $r \in N_G(A)$. Now we have that
               \begin{align*}
                  sAs^{-1} &= sAs \\
                     &= \{s1s, srs, sr^2s, sr^3s, sr^4s\} \\
                     &= \{s^2, s^2r^{-1}, s^2r^{-2}, s^2r^{-3}, s^2r^{-4}\} \\
                     &= \{1, r^4, r^3, r^2, r^1\} \\
                     &= A.
               \end{align*}
               This says that $s \in N_G(A)$. That is, $N_G(A)$ contains the
               generators of $G$, namely $r$ and $s$. Hence since $N_G(A) \le G$, it
               follows that $N_G(A) = G$. \qed
      \end{enumerate}
%%%%%%%%%%%%%%%%%%%%%%%%%%%%%%%%%%%%%2.2.8%%%%%%%%%%%%%%%%%%%%%%%%%%%%%%%%%%%%%%
   \item[2.2.8]   Let $G = S_n$, fix an $i \in \{1, 2, \ldots, n\}$ and let
                  $G_i = \{\sigma \in G : \sigma(i) = i\}$ (the stabilizer of
                  $i$ in $G$). Use group actions to prove that $G_i$ is a
                  subgroup of $G$. Find $|G_i|$.

      \textbf{Proof.} Let $G = S_n$ and $I_n = \{1, 2, \ldots, n\}$. Now
      consider the map $f: G \times I_n \rightarrow I_n$ defined by 
      $(g, a) \mapsto g(a)$. We want to first show that $f$ is a group action of
      $G$ on $I_n$. Let $\sigma_1$, $\sigma_2 \in G$ and $a \in I_n$. So it
      follows that $f(1, a) = 1(a) = a$ and
      \begin{align*}
         f(\sigma_1, f(\sigma_2, a)) &= f(\sigma_1, \sigma_2(a)) \\
            &= \sigma_1(\sigma_2(a)) \\
            &= (\sigma_1 \circ \sigma_2)(a) \\
            &= f(\sigma_1 \circ \sigma_2, a),
      \end{align*}
      so that $f$ is a group action. Now fix $i \in I_n$ and consider the set
      $G_i = \{\sigma \in G : \sigma(i) = i\}$. That is, $G_i$ is the stabilizer
      of $i$ in $G$ under $f$; thus $|G_i| \le G$ by Exercise 1.7.4(b). By
      counting we have that $|G_i| = (n - 1)!$ \qed
%%%%%%%%%%%%%%%%%%%%%%%%%%%%%%%%%%%%%2.3.2%%%%%%%%%%%%%%%%%%%%%%%%%%%%%%%%%%%%%%
   \item[2.3.2]   If $x$ is an element of the finite group $G$ and $|x| = |G|$,
                  prove that $G = \cyc{x}$. Give an explicit example to show 
                  that this result need not be true if $G$ is an infinite group.
                  
      \textbf{Proof.} Let $G$ be a finite group, so that $|G| = n \in \Z^+$.
      Suppose that there exists $x \in G$ such that $|x| = n$. Now
      $\cyc{x} \subseteq G$ by the closure property of a group. But we showed in
      class that if $|y| = m \in \Z^+$ for some element $y$ in a group $H$, then the
      cardinality of the group that $y$ generates is $m$. Thus $|\cyc{x}| = n$
      since $|x| = n$; Since $G$ is finite and since $|\cyc{x}| = |G| = n$, it
      follows that $\cyc{x} = G$. If $G$ were an infinite group, say $G = \Z$,
      then we have that $\cyc{2} \neq \Z$, although $|\Z| = |\cyc{2}|$.      \qed
%%%%%%%%%%%%%%%%%%%%%%%%%%%%%%%%%%%%%2.3.4%%%%%%%%%%%%%%%%%%%%%%%%%%%%%%%%%%%%%%
   \item[2.3.4]   Find all generators for $\Z/202\Z$.
   
      \textbf{Solution.} Let $S$ be the set of generators for $\Z/202\Z$. Then
      $|S| = 100$ since
      $$S = \{\overline{x} \in \Z/202\Z: x \text{ is odd and positive},
        x \neq 101, \text{ and } x < 202\}.$$
%%%%%%%%%%%%%%%%%%%%%%%%%%%%%%%%%%%%%2.3.6%%%%%%%%%%%%%%%%%%%%%%%%%%%%%%%%%%%%%%
   \item[2.3.6]   In $\Z/48\Z$ write out all elements of $\cyc{\overline{a}}$
                  for every $\overline{a}$. Find all inclusions between
                  subgroups in $\Z/48\Z$.
      
      \textbf{Solution.}
      $$
         \begin{tabular}{|c|c|} \hline
            \textbf{Generators} & \textbf{Subgroups in} $\Z/48\Z$ \\ \hline
            0 & $\{0\}$ \\ \hline
            24 & $\{0, 24\}$ \\ \hline
            16, 32 & $\{0, 16, 32\}$ \\ \hline
            12, 36 & $\{0, 12, 24, 36\}$ \\ \hline
            8, 40 & $\{0, 8, 16, 24, 32, 40\}$ \\ \hline
            6, 18, 30, 42 & $\{0, 6, 12, 18, 24, 30, 36, 42\}$ \\ \hline
            4,20,28,44 & $\{0,4,8,12,16, 20, 24, 28, 32, 36, 40, 44\}$ \\ \hline
            3, 9, 15, 21, 27, 33, 39, 45 & $\{0, 3, 6, 9, 12, 15, 18, 21, 24,
            27, 30, 33, 36, 39, 42, 45\}$ \\ \hline            
            2, 10, 14, 22, 26, 34, 38, 46 & $\{x : 0 \le x \le 46,
            x \text{ is even}\}$ \\ \hline
            \text{See Below } & $\Z/48\Z$ \\ \hline
         \end{tabular}
      $$
      The generators for $\Z/48\Z$ are: 1, 5, 7, 11, 13, 17, 19, 23, 25, 29, 31,
      35, 37, 41, 43, and 47.
\end{enumerate}
\end{document}
