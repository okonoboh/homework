\documentclass[9pt]{article}

\usepackage{amssymb}
\usepackage{amsmath}
\usepackage{amsfonts}
\usepackage{comment}
\usepackage{fancyhdr}
\usepackage{mathrsfs}
\usepackage{enumitem}
%\usepackage[retainorgcmds]{IEEEtrantools}

\everymath{\displaystyle}

\usepackage{tikz}

\voffset = -50pt
%\textheight = 700pt
\addtolength{\textwidth}{60pt}
\addtolength{\evensidemargin}{-30pt}
\addtolength{\oddsidemargin}{-30pt}
%\setlength{\headheight}{44pt}

\pagestyle{fancy}
\fancyhf{} % clear all fields
\fancyhead[R]{%
  \scshape
  \begin{tabular}[t]{@{}r@{}}
  MATH 540, Fall 2015\\Section 1 (9932)\\
  HW \#6, DUE: 2015, October 08
  \end{tabular}}
\fancyhead[L]{%
  \scshape
  \begin{tabular}[t]{@{}r@{}}
  JOSEPH OKONOBOH\\Mathematics\\Cal State Long Beach
  \end{tabular}}
\fancyfoot[C]{\thepage}

\newcommand{\qed}{\hfill \ensuremath{\Box}}


\newcommand*\circled[1]{\tikz[baseline=(char.base)]{
            \node[shape=circle,draw,inner sep=2pt] (char) {#1};}}


\newcommand{\cyc}[1]{\langle #1 \rangle}
\newcommand{\Z}{\mathbb{Z}}
\newcommand{\C}{\mathbb{C}}
\newcommand{\F}{\mathbb{F}}
\newcommand{\M}{\mathbb{M}}
\newcommand{\R}{\mathbb{R}}
\newcommand{\Q}{\mathbb{Q}}
\everymath{\displaystyle}
\newcommand{\CYC}[1]{\left\langle #1 \right\rangle}
%\setcounter{section}{-1}

\begin{document}
\begin{enumerate}
%%%%%%%%%%%%%%%%%%%%%%%%%%%%%%%%%%%%%3.2.04%%%%%%%%%%%%%%%%%%%%%%%%%%%%%%%%%%%%%
   \item[3.2.4]   Show that if $|G| = pq$ for some primes $p$ and $q$ (not
                  necessarily distinct) then either $G$ is abelian or
                  $Z(G) = 1$. [See Exercise 3.1.36]

      \textbf{Proof.} Let $G$ be a group. Suppose that $|G| = pq$, for some
      primes $p$ and $q$. If $Z(G) = 1$, then we are done, so suppose that
      $Z(G) \neq 1$. In order to complete the proof, it suffices to show that
      $G$ is abelian. We recall that $Z(G) \le G$, so it follows by Lagrange's 
      Theorem that $|Z(G)|$ divides $|G|$. Since $|G| = pq$ and since $p$ and
      $q$ are both primes, it follows that
      $$|Z(G)| \in \{p, q, pq\}.$$
      Observe that we did not include 1 in the set above because we assumed that
      $Z(G) \neq 1$, so that $|Z(G)| > 1$. Now let us investigate the possible
      cases:

      \textbf{Case 1.} $|Z(G)| = pq$. That is, $Z(G)$ must equal $G$, so that
      $G$ is abelian.

      \textbf{Case 2.} $|Z(G)| = p$. So it follows by Theorem 8 that
      $|G/Z(G)| = \frac{pq}{p} = q$. Hence, $G/Z(G)$ is cyclic by Corollary 10, 
      so it follows by Exercise 3.1.36 that $G$ is abelian.

      \textbf{Case 3.} $|Z(G)| = q$. Interchange the roles of $p$ and $q$ in
      Case 2, and conclude that $G$ is abelian.

      We have thus shown that $G$ is abelian if $Z(G)$ is nontrivial, and the
      proof is done. \qed      
%%%%%%%%%%%%%%%%%%%%%%%%%%%%%%%%%%%%%3.2.05%%%%%%%%%%%%%%%%%%%%%%%%%%%%%%%%%%%%%
   \item[3.2.5]   Let $H$ be a subgroup of $G$ and fix some element $g \in G$.
                  \begin{enumerate}
                     \item Prove that $gHg^{-1}$ is a subgroup of $G$ of the
                           same order as $H$.
                     \item Deduce that if $n \in \Z^+$ and $H$ is the unique
                           subgroup of $G$ of order $n$ then
                           $H \trianglelefteq G$.
                  \end{enumerate}

      \textbf{Proof.}

      \begin{enumerate}
         \item The set $gHg^{-1}$ is nonempty because $1 \in H$, so that
               $1 = gg^{-1} = g1g^{-1} \in gHg^{-1}$. Now let
               $x, y \in gHg^{-1}$, so that $x = gh_1g^{-1}$ and
               $y = gh_2g^{-1}$, for some $h_1, h_2 \in H$. We have that
               \begin{align*}
                  xy^{-1} &= gh_1g^{-1}(gh_2g^{-1})^{-1} \\
                     &= gh_1g^{-1}((g^{-1})^{-1}{h_2}^{-1}g^{-1}) \\
                     &= gh_1g^{-1}(g{h_2}^{-1}g^{-1}) \\
                     &= gh_1(g^{-1}g){h_2}^{-1}g^{-1} \\
                     &= g(h_1{h_2}^{-1})g^{-1}.
               \end{align*}
               By the closure $H$, we have that $h_1{h_2}^{-1}$; that is,
               $xy^{-1} = g(h_1{h_2}^{-1})g^{-1} \in gHg^{-1}$, and we conclude
               by the Subgroup Criterion that $gHg^{-1} \le G$. Now consider the
               map
               $$\phi : H \rightarrow gHg^{-1}, \text{ defined by }
                   h \mapsto ghg^{-1}.$$
               To show injectivity of $\phi$, we suppose that
               $\phi(x) = \phi(y)$, for some $x, y \in H$. That is,
               $gxg^{-1} = gyg^{-1}$, so that $x = y$ by left and right 
               cancellations, and thus, $\phi$ is injective. Now let
               $z \in gHg^{-1}$, so that $z = gx'g^{-1}$, for some $x' \in H$. 
               Then since $\phi(x') = gx'g^{-1} = z$, we conclude that $\phi$ is 
               surjective, and thus, $\phi$ is a one to one correspondence 
               between $H$ and $gHg^{-1}$; in order words, $|H| = |gHg^{-1}|$.
               \qed
         \item Suppose that there exists a positive integer $n$ such that
               $H$ is the unique subgroup of $G$ of order $n$. Let $r \in G$ and
               $h \in H$. Thus $rhr^{-1} \in rHr^{-1}$. Now $rHr^{-1} \le G$
               and $|rHr^{-1}| = |H|$ by part (a); but $|H| = n$, so that
               $|rHr^{-1}| = n$, and since $H$ is the unique subgroup of order
               $n$, it must be the case that $rHr^{-1} = H$; thus, since
               $rhr^{-1} \in rHr^{-1}$, we must have that $rhr^{-1} \in H$.
               Since $r$ and $h$ were arbitrary, it follows that
               $rhr^{-1} \in H$ for all $r \in G$ and $h \in H$, so that
               $H \trianglelefteq G$ by Theorem 6 (5).
      \end{enumerate}
      \qed
%%%%%%%%%%%%%%%%%%%%%%%%%%%%%%%%%%%%%3.2.08%%%%%%%%%%%%%%%%%%%%%%%%%%%%%%%%%%%%%
   \item[3.2.8]   Prove that if $H$ and $K$ are finite subgroups of $G$ whose
                  orders are relatively prime then $H \cap K = 1$.

      \textbf{Proof.} Suppose $H$ and $K$ are finite subgroups of some group $G$
      such that
      $$\gcd(|H|, |K|) = 1.$$
      The set $H \cap K$ is nonempty because it is a subgroup of $G$, so let
      $x \in H \cap K$. It follows by definition that $x \in H$ and $x \in K$,
      so, by Corollary 9, $|x|$ divides both $|H|$ and $|K|$, so that $|x|$ must 
      also divide the greatest commond divisor of $|H|$ and $|K|$. Hence
      $|x| \mid 1$, and we conclude that $|x| = 1$. Since the identity element 
      is the unique element of order 1, we conclude that $x = 1$. We have thus 
      shown that if $y \in H \cap K$, then $y = 1$; thus $H \cap K = 1$. \qed
%%%%%%%%%%%%%%%%%%%%%%%%%%%%%%%%%%%%%3.2.12%%%%%%%%%%%%%%%%%%%%%%%%%%%%%%%%%%%%%%
   \item[3.2.12]  Let $H \le G$. Prove that the map $x \mapsto x^{-1}$ sends
                  each left coset of $H$ in $G$ onto a right coset of $H$ and
                  gives a bijection between the set of left cosets and the set
                  of right cosets of $H$ in $G$ (hence the number of left cosets
                  of $H$ in $G$ equals the number of right cosets).

      \textbf{Proof.} Let $L_H$ be the set of left cosets of $H$ in $G$ and
      $R_H$ the set of right cosets of $H$ in $G$. Consider the map
      $\varphi : L_H \rightarrow R_H$, where if $g \in G$, then
      $\varphi(gH) = Hg^{-1}$. First we must show that $\varphi$ is
      well-defined. So suppose that $g_1H = g_2H$ for some $g_1H, g_2H \in L_H$.
      To show that $\varphi$ is well defined, it suffices to show that
      $\varphi(g_1H) = \varphi(g_2H)$; that is, we must show that
      $Hg_1^{-1} = Hg_2^{-1}$. Let $r \in Hg_1^{-1}$, so that $r = h_1g_1^{-1}$
      for some $h_1 \in H$. Since $g_1H = g_2H$, Proposition 4 says that
      $g_1^{-1}g_2 \in H$, so that $g_1^{-1}g_2 = h_2$, for some $h_2 \in H$;
      that is, $g_1^{-1} = h_2g_2^{-1}$. Thus
      $r = h_1g_1^{-1} = (h_1h_2)g_2^{-1}$, so that $r \in Hg_2^{-1}$, and we
      conclude that $Hg_1^{-1} \subseteq Hg_2^{-1}$. Now let $s$ be a member
      of $Hg_2^{-1}$, so that $s = h_3g_2^{-1}$ for some $h_3 \in H$. Recall
      that $g_1^{-1}g_2 = h_2$, so that $g_2^{-1} = h_2^{-1}g_1^{-1}$, and thus
      $s = h_3g_2^{-1} = (h_3h_2^{-1})g_1^{-1}$; hence, $s \in Hg^{-1}$, and it
      follows  that $Hg_2^{-1} \subseteq Hg_1^{-1}$. We have thus shown that
      $Hg_1^{-1} = Hg_2^{-1}$; that is, $\varphi$ is well-defined. To show
      injectivity, we suppose that $\varphi(g_3H) = \varphi(g_4H)$, for some
      $g_3H, g_4H \in L_H$. That is, $Hg_3^{-1} = Hg_4^{-1}$. Let
      $t \in Hg_3^{-1}$, so that $t = h_4g_3^{-1} = h_5g_4^{-1}$, for some
      $h_4, h_5 \in H$. Now the equality $h_4g_3^{-1} = h_5g_4^{-1}$ implies
      that $g_3^{-1}g_4 = h_4^{-1}h_5$, so that $g_3^{-1}g_4 \in H$, and we use
      Proposition 4 to claim that $g_3H = g_4H$, so that $\varphi$ is injective.
      The map $\varphi$ is surjective because if $Hg_5^{-1} \in R_H$, then
      $\varphi(g_5H) = Hg_5^{-1}$. Thus $\varphi$ is bijective and it follows
      that $|L_H| = |R_H|$. \qed
%%%%%%%%%%%%%%%%%%%%%%%%%%%%%%%%%%%%%3.2.16%%%%%%%%%%%%%%%%%%%%%%%%%%%%%%%%%%%%%
   \item[3.2.16]  Use Lagrange's Theorem in the multiplicative group
                  $(\Z/p\Z)^\times$ to prove \textit{Fermat's Little Theorem: }
                  if $p$ is a prime then $a^p \equiv a$ (mod $p$) for all
                  $a \in \Z$.

      \textbf{Proof.} Let $p$ be a prime and $a$ an integer. Suppose first that
      $\gcd(a, p) \neq 1$. Then it follows that $a$ must have at least one
      factor of $p$, so that there exists $k \in \Z$, such that $a = kp$. Thus
      $a^p - a = (kp)^p - kp = p(k^pp^{p-1} - k)$. That is, $p \mid a^p - a$,
      or equivalently, $a^p \equiv a$ (mod $p$). Now suppose that $a$ and $p$
      are relatively prime. It follows by Proposition 0.4 (Page 10) that
      $\overline{a} \in (\Z/p\Z)^\times$. Also recall that if $\varphi$ is the
      Euler's totient function, then $|(\Z/p\Z)^\times| = \varphi(p) = p - 1$.
      Notice that the order of $\overline{a}$ is finite since $(\Z/p\Z)^\times$ 
      is finite; hence $|\overline{a}| = |\cyc{\overline{a}}|$ divides
      $p - 1$ by Lagrange's Theorem; that is, $p - 1 = |\overline{a}|s$ for some 
      integer $s$. Hence,
      $$\overline{a}^{p-1} = \overline{a}^{|\overline{a}|s} =
        (\overline{a}^{|\overline{a}|})^s = \overline{1}^s = \overline{1},$$
      so that
      $$\overline{a^p} = \overline{a}^p = \overline{a}\cdot
        \overline{a}^{p-1} = \overline{a} \cdot \overline{1} = \overline{a}.$$
      Using the equality $\overline{a^p} = \overline{a}$, we conclude that
      $a^p \equiv a$ (mod $p$), as desired. \qed
%%%%%%%%%%%%%%%%%%%%%%%%%%%%%%%%%%%%%3.2.22%%%%%%%%%%%%%%%%%%%%%%%%%%%%%%%%%%%%%
   \item[3.2.22]  Use Lagrange's Theorem in the multiplicative group
                  $(\Z/n\Z)^\times$ to prove \textit{Euler's Theorem: }
                  $a^{\varphi(n)} \equiv 1$ (mod $n$) for every integer $a$
                  relatively prime to $n$, where $\varphi$ denotes Euler's
                  $\varphi$-function.

      \textbf{Proof.} Let $\varphi$ denote the Euler's totient function and $n$
      a positive integer. Suppose $a$ is an integer that is relatively prime to
      $n$. So $\overline{a} \in (\Z/n\Z)^\times$ by Proposition 0.4. Since
      $(\Z/n\Z)^\times$ is finite, it follows that $\cyc{\overline{a}}$ must
      also be finite. By Lagrange's Theorem,
      $|\overline{a}| = |\cyc{\overline{a}}|$ divides
      $|(\Z/n\Z)^\times| = \varphi(n)$. So we can write
      $\varphi(n) = |\overline{a}|s$, for some integer $s$. Thus
      $$\overline{a^{\varphi(n)}} = \overline{a}^{\varphi(n)} =
        \overline{a}^{|\overline{a}|s} = (\overline{a}^{|\overline{a}|})^s =
        \overline{1}^s = \overline{1}.$$
      Using the equality $\overline{a^{\varphi(n)}} = \overline{1}$,
      we conclude that $a^{\varphi(n)} \equiv 1$ (mod $n$), which is what we
      wanted to prove. \qed
%%%%%%%%%%%%%%%%%%%%%%%%%%%%%%%%%%%%%3.4.02%%%%%%%%%%%%%%%%%%%%%%%%%%%%%%%%%%%%%
   \item[3.4.2]   Exhibit all 3 composition series for $Q_8$ and all 7 
                  composition series for $D_8$. List the composition factors in
                  each case.
%%%%%%%%%%%%%%%%%%%%%%%%%%%%%%%%%%%%%3.4.05%%%%%%%%%%%%%%%%%%%%%%%%%%%%%%%%%%%%%
   \item[3.4.5]   Prove that subgroups and quotient groups of a solvable group
                  are solvable.
%%%%%%%%%%%%%%%%%%%%%%%%%%%%%%%%%%%%%3.4.06%%%%%%%%%%%%%%%%%%%%%%%%%%%%%%%%%%%%%
   \item[3.4.6]   Prove part(1) of the Jordan-H\"{o}lder Theorem by induction on
                  $|G|$.
\end{enumerate}
\end{document}
