\documentclass[9pt]{article}

\usepackage{amssymb}
\usepackage{amsmath}
\usepackage{amsfonts}
\usepackage{comment}
\usepackage{fancyhdr}
\usepackage{mathrsfs}
\usepackage{enumitem}
%\usepackage[retainorgcmds]{IEEEtrantools}

\everymath{\displaystyle}

\usepackage{tikz}

\voffset = -50pt
%\textheight = 700pt
\addtolength{\textwidth}{60pt}
\addtolength{\evensidemargin}{-30pt}
\addtolength{\oddsidemargin}{-30pt}
%\setlength{\headheight}{44pt}

\pagestyle{fancy}
\fancyhf{} % clear all fields
\fancyhead[R]{%
  \scshape
  \begin{tabular}[t]{@{}r@{}}
  MATH 540, Fall 2015\\Section 1 (9932)\\
  HW \#6, DUE: 2015, October 08
  \end{tabular}}
\fancyhead[L]{%
  \scshape
  \begin{tabular}[t]{@{}r@{}}
  JOSEPH OKONOBOH\\Mathematics\\Cal State Long Beach
  \end{tabular}}
\fancyfoot[C]{\thepage}

\newcommand{\qed}{\hfill \ensuremath{\Box}}


\newcommand*\circled[1]{\tikz[baseline=(char.base)]{
            \node[shape=circle,draw,inner sep=2pt] (char) {#1};}}


\newcommand{\cyc}[1]{\langle #1 \rangle}
\newcommand{\Z}{\mathbb{Z}}
\newcommand{\C}{\mathbb{C}}
\newcommand{\F}{\mathbb{F}}
\newcommand{\M}{\mathbb{M}}
\newcommand{\R}{\mathbb{R}}
\newcommand{\Q}{\mathbb{Q}}
\everymath{\displaystyle}
\newcommand{\CYC}[1]{\left\langle #1 \right\rangle}
%\setcounter{section}{-1}

\begin{document}
\begin{enumerate}
%%%%%%%%%%%%%%%%%%%%%%%%%%%%%%%%%%%%%3.2.04%%%%%%%%%%%%%%%%%%%%%%%%%%%%%%%%%%%%%
   \item[3.2.4]   Show that if $|G| = pq$ for some primes $p$ and $q$ (not
                  necessarily distinct) then either $G$ is abelian or
                  $Z(G) = 1$. [See Exercise 3.1.36]

      \textbf{Proof.} Let $G$ be a group. Suppose that $|G| = pq$, for some
      primes $p$ and $q$. If $Z(G) = 1$, then we are done, so suppose that
      $Z(G) \neq 1$. In order to complete the proof, it suffices to show that
      $G$ is abelian. We recall that $Z(G) \le G$, so it follows by Lagrange's 
      Theorem that $|Z(G)|$ divides $|G|$. Since $|G| = pq$ and since $p$ and
      $q$ are both primes, it follows that
      $$|Z(G)| \in \{p, q, pq\}.$$
      Observe that we did not include 1 in the set above because we assumed that
      $Z(G) \neq 1$, so that $|Z(G)| > 1$. Now let us investigate the possible
      cases:

      \textbf{Case 1.} $|Z(G)| = pq$. That is, $Z(G)$ must equal $G$, so that
      $G$ is abelian.

      \textbf{Case 2.} $|Z(G)| = p$. So it follows by Theorem 8 that
      $|G/Z(G)| = \frac{pq}{p} = q$. Let $\overline{h}$ be a nonidentity element 
      in $G/Z(G)$. That is, $|\overline{h}| > 1$. By Lagrange's Theorem,
      $|\cyc{\overline{h}}|$ divides $|G/Z(G)|$. Thus $|\cyc{\overline{h}}| = 1$ 
      or $|\cyc{\overline{h}}| = q$. But the former is invalid because
      $|\cyc{\overline{h}}| = |\overline{h}| > 1$, so that
      $|\cyc{\overline{h}}| = q = |G/Z(G)|$, and we conclude
      that $\cyc{\overline{h}} = G/Z(G)$. Hence, $G/Z(G)$ is cyclic, so it
      follows by Exercise 3.1.36 that $G$ is abelian.

      \textbf{Case 3.} $|Z(G)| = q$. Interchange the roles of $p$ and $q$ in
      Case 2, and conclude that $G$ is abelian.

      We have thus shown that $G$ is abelian if $Z(G)$ is nontrivial, and the
      proof is done. \qed      
%%%%%%%%%%%%%%%%%%%%%%%%%%%%%%%%%%%%%3.2.05%%%%%%%%%%%%%%%%%%%%%%%%%%%%%%%%%%%%%
   \item[3.2.5]   Let $H$ be a subgroup of $G$ and fix some element $g \in G$.
                  \begin{enumerate}
                     \item Prove that $gHg^{-1}$ is a subgroup of $G$ of the
                           same order as $H$.
                     \item Deduce that if $n \in \Z^+$ and $H$ is the unique
                           subgroup of $G$ of order $n$ then
                           $H \trianglelefteq G$.
                  \end{enumerate}

      \textbf{Proof.}

      \begin{enumerate}
         \item The set $gHg^{-1}$ is nonempty because $1 \in H$, so that
               $1 = gg^{-1} = g1g^{-1} \in gHg^{-1}$. Now let
               $x, y \in gHg^{-1}$, so that $x = gh_1g^{-1}$ and
               $y = gh_2g^{-1}$, for some $h_1, h_2 \in H$. We have that
               \begin{align*}
                  xy^{-1} &= gh_1g^{-1}(gh_2g^{-1})^{-1} \\
                     &= gh_1g^{-1}((g^{-1})^{-1}{h_2}^{-1}g^{-1}) \\
                     &= gh_1g^{-1}(g{h_2}^{-1}g^{-1}) \\
                     &= gh_1(g^{-1}g){h_2}^{-1}g^{-1} \\
                     &= g(h_1{h_2}^{-1})g^{-1}.
               \end{align*}
               By the closure $H$, we have that $h_1{h_2}^{-1}$; that is,
               $xy^{-1} = g(h_1{h_2}^{-1})g^{-1} \in gHg^{-1}$, and we conclude
               by the Subgroup Criterion that $gHg^{-1} \le G$. Now consider the
               map
               $$\phi : H \rightarrow gHg^{-1}, \text{ defined by }
                   h \mapsto ghg^{-1}.$$
               To show injectivity of $\phi$, we suppose that
               $\phi(x) = \phi(y)$, for some $x, y \in H$. That is,
               $gxg^{-1} = gyg^{-1}$, so that $x = y$ by left and right 
               cancellations, and thus, $\phi$ is injective. Now let
               $z \in gHg^{-1}$, so that $z = gx'g^{-1}$, for some $x' \in H$. 
               Then since $\phi(x') = gx'g^{-1} = z$, we conclude that $\phi$ is 
               surjective, and thus, $\phi$ is a one to one correspondence 
               between $H$ and $gHg^{-1}$; in order words, $|H| = |gHg^{-1}|$.
               \qed
         \item Suppose that there exists a positive integer $n$ such that
               $H$ is the unique subgroup of $G$ of order $n$. Let $r \in G$ and
               $h \in H$. Thus $rhr^{-1} \in rHr^{-1}$. Now $rHr^{-1} \le G$
               and $|rHr^{-1}| = |H|$ by part (a); but $|H| = n$, so that
               $|rHr^{-1}| = n$, and since $H$ is the unique subgroup of order
               $n$, it must be the case that $rHr^{-1} = H$; thus, since
               $rhr^{-1} \in rHr^{-1}$, we must have that $rhr^{-1} \in H$.
               Since $r$ and $h$ were arbitrary, it follows that
               $rhr^{-1} \in H$ for all $r \in G$ and $h \in H$, so that
               $H \trianglelefteq G$ by Theorem 6 (5).
      \end{enumerate}
      \qed
%%%%%%%%%%%%%%%%%%%%%%%%%%%%%%%%%%%%%3.2.08%%%%%%%%%%%%%%%%%%%%%%%%%%%%%%%%%%%%%
   \item[3.2.8]   Prove that if $H$ and $K$ are finite subgroups of $G$ whose
                  orders are relatively prime then $H \cap K = 1$.      
%%%%%%%%%%%%%%%%%%%%%%%%%%%%%%%%%%%%%3.2.12%%%%%%%%%%%%%%%%%%%%%%%%%%%%%%%%%%%%%%
   \item[3.2.12]  Let $H \le G$. Prove that the map $x \mapsto x^{-1}$ sends
                  each left coset of $H$ in $G$ onto a right coset of $H$ and
                  gives a bijection between the set of left cosets and the set
                  of right cosets of $H$ in $G$ (hence the number of left cosets
                  of $H$ in $G$ equals the number of right cosets).
%%%%%%%%%%%%%%%%%%%%%%%%%%%%%%%%%%%%%3.2.16%%%%%%%%%%%%%%%%%%%%%%%%%%%%%%%%%%%%%
   \item[3.2.16]  Use Lagrange's Theorem in the multiplicative group
                  $(\Z/p\Z)^\times$ to prove \textit{Fermat's Little Theorem: }
                  if $p$ is a prime then $a^p \equiv a$ (mod $p$) for all
                  $a \in \Z$.
%%%%%%%%%%%%%%%%%%%%%%%%%%%%%%%%%%%%%3.2.22%%%%%%%%%%%%%%%%%%%%%%%%%%%%%%%%%%%%%
   \item[3.2.22]  Use Lagrange's Theorem in the multiplicative group
                  $(\Z/n\Z)^\times$ to prove \textit{Euler's Theorem: }
                  $a^{\varphi(n)} \equiv 1$ (mod $n$) for every integer $a$
                  relatively prime to $n$, where $\varphi$ denotes Euler's
                  $\varphi$-function.
%%%%%%%%%%%%%%%%%%%%%%%%%%%%%%%%%%%%%3.4.02%%%%%%%%%%%%%%%%%%%%%%%%%%%%%%%%%%%%%
   \item[3.4.2]   Exhibit all 3 composition series for $Q_8$ and all 7 
                  composition series for $D_8$. List the composition factors in
                  each case.
%%%%%%%%%%%%%%%%%%%%%%%%%%%%%%%%%%%%%3.4.05%%%%%%%%%%%%%%%%%%%%%%%%%%%%%%%%%%%%%
   \item[3.4.5]   Prove that subgroups and quotient groups of a solvable group
                  are solvable.
%%%%%%%%%%%%%%%%%%%%%%%%%%%%%%%%%%%%%3.4.06%%%%%%%%%%%%%%%%%%%%%%%%%%%%%%%%%%%%%
   \item[3.4.6]   Prove part(1) of the Jordan-H\"{o}lder Theorem by induction on
                  $|G|$.
\end{enumerate}
\end{document}
