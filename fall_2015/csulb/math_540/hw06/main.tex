\documentclass[9pt]{article}

\usepackage{amssymb}
\usepackage{amsmath}
\usepackage{amsfonts}
\usepackage{comment}
\usepackage{fancyhdr}
\usepackage{mathrsfs}
\usepackage{enumitem}
%\usepackage[retainorgcmds]{IEEEtrantools}

\everymath{\displaystyle}

\usepackage{tikz}

\voffset = -50pt
%\textheight = 700pt
\addtolength{\textwidth}{60pt}
\addtolength{\evensidemargin}{-30pt}
\addtolength{\oddsidemargin}{-30pt}
%\setlength{\headheight}{44pt}

\pagestyle{fancy}
\fancyhf{} % clear all fields
\fancyhead[R]{%
  \scshape
  \begin{tabular}[t]{@{}r@{}}
  MATH 540, Fall 2015\\Section 1 (9932)\\
  HW \#6, DUE: 2015, October 08
  \end{tabular}}
\fancyhead[L]{%
  \scshape
  \begin{tabular}[t]{@{}r@{}}
  JOSEPH OKONOBOH\\Mathematics\\Cal State Long Beach
  \end{tabular}}
\fancyfoot[C]{\thepage}

\newcommand{\qed}{\hfill \ensuremath{\Box}}


\newcommand*\circled[1]{\tikz[baseline=(char.base)]{
            \node[shape=circle,draw,inner sep=2pt] (char) {#1};}}


\newcommand{\cyc}[1]{\langle #1 \rangle}
\newcommand{\Z}{\mathbb{Z}}
\newcommand{\C}{\mathbb{C}}
\newcommand{\F}{\mathbb{F}}
\newcommand{\M}{\mathbb{M}}
\newcommand{\R}{\mathbb{R}}
\newcommand{\Q}{\mathbb{Q}}
\everymath{\displaystyle}
\newcommand{\CYC}[1]{\left\langle #1 \right\rangle}
%\setcounter{section}{-1}

\begin{document}
\begin{enumerate}
%%%%%%%%%%%%%%%%%%%%%%%%%%%%%%%%%%%%%3.2.04%%%%%%%%%%%%%%%%%%%%%%%%%%%%%%%%%%%%%
   \item[3.2.4]   Show that if $|G| = pq$ for some primes $p$ and $q$ (not
                  necessarily distinct) then either $G$ is abelian or
                  $Z(G) = 1$. [See Exercise 3.1.36]
%%%%%%%%%%%%%%%%%%%%%%%%%%%%%%%%%%%%%3.2.05%%%%%%%%%%%%%%%%%%%%%%%%%%%%%%%%%%%%%
   \item[3.2.5]   Let $H$ be a subgroup of $G$ and fix some element $g \in G$.
                  \begin{enumerate}
                     \item Prove that $gHg^{-1}$ is a subgroup of $G$ of the
                           same order as $H$.
                     \item Deduce that if $n \in \Z^+$ and $H$ is the unique
                           subgroup of $G$ of order $n$ then
                           $H \trianglelefteq G$.
                  \end{enumerate}
%%%%%%%%%%%%%%%%%%%%%%%%%%%%%%%%%%%%%3.2.08%%%%%%%%%%%%%%%%%%%%%%%%%%%%%%%%%%%%%
   \item[3.2.8]   Prove that if $H$ and $K$ are finite subgroups of $G$ whose
                  orders are relatively prime then $H \cap K = 1$.      
%%%%%%%%%%%%%%%%%%%%%%%%%%%%%%%%%%%%%3.2.12%%%%%%%%%%%%%%%%%%%%%%%%%%%%%%%%%%%%%%
   \item[3.2.12]  Let $H \le G$. Prove that the map $x \mapsto x^{-1}$ sends
                  each left coset of $H$ in $G$ onto a right coset of $H$ and
                  gives a bijection between the set of left cosets and the set
                  of right cosets of $H$ in $G$ (hence the number of left cosets
                  of $H$ in $G$ equals the number of right cosets).
%%%%%%%%%%%%%%%%%%%%%%%%%%%%%%%%%%%%%3.2.16%%%%%%%%%%%%%%%%%%%%%%%%%%%%%%%%%%%%%
   \item[3.2.16]  Use Lagrange's Theorem in the multiplicative group
                  $(\Z/p\Z)^\times$ to prove \textit{Fermat's Little Theorem: }
                  if $p$ is a prime then $a^p \equiv a$ (mod $p$) for all
                  $a \in \Z$.
%%%%%%%%%%%%%%%%%%%%%%%%%%%%%%%%%%%%%3.2.22%%%%%%%%%%%%%%%%%%%%%%%%%%%%%%%%%%%%%
   \item[3.2.22]  Use Lagrange's Theorem in the multiplicative group
                  $(\Z/n\Z)^\times$ to prove \textit{Euler's Theorem: }
                  $a^{\varphi(n)} \equiv 1$ (mod $n$) for every integer $a$
                  relatively prime to $n$, where $\varphi$ denotes Euler's
                  $\varphi$-function.
%%%%%%%%%%%%%%%%%%%%%%%%%%%%%%%%%%%%%3.4.02%%%%%%%%%%%%%%%%%%%%%%%%%%%%%%%%%%%%%
   \item[3.4.2]   Exhibit all 3 composition series for $Q_8$ and all 7 
                  composition series for $D_8$. List the composition factors in
                  each case.
%%%%%%%%%%%%%%%%%%%%%%%%%%%%%%%%%%%%%3.4.05%%%%%%%%%%%%%%%%%%%%%%%%%%%%%%%%%%%%%
   \item[3.4.5]   Prove that subgroups and quotient groups of a solvable group
                  are solvable.
%%%%%%%%%%%%%%%%%%%%%%%%%%%%%%%%%%%%%3.4.06%%%%%%%%%%%%%%%%%%%%%%%%%%%%%%%%%%%%%
   \item[3.4.6]   Prove part(1) of the Jordan-H\"{o}lder Theorem by induction on
                  $|G|$.
\end{enumerate}
\end{document}
