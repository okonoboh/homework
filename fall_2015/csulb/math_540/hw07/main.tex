\documentclass[9pt]{article}

\usepackage{amssymb}
\usepackage{amsmath}
\usepackage{amsfonts}
\usepackage{comment}
\usepackage{fancyhdr}
\usepackage{mathrsfs}
\usepackage{enumitem}
%\usepackage[retainorgcmds]{IEEEtrantools}

\everymath{\displaystyle}

\usepackage{tikz}

\voffset = -50pt
%\textheight = 700pt
\addtolength{\textwidth}{60pt}
\addtolength{\evensidemargin}{-30pt}
\addtolength{\oddsidemargin}{-30pt}
%\setlength{\headheight}{44pt}

\pagestyle{fancy}
\fancyhf{} % clear all fields
\fancyhead[R]{%
  \scshape
  \begin{tabular}[t]{@{}r@{}}
  MATH 540, Fall 2015\\Section 1 (9932)\\
  HW \#7, DUE: 2015, October 22
  \end{tabular}}
\fancyhead[L]{%
  \scshape
  \begin{tabular}[t]{@{}r@{}}
  JOSEPH OKONOBOH\\Mathematics\\Cal State Long Beach
  \end{tabular}}
\fancyfoot[C]{\thepage}

\newcommand{\qed}{\hfill \ensuremath{\Box}}


\newcommand*\circled[1]{\tikz[baseline=(char.base)]{
            \node[shape=circle,draw,inner sep=2pt] (char) {#1};}}


\newcommand{\cyc}[1]{\langle #1 \rangle}
\newcommand{\Z}{\mathbb{Z}}
\newcommand{\C}{\mathbb{C}}
\newcommand{\F}{\mathbb{F}}
\newcommand{\M}{\mathbb{M}}
\newcommand{\R}{\mathbb{R}}
\newcommand{\Q}{\mathbb{Q}}
\everymath{\displaystyle}
\newcommand{\CYC}[1]{\left\langle #1 \right\rangle}
%\setcounter{section}{-1}

\begin{document}
\begin{enumerate}
   \item[\textbf{Lemma 1.}] \textit{$A_4$ has no subgroup of order 6}.
   
   \textbf{Proof.} Suppose to the contrary that $K \le A_4$ such that $|K| = 6$.
   Thus $|A_4 : K| = 2$, so that, as shown in class, $K$ is normal in $A_4$. Now
   let $\sigma \in A_4$. It follows by Corollary 3.9 that
   $\sigma^2K = (\sigma K)^2 = K$; that is, $\sigma^2 \in K$, and we have that
   $\sigma^2 \in K$ for all $\sigma \in A_4$. Let $\alpha$ be some 3-cycle in
   $A_4$. Since $|\alpha| = 3$, we have that
   $\alpha = \alpha^4 = (\alpha^2)^2 \in K$. Hence every 3-cycle in $A_4$ is
   an element of $K$; but there are 8 3-cyles in $A_4$, so that $|K| \ge 8$,
   a contradiction since we assumed that $|K| = 6$. Thus $A_4$ has no subgroup
   of order 6. \qed
%%%%%%%%%%%%%%%%%%%%%%%%%%%%%%%%%%%%%3.5.10%%%%%%%%%%%%%%%%%%%%%%%%%%%%%%%%%%%%%
   \item[3.5.10]  Find a composition series for $A_4$. Deduce that $A_4$ is
                  solvable.
                  
      \textbf{Solution.} Let
      $$V = \{1, (1\;2)(3\;4), (1\;3)(2\;4), (1\;4)(2\;3)\}.$$
      A few computations will show us that $V$ is closed under multiplication,
      and since $V$ is finite, it follows that $V \le A_4$. We now claim that
      $V$ is the only subgroup of $A_4$ of order 4. If this is not the case,
      then any other subgroup of $A_4$ of order 4, say $V' \neq V$ must contain
      a 3-cycle. This is so since every other element of $A_4$ that is not in
      $V$ is a 3-cycle. But Lagrange's Theorem says that the order of this
      3-cycle must divide $|V'|$; that is, $3 \mid 4$, an impossbility; thus $V$
      is the only subgroup of $A_4$ of order 4. It follows by Exercise 3.2.5(b)
      that $V \trianglelefteq A_4$. Now let $Z = \{1, (1\;2)(3\;4)\} \le V$;
      since $V$ is abelian, all its subgroups must be normal, so that
      $Z \trianglelefteq V$. Since $|Z/{1}|$, $|V/Z|$, and $|A_4/V|$ are all
      prime, it follows that $Z/{1}$, $V/Z$, and $A_4/V$ are all simple groups;
      thus we have the composition series
      $$\{1\} \trianglelefteq Z \trianglelefteq V \trianglelefteq A_4.$$
      By Corollary 3.10, it follows that $Z/{1}$, $V/Z$, and $A_4/V$ are also
      cyclic, and thus, abelian. Hence, using the composition series above, we
      conclude that $A_4$ is solvable.
%%%%%%%%%%%%%%%%%%%%%%%%%%%%%%%%%%%%%3.5.14%%%%%%%%%%%%%%%%%%%%%%%%%%%%%%%%%%%%%
   \item[3.5.14]  Prove that the subgroup of $A_4$ generated by any element of
                  order 2 and any element of order 3 is all of $A_4$.
                  
      \textbf{Proof.} Let $\alpha, \sigma \in A_4$, such that $|\alpha| = 2$
      and $|\sigma| = 3$, and $S = \cyc{\alpha, \sigma}$. By Lagrange's
      Theorem, it follows that
      $$|S| \in \{1, 2, 3, 4, 6, 12\}.$$
      Observe that $\cyc{\sigma} = \{1, \sigma, \sigma^2\} \le S$, where
      $\sigma^2$ is also a 3-cycle; thus $1$, $\sigma$, $\sigma^2$, and
      $\alpha$ are all four different elements in $S$; that is, $|S| \ge 4$.
      Lemma 1 says that $|S| \neq 6$. Thus we are left with two options:
      $|S| = 4$ or $|S| = 12$. Since $|\cyc{\sigma}| = 3$, it follows by
      Lagrange's Theorem that $3$ must divide the order of $S$; that is, we
      must immediately eliminate the option $|S| = 4$. We are now forced to
      conclude that $|S| = 12$. Since $S \le A_4$, and $|S| = |A_4| = 12$, it
      follows that $S = A_4$. Thus an element of order 2 and an element of order
      3 in $A_4$ both generate all of $A_4$. \qed
%%%%%%%%%%%%%%%%%%%%%%%%%%%%%%%%%%%%%3.5.15%%%%%%%%%%%%%%%%%%%%%%%%%%%%%%%%%%%%%
   \item[3.5.15]  Prove that if $x$ and $y$ are distinct 3-cycles in $S_4$ with
                  $x \neq y^{-1}$, then the subgroup of $S_4$ generated by $x$
                  and $y$ is $A_4$.
                  
      \textbf{Proof.} Let $x$ and $y$ be distinct 3-cycles in $S_4$ with
      $x \neq y^{-1}$. We want to show that $\cyc{x, y} = A_4$. First note that
      $x, y \in A_4$, so that $\cyc{x, y} \le A_4$. If we can show that there
      exists $z \in \cyc{x, y}$, with $|z| = 2$, then by Exercise
      3.5.14, we will have $A_4 = \cyc{x, z} \le \cyc{x, y}$, so that
      $\cyc{x, y} = A_4$. Since $x$ and $y$ are distinct 3-cycles and
      $x \neq y^{-1}$, it follows that $x$ and $y$ must have exactly two
      elements of $\{1, 2, 3, 4\}$ in common. Let $a$, $b$, $c$, and $d$ be
      different members of $\{1, 2, 3, 4\}$. Assume without loss of generatlity
      that $x$ and $y$ have $a$ and $b$ in common. Let us now investigate the
      following cases. 
      
      \textbf{Case 1.} $x = (a \quad b \quad c)$ and $y = (a \quad b \quad d)$.
      So let $z_1 = xy$.
      
      \textbf{Case 2.} $x = (a \quad b \quad c)$ and $y = (a \quad d \quad b)$.      
      So let $z_2 = xy^{-1}$.
      
      \textbf{Case 3.} $x = (a \quad c \quad b)$ and $y = (a \quad b \quad d)$.      
      So let $z_3 = x^{-1}y$.
      
      \textbf{Case 4.} $x = (a \quad c \quad b)$ and $y = (a \quad d \quad b)$.      
      So let $z_4 = x^{-1}y^{-1}$.
      
      In each case we have that $z_i =  (a \quad c)(b \quad d) \in \cyc{x, y}$
      has order 2, so it follows by our previous argument that
      $A_4 = \cyc{x, y}$. \qed
%%%%%%%%%%%%%%%%%%%%%%%%%%%%%%%%%%%%%ExamB3%%%%%%%%%%%%%%%%%%%%%%%%%%%%%%%%%%%%%
   \item[Exam B3] Let $G$ be a group, $x, y \in G$, $N = \cyc{x, y}$. Suppose
                  further that for all $g \in G$, $gxg^{-1} \in N$ and
                  $gyg^{-1} \in N$. Prove that $N \trianglelefteq G$.
                  
      \textbf{Proof.} Let $g \in G$ and $n \in N$. If $n$ is the identity, then
      $gng^{-1} = 1 \in N$. So assume that $n \neq 1$; thus
      $$n = a_1^{\epsilon_1}a_2^{\epsilon_2}\cdots a_m^{\epsilon_m},$$
      where $a_i \in \{x, y\}$, $\epsilon_i \in \{\pm1\}$ and $m \in \Z^+$. It
      follows that
      \begin{align*}
         gng^{-1} &= g(a_1^{\epsilon_1}a_2^{\epsilon_2}\cdots
            a_m^{\epsilon_m})g^{-1} \\
            &= (ga_1^{\epsilon_1}g^{-1})(ga_2^{\epsilon_2}g^{-1})\cdots
            (ga_m^{\epsilon_m}g^{-1}) &[\text{Apply Exercise 3.1.26(a) } m
               \text{ times}] \\
            &= (ga_1g^{-1})^{\epsilon_1}(ga_2g^{-1})^{\epsilon_2}\cdots
            (ga_mg^{-1})^{\epsilon_m} &[\text{Exercise 3.1.26(b)}]
      \end{align*}
      By hypothesis, we have that $ga_ig^{-1} \in N$; thus, it follows by
      closure that $(ga_ig^{-1})^{\epsilon_i} \in N$, and, by extension, we have
      $gng^{-1} = (ga_1g^{-1})^{\epsilon_1}(ga_2g^{-1})^{\epsilon_2}\cdots
            (ga_mg^{-1})^{\epsilon_m} \in N$. Hence $gng^{-1} \in N$ for all
      $g \in G$ and $n \in N$, so we conclude by Theorem 6 (5) that
      $N \trianglelefteq G$. \qed
%%%%%%%%%%%%%%%%%%%%%%%%%%%%%%%%%%%%%ExtraC%%%%%%%%%%%%%%%%%%%%%%%%%%%%%%%%%%%%%
   \item[Extra Credit]  Prove part (1) of the Jordan-H$\ddot{\text{o}}$lder Theorem by induction
                        on $|G|$.
\end{enumerate}
\end{document}
