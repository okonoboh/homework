\documentclass[9pt]{article}

\usepackage{amssymb}
\usepackage{amsmath}
\usepackage{amsfonts}
\usepackage{comment}
\usepackage{fancyhdr}
\usepackage{mathrsfs}
\usepackage{enumitem}
%\usepackage[retainorgcmds]{IEEEtrantools}


\usepackage{tikz}

\voffset = -50pt
%\textheight = 700pt
\addtolength{\textwidth}{60pt}
\addtolength{\evensidemargin}{-30pt}
\addtolength{\oddsidemargin}{-30pt}
%\setlength{\headheight}{44pt}

\pagestyle{fancy}
\fancyhf{} % clear all fields
\fancyhead[R]{%
  \scshape
  \begin{tabular}[t]{@{}r@{}}
  MATH 364A, Fall 2015\\Section 1 (2226)\\
  Chapter 2.3, DUE: N/A
  \end{tabular}}
\fancyhead[L]{%
  \scshape
  \begin{tabular}[t]{@{}r@{}}
  JOSEPH OKONOBOH\\Mathematics\\Cal State Long Beach
  \end{tabular}}
\fancyfoot[C]{\thepage}

\newcommand{\qed}{\hfill \ensuremath{\Box}}


\newcommand*\circled[1]{\tikz[baseline=(char.base)]{
            \node[shape=circle,draw,inner sep=2pt] (char) {#1};}}

\newcommand{\Z}{\mathbb{Z}}
\newcommand{\I}{\mathbb{I}}
\newcommand{\M}{\mathbb{M}}
\newcommand{\R}{\mathbb{R}}
\newcommand{\D}{\displaystyle}
%\setcounter{section}{-1}

\begin{document}
\begin{enumerate}
%%%%%%%%%%%%%%%%%%%%%%%%%%%%%%%%%%%%%%%01%%%%%%%%%%%%%%%%%%%%%%%%%%%%%%%%%%%%%%%
   \item[1.]   Consider a tank used in certain hydrodynamic experiments. After
               one experiment the tank contains 200 $L$ of a dye solution with a
               concentration of 1 $g/L$. To prepare for the next experiment, the  
               tank is to be rinsed with fresh water flowing in at a rate of 2
               $L$/min, the well-stirred solution flowing out at the same rate. 
               Find the time that will elapse before the concentration of dye in 
               the tank reaches 1\% of its original value.
               
      \textbf{Solution.} Let $Q(t)$ be the amount (in grams) of dye solution
      after $t$ minutes. Thus it follows that
      \begin{align*}
         Q(0) &= \frac{200\;L}{1} \cdot \frac{1\;g}{L} = 200\;g \\
         \text{Rate of dye in}&=\frac{0\;g}{L}\cdot\frac{2\;L}{\text{min}}=0 \\
         \text{Rate of dye out}&=\frac{Q(t)}{200\;L}\cdot\frac{2\;L}{\text{min}}
            = \frac{Q(t)}{100}.
      \end{align*}
      
      Since
      $$\frac{dQ}{dt} = \text{Rate of dye in} - \text{Rate of dye out},$$
      it follows that
      $$\frac{dQ}{dt} = 0 - \frac{Q(t)}{100} \Rightarrow$$
      $$\frac{dQ}{dt} + \frac{Q(t)}{100} = 0.$$
      Multiply the above equation by the integrating factor,
      $\mu(t) = e^{\int(1/100)\;dt} = e^{(1/100)t}$, and our differential
      equation becomes
      $$\frac{d}{dt}\left(Q(t)e^{(1/100)t}\right) = 0.$$
      Taking the integral of the above equation and using the initial condition
      $Q(0) = 200$ will give us
      $$Q(t) = 200e^{-(1/100)t}.$$
      
      Let $t_2$ be the time in minutes after which concentration of the dye
      reaches 1\% of its original value. Thus $Q(t_2) = 2$. Solving this
      equation will give us
      $$t_2 = 200\ln(10) \approx 461 \text{ minutes}.$$
      
%%%%%%%%%%%%%%%%%%%%%%%%%%%%%%%%%%%%%%%02%%%%%%%%%%%%%%%%%%%%%%%%%%%%%%%%%%%%%%%
   \item[2.]   A tank initially contains 120 L of pure water. A mixture
               containing a concentration of $\gamma$ g/L of salt enters the
               tank at a rate of 2 L/min, and the well-stirred mixture leaves
               the tank at the same rate. Find an expression in terms of
               $\gamma$ for the amount of salt in the tank at any time $t$. Also
               find the limiting amount of salt in the tank as
               $t \rightarrow \infty$.
               
      \textbf{Solution.} Let $Q(t)$ be the amount (in grams) of salt in the tank
      after $t$ minutes. Thus it follows that
      \begin{align*}
         Q(0) &= 0 \\
         \text{Rate of dye in}&=\frac{\gamma\;g}{L}\cdot\frac{2\;L}
            {\text{min}}=2\gamma \\
         \text{Rate of dye out}&=\frac{Q(t)}{120\;L}\cdot\frac{2\;L}{\text{min}}
            = \frac{Q(t)}{60}.
      \end{align*}
      
      Since
      $$\frac{dQ}{dt} = \text{Rate of salt in} - \text{Rate of salt out},$$
      it follows that
      $$\frac{dQ}{dt} = 2\gamma - \frac{Q(t)}{60} \Rightarrow$$
      $$\frac{dQ}{dt} + \frac{Q(t)}{60} = 2\gamma.$$
      Solving with integrating factor $e^{(1/60)t}$ and the initial condition
      $Q(0) = 0$ will give us
      $$120\gamma(1 - e^{-(1/60)t}).$$
      That is, the limiting amount of salt is $120\gamma$.
%%%%%%%%%%%%%%%%%%%%%%%%%%%%%%%%%%%%%%%04%%%%%%%%%%%%%%%%%%%%%%%%%%%%%%%%%%%%%%%
   \item[4.]   A tank with a capacity of 500 gal originally contains 200 gal of
               water with 100 lb of salt in solution. Water containing 1 lb of
               salt per gallon is entering at a rate of 9 gal/min, and the
               mixture is allowed to flow out of the tank at a rate of
               6 gal/min. Find the amount of salt in the tank at any time prior
               to the instant when the solution begins to overflow. Find the
               concentration (in pounds per gallon) of salt in the tank when it
               is on the point of overflowing. Compare this concentration with
               the theoretical limiting concentration if the tank had infinite
               capacity.

      \textbf{Solution.} Let $Q(t)$ be the amount of salt after $t$ minutes. It
      follows that
      \begin{align*}
         \frac{dQ}{dt} &= \text{rate of salt in} - \text{rate of salt out} \\
            &= \frac{1 \text{ lb}}{1 \text{ gal}} \cdot
               \frac{9 \text{ gal}}{1 \text{ min}} -
               \frac{Q(t)}{200 + 3t} \cdot
               \frac{6 \text{ gal}}{1 \text{ min}}.
      \end{align*}
      That is,
      $$\frac{dQ}{dt} + \frac{6Q(t)}{200+3t} = 9.$$
      Multiplying the by the integrating factor $e^{\int6 dt/(200+3t)}$,
      solving, and then using the initial condition $Q(0) = 100$ will yield
      $$Q(t) = 200 + 3t - \frac{4000000}{(200 + 3t)^2},$$
      which is the amount of salt prior to the time before the tank starts to
      overflow. The volume of the solution becomes 500 at $t = 100$, so the
      concentration of the salt at this point is
      $$\frac{Q(100)}{500} = 0.968 \text{ lb/gal}.$$
      The theoretical limiting concentration if the tank had infinite capacity
      is
      $$\lim_{t\rightarrow\infty}\frac{Q(t)}{200+3t} = 1\text{ lb/gal}.$$
%%%%%%%%%%%%%%%%%%%%%%%%%%%%%%%%%%%%%%%10%%%%%%%%%%%%%%%%%%%%%%%%%%%%%%%%%%%%%%%
   \item[10.]  A home buyer can afford to spend no more than \$1500/month on
               mortgage payments. Suppose that the interest rate is 6\%, that
               interest is compounded continuously, and that payments are also
               made continuously.

               \begin{enumerate}
                  \item Determine the maximum amount that this buyer can afford
                        to borrow on a 20-year mortage; on a 30-year mortage.
                  \item Determine the total interest paid during the term of the
                        mortgage in each of the cases in part (a).
               \end{enumerate}

      \textbf{Solution.}

      \begin{enumerate}
         \item Let $S(t)$ be the amount of the loan that is left after $t$
               years. Mortage payments of \$1500/month = \$18000/year. So our
               differential equation is
               $$\frac{dS}{dt} = rS - k = \frac{6S}{100} - 18000,$$
               so that
               $$\frac{dS}{dt} - \frac{6S}{100} = - 18000.$$
               Multiply by the integrating factor $e^{\int-(6/100)dt}$ and solve 
               to get
               $$S(t) = 300000 + Ce^{0.06t}.$$
               The amount that the buyer can afford is $S(0)$, so we use the
               initial conditions $S(20) = 0$ and $S(30) = 0$ for the 20-year
               and 30-year mortgages respectively. Thus we get that
               \$209641.74 and \$250410.33 for the 20-year and 30-year
               mortgages.
         \item Interest for the 20-year mortgage is: 
               $$\$18000 \cdot 20 - \$209641.74 = \$150358.26$$
               and interest for the 30-year mortgage is:
               $$\$18000 \cdot 30 - \$250410.33 = \$289589.67.$$

      \end{enumerate}
%%%%%%%%%%%%%%%%%%%%%%%%%%%%%%%%%%%%%%%21%%%%%%%%%%%%%%%%%%%%%%%%%%%%%%%%%%%%%%%
   \item[21.]  Assume that the conditions are as in Problem 20 except there is a
               force due to air resistance of magnitude $|v|/30$ directed
               opposite to the velocity, where the velocity $|v|$ is measured in
               m/s.

               \begin{enumerate}
                  \item Find the maximum height above the ground that the ball
                        reaches.
               \end{enumerate}
%%%%%%%%%%%%%%%%%%%%%%%%%%%%%%%%%%%%%%%23%%%%%%%%%%%%%%%%%%%%%%%%%%%%%%%%%%%%%%%
   \item[23.]  A skydiver weighing 180 lb (including equipment) falls vertically
               downward from an altitude of 5000 ft and opens the parachute
               after 10s of free fall. Assume that the force of air resistance,
               which is directed opposite to the velocity, is of magnitude
               $0.75|v|$ when the parachute is closed and is of magnitude
               $12|v|$ when the parachute is open, where the velocity $v$ is
               measured in ft/s.

               \begin{enumerate}
                  \item Find the speed of the skydiver when the parachute opens.
                  \item Find the distance fallen before the parachute opens.
                  \item What is the limiting velocity $v_L$ after the parachute
                        opens?
               \end{enumerate}
%%%%%%%%%%%%%%%%%%%%%%%%%%%%%%%%%%%%%%%26%%%%%%%%%%%%%%%%%%%%%%%%%%%%%%%%%%%%%%%
   \item[26.]  A body of mass $m$ is projected vertically upward with an initial
               velocity $v_0$ in a medium offering a resistance $k|v|$, where
               $k$ is a constant. Assume that the gravitational attraction of
               the earth is constant.

               \begin{enumerate}
                  \item Find the velocity $v(t)$ of the body at any time.
               \end{enumerate}
      
\end{enumerate}
\end{document}
