In Problems 2, 4, and 7, solve the given differential equation.
\begin{enumerate}
%%%%%%%%%%%%%%%%%%%%%%%%%%%%%%%%%%%%%2.2.2%%%%%%%%%%%%%%%%%%%%%%%%%%%%%%%%%%%%%%
   \item[2.2.2]   $y' = \frac{x^2}{y(1 + x^3)}$.

      \textbf{Solution.} Given the separable differential equation
      $$\frac{dy}{dx} = \frac{x^2}{y(1+x^3)},$$
      it follows that
      $y\;dy = \frac{x^2}{1+x^3} dx$. Integrate to get
      $\frac{y^2}{2} = \frac{1}{3}\ln|1 + x^3| + C$. That is,
      $$y(x) = \pm\sqrt{\frac{2}{3}\ln|1 + x^3| + C'}.$$
%%%%%%%%%%%%%%%%%%%%%%%%%%%%%%%%%%%%%2.2.4%%%%%%%%%%%%%%%%%%%%%%%%%%%%%%%%%%%%%%
   \item[2.2.4]   $y' = \frac{3x^2 - 1}{3 + 2y}$.

      \textbf{Solution.} We have
      $$\frac{dy}{dx} = \frac{3x^2 - 1}{3 + 2y}.$$
      That is, $(3 + 2y)\;dy = (3x^2 - 1)\;dx$. Integrate to get
      $3y + y^2 = x^3 - x + C$. Using the quadratic formula, it follows that
      $$y(x) = \frac{-3 \pm \sqrt{4x^3 - 4x + C'}}{2}$$
%%%%%%%%%%%%%%%%%%%%%%%%%%%%%%%%%%%%%2.2.7%%%%%%%%%%%%%%%%%%%%%%%%%%%%%%%%%%%%%%
   \item[2.2.7]   $\frac{dy}{dx} = \frac{x - e^{-x}}{y + e^y}$.

      \textbf{Solution.} Separate the equation so that
      $(y + e^y)\;dy = (x - e^{-x})\;dx$, and integrate to get
      $\frac{y^2}{2} + e^y = \frac{x^2}{2} + e^{-x} + C$. Multiply by 2 to get
      $y^2 + 2e^y = x^2 + 2e^{-x} + C'$. No further simplification is possible.
\end{enumerate}

In Problems 9, 10, 12, 14, 15, 16, and 20, find the solution of the given
initial value problem in explicit form and determine (at least approximately)
the interval in which the solution is defined.
\begin{enumerate}
%%%%%%%%%%%%%%%%%%%%%%%%%%%%%%%%%%%%%2.2.9%%%%%%%%%%%%%%%%%%%%%%%%%%%%%%%%%%%%%%
   \item[2.2.9]   $y' = (1 - 2x)y^2, \qquad y(0) = -1/6$.

      \textbf{Solution.} Separate the variables to get
      $\frac{1}{y^2}dy = (1 - 2x)\;dx \quad (y \neq 0)$. Integrate to get
      $-\frac{1}{y} = x - x^2 + C$. That is, $y(x) = \frac{1}{x^2 - x + C'}$.
      Use the initial condition to now conclude that
      $y(x) = \frac{1}{x^2 - x - 6}$. For $y$ to be defined, its denominator
      must be nonzero, so this solution is defined on the intervals
      $(-\infty, -2)$, $(-2, 3)$, and $(3, \infty)$. But since the initial
      condition (where $x = 0$) is contained only in the interval $(-2, 3)$, it
      follows that $y(x)$ is defined  in the interval $(-2, 3)$.
%%%%%%%%%%%%%%%%%%%%%%%%%%%%%%%%%%%%%2.2.10%%%%%%%%%%%%%%%%%%%%%%%%%%%%%%%%%%%%%
   \item[2.2.10]  $y' = \frac{1-2x}{y} \qquad y(1) = -2$.

      \textbf{Solution.} Separate and integrate to obtain
      $y(x) = \pm\sqrt{2x-2x^2+C}$. Use initial condition to conclude that
      $y(x) = -\sqrt{2x-2x^2+4} = -\sqrt{-2(x+1)(x-2)}$. By observing the
      differential equation, we see that $y \neq 0$. Thus the interval in which
      the solution is defined is $(-1, 2)$.
%%%%%%%%%%%%%%%%%%%%%%%%%%%%%%%%%%%%%2.2.12%%%%%%%%%%%%%%%%%%%%%%%%%%%%%%%%%%%%%
   \item[2.2.12]  $\frac{dr}{d\theta} = \frac{r^2}{\theta}, \quad r(1) = 2$.

      \textbf{Solution.} Separate the variables, integrate and use the initial
      condition to get
      $$r(\theta) = \frac{-2}{2\ln|\theta| - 1}.$$
      The interval on which the solution is defined is $(0, \sqrt{e})$.
%%%%%%%%%%%%%%%%%%%%%%%%%%%%%%%%%%%%%2.2.14%%%%%%%%%%%%%%%%%%%%%%%%%%%%%%%%%%%%%
   \item[2.2.14]  $y' = xy^3(1+x^2)^{-1/2}, \qquad y(0) = 1$.

      \textbf{Solution.} By separating the variables and integrating, we shall
      get
      $$y = \pm\sqrt{\frac{1}{C - 2\sqrt{1+x^2}}}.$$
      Use the initial condition to conclude that
      $$y = \sqrt{\frac{1}{3 - 2\sqrt{1+x^2}}}.$$
      The interval on which the solution is defined is thus
      $(-\sqrt{5}/2, \sqrt{5}/2)$.
%%%%%%%%%%%%%%%%%%%%%%%%%%%%%%%%%%%%%2.2.15%%%%%%%%%%%%%%%%%%%%%%%%%%%%%%%%%%%%%
   \item[2.2.15]  $y' = \frac{2x}{1+2y}, \quad y(2) = 0$.

      \textbf{Solution.} Separate the variable and solve to get
      $y^2 + y = x^2 + C$. Now use the quadratic formula to conclude that
      $$y(x) = \frac{-1 \pm \sqrt{4x^2 + C'}}{2}$$
      and that
      $$y(x) = \frac{-1 + \sqrt{4x^2 - 15}}{2}$$
      by the initial condition. From the differential equation, we observe that
      $1 + 2y \neq 0$, so that $y \neq -1/2$. That is, for our solution, we must
      have that $\sqrt{4x - 15} > 0$. So the interval of our solution is
      $(\sqrt{15}/2, \infty)$.
%%%%%%%%%%%%%%%%%%%%%%%%%%%%%%%%%%%%%2.2.16%%%%%%%%%%%%%%%%%%%%%%%%%%%%%%%%%%%%%
   \item[2.2.16]  $y' = \frac{x(x^2 + 1)}{4y^3}, \qquad
                        y(0) = -\frac{1}{\sqrt{2}}$.


      \textbf{Solution.} After solving we shall get
      $$y(x) = \pm\left(\frac{x^4}{4} + \frac{x^2}{2} + C\right)^{1/4},$$
      and, thus,
      $$y(x) = -\left(\frac{x^4}{4} + \frac{x^2}{2} +\frac{1}{4}\right)^{1/4} =
        -\left[\left(\frac{x^2}{2} + \frac{1}{2}\right)^2\right]^{1/4} = 
         -\left(\frac{x^2}{2} + \frac{1}{2}\right)^{1/2},$$
      by the initial condition. The solution is defined on all the real
      numbers.
%%%%%%%%%%%%%%%%%%%%%%%%%%%%%%%%%%%%%2.2.20%%%%%%%%%%%%%%%%%%%%%%%%%%%%%%%%%%%%%
   \item[2.2.20]  $y^2(1-x^2)^{1/2}\;dy = \arcsin(x)\;dx, \quad y(0) = 1$.

      \textbf{Solution.} Separate the variables and use the initial condition to
      get:
      $$y(x) = \left[\frac{3}{2}(\arcsin(x))^2 + 1\right]^{1/3}.$$
      The interval in which the solution is defined is thus $-1 < x < 1$.
\end{enumerate}
