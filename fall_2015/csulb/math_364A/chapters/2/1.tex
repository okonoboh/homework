In Problems 1, 3, 7, 8, and 11, find the general solution of the differential
equation, and use it to determine how solutions behave as
$t \rightarrow \infty$.
\begin{enumerate}
%%%%%%%%%%%%%%%%%%%%%%%%%%%%%%%%%%%%%2.1.1%%%%%%%%%%%%%%%%%%%%%%%%%%%%%%%%%%%%%%
   \item[2.1.1]   $y' + 3y = t + e^{-2t}$.

      \textbf{Solution.} Multiply the equation above by the integrating factor
      $e^{\int 3\;dt} = e^{3t}$ to get
      $$e^{3t}y' + 3e^{3t}y = te^{3t} + e^t,$$
      so that
      $$(ye^{3t})' = te^{3t} + e^t.$$
      Differentiate the preceding equation to get
      $$ye^{3t} = \int (te^{3t} + e^t)\;dt = \frac{1}{9}e^{3t}(3t - 1) + e^t +
        C.$$
      That is, $y(t) = \frac{1}{9}(3t - 1) + e^{-2t} + Ce^{-3t}$, and
      $y \rightarrow \infty$ as $t \rightarrow \infty$.
%%%%%%%%%%%%%%%%%%%%%%%%%%%%%%%%%%%%%2.1.3%%%%%%%%%%%%%%%%%%%%%%%%%%%%%%%%%%%%%%
   \item[2.1.3]   $y' + y = te^{-t} + 1$.

      \textbf{Solution.} The integrating factor is $e^{\int dt} = e^t$. Now
      multiply the differential equation by $e^t$ to get
      $$e^ty' + e^ty = t + e^t,$$
      so that
      $$(e^ty)' = t + e^t.$$
      That is $e^ty = \int(t + e^t) = \frac{1}{2}t^2 + e^t + C$, and we conclude
      that $y(t) = \frac{1}{2}e^{-t}(t^2 + C') + 1$, and $y \rightarrow 1$ as
      $t \rightarrow \infty$.
%%%%%%%%%%%%%%%%%%%%%%%%%%%%%%%%%%%%%2.1.7%%%%%%%%%%%%%%%%%%%%%%%%%%%%%%%%%%%%%%
   \item[2.1.7]   $y' + 2ty = 2te^{-t^2}$.

      \textbf{Solution.} The integrating factor is
      $e^{\int 2t\;dt} = e^{t^2}$. Multiply by the integrating factor
      to transform the equation into
      $$(e^{t^2}y)' = 2t.$$
      Integrate and simplify to get $y(t) = e^{-t^2}(t^2 + C)$, and thus,
      $y \rightarrow 0$ as $t \rightarrow \infty$.
%%%%%%%%%%%%%%%%%%%%%%%%%%%%%%%%%%%%%2.1.8%%%%%%%%%%%%%%%%%%%%%%%%%%%%%%%%%%%%%%
   \item[2.1.8]   $(1 + t^2)y' + 4ty = (1 + t^2)^{-2}$.

      \textbf{Solution.} First we rewrite the equation in standard form to get:
      $$y' + \frac{4t}{1+t^2}y = (1 + t^2)^{-3}.$$
      The integrating factor is thus $e^{\int((4t/1+t^2))\;dt} = (1 + t^2)^2$.
      So multiply the standard form by the integrating factor to get
      $$(y(1+t^2)^2)' = (1+t^2)^{-1}.$$
      Now integrate and simplify to get
      $$y(t) = \frac{\arctan(t)}{(1+t^2)^2} + \frac{C}{(1+t^2)^2}.$$
      Thus $y \rightarrow 0$ as $t \rightarrow \infty$.
%%%%%%%%%%%%%%%%%%%%%%%%%%%%%%%%%%%%%2.1.11%%%%%%%%%%%%%%%%%%%%%%%%%%%%%%%%%%%%%
   \item[2.1.11]  $y' + y = 5\sin(2t)$.

      \textbf{Solution.} The integrating factor is $e^{\int dt} = e^t$. So
      multiply the equation by $e^t$ to arrive at
      $$(e^ty)' = 5e^t\sin(2t).$$
      Integrate (use by parts on the right) and divide the result by $e^t$ to
      get
      $$y(t) = \sin(2t) - 2\cos(2t) + Ce^{-t}.$$
      It follows that $y$ oscillates as $t$ approaches infinity.
      
\end{enumerate}

In Problems 13, 15, 16, 17, 18, and 20, find the solution of the given initial
value problem.
\begin{enumerate}
%%%%%%%%%%%%%%%%%%%%%%%%%%%%%%%%%%%%%2.1.13%%%%%%%%%%%%%%%%%%%%%%%%%%%%%%%%%%%%%
   \item[2.1.13]  $y' - y = 2te^{2t}, \quad y(0) = 1$

      \textbf{Solution.} Using the integrating factor
      $e^{\int-t\;dt} = -e^{-t}$, it follows that
      $$y(t) = 2te^{2t} - Ce^t - 2e^{2t}.$$
      Now use the initial condition $y(0) = 1$ to get
      $$y(t) = 2te^{2t} + 3e^t - 2e^{2t}$$
      
%%%%%%%%%%%%%%%%%%%%%%%%%%%%%%%%%%%%%2.1.15%%%%%%%%%%%%%%%%%%%%%%%%%%%%%%%%%%%%%
   \item[2.1.15]  $ty' + 2y = t^2 - t + 1, \quad y(1) = \frac{1}{2},
                                           \quad t > 0$

      \textbf{Solution.} Put the equation in standard form:
      $$y' + \frac{2}{t}y = t - 1 + \frac{1}{t}.$$
      Now use the integrating factor $e^{\int(2/t)\;dt} = t^2$ to solve;
      thus
      $$y(t) = \frac{t^2}{4} - \frac{t}{3} + \frac{1}{2} + \frac{C}{t^2}.$$
      Finally use the initial condition $y(1) = 1/2$ to get
      $$y(t) = \frac{t^2}{4} - \frac{t}{3} + \frac{1}{2} + \frac{1}{12t^2}$$.
%%%%%%%%%%%%%%%%%%%%%%%%%%%%%%%%%%%%%2.1.16%%%%%%%%%%%%%%%%%%%%%%%%%%%%%%%%%%%%%
   \item[2.1.16]  $y' + (2/t)y = (\cos(t))/t^2, \quad y(\pi) = 0, \quad t > 0$

      \textbf{Solution.} The general solution using the integrating factor
      $e^{\int(2/t)\;dt} = t^2$ is
      $$y(t) = \frac{\sin(t)}{t^2} + \frac{C}{t^2},$$
      and, thus, $y(t) = \frac{\sin(t)}{t^2}$, after using the initial
      condition.
%%%%%%%%%%%%%%%%%%%%%%%%%%%%%%%%%%%%%2.1.17%%%%%%%%%%%%%%%%%%%%%%%%%%%%%%%%%%%%%
   \item[2.1.17]  $y' - 2y = e^{2t}, \quad y(0) = 2$

      \textbf{Solution.} Using the initial conditipn and the
      integrating factor $e^{\int-2\;dt} = e^{-2t}$, it follows that
      $$y(t) = te^{2t} + Ce^{2t},$$
      and thus $y(t) = te^{2t} + 2e^{2t}$.
%%%%%%%%%%%%%%%%%%%%%%%%%%%%%%%%%%%%%2.1.18%%%%%%%%%%%%%%%%%%%%%%%%%%%%%%%%%%%%%
   \item[2.1.18]  $ty' + 2y = \sin(t), \quad y(\pi/2) = 1, \quad t > 0$

      \textbf{Solution.} The standard form of the equation is
      $$y' + \frac{2}{t}y = \frac{\sin(t)}{t}.$$
      Now use the integrating factor $e^{\int(2/t)\;dt} = t^2$ to get
      $$y(t) = -\frac{\cos(t)}{t} + \frac{\sin(t)}{t^2} + \frac{C}{t^2},$$
      and thus
      $$y(t) = -\frac{\cos(t)}{t} + \frac{\sin(t)}{t^2} +
                \frac{\pi^2 - 4}{4t^2}$$
      by the initial conditon $y(\pi/2) = 1$.
%%%%%%%%%%%%%%%%%%%%%%%%%%%%%%%%%%%%%2.1.18%%%%%%%%%%%%%%%%%%%%%%%%%%%%%%%%%%%%%
   \item[2.1.20]  $ty' + (t + 1)y = t, \quad y(\ln(2)) = 1, \quad t > 0$

      \textbf{Solution.} The standard form of the equation is
      $$y' + \frac{t+1}{t}y = 1,$$
      so that the integrating factor is $e^{\int(t+1/t)\;dt} = te^t$, and thus,
      $$y(t) = \frac{C}{te^t} - \frac{1}{t} + 1,$$
      so that
      $$y(t) = \frac{2}{te^t} - \frac{1}{t} + 1$$
      after using the initial condition.
\end{enumerate}
