\documentclass[9pt]{article}

\usepackage{amssymb}
\usepackage{amsmath}
\usepackage{amsfonts}
\usepackage{comment}
\usepackage{fancyhdr}
\usepackage{mathrsfs}
\usepackage{enumitem}
%\usepackage[retainorgcmds]{IEEEtrantools}


\usepackage{tikz}

\voffset = -50pt
%\textheight = 700pt
\addtolength{\textwidth}{60pt}
\addtolength{\evensidemargin}{-30pt}
\addtolength{\oddsidemargin}{-30pt}
%\setlength{\headheight}{44pt}

\pagestyle{fancy}
\fancyhf{} % clear all fields
\fancyhead[R]{%
  \scshape
  \begin{tabular}[t]{@{}r@{}}
  MATH 491, Fall 2015\\Section 1 (4847)\\
  HW \#0, DUE: 2015, August 25
  \end{tabular}}
\fancyhead[L]{%
  \scshape
  \begin{tabular}[t]{@{}r@{}}
  JOSEPH OKONOBOH\\Mathematics\\Cal State Long Beach
  \end{tabular}}
\fancyfoot[C]{\thepage}

\newcommand{\qed}{\hfill \ensuremath{\Box}}


\newcommand*\circled[1]{\tikz[baseline=(char.base)]{
            \node[shape=circle,draw,inner sep=2pt] (char) {#1};}}

\newcommand{\Z}{\mathbb{Z}}
\newcommand{\I}{\mathbb{I}}
\newcommand{\M}{\mathbb{M}}
\newcommand{\R}{\mathbb{R}}
\newcommand{\D}{\displaystyle}
%\setcounter{section}{-1}

\begin{document}
\begin{enumerate}
%%%%%%%%%%%%%%%%%%%%%%%%%%%%%%%%%%%%%%%01%%%%%%%%%%%%%%%%%%%%%%%%%%%%%%%%%%%%%%%
   \item Let $f$ be the function defined by $f(x) = x^3 - 49x^2 + 623x - 2015$,
         and let $g(x) = f(x + 5)$. Compute the sum of the roots of $g$.

      \textbf{Solution.} First we observe that $c$ is a root of $f$ if and only
      if $c-5$ is a root of $g$. If $f$ has a rational root, then the Rational
      Root Theorem says that it must be a divisor of $2015$. Since
      $2015 = 5 \cdot 13 \cdot 31 \cdot 403$, it follows that the candidates for 
      the rational roots of $f$ are
      $$\pm1, \pm5, \pm13, \pm31, \pm65, \pm155, \pm403, \text{ and } \pm2015.$$
      Now $f(1) \neq -1440$ and $f(5) = 0$. Thus 5 is a root of $f$. By
      division and further factorization, we get
      $$f(x) = (x - 5)(x^2 -44x + 403) = (x - 5)(x - 13)(x - 31),$$
      so that the roots of $f$ are 5, 13, and 31. Thus the sum of the roots of
      $g$ is
      $$(5 - 5) + (13 - 5) + (31 - 5) = 34.$$
%%%%%%%%%%%%%%%%%%%%%%%%%%%%%%%%%%%%%%%02%%%%%%%%%%%%%%%%%%%%%%%%%%%%%%%%%%%%%%%
   \item Lay six long strings in parallel. Two people (each of whom cannot see
         what the other is doing) work at opposite ends of the bundle of
         strings. Each randomly pairs off the strings at his or her end and
         glues together the ends of each pair. Then stretch the strings out.
         What is the probability that the strings now form a single large loop?

      \textbf{Solution.}
%%%%%%%%%%%%%%%%%%%%%%%%%%%%%%%%%%%%%%%03%%%%%%%%%%%%%%%%%%%%%%%%%%%%%%%%%%%%%%%
   \item Compute the product $\D\cos\left(\frac{\pi}{5}\right)\cos\left(
         \frac{2\pi}{5}\right)\cos\left(\frac{3\pi}{5}\right)\cos\left(
         \frac{4\pi}{5}\right)$.

      \textbf{Solution.}
%%%%%%%%%%%%%%%%%%%%%%%%%%%%%%%%%%%%%%%04%%%%%%%%%%%%%%%%%%%%%%%%%%%%%%%%%%%%%%%
   \item Compute the number of ordered pairs of integers $(x, y)$ such that
         \begin{equation} \label{4_2}
            \frac{1}{x} + \frac{540}{xy} = 2.
         \end{equation}

      \textbf{Solution.} Clearly neither of $x$ and $y$ is 0, so multiply
      Equation \eqref{4_2} by $xy$ to get $y + 540 = 2xy$. Rearrange to get
      \begin{equation} \label{4_1}
         y(2x - 1) = 540.
      \end{equation}
      Equation \eqref{4_1} tells us that $(2x - 1)$ is an odd factor of 540.
      Since $540 = 2^2 \cdot 3^3 \cdot 5$, it follows that the odd factors of
      540 are
      $$\pm1, \pm3, \pm9, \pm27, \pm5, \pm 15, \pm45, \text{ and } \pm135.$$
      That is, $(2x - 1)$ must be one of the factors above, so that Equation
      \eqref{4_1} has 16 ordered pairs of solutions. Now in each of the 16
      solutions the $y$ component cannot be 0 since Equation \eqref{4_1} says
      that the $y$ component is always a factor of 540; however, $(0, -540)$ is
      a solution of Equation \eqref{4_1}, so we shall exclude it. Thus there are
      15 solutions to Equation \eqref{4_2}. Listed explicitly, they are:
      $(-67, -4)$, $(-22, -12)$, $(-13, -20)$, $(-7, -36)$, $(-4, -60)$,
      $(-2, -108)$, $(-1, -180)$, $(1, 540)$, $(2, 180)$, $(3, 108)$, $(5, 60)$, 
      $(8, 36), (14, 20)$, $(23, 12)$, and $(68, 4)$.
%%%%%%%%%%%%%%%%%%%%%%%%%%%%%%%%%%%%%%%05%%%%%%%%%%%%%%%%%%%%%%%%%%%%%%%%%%%%%%%
   \item Compute the smallest positive integer $n$ such that $n + i$,
         $(n + i)^2$, and $(n + i)^3$ are the vertices of a triangle in the
         complex plane whose area is greater than 2015.

      \textbf{Solution.} Let $A(x_1, y_1)$, $B(x_2, y_2)$, and $C(x_3, y_3)$ be
      points in the $xy$-plane; then it can be shown that
      $$\text{Area of } \triangle ABC = \frac{1}{2}\left|\det\left(
        \begin{tabular}{@{}ccc@{}}
           1 & 1 & 1 \\
           $x_1$ & $x_2$ & $x_3$ \\
           $y_1$ & $y_2$ & $y_3$
        \end{tabular}\right)\right|,
      $$
      where $\det(M)$ is the determinant of a matrix $M$. Now suppose
      $$z_1 = a_1 + b_1i, z_2 = a_2 + b_2i, \text{ and } z_3 = a_3 + b_3i$$
      are the vertices of a triangle in the complex plane; also suppose that $s$
      is the area of this triangle. Letting $\overline{z}$ denote the conjugate
      of the complex number $z$ and using the following facts:
      \begin{itemize}
         \item adding a multiple of a row to another does not change the 
               determinant of a matrix and
         \item multiplying a row by a scalar multiplies the determinant by the 
               same scalar,
      \end{itemize}
      it follows that
      \begin{align*}
         s &= \frac{1}{2}\left|\det\left(\begin{tabular}{@{}ccc@{}}
                  1 & 1 & 1 \\
                  $a_1$ & $a_2$ & $a_3$ \\
                  $b_1$ & $b_2$ & $b_3$
              \end{tabular}\right)\right| \\
           &= \frac{1}{2i}\left|\det\left(\begin{tabular}{@{}ccc@{}}
                  1 & 1 & 1 \\
                  $a_1$ & $a_2$ & $a_3$ \\
                  $b_1i$ & $b_2i$ & $b_3i$
              \end{tabular}\right)\right| \\
           &= \frac{1}{2i}\left|\det\left(\begin{tabular}{@{}ccc@{}}
                  1 & 1 & 1 \\
                  $a_1+b_1i$ & $a_2+b_2i$ & $a_3+b_3i$ \\
                  $b_1i$ & $b_2i$ & $b_3i$
              \end{tabular}\right)\right| \\
           &= \frac{1}{-4i}\left|\det\left(\begin{tabular}{@{}ccc@{}}
                  1 & 1 & 1 \\
                  $a_1+b_1i$ & $a_2+b_2i$ & $a_3+b_3i$ \\
                  $-2b_1i$ & $-2b_2i$ & $-2b_3i$
              \end{tabular}\right)\right| \\
           &= \frac{1}{-4i}\left|\det\left(\begin{tabular}{@{}ccc@{}}
                  1 & 1 & 1 \\
                  $a_1+b_1i$ & $a_2+b_2i$ & $a_3+b_3i$ \\
                  $a_1-b_1i$ & $a_2-b_2i$ & $a_3-b_3i$
              \end{tabular}\right)\right| \\
           &= \frac{i}{4}\left|\det\left(\begin{tabular}{@{}ccc@{}}
                  1 & 1 & 1 \\
                  $z_1$ & $z_2$ & $z_3$ \\
                  $\overline{z_1}$ & $\overline{z_2}$ & $\overline{z_3}$
              \end{tabular}\right)\right|.
      \end{align*}

      Now, for a positive integer $n$, let $s_n$ denote the area of a triangle 
      with vertices $n + i$, $(n + i)^2$, and $(n + i)^3$ in the complex plane.
      As previously shown, it follows that
      \begin{align*}
         s_n &= \frac{i}{4}\left|\det\left(\begin{tabular}{@{}ccc@{}}
                   1 & 1 & 1 \\
                   $n+i$ & $(n+i)^2$ & $(n+i)^3$ \\
                   $\overline{n+i}$& $\overline{(n+i)^2}$ & $\overline{(n+i)^3}$
                \end{tabular}\right)\right| \\
             &= \frac{i}{4}\left|\det\left(\begin{tabular}{@{}ccc@{}}
                   1 & 1 & 1 \\
                   $n+i$ & $(n+i)^2$ & $(n+i)^3$ \\
                   $n-i$ & $(n-i)^2$ & $(n-i)^3$
                \end{tabular}\right)\right| \\
            &= \frac{i}{4} \left|(n+i)^2(n-i)^3-(n+i)^3(n-i)^2+ 
                  (n+i)^3(n-i)-(n+i)(n-i)^3+ \right.\\
                  &\quad \left.\quad (n+i)(n-i)^2-(n+i)^2(n-i)\right| \\
            &= \frac{i}{4} \left|(n+i)^2(n-i)^2[(n-i)-(n+i)]+ 
                  (n+i)(n-i)[(n+i)^2-(n-i)^2]+ \right.\\
                  &\quad \left.\quad (n+i)(n-i)[(n-i)-(n+i)]\right| \\
            &= \frac{i}{4} \left|(n+i)^2(n-i)^2(-2i)+ 
                  (n^2+1)(4ni)+ (n^2+1)(-2i)\right| \\
            &= \frac{i}{4} \left|(n^2+1)^2(-2i)+ 
                  (n^2+1)(4ni)+ (n^2+1)(-2i)\right| \\
            &= \frac{1}{4} \left|2(n^2+1)^2-4n(n^2+1)+2(n^2+1)\right| \\
            &= \frac{1}{4} (n^2+1)(2(n^2+1)-4n+2) \\
            &= \frac{1}{2} (n^2+1)(n^2-2n+2).
      \end{align*}
      
      Since $s_8 = 1625 < 2015$ and $s_9 = 2665 > 2015$, it follows that the
      desired answer is $n = 9$.
%%%%%%%%%%%%%%%%%%%%%%%%%%%%%%%%%%%%%%%06%%%%%%%%%%%%%%%%%%%%%%%%%%%%%%%%%%%%%%%
   \item A loop is made connecting rods of lengths 1, 2, \ldots, 90 in that
         order. (The rod of length 90 is connected to the rods of lengths 89 and
         1.) The loop is laid in the shape of an equilateral triangle of
         perimeter 4095. Rods cannot be bent or broken. Compute the sum of the
         shortest rods on each side of the triangle. (For example, the loop with
         rods 1, 2, \ldots, 9 can be arranged into an equilateral triangle
         because 4 + 5 + 6 = 7 + 8 = 9 + 1 + 2 + 3.)
         
      \textbf{Solution.} We shall be using the fact that the $n$-th sum of an
      arithmetic sequence whose first term is $a$ and common difference is $d$
      is $$\frac{n}{2}(2a + (n-1)d).$$
      Suppose the side that contains the rod of length 1 is of the form
      $$1 + 2 + \cdots + k$$
      for some $1 < k < 90$. Thus it follows that
      $$1 + 2 + \cdots + k = \frac{k(k+1)}{2} = 1365,$$
      so that $k^2+k-2730=0$, a quadratic equation with no integral solution.
      Thus no such $k$ exists. We then hypothesize that there exist positive
      integers $n$, $h$, and $m$ such that the sides of the triangle are of the
      following forms (the sides have been arbitrarily labelled):
      
      \textbf{Side 1.}
      $$n + (n + 1) + \cdots + 90 + 1 + 2 + \cdots + h = 1365.$$
      
      \textbf{Side 2.}
      $$(h + 1) + (h + 2) + \cdots + (h + m) = 1365.$$
      
      \textbf{Side 3.}
      $$(h + m + 1) + \cdots (n - 1) = 1365.$$
      
      For side 1, we have
      \begin{align*}
         1365 &= n + (n + 1) + \cdots + 90 + 1 + 2 + \cdots + h \\
              &= \frac{(90 + n)(91 - n)}{2} + \frac{h(h+1)}{2},
      \end{align*}
      so that
      \begin{equation} \label{6_1}
         n^2 - n - (h^2+h+5460) = 0.
      \end{equation}
      For side 2, we have
      $$1365 = (h + 1) + (h + 2) + \cdots + (h + m) = \frac{m}{2}(2h+m+1),$$
      so that
      \begin{equation} \label{6_2}
         m(2h+m+1) = 2730.
      \end{equation}
      Equation \eqref{6_2} tells us that $m$ is a factor of 2730. Now since
      $2730 = 2 \cdot 3 \cdot 5 \cdot 7 \cdot 13$, it follows that there are
      at most 32 pairs of $m$ and $h$ that satisfy Equation \eqref{6_2}.
      Observe that $h$ and $m$ are both less than 90 and $h + m < 90$; hence we
      can whittle down the solutions. A few computations will lead us to the
      following candidates ($m$ is first component and $h$ is second) for the
      solutions of Equation  \eqref{6_2}:
      $$(21, 54), (26, 39), (30, 30), (35, 21), (39, 15), \text{ and }
        (42, 11).$$
      Now if we plug in the values of $h$ above in Equation \eqref{6_1}, we
      shall notice that only $h = 15$ produces an integral value of $n$;
      specifically, we shall get $n = 76$. Thus the sides of our triangle are:
      
      \textbf{Side 1.}
      $$76 + 77 + \cdots + 90 + 1 + 2 + \cdots + 15 = 1365.$$
      
      \textbf{Side 2.}
      $$16 + 17 + \cdots + 54 = 1365.$$
      
      \textbf{Side 3.}
      $$55 + 56 + \cdots + 75 = 1365.$$
      
      Thus the sum of the shortest rods on each side is $1 + 16 + 55 = 72$.
%%%%%%%%%%%%%%%%%%%%%%%%%%%%%%%%%%%%%%%07%%%%%%%%%%%%%%%%%%%%%%%%%%%%%%%%%%%%%%%
   \item Let $a_1 = a_2 = a_3 = 1$. For $n > 3$, let $a_n$ be the number of
         real numbers $x$ such that
         $$x^4 - 2a_{n-1}x^2+a_{n-2}a_{n-3} = 0.$$
         Compute the sum $a_1 + a_2 + a_3 + \cdots + a_{1000}$.
         
      \textbf{Solution.} We can solve any quartic equation of the form
      $$a_1x^4 + a_2x^2 + a_3 = 0$$
      by making the substitution $y = x^2$, and consequently, reducing it to a
      quadratic equation. So now we shall iteratively compute some of the
      terms in the sequence $\{a_n\}$ to see if there is a repetition. From the
      table below,
      $$
         \begin{tabular}{|c|c|c|} \hline
            $n$ & Quartic Polynomial & $a_n = $ \#unique solutions \\ \hline
            4  & $x^4 - 2x^2 + 1$  & 2 \\ \hline
            5  & $x^4 - 4x^2 + 1$  & 4 \\ \hline
            6  & $x^4 - 8x^2 + 2$  & 4 \\ \hline
            7  & $x^4 - 8x^2 + 8$  & 4 \\ \hline
            8  & $x^4 - 8x^2 + 16$ & 2 \\ \hline
            9  & $x^4 - 4x^2 + 16$ & 0 \\ \hline
            10 & $x^4 + 8$         & 0 \\ \hline
            11 & $x^4$             & 1 \\ \hline
            12 & $x^4 - 2x^2$      & 3 \\ \hline
            13 & $x^4 - 6x^2$      & 3 \\ \hline
            14 & $x^4 - 6x^2 + 3$  & 4 \\ \hline
            15 & $x^4 - 8x^2 + 9$  & 4 \\ \hline
            16 & $x^4 - 8x^2 + 12$ & 4 \\ \hline
            17 & $x^4 - 8x^2 + 16$ & 2 \\ \hline
         \end{tabular}
      $$
      we see that the terms $a_5$, $a_6$, $\cdots$, $a_{13}$ repeat. There are 9
      of these terms and their sum is 21. Thus
      \begin{align*}
         \sum_{i=1}^{1000} a_i &= \sum_{i=1}^{4} a_i + \sum_{i=5}^{994} a_i +
            \sum_{i=995}^{1000} a_i \\
            &= 5 + 21 \cdot 110 + 14 \\
            &= 2329.
      \end{align*}
%%%%%%%%%%%%%%%%%%%%%%%%%%%%%%%%%%%%%%%08%%%%%%%%%%%%%%%%%%%%%%%%%%%%%%%%%%%%%%%
   \item Compute the smallest positive integer $n$ such that
         \begin{equation} \label{8_1}
            \sum_{k=0}^n\log_2\left(1 + \frac{1}{2^{2^k}}\right) \ge 1 +
              \log_2\left(\frac{2014}{2015}\right).
         \end{equation}
              
      \textbf{Solution.} Let $n$ be a positive integer. It follows that
      \begin{align*}
         \sum_{k=0}^n\log_2\left(1 + \frac{1}{2^{2^k}}\right) &=
            \sum_{k=0}^n\log_2\left(\frac{2^{2^k}+1}{2^{2^k}}\right) \\
            &= \log_2\left(\frac{3}{2^1}\right) +
               \log_2\left(\frac{5}{2^2}\right) + \cdots +
               \log_2\left(\frac{2^{2^n}+1}{2^{2^n}}\right) \\
            &= \log_2\left(\frac{3\cdot5\cdot17\cdots(2^{2^n}+1)}
                  {2^1\cdot\cdot2^2\cdot2^4\cdots2^{2^n}}\right) \\
            &= \log_2\left(\frac{3\cdot5\cdot17\cdots(2^{2^n}+1)}
                  {2^{1+2+4+\cdots+2^n}}\right) \\
            &= \log_2\left(\frac{3\cdot5\cdot17\cdots(2^{2^n}+1)}
                  {2^{2^{n+1}-1}}\right).
      \end{align*}
      In the above notice that $1 + 2 + \cdots + 2^n$ is a geometric sum. So we
      used the following fact: the $m$-th sum of a geometric series with first
      term $a$ and common ration $r$ is given by
      $$\frac{a(r^m - 1)}{r - 1}$$
      to get $1 + 2 + \cdots + 2^n = 2^{2^{n+1}-1}$. Now we want to find a
      closed form for the expression
      $$f(n) = 3\cdot5\cdot17\cdots(2^{2^n}+1).$$
      We have $f(0) = 3$, $f(1) = 15$, $f(2) = 255$, so we conjecture that
      $f(n) = 2^{2^{n+1}} - 1$. We shall now prove this by induction. It is
      clear that our conjecture holds whenever $n$ is 0, 1, or 2. So assume that
      $f(j) = 2^{2^{j+1}} - 1$ for some positive integer $j$. So we have that
      \begin{align*}
         f(j+1) &= 3\cdot5\cdot17\cdots(2^{2^j}+1)\cdot(2^{2^{j+1}}+1) \\
            &= f(j) \cdot (2^{2^{j+1}}+1) \\
            &= (2^{2^{j+1}} - 1)\cdot (2^{2^{j+1}}+1)
               &[\text{Inductive hypothesis}] \\
            &= (2^{2^{j+1}})^2 - 1\\
            &= 2^{2^{(j+1)+1}} - 1.
      \end{align*}
      We have just shown that our conjecture holds for $j + 1$. Thus it follows
      by Mathematical Induction that it holds for all non-negative integers, and
      we have that
      $$\sum_{k=0}^n\log_2\left(1 + \frac{1}{2^{2^k}}\right) =
           \log_2\left(\frac{2^{2^{n+1}} - 1}{2^{2^{n+1}-1}}\right).$$
           
      Since
      $$1 + \log_2\left(\frac{2014}{2015}\right) = 
        \log_22 + \log_2\left(\frac{2014}{2015}\right) =
              \log_2\left(\frac{4028}{2015}\right),$$
      it follows that \eqref{8_1} holds if and only if
      $$
         \log_2\left(\frac{2^{2^{n+1}} - 1}{2^{2^{n+1}-1}}\right) \ge
            \log_2\left(\frac{4028}{2015}\right)
      $$
      if and only if
      $$\frac{2^{2^{n+1}} - 1}{2^{2^{n+1}-1}} \ge \frac{4028}{2015}$$
      if and only if
      $$2015(2^{2^{n+1}}-1) \ge 4028(2^{2^{n+1}-1})$$
      if and only if
      $$2015(2^{2^{n+1}}-1) \ge 2014(2^{2^{n+1}})$$
      if and only if
      $$2^{2^{n+1}} \ge 2015$$
      if and only if
      $$n \ge \log_2(\log_22015) - 1.$$
      Thus the smallest $n$ is 3.
            
\end{enumerate}
\end{document}
