\documentclass[9pt]{article}

\usepackage{amssymb}
\usepackage{amsmath}
\usepackage{amsfonts}
\usepackage{comment}
\usepackage{fancyhdr}
\usepackage{mathrsfs}
\usepackage{enumitem}
\usepackage[retainorgcmds]{IEEEtrantools}


\usepackage{tikz}

\voffset = -50pt
%\textheight = 700pt
\addtolength{\textwidth}{60pt}
\addtolength{\evensidemargin}{-30pt}
\addtolength{\oddsidemargin}{-30pt}
%\setlength{\headheight}{44pt}

\pagestyle{fancy}
\fancyhf{} % clear all fields
\fancyhead[R]{%
  \scshape
  \begin{tabular}[t]{@{}r@{}}
  MATH 491, Fall 2015\\Section 1 (4847)\\
  HW \#0, DUE: 2015, August 25
  \end{tabular}}
\fancyhead[L]{%
  \scshape
  \begin{tabular}[t]{@{}r@{}}
  JOSEPH OKONOBOH\\Mathematics\\Cal State Long Beach
  \end{tabular}}
\fancyfoot[C]{\thepage}

\newcommand{\qed}{\hfill \ensuremath{\Box}}


\newcommand*\circled[1]{\tikz[baseline=(char.base)]{
            \node[shape=circle,draw,inner sep=2pt] (char) {#1};}}

\newcommand{\Z}{\mathbb{Z}}
\newcommand{\I}{\mathbb{I}}
\newcommand{\M}{\mathbb{M}}
\newcommand{\R}{\mathbb{R}}
\newcommand{\D}{\displaystyle}
%\setcounter{section}{-1}

\begin{document}
\begin{enumerate}
%%%%%%%%%%%%%%%%%%%%%%%%%%%%%%%%%%%%%%%01%%%%%%%%%%%%%%%%%%%%%%%%%%%%%%%%%%%%%%%
   \item Let $f$ be the function defined by $f(x) = x^3 - 49x^2 + 623x - 2015$,
         and let $g(x) = f(x + 5)$. Compute the sum of the roots of $g$.

      \textbf{Solution.} First we observe that $c$ is a root of $f$ if and only
      if $c-5$ is a root of $g$. If $f$ has a rational root, then the Rational
      Root Theorem says that it must be a divisor of $2015$. Since
      $2015 = 5 \cdot 13 \cdot 31 \cdot 403$, it follows that the candidates for 
      the rational roots of $f$ are
      $$\pm1, \pm5, \pm13, \pm31, \pm65, \pm155, \pm403, \text{ and } \pm2015.$$
      Now $f(1) \neq -1440$ and $f(5) = 0$. Thus 5 is a root of $f$. By
      division and further factorization, we get
      $$f(x) = (x - 5)(x^2 -44x + 403) = (x - 5)(x - 13)(x - 31),$$
      so that the roots of $f$ are 5, 13, and 31. Thus the sum of the roots of
      $g$ is
      $$(5 - 5) + (13 - 5) + (31 - 5) = 34.$$
%%%%%%%%%%%%%%%%%%%%%%%%%%%%%%%%%%%%%%%02%%%%%%%%%%%%%%%%%%%%%%%%%%%%%%%%%%%%%%%
   \item Lay six long strings in parallel. Two people (each of whom cannot see
         what the other is doing) work at opposite ends of the bundle of
         strings. Each randomly pairs off the strings at his or her end and
         glues together the ends of each pair. Then stretch the strings out.
         What is the probability that the strings now form a single large loop?

      \textbf{Solution.}
%%%%%%%%%%%%%%%%%%%%%%%%%%%%%%%%%%%%%%%03%%%%%%%%%%%%%%%%%%%%%%%%%%%%%%%%%%%%%%%
   \item Compute the product $\D\cos\left(\frac{\pi}{5}\right)\cos\left(
         \frac{2\pi}{5}\right)\cos\left(\frac{3\pi}{5}\right)\cos\left(
         \frac{4\pi}{5}\right)$.

      \textbf{Solution.}
%%%%%%%%%%%%%%%%%%%%%%%%%%%%%%%%%%%%%%%04%%%%%%%%%%%%%%%%%%%%%%%%%%%%%%%%%%%%%%%
   \item Compute the number of ordered pairs of integers $(x, y)$ such that
         \begin{equation} \label{4_2}
            \frac{1}{x} + \frac{540}{xy} = 2.
         \end{equation}

      \textbf{Solution.} Clearly neither of $x$ and $y$ is 0, so multiply
      Equation \eqref{4_2} by $xy$ to get $y + 540 = 2xy$. Rearrange to get
      \begin{equation} \label{4_1}
         y(2x - 1) = 540.
      \end{equation}
      Equation \eqref{4_1} tells us that $(2x - 1)$ is an odd factor of 540.
      Since $540 = 2^2 \cdot 3^3 \cdot 5$, it follows that the odd factors of
      540 are
      $$\pm1, \pm3, \pm9, \pm27, \pm5, \pm 15, \pm45, \text{ and } \pm135.$$
      That is, $(2x - 1)$ must be one of the factors above, so that Equation
      \eqref{4_1} has 16 ordered pairs of solutions. Now in each of the 16
      solutions the $y$ component cannot be 0 since Equation \eqref{4_1} says
      that the $y$ component is always a factor of 540; however, $(0, -540)$ is
      a solution of Equation \eqref{4_1}, so we shall exclude it. Thus there are
      15 solutions to Equation \eqref{4_2}. Listed explicitly, they are:
      $(-67, -4)$, $(-22, -12)$, $(-13, -20)$, $(-7, -36)$, $(-4, -60)$,
      $(-2, -108)$, $(-1, -180)$, $(1, 540)$, $(2, 180)$, $(3, 108)$, $(5, 60)$, 
      $(8, 36), (14, 20)$, $(23, 12)$, and $(68, 4)$.
%%%%%%%%%%%%%%%%%%%%%%%%%%%%%%%%%%%%%%%05%%%%%%%%%%%%%%%%%%%%%%%%%%%%%%%%%%%%%%%
   \item Compute the smallest positive integer $n$ such that $n + i$,
         $(n + i)^2$, and $(n + i)^3$ are the vertices of a triangle in the
         complex plane whose area is greater than 2015.

      \textbf{Solution.} Let $A(x_1, y_1)$, $B(x_2, y_2)$, and $C(x_3, y_3)$ be
      points in the $xy$-plane; then it can be shown that
      $$\text{Area of } \triangle ABC = \frac{1}{2}\left|\det\left(
        \begin{tabular}{@{}ccc@{}}
           1 & 1 & 1 \\
           $x_1$ & $x_2$ & $x_3$ \\
           $y_1$ & $y_2$ & $y_3$
        \end{tabular}\right)\right|,
      $$
      where $\det(M)$ is the determinant of a matrix $M$. Now suppose
      $$z_1 = a_1 + b_1i, z_2 = a_2 + b_2i, \text{ and } z_3 = a_3 + b_3i$$
      are the vertices of a triangle in the complex plane; also suppose that $s$
      is the area of this triangle. Letting $\overline{z}$ denote the conjugate
      of the complex number $z$ and using the following facts:
      \begin{itemize}
         \item adding a multiple of a row to another does not change the 
               determinant of a matrix and
         \item multiplying a row by a scalar multiplies the determinant by the 
               same scalar,
      \end{itemize}
      it follows that
      \begin{align*}
         s &= \frac{1}{2}\left|\det\left(\begin{tabular}{@{}ccc@{}}
                  1 & 1 & 1 \\
                  $a_1$ & $a_2$ & $a_3$ \\
                  $b_1$ & $b_2$ & $b_3$
              \end{tabular}\right)\right| \\
           &= \frac{1}{2i}\left|\det\left(\begin{tabular}{@{}ccc@{}}
                  1 & 1 & 1 \\
                  $a_1$ & $a_2$ & $a_3$ \\
                  $b_1i$ & $b_2i$ & $b_3i$
              \end{tabular}\right)\right| \\
           &= \frac{1}{2i}\left|\det\left(\begin{tabular}{@{}ccc@{}}
                  1 & 1 & 1 \\
                  $a_1+b_1i$ & $a_2+b_2i$ & $a_3+b_3i$ \\
                  $b_1i$ & $b_2i$ & $b_3i$
              \end{tabular}\right)\right| \\
           &= \frac{1}{-4i}\left|\det\left(\begin{tabular}{@{}ccc@{}}
                  1 & 1 & 1 \\
                  $a_1+b_1i$ & $a_2+b_2i$ & $a_3+b_3i$ \\
                  $-2b_1i$ & $-2b_2i$ & $-2b_3i$
              \end{tabular}\right)\right| \\
           &= \frac{1}{-4i}\left|\det\left(\begin{tabular}{@{}ccc@{}}
                  1 & 1 & 1 \\
                  $a_1+b_1i$ & $a_2+b_2i$ & $a_3+b_3i$ \\
                  $a_1-b_1i$ & $a_2-b_2i$ & $a_3-b_3i$
              \end{tabular}\right)\right| \\
           &= \frac{i}{4}\left|\det\left(\begin{tabular}{@{}ccc@{}}
                  1 & 1 & 1 \\
                  $z_1$ & $z_2$ & $z_3$ \\
                  $\overline{z_1}$ & $\overline{z_2}$ & $\overline{z_3}$
              \end{tabular}\right)\right|.
      \end{align*}

      Now, for a positive integer $n$, let $s_n$ denote the area of a triangle 
      with vertices $n + i$, $(n + i)^2$, and $(n + i)^3$ in the complex plane.
      As previously shown, it follows that
      \begin{align*}
         s_n &= \frac{i}{4}\left|\det\left(\begin{tabular}{@{}ccc@{}}
                   1 & 1 & 1 \\
                   $n+i$ & $(n+i)^2$ & $(n+i)^3$ \\
                   $\overline{n+i}$& $\overline{(n+i)^2}$ & $\overline{(n+i)^3}$
                \end{tabular}\right)\right| \\
             &= \frac{i}{4}\left|\det\left(\begin{tabular}{@{}ccc@{}}
                   1 & 1 & 1 \\
                   $n+i$ & $(n+i)^2$ & $(n+i)^3$ \\
                   $n-i$ & $(n-i)^2$ & $(n-i)^3$
                \end{tabular}\right)\right| \\
      \end{align*}
\end{enumerate}
\end{document}
