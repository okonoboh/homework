\documentclass[9pt]{article}

\usepackage{amssymb}
\usepackage{amsmath}
\usepackage{amsfonts}
\usepackage{comment}
\usepackage{fancyhdr}
\usepackage{mathrsfs}
\usepackage{enumitem}
\usepackage[retainorgcmds]{IEEEtrantools}


\usepackage{tikz}

\voffset = -50pt
%\textheight = 700pt
\addtolength{\textwidth}{60pt}
\addtolength{\evensidemargin}{-30pt}
\addtolength{\oddsidemargin}{-30pt}
%\setlength{\headheight}{44pt}

\pagestyle{fancy}
\fancyhf{} % clear all fields
\fancyhead[R]{%
  \scshape
  \begin{tabular}[t]{@{}r@{}}
  MATH 491, Fall 2015\\Section 1 (4847)\\
  HW \#0, DUE: 2015, August 25
  \end{tabular}}
\fancyhead[L]{%
  \scshape
  \begin{tabular}[t]{@{}r@{}}
  JOSEPH OKONOBOH\\Mathematics\\Cal State Long Beach
  \end{tabular}}
\fancyfoot[C]{\thepage}

\newcommand{\qed}{\hfill \ensuremath{\Box}}


\newcommand*\circled[1]{\tikz[baseline=(char.base)]{
            \node[shape=circle,draw,inner sep=2pt] (char) {#1};}}

\newcommand{\Z}{\mathbb{Z}}
\newcommand{\I}{\mathbb{I}}
\newcommand{\M}{\mathbb{M}}
\newcommand{\R}{\mathbb{R}}
\newcommand{\D}{\displaystyle}
%\setcounter{section}{-1}

\begin{document}
\begin{enumerate}
%%%%%%%%%%%%%%%%%%%%%%%%%%%%%%%%%%%%%%%01%%%%%%%%%%%%%%%%%%%%%%%%%%%%%%%%%%%%%%%
   \item Let $f$ be the function defined by $f(x) = x^3 - 49x^2 + 623x - 2015$,
         and let $g(x) = f(x + 5)$. Compute the sum of the roots of $g$.

      \textbf{Solution.} Using the Rational Root Theorem, we get the following
      factorization
      $$f(x) = (x - 5)(x-13)(x-31),$$
      so that the roots of $f$ are 5, 13, and 31. Now observe that $c$ is a root 
      of $f$ if and only if $c-5$ is a root of $g$; thus the sum of the roots of 
      $g$ is $(5 - 5) + (13 - 5) + (31 - 5) = 34$.
%%%%%%%%%%%%%%%%%%%%%%%%%%%%%%%%%%%%%%%02%%%%%%%%%%%%%%%%%%%%%%%%%%%%%%%%%%%%%%%
   \item Lay six long strings in parallel. Two people (each of whom cannot see
         what the other is doing) work at opposite ends of the bundle of
         strings. Each randomly pairs off the strings at his or her end and
         glues together the ends of each pair. Then stretch the strings out.
         What is the probability that the strings now form a single large loop?

      \textbf{Solution.}
%%%%%%%%%%%%%%%%%%%%%%%%%%%%%%%%%%%%%%%03%%%%%%%%%%%%%%%%%%%%%%%%%%%%%%%%%%%%%%%
   \item Compute the product $\D\cos\left(\frac{\pi}{5}\right)\cos\left(
         \frac{2\pi}{5}\right)\cos\left(\frac{3\pi}{5}\right)\cos\left(
         \frac{4\pi}{5}\right)$.

      \textbf{Solution.}
\end{enumerate}
\end{document}
