\documentclass[9pt]{article}

\usepackage{amssymb}
\usepackage{amsmath}
\usepackage{amsfonts}
\usepackage{comment}
\usepackage{fancyhdr}
\usepackage{mathrsfs}
\usepackage{enumitem}
%\usepackage[retainorgcmds]{IEEEtrantools}


\usepackage{tikz}

\voffset = -50pt
%\textheight = 700pt
\addtolength{\textwidth}{60pt}
\addtolength{\evensidemargin}{-30pt}
\addtolength{\oddsidemargin}{-30pt}
%\setlength{\headheight}{44pt}

\pagestyle{fancy}
\fancyhf{} % clear all fields
\fancyhead[R]{%
  \scshape
  \begin{tabular}[t]{@{}r@{}}
  MATH 491, Fall 2015\\Section 1 (4847)\\
  HW \#1, DUE: 2015, September 01
  \end{tabular}}
\fancyhead[L]{%
  \scshape
  \begin{tabular}[t]{@{}r@{}}
  JOSEPH OKONOBOH\\Mathematics\\Cal State Long Beach
  \end{tabular}}
\fancyfoot[C]{\thepage}

\newcommand{\qed}{\hfill \ensuremath{\Box}}


\newcommand*\circled[1]{\tikz[baseline=(char.base)]{
            \node[shape=circle,draw,inner sep=2pt] (char) {#1};}}

\newcommand{\Z}{\mathbb{Z}}
\newcommand{\I}{\mathbb{I}}
\newcommand{\M}{\mathbb{M}}
\newcommand{\R}{\mathbb{R}}
\everymath{\displaystyle}
%\setcounter{section}{-1}

\begin{document}
\begin{enumerate}
%%%%%%%%%%%%%%%%%%%%%%%%%%%%%%%%%%%%%%%01%%%%%%%%%%%%%%%%%%%%%%%%%%%%%%%%%%%%%%%
   \item Assume $m\ge 3$ is an odd  integer. Compute, in general,
         $\prod_{j=1}^{m-1}\cos\left(\frac{j\pi}m\right).$
%%%%%%%%%%%%%%%%%%%%%%%%%%%%%%%%%%%%%%%02%%%%%%%%%%%%%%%%%%%%%%%%%%%%%%%%%%%%%%%
   \item Assume $m\ge 2$ is an  integer. Compute, in general,
         $\prod_{j=1}^{m-1}\sin\left(\frac{j\pi}m\right).$

         Hint: both \#1 and \#2 have answers that are always rational and
         ``nice."
%%%%%%%%%%%%%%%%%%%%%%%%%%%%%%%%%%%%%%%03%%%%%%%%%%%%%%%%%%%%%%%%%%%%%%%%%%%%%%%
   \item The number $x=\left(4+\sqrt{17}\right)^{714}$ has a decimal expansion 
         which has exactly $N$ digits to the the left of the decimal point and 
         infinitely many digits to the right of the decimal point. The leading 
         digits are $34800038\dots.$ What is the $N$th decimal digit of $x$ to 
         the right of the decimal point?
%%%%%%%%%%%%%%%%%%%%%%%%%%%%%%%%%%%%%%%04%%%%%%%%%%%%%%%%%%%%%%%%%%%%%%%%%%%%%%%
   \item A great circle is a circle drawn on a sphere whose center is the center 
         of the sphere (an ``equator"). Given $n$ great circles on a sphere, no 
         three of which meet at a point, into how many regions do they divide
         the sphere?
%%%%%%%%%%%%%%%%%%%%%%%%%%%%%%%%%%%%%%%05%%%%%%%%%%%%%%%%%%%%%%%%%%%%%%%%%%%%%%%
   \item We are given the numbers $1,2,3,\cdots,100.$ Every minute, we erase
         (any) two numbers $u$ and $v$ and replace them with the number
         $u+v+uv.$ After 99 minutes, we are left with just one number. What is 
         it, and why?
%%%%%%%%%%%%%%%%%%%%%%%%%%%%%%%%%%%%%%%06%%%%%%%%%%%%%%%%%%%%%%%%%%%%%%%%%%%%%%%
   \item Is it true that for any prime $p,$ there is a Fibonacci number which is 
         a multiple of $p?$ Prove your answer, one way or the other.
\end{enumerate}
\end{document}
