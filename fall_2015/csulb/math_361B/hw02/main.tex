\documentclass[9pt]{article}

\usepackage{amssymb}
\usepackage{amsmath}
\usepackage{amsfonts}
\usepackage{comment}
\usepackage{fancyhdr}
\usepackage{mathrsfs}
\usepackage{enumitem}
%\usepackage[retainorgcmds]{IEEEtrantools}


\usepackage{tikz}

\voffset = -50pt
%\textheight = 700pt
\addtolength{\textwidth}{60pt}
\addtolength{\evensidemargin}{-30pt}
\addtolength{\oddsidemargin}{-30pt}
%\setlength{\headheight}{44pt}

\pagestyle{fancy}
\fancyhf{} % clear all fields
\fancyhead[R]{%
  \scshape
  \begin{tabular}[t]{@{}r@{}}
  MATH 361B, Fall 2015\\Section 1 (2225)\\
  HW \#1, DUE: 2015, August 26
  \end{tabular}}
\fancyhead[L]{%
  \scshape
  \begin{tabular}[t]{@{}r@{}}
  JOSEPH OKONOBOH\\Mathematics\\Cal State Long Beach
  \end{tabular}}
\fancyfoot[C]{\thepage}

\newcommand{\qed}{\hfill \ensuremath{\Box}}


\newcommand*\circled[1]{\tikz[baseline=(char.base)]{
            \node[shape=circle,draw,inner sep=2pt] (char) {#1};}}

\newcommand{\Z}{\mathbb{Z}}
\newcommand{\I}{\mathbb{I}}
\newcommand{\Q}{\mathbb{Q}}
\newcommand{\R}{\mathbb{R}}
\newcommand{\D}{\displaystyle}
%\setcounter{section}{-1}

\begin{document}
\begin{enumerate}
%%%%%%%%%%%%%%%%%%%%%%%%%%%%%%%%%%%%%%%09%%%%%%%%%%%%%%%%%%%%%%%%%%%%%%%%%%%%%%%
   \item[6.9.] Let    
               \begin{equation*}
                  f(x) = \left\{
                     \begin{array}{ll}
                        x^2  & \text{if } x \in \Q \cap [a, b] \\
                        0    & \text{if } x \in (\R - \Q) \cap [a, b]
                     \end{array} \right.
               \end{equation*}
               where $a > 0$. Prove that
               $\D\overline{\int} f = \frac{b^3-a^3}{3}$.
               
      \textbf{Proof.} Let $P_n$ be a regular partition of $[a, b]$. Since the
      right endpoints of $P_n$ occur at rational numbers and since
      $f(q_1) < f(q_2)$ for positive rational numbers $q_1$ and $q_2$, with
      $q_1 < q_2$, it follows that the supremum of $f$ on each subinterval must
      occur at the right endpoint. Thus
      \begin{align*}
         U(P_n, f) &= \sum_i^nM_i \cdot \Delta x_i \\
            &= \sum_i^n\left(a + \frac{i}{n}(b-a)\right)^2 \frac{b-a}{n} \\
            &= \frac{b-a}{n}\sum_i^n\left(a^2+\frac{2a}{n}(b-a)i+
                  \frac{(b-a)^2}{n^2}i^2\right) \\
            &= \frac{b-a}{n}\sum_i^na^2 + \frac{2a(b-a)^2}{n^2}\sum_i^ni +
                  \frac{(b-a)^3}{n^3}\sum_i^ni^2 \\
            &= a^2(b-a) + \frac{2a(b-a)^2}{n^2}\frac{n(n+1)}{2}+
                  \frac{(b-a)^3}{n^3}\frac{n(n+1)(2n+1)}{6} \\
            &= a^2(b-a) + a(b-a)^2\frac{n+1}{n}+
                  \frac{(b-a)^3}{6}\frac{(n+1)(2n+1)}{n^2} \\
            &= a^2(b-a) + a(b-a)^2\left(1 + \frac{1}{n}\right)+
                  \frac{(b-a)^3}{6}\left(1+\frac{1}{n}\right)
                  \left(2+\frac{1}{n}\right) \\
            &= a^2(b-a) + a(b-a)^2 + \frac{a(b-a)^2}{n}+
                  \frac{(b-a)^3}{6}\left(2+\frac{3}{n}+\frac{1}{n^2}\right) \\
            &= ab^2-a^2b + \frac{a(b-a)^2}{n}+\frac{(b-a)^3}{3}+
                  \frac{(b-a)^3}{2n}+\frac{(b-a)^3}{6n^2} \\
            &= \frac{b^3-a^3}{3} +\frac{(b-a)^2(b+a)}{2n}+\frac{(b-a)^3}{6n^2}\\            
            &= \frac{b^3-a^3}{3} +q(n),
      \end{align*}
      where
      $$q(n) = \frac{(b-a)^2(b+a)}{2n}+\frac{(b-a)^3}{6n^2}.$$
      Since $\D\overline{\int} f$ is the infimum of all the upper sums of
      partitions of $f$, it follows that
      $$\overline{\int} f \le \frac{b^3-a^3}{3}+q(n)$$
      for each positive integer $n$. Now suppose to the contrary that
      $$\overline{\int} f > \frac{b^3-a^3}{3}.$$
      Choose $N_1$ such that
      $$q(N_1) < \left(\overline{\int}f - \frac{b^3-a^3}{3}\right).$$
      Thus
      \begin{align*}
         \overline{\int}f &= \left(\overline{\int}f - \frac{b^3-a^3}{3}\right) + \frac{b^3-a^3}{3} \\
           &> q(N_1) + \frac{b^3-a^3}{3},
      \end{align*}
      a contradiction, since we just showed that 
      $$\overline{\int} f \le \frac{b^3-a^3}{3}+q(n)$$
      for each positive integer $n$. Thus
      $$\overline{\int} f \le \frac{b^3-a^3}{3}.$$
      
      Let $\varepsilon$ be a positive real number. It follows by Theorem 6.4
      that there exists a $\delta > 0$ such that
      $U(P, f) < \varepsilon + \overline{\int} f$, whenever $||P|| < \delta$. So
      choose $N'$ such that $||P_{N'}|| < \delta$. Thus it follows that
      $$U(P_{N'}, f) < \varepsilon + \overline{\int} f$$
      and thus
      $$\frac{b^3-a^3}{3} + q(n) < \varepsilon + \overline{\int} f$$      
      for all $n \ge N'$. We can similarly argue by contradiction that
      \begin{equation}\label{9_1}
         \frac{b^3-a^3}{3} < \varepsilon + \overline{\int} f.
      \end{equation}
      Finally suppose to the contrary that
      $\D\frac{b^3-a^3}{3} > \overline{\int} f$. So it follows from \eqref{9_1}
      that
      \begin{align*}
         \varepsilon > \frac{b^3-a^3}{3} - \overline{\int} f > 0
      \end{align*}
      contradicting our assumption that $\varepsilon$ was an arbitrary positive
      real number. Thus we have that
      $$\frac{b^3-a^3}{3} \le \overline{\int} f$$
      and since
      $$\overline{\int} f \le \frac{b^3-a^3}{3}$$
      we conclude that
      $$\overline{\int} f = \frac{b^3-a^3}{3}.$$ \qed      
\end{enumerate}
\end{document}
