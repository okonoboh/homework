\documentclass[9pt]{article}

\usepackage{amssymb}
\usepackage{amsmath}
\usepackage{amsfonts}
\usepackage{comment}
\usepackage{fancyhdr}
\usepackage{mathrsfs}
\usepackage{enumitem}
%\usepackage[retainorgcmds]{IEEEtrantools}


\usepackage{tikz}

\voffset = -50pt
%\textheight = 700pt
\addtolength{\textwidth}{60pt}
\addtolength{\evensidemargin}{-30pt}
\addtolength{\oddsidemargin}{-30pt}
%\setlength{\headheight}{44pt}

\pagestyle{fancy}
\fancyhf{} % clear all fields
\fancyhead[R]{%
  \scshape
  \begin{tabular}[t]{@{}r@{}}
  MATH 361B, Fall 2015\\Section 1 (2225)\\
  HW \#1, DUE: 2015, August 26
  \end{tabular}}
\fancyhead[L]{%
  \scshape
  \begin{tabular}[t]{@{}r@{}}
  JOSEPH OKONOBOH\\Mathematics\\Cal State Long Beach
  \end{tabular}}
\fancyfoot[C]{\thepage}

\newcommand{\qed}{\hfill \ensuremath{\Box}}


\newcommand*\circled[1]{\tikz[baseline=(char.base)]{
            \node[shape=circle,draw,inner sep=2pt] (char) {#1};}}

\newcommand{\Z}{\mathbb{Z}}
\newcommand{\I}{\mathbb{I}}
\newcommand{\M}{\mathbb{M}}
\newcommand{\R}{\mathbb{R}}
\newcommand{\D}{\displaystyle}
%\setcounter{section}{-1}

\begin{document}
\begin{enumerate}
%%%%%%%%%%%%%%%%%%%%%%%%%%%%%%%%%%%%%%%01%%%%%%%%%%%%%%%%%%%%%%%%%%%%%%%%%%%%%%%
   \item Prove that if $f$ is bounded on $[a, b]$ and has exactly one
         discontinuity in $[a, b]$ then $f$ is Riemann-integrable on $[a, b]$.

      \textbf{Proof.} Suppose that $f$ is bounded on $[a, b]$ and has exactly
      one discontinuity at $c \in [a, b]$. Let $\varepsilon$ be a positive real
      number, and let $M$ and $m$ be the supremum and infimum of $f$ on
      $[a, b]$. We want to find a partition $P$ of $[a, b]$ such that
      $U(P, f) - L(P, f) < \varepsilon$. Now we shall consider the following
      cases:
      
      \textbf{Case 1.} $c \in (a, b)$. Consider any subinterval $[x, y]$ of
      $[a, b]$, where $a < x < y < b$, such that $c \in [x, y]$ and
      $$y - x < \frac{\varepsilon}{2(M-m+1)}.$$
      By our hypothesis, $f$ is continuous on $[a, x]$ and $[y, b]$. Thus it follows
      by Theorem 6.2 that $f$ is Riemann-integrable on $[a, x]$ and $[y, b]$.
      So, by Theorem 6.1, there exist partitions $P_x$ of $[a, x]$ and $P_y$ of
      $[y, b]$ such that
      $$U(P_x, f) - L(P_x, f) < \frac{\varepsilon}{4} \text{ and }
        U(P_y, f) - L(P_y, f) < \frac{\varepsilon}{4}.$$
      Let $P = P_x \cup P_y$, a partition of $[a, b]$. Now let $M_c$ and $m_c$
      be the supremum and infimum of $f$ on $[x, y]$ and observe that
      $$U(P, f) = U(P_x, f) + U(P_y, f) + M_c(y - x) \text{ and }
        L(P, f) = L(P_x, f) + L(P_y, f) + m_c(y - x).$$
      Thus
      \begin{align*}
         U(P, f) - L(P, f) &= U(P_x, f) - L(P_x, f) + U(P_y, f) - L(P_y, f) +
          (M_c - m_c)(y - x) \\
          &\le \frac{\varepsilon}{4} + \frac{\varepsilon}{4} + (M - m)(y - x) \\
          &<\frac{\varepsilon}{2} + (M - m)\frac{\varepsilon}{2(M-m+1)} \\
          &< \varepsilon.
      \end{align*}
      
      \textbf{Case 2.} $c = a$. So let $[y, b]$ be a subinterval of $[a, b]$,
      with $y > a$, such that
      $$y - a < \frac{\varepsilon}{2(M-m+1)};$$
      also let $M_c$ and $m_c$ denote the supremum and infimum of $f$ on
      $[a, y]$. Because it is continuous on $[y, b]$, $f$ is Riemann-integrable by Theorem
      6.2, and thus by Theorem 6.1, there exists a partition $P_y$ of $[y, b]$
      such that $$U(P_y, f) - L(P_y, f) < \frac{\varepsilon}{2}.$$
      
      Let $P = \{a\} \cup P_y$, a partition of $[a, b]$. Thus
      \begin{align*}
         U(P, f) - L(P, f) &= [M_c(y - a) + U(P_y, f)] -
            [m_c(y - a) + L(P_y, f)] \\
            &= (M_c - m_c)(y - a) + U(P_y, f) - L(P_y, f) \\
          &\le (M - m)(y - a) + \frac{\varepsilon}{2} \\
          &<(M - m)\frac{\varepsilon}{2(M-m+1)} + \frac{\varepsilon}{2} \\
          &< \varepsilon.
      \end{align*}
      
      \textbf{Case 3.} $c = b$. This is similar to Case 2. So we can exploit the
      symmetry to conclude that $U(P, f) - L(P, f) < \varepsilon$ for some
      partition $P$ of $[a, b]$. \\
      
      In all cases, we showed that $U(P, f) - L(P, f) < \varepsilon$ for some
      partition $P$ of $[a, b]$. We can then conclude by Theorem 6.1 that $f$ is
      Riemann-integrable on $[a, b]$. \qed
      
      
      
\end{enumerate}
\end{document}
