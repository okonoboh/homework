\documentclass[9pt]{article}

\usepackage{amssymb}
\usepackage{amsmath}
\usepackage{amsfonts}
\usepackage{comment}
\usepackage{fancyhdr}
\usepackage{mathrsfs}
\usepackage{enumitem}
\usepackage{graphicx}
%\usepackage[retainorgcmds]{IEEEtrantools}


\usepackage{tikz}

\voffset = -50pt
%\textheight = 700pt
\addtolength{\textwidth}{60pt}
\addtolength{\evensidemargin}{-30pt}
\addtolength{\oddsidemargin}{-30pt}
%\setlength{\headheight}{44pt}

\pagestyle{fancy}
\fancyhf{} % clear all fields
\fancyhead[R]{%
  \scshape
  \begin{tabular}[t]{@{}r@{}}
  MATH 562, Fall 2015\\Section 1 (9934)\\
  HW \#13, DUE: 2015, October 
  \end{tabular}}
\fancyhead[L]{%
  \scshape
  \begin{tabular}[t]{@{}r@{}}
  JOSEPH OKONOBOH\\Mathematics\\Cal State Long Beach
  \end{tabular}}
\fancyfoot[C]{\thepage}

\newcommand{\qed}{\hfill \ensuremath{\Box}}


\newcommand*\circled[1]{\tikz[baseline=(char.base)]{
            \node[shape=circle,draw,inner sep=2pt] (char) {#1};}}

\newcommand{\Z}{\mathbb{Z}}
\newcommand{\I}{\mathbb{I}}
\newcommand{\M}{\mathbb{M}}
\newcommand{\R}{\mathbb{R}}
\newcommand{\C}{\mathbb{C}}
\everymath{\displaystyle}
%\setcounter{section}{-1}

\begin{document}
\begin{enumerate}
%%%%%%%%%%%%%%%%%%%%%%%%%%%%%%%%%%%%%4.1.3%%%%%%%%%%%%%%%%%%%%%%%%%%%%%%%%%%%%%%
   \item[4.1.3.]  Let $\gamma$ be the circle $\{|z| = R\}$, with the usual
                  counterclockwise orientation. Evaluate the following
                  integrals, for $m = 0, \pm 1, \pm2, \ldots$.

                  \begin{enumerate}
                     \item $\int_\gamma |z^m|\;dz$
                     \item $\int_\gamma |z^m|\;|dz|$
                     \item $\int_\gamma \overline{z}^m\;dz$
                  \end{enumerate}
%%%%%%%%%%%%%%%%%%%%%%%%%%%%%%%%%%%%%4.1.5%%%%%%%%%%%%%%%%%%%%%%%%%%%%%%%%%%%%%%
   \item[4.1.5.]  Show that
                  $$\left|\oint_{|z-1|=1}\frac{e^z}{z+3}dz\right|\le 2\pi e^2.$$
%%%%%%%%%%%%%%%%%%%%%%%%%%%%%%%%%%%%%4.2.5%%%%%%%%%%%%%%%%%%%%%%%%%%%%%%%%%%%%%%
   \item[4.2.5.]  Show that an analytic function $f(z)$ has a primitive in $D$
                  if and only if $\int_\gamma f(z)dz = 0$ for every closed path
                  $\gamma$ in $D$.
%%%%%%%%%%%%%%%%%%%%%%%%%%%%%%%%%%%%%4.3.1%%%%%%%%%%%%%%%%%%%%%%%%%%%%%%%%%%%%%%
   \item[4.3.1.]  By integrating $e^{-z^2/2}$ around a rectangle with vertices
                  $\pm R$, $it \pm R$, and sending $R$ to $\infty$, show that
                  $$\frac{1}{\sqrt{2\pi}}\int_{-\infty}^\infty e^{-x^2/2}
                    e^{-itx}dx = e^{-t^2/2}, \quad -\infty < t < \infty.$$
                  Use the known value of the integral for $t = 0$.
                  \textit{Remark.} This shows that $e^{-x^2/2}$ is an
                  eigenfunction of the Fourier transform with eigenvalue 1. For
                  more, see the next exercise.
%%%%%%%%%%%%%%%%%%%%%%%%%%%%%%%%%%%%%4.3.4%%%%%%%%%%%%%%%%%%%%%%%%%%%%%%%%%%%%%%
   \item[4.3.4.]  Prove that a polynomial in $z$ without zeros is constant (the
                  fundamental theorem of algebra) using Cauchy's theorem, along
                  the following lines. If $P(z)$ is a polynomial that is not a
                  constant, write $P(z) = P(0) + zQ(z)$, divide by $zP(z)$, and
                  integrate around a large circle. This will lead to a
                  contradiction if $P(z)$ has no zeros.
%%%%%%%%%%%%%%%%%%%%%%%%%%%%%%%%%%%%%4.4.1%%%%%%%%%%%%%%%%%%%%%%%%%%%%%%%%%%%%%%
   \item[4.4.1.]  \begin{enumerate}
                     \item[(c)]  $\oint_{|z|=1} \frac{\sin(z)}{z}\;dz$
                     \item[(f)]  $\int_{|z-1-i|=5/4}
                                  \frac{\text{Log}(z)}{(z-1)^2}\;dz$
                     \item[(g)]  $\oint_{|z|=1} \frac{dz}{z^2(z^2-4)e^z}$
                     \item[(h)]  $\oint_{|z-1|=3} \frac{dz}{z(z^2-4)e^z}$
                     
                  \end{enumerate}
%%%%%%%%%%%%%%%%%%%%%%%%%%%%%%%%%%%%%4.4.4%%%%%%%%%%%%%%%%%%%%%%%%%%%%%%%%%%%%%%
   \item[4.4.4.]  Let $D$ be a bounded domain with smooth boundary $\partial D$,
                  and let $z_0 \in D$. Using the Cauchy integral formula, show
                  that there is a constant $C$ such that
                  $$|f(z_0)| \le C \sup\{|f(z)| : z \in \partial D\}$$
                  for any function $f(z)$ analytic on $D\cup\partial D$. By
                  applying this estimate to $f(z)^n$, taking $n$th roots, and
                  letting $n \rightarrow \infty$, show that the estimate holds
                  with $C = 1$. \textit{Remark.} This provides an alternative
                  proof of the maximum principle for analytic functions.
%%%%%%%%%%%%%%%%%%%%%%%%%%%%%%%%%%%%%4.5.1%%%%%%%%%%%%%%%%%%%%%%%%%%%%%%%%%%%%%%
   \item[4.5.1.]  Show that if $u$ is a harmonic function $\C$ that is bounded
                  above, then $u$ is constant. \textit{Hint.} Express $u$ as the
                  real part of an analytic function, and exponentiate.
%%%%%%%%%%%%%%%%%%%%%%%%%%%%%%%%%%%%%4.5.4%%%%%%%%%%%%%%%%%%%%%%%%%%%%%%%%%%%%%%
   \item[4.5.4.]  Suppose that $f(z)$ is an entire function such that $f(z)/z^n$
                  is bounded for $|z| \ge R$. Show that $f(z)$ is a polynomial
                  of degree at most $n$. What can be said if $f(z)/z^n$ is
                  bounded on the entire complex plane?
%%%%%%%%%%%%%%%%%%%%%%%%%%%%%%%%%%%%%4.6.1%%%%%%%%%%%%%%%%%%%%%%%%%%%%%%%%%%%%%%
   \item[4.6.1.]  Let $L$ be a line in the complex plane. Suppose $f(z)$ is a
                  continuous complex-valued function on a domain $D$ that is
                  analytic on $D\backslash L$. Show that $f(z)$ is analytic on
                  $D$.
\end{enumerate}
\end{document} 
