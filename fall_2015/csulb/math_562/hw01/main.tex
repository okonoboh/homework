\documentclass[9pt]{article}

\usepackage{amssymb}
\usepackage{amsmath}
\usepackage{amsfonts}
\usepackage{comment}
\usepackage{fancyhdr}
\usepackage{mathrsfs}
\usepackage{enumitem}
%\usepackage[retainorgcmds]{IEEEtrantools}


\usepackage{tikz}

\voffset = -50pt
%\textheight = 700pt
\addtolength{\textwidth}{60pt}
\addtolength{\evensidemargin}{-30pt}
\addtolength{\oddsidemargin}{-30pt}
%\setlength{\headheight}{44pt}

\pagestyle{fancy}
\fancyhf{} % clear all fields
\fancyhead[R]{%
  \scshape
  \begin{tabular}[t]{@{}r@{}}
  MATH 562, Fall 2015\\Section 1 (9934)\\
  HW \#1, DUE: 2015, September 10
  \end{tabular}}
\fancyhead[L]{%
  \scshape
  \begin{tabular}[t]{@{}r@{}}
  JOSEPH OKONOBOH\\Mathematics\\Cal State Long Beach
  \end{tabular}}
\fancyfoot[C]{\thepage}

\newcommand{\qed}{\hfill \ensuremath{\Box}}


\newcommand*\circled[1]{\tikz[baseline=(char.base)]{
            \node[shape=circle,draw,inner sep=2pt] (char) {#1};}}

\newcommand{\Z}{\mathbb{Z}}
\newcommand{\I}{\mathbb{I}}
\newcommand{\M}{\mathbb{M}}
\newcommand{\R}{\mathbb{R}}
\newcommand{\C}{\mathbb{C}}
\everymath{\displaystyle}
%\setcounter{section}{-1}

\begin{document}
\begin{enumerate}
%%%%%%%%%%%%%%%%%%%%%%%%%%%%%%%%%%%%%2.2.3%%%%%%%%%%%%%%%%%%%%%%%%%%%%%%%%%%%%%%
   \item[2.2.3.]  Show from the definition that the functions $x = \text{Re }z$
                  and $y = \text{Im }z$ are not complex differentiable at any 
                  point.
                  
      \textbf{Proof.} Let $x(z) = \text{Re}(z)$. It follows by definition that
      \begin{align*}
         \frac{dx}{dz} &= \lim_{\Delta z\rightarrow 0}
            \frac{x(z + \Delta z) - x(z)}{\Delta z} \\
            &=  \lim_{\Delta z\rightarrow 0}
            \frac{\text{Re}(z + \Delta z) - \text{Re}(z)}{\Delta z} \\
            &=  \lim_{\Delta z\rightarrow 0}
            \frac{\text{Re}(z) + \text{Re}(\Delta z) - \text{Re}(z)}{\Delta z} \\
            &=  \lim_{\Delta z\rightarrow 0}
            \frac{\text{Re}(\Delta z)}{\Delta z}.
      \end{align*}      
      If we let $\Delta z$ approach 0 along the real axis, then $\Delta z$
      is real, so that 
      $$\lim_{\Delta z\rightarrow 0}\frac{\text{Re}(\Delta z)}{\Delta z}
        = \frac{\Delta z}{\Delta z} = 1.$$     
      However, if we let $\Delta z$ approach 0 along the imaginary axis, then
      $\Delta z$ is pure imaginary, so that 
      $$\lim_{\Delta z\rightarrow 0}\frac{\text{Re}(\Delta z)}{\Delta z}
        = \frac{0}{\Delta z} = 0.$$
      Thus $x$ is not complex differentiable since the limit has two different
      values for two different paths. Similarly, we Let $y(z) = \text{Im}(z)$.
      It follows by definition that
      \begin{align*}
         \frac{dy}{dz} &= \lim_{\Delta z\rightarrow 0}
            \frac{y(z + \Delta z) - y(z)}{\Delta z} \\
            &=  \lim_{\Delta z\rightarrow 0}
            \frac{\text{Im}(z + \Delta z) - \text{Im}(z)}{\Delta z} \\
            &=  \lim_{\Delta z\rightarrow 0}
            \frac{\text{Im}(z) + \text{Im}(\Delta z) - \text{Im}(z)}{\Delta z} \\
            &=  \lim_{\Delta z\rightarrow 0}
            \frac{\text{Im}(\Delta z)}{\Delta z}.
      \end{align*}      
      If we let $\Delta z$ approach 0 along the real axis, then $\Delta z$
      is real, so that 
      $$\lim_{\Delta z\rightarrow 0}\frac{\text{Im}(\Delta z)}{\Delta z}
        = \frac{0}{\Delta z} = 0.$$     
      However, if we let $\Delta z$ approach 0 along the imaginary axis, then
      $\Delta z$ is pure imaginary, so that 
      $$\lim_{\Delta z\rightarrow 0}\frac{\text{Im}(\Delta z)}{\Delta z}
        = \frac{\Delta z}{\Delta z} = 1.$$
      Thus $y$ is not complex differentiable since the limit has two different
      values for two different paths. \qed
%%%%%%%%%%%%%%%%%%%%%%%%%%%%%%%%%%%%%2.2.5%%%%%%%%%%%%%%%%%%%%%%%%%%%%%%%%%%%%%%
   \item[2.2.5.]  Show that if $f$ is analytic on $D$, then
                  $g(z) = \overline{f(\overline{z})}$ is analytic on the
                  reflected domain $D^* = \{\overline{z} : z \in D\}$, and
                  $g'(z) = \overline{f'(\overline{z})}$.
                  
      \textbf{Proof.} Suppose that $f$ is analytic on $D$. Let
      $D^* = \{\overline{z} : z \in D\}$ and
      $$g : D^* \rightarrow \C, \quad z \mapsto\overline{f(\overline{z})}.$$
      For readiblity sake, we let $C(z)$ denote the complex conjugate of a
      complex number $z$. Let $z \in D^*$ (so that $\overline{z} \in D$ and,
      thus, $f$ is differentaible at $\overline{z}$). It follows that
      \begin{align*}
         \frac{dg}{dz}(z) &= \lim_{\Delta z \rightarrow 0}
            \frac{g(z + \Delta z) - g(z)}{\Delta z} \\
            &= \lim_{\Delta z \rightarrow 0}
            \frac{C(f(\overline{z + \Delta z})) -
               C(f(\overline{z}))}{\Delta z} \\
            &= \lim_{\Delta z \rightarrow 0}
            \frac{C(f(\overline{z + \Delta z}) -
               f(\overline{z}))}{C(C(\Delta z))} \\
            &= \lim_{\Delta z \rightarrow 0}
            C\left(\frac{f(\overline{z} + \overline{\Delta z}) -
               f(\overline{z})}{\overline{\Delta z}}\right) \\
            &= C\left(\lim_{\Delta z \rightarrow 0}
            \frac{f(\overline{z} + \overline{\Delta z}) -
               f(\overline{z})}{\overline{\Delta z}}\right) \\
            &= C\left(\lim_{\overline{\Delta z} \rightarrow 0}
            \frac{f(\overline{z} + \overline{\Delta z}) -
               f(\overline{z})}{\overline{\Delta z}}\right)
               &[\overline{\Delta z} \rightarrow 0 \text{ iff }
                 \Delta z \rightarrow 0] \\
            &= C(f'(\overline{z})) = \overline{f'(\overline{z})}.
               &[f \text{ is differentiable at }\overline{z}]
      \end{align*}
      That is, $g$ is differentable for all $z \in D^*$, and we conclude that
      $g$ is analytic on $D^*$. \qed
%%%%%%%%%%%%%%%%%%%%%%%%%%%%%%%%%%%%%2.3.3%%%%%%%%%%%%%%%%%%%%%%%%%%%%%%%%%%%%%%
   \item[2.3.3.]  Show that if $f$ and $\overline{f}$ are both analytic on a
                  domain $D$, then $f$ is constant.
                  
      \textbf{Proof.} Suppose that $f$ and $\overline{f}$ are both analytic on a
      domain $D$. Let $f = u + iv$, so that $\overline{f} = u - iv$. Because
      $f$ is analytic, it follows by the Cauchy-Riemann equations that
      \begin{equation} \label{2_3_3_1}
         \frac{\partial u}{\partial x} = \frac{\partial v}{\partial y}
            \text{ and }
         \frac{\partial u}{\partial y} = -\frac{\partial v}{\partial x}
      \end{equation}
      on $D$. Similarly, since  $\overline{f}$ is also analytic, it follows
      by the Cauchy-Riemann equations that
      \begin{equation} \label{2_3_3_2}
      \frac{\partial u}{\partial x} = -\frac{\partial v}{\partial y}
        \text{ and }
        \frac{\partial u}{\partial y} = \frac{\partial v}{\partial x}
      \end{equation}
      on $D$. By solving \eqref{2_3_3_1} and \eqref{2_3_3_2}, we shall get
      $$\frac{\partial u}{\partial x} = \frac{\partial u}{\partial y} =
        \frac{\partial v}{\partial x} = \frac{\partial u}{\partial y} = 0$$
      on $D$; that is, $\Delta u = \Delta v = 0$. It follows by the Theorem on
      Page 38 that $u$ and $v$ are both constant on $D$, so that $f$ is constant
      on $D$. \qed
%%%%%%%%%%%%%%%%%%%%%%%%%%%%%%%%%%%%%2.3.4%%%%%%%%%%%%%%%%%%%%%%%%%%%%%%%%%%%%%%
   \item[2.3.4.]  Show that if $f$ is analytic on a domain $D$, and if $|f|$ is
                  constant, then $f$ is constant. \textit{Hint.} Write
                  $\overline{f} = |f|^2/f$.
                  
      \textbf{Proof.} Let $f$ be analytic on a domain $D$. Suppose that $|f|$ is
      constant. If $f$ is 0 at any point in $D$, then $f$ must be identically
      zero at all points in $D$ because $|f|$ is constant. So further suppose
      that $f$ is never zero. Let $r = |f|^2$, where $r$ is some positive
      real number. Since the constant funtion $g(z) = r$ and $f$ are both
      differentiable (since $f$ is analytic), it follows that $g/f$ is also
      differentiable. Observe that $\overline{f} = |f|^2/f = r/f = g/f$. That
      is, $\overline{f}$ is also differentiable on $D$. So it follows by
      Exercise 2.3.3 that $f$ is constant on $D$. \mbox{ }\qed
%%%%%%%%%%%%%%%%%%%%%%%%%%%%%%%%%%%%%2.3.6%%%%%%%%%%%%%%%%%%%%%%%%%%%%%%%%%%%%%%
   \item[2.3.6.]  If $f = u + iv$ is analytic on $D$, then $\nabla v$ is
                  obtained by rotating $\nabla u$ by $90^\circ$. In particular,
                  $\nabla u$ and $\nabla v$ are orthogonal.
                  
      \textbf{Solution.} Suppose that $f = u + iv$ is analytic on $D$. By
      definition, we have 
      $$\Delta u = \left(\frac{\partial u}{\partial x},
           \frac{\partial u}{\partial y}\right) \text{ and }
        \Delta v = \left(\frac{\partial v}{\partial x},
            \frac{\partial v}{\partial y}\right).$$
      Since $f$ is analytic, apply the Cauchy-Riemann equations to $\Delta v$,
      so that
      $$\Delta v = \left(\frac{\partial v}{\partial x},
            \frac{\partial v}{\partial y}\right) =
        \left(-\frac{\partial u}{\partial y},
            \frac{\partial u}{\partial x}\right).$$
      First we observe that $\Delta u$ are $\Delta v$ orthogonal because
      $$\Delta u \cdot \Delta v = \left(\frac{\partial u}{\partial x},
           \frac{\partial u}{\partial y}\right) \cdot
        \left(-\frac{\partial u}{\partial y},
            \frac{\partial u}{\partial x}\right) =
        -\frac{\partial u}{\partial x}\frac{\partial u}{\partial y} +
            \frac{\partial u}{\partial x} \frac{\partial u}{\partial y} = 0.$$
            
      Recall: to rotate a 2-column vector by $90^\circ$, we can (left) multiply
      the vector by the matrix $A = \left[\begin{tabular}{@{}rr@{}}
         0 & $-1$ \\
         1 & 0
      \end{tabular}\right]$. Thus
      $$
         A(\Delta u)^T = A\left[\begin{tabular}{@{}rr@{}}
         $\frac{\partial u}{\partial x}$ \\ \\
         $\frac{\partial u}{\partial y}$
      \end{tabular}\right] = \left[\begin{tabular}{@{}rr@{}}
         $-\frac{\partial u}{\partial y}$ \\ \\
         $\frac{\partial u}{\partial x}$
      \end{tabular}\right] = (\Delta v)^T,
      $$
      so that $\Delta v$ is obtained by rotating $\Delta u$ by $90^\circ$. \qed
%%%%%%%%%%%%%%%%%%%%%%%%%%%%%%%%%%%%%2.4.8%%%%%%%%%%%%%%%%%%%%%%%%%%%%%%%%%%%%%%
   \item[2.4.8.]  Sketch the image of the circle $\{|z-1| \le 1\}$ under the map
                  $w = z^2$. Compute the area of the image.
                  
      \textbf{Solution.} Observe that the region $D$ is given by
      $$r = 2 \cos(\theta), 0 \le \theta \le \pi$$
      in polar coordinates. Thus using Exercise 2.4.7, it follows that the area
      of the region is
      \begin{align*}
         \int\int_D|f'(z)|^2dx\;dy &= \int\int_D|2z|^2dx\;dy \\
            &= 4\int\int_D|z|^2dx\;dy \\
            &= 4\int\int_D x^2 + y^2 dx\;dy  \\
            &= 8\int_0^{\pi/2}\int_0^{2\cos(\theta)} r^3 dr\;d\theta  \\
            &= 8\int_0^{\pi/2} \frac{r^4}{4} \bigg|_0^{2\cos(\theta)}\;d\theta  \\
            &= 32\int_0^{\pi/2} \cos^4(\theta)\;d\theta  \\
            &= 32\left(\frac{3x}{8}+ \frac{\sin(2x)}{4}+\frac{\sin(4x)}{32}\right)\bigg|_0^{\pi/2} \\
            &= 32 \cdot \frac{3\pi}{16} \\
            &= 6\pi.
      \end{align*} 
      
%%%%%%%%%%%%%%%%%%%%%%%%%%%%%%%%%%%%%2.4.9%%%%%%%%%%%%%%%%%%%%%%%%%%%%%%%%%%%%%%
   \item[2.4.9.]  Compute
                  $$\int\int_D|f'(z)|^2dx\;dy,$$
                  for $f(z) = z^2$ and $D$ the open unit disk $\{|z| \le 1\}$.
                  Interpret your answer in terms of areas.
                  
      \textbf{Solution.} We have that
      \begin{align*}
         \int\int_D|f'(z)|^2dx\;dy &= \int\int_D|2z|^2dx\;dy \\
            &= 4\int\int_D|z|^2dx\;dy \\
            &= 4\int\int_D x^2 + y^2 dx\;dy  \\
            &= 4\int_0^{2\pi}\int_0^1 r^3 dr\;d\theta  \\
            &= 4\int_0^1 r^3 dr\;\int_0^{2\pi}d\theta  \\
            &= 2\pi.
      \end{align*}
      According to Exercise 2.4.7, the area of $f(D)$ is $2\pi$. 
%%%%%%%%%%%%%%%%%%%%%%%%%%%%%%%%%%%%%2.5.2%%%%%%%%%%%%%%%%%%%%%%%%%%%%%%%%%%%%%%
   \item[2.5.2.]  Show that if $v$ is a harmonic conjugate for $u$, then $-u$ is
                  a harmonic conjugate for $v$.
                  
      \textbf{Proof.} Suppose $v$ is a harmonic conjugate for $u$. Then it
      follows by definition that $u + iv$ is analytic. Since $u$ is harmonic, it
      follows that
      $$\frac{\partial^2 u}{\partial x^2} + \frac{\partial^2 u}{\partial y^2}
        = 0,$$
      so that
      $$-\frac{\partial^2 u}{\partial x^2} - \frac{\partial^2 u}{\partial y^2}
        = 0.$$
      That is, $-u$ is also harmonic. To show that $-u$ is a harmonic conjugate
      for $v$, it suffices to show that $v - iu$ is analytic. The constant
      function $g(z) = -i$ is complex differentiable and since $u + iv$ is
      analytic, it must also be complex differentiable; thus the product
      $$g \cdot (u + iv) = -i(u + iv) = v - iu$$
      must be also be complex differentiable, and thus analytic. We conclude
      that $-u$ is a harmonic conjugate for $v$. \qed
%%%%%%%%%%%%%%%%%%%%%%%%%%%%%%%%%%%%%2.5.4%%%%%%%%%%%%%%%%%%%%%%%%%%%%%%%%%%%%%%
   \item[2.5.4.]  Show that if $h(z)$ is a complex-valued harmonic function 
                  (solution of Laplace's equation) such that $zh(z)$ is also
                  harmonic, then $h(z)$ is analytic.
                  
      \textbf{Proof.} Suppose $h(z)$ is a complex-valued harmonic function such
      that $zh(z)$ is also harmonic. First recall that $z = x + iy$ so that
      \begin{align*}
         \frac{\partial z}{\partial x} = 1, \quad
         \frac{\partial^2 z}{\partial x^2} = 0, \quad
         \frac{\partial z}{\partial y} = i, \quad \text{and} \quad
         \frac{\partial^2 z}{\partial y^2} = 0.
      \end{align*}
      
      Now since $h(z)$ is harmonic, it follows that
      \begin{equation} \label{2_5_4_1}
         0 = \frac{\partial^2 h}{\partial x^2} +
             \frac{\partial^2 h}{\partial y^2}.
      \end{equation}
      
      Also, since $zh(z)$ is harmonic, we have that
      \begin{align*}
         0 &= \frac{\partial^2 (zh)}{\partial x^2} +
              \frac{\partial^2 (zh)}{\partial y^2} \\
           &=  \frac{\partial}{\partial x}
               \left(\frac{\partial (zh)}{\partial x}\right) +
               \frac{\partial}{\partial y}
               \left(\frac{\partial (zh)}{\partial y}\right) \\
           &=  \frac{\partial}{\partial x}
               \left(\frac{\partial z}{\partial x}h +
               \frac{\partial h}{\partial x}z\right) +
               \frac{\partial}{\partial y}
               \left(\frac{\partial z}{\partial y}h +
               \frac{\partial h}{\partial y}z\right) \\
           &=  \frac{\partial^2z}{\partial x^2}h +2\frac{\partial z}{\partial x}
               \frac{\partial h}{\partial x} +\frac{\partial^2h}{\partial x^2}z+
               \frac{\partial^2z}{\partial y^2}h+2\frac{\partial z}{\partial y}
               \frac{\partial h}{\partial y}+\frac{\partial^2h}{\partial y^2}z\\
           &=  \frac{\partial^2z}{\partial x^2}h +2\frac{\partial z}{\partial x}
               \frac{\partial h}{\partial x} +
               \frac{\partial^2z}{\partial y^2}h+2\frac{\partial z}{\partial y}
               \frac{\partial h}{\partial y}+
               \left(\frac{\partial^2h}{\partial x^2}+
               \frac{\partial^2h}{\partial y^2}\right)z \\
           &=  0 \cdot h+2\cdot1\cdot
               \frac{\partial h}{\partial x} +
               0 \cdot h+2\cdot i \cdot
               \frac{\partial h}{\partial y}+ 0 \cdot z \\
           &=  2\frac{\partial h}{\partial x} +
               2i\frac{\partial h}{\partial y} \\
           &=  2\frac{\partial u}{\partial x} +2i\frac{\partial v}{\partial x}+
               2i\frac{\partial u}{\partial y}-2\frac{\partial v}{\partial y} \\
           &=  \frac{\partial u}{\partial x}-\frac{\partial v}{\partial y}+
               i\left(\frac{\partial u}{\partial y}+
               \frac{\partial v}{\partial x}\right).               
      \end{align*}
      
      That is
      $$\frac{\partial u}{\partial x} = \frac{\partial v}{\partial y}
        \text{ and }
        \frac{\partial u}{\partial y} = -\frac{\partial v}{\partial x},$$
      so that $u$ and $v$ satisfy the Cauchy-Riemann equations. Thus, $f$ is
      analytic. \qed
%%%%%%%%%%%%%%%%%%%%%%%%%%%%%%%%%%%%%2.6.3%%%%%%%%%%%%%%%%%%%%%%%%%%%%%%%%%%%%%%
   \item[2.6.3.]  Sketch the families of level curves of $u$ and $v$ for the
                  functions $f = u + iv$ given by (a) $f(z) = e^z$, (b)
                  $f(z) = e^{\alpha z}$, where $\alpha$ is complex. Determine
                  which $f(z)$ is conformal and where it is not conformal.
%%%%%%%%%%%%%%%%%%%%%%%%%%%%%%%%%%%%%2.7.3%%%%%%%%%%%%%%%%%%%%%%%%%%%%%%%%%%%%%%
   \item[2.7.3.]  Consider the fractional linear transformation that maps 1 to
                  $i$, 0 to $1 + i$, and $-1$ to 1. Determine the image of the
                  unit circle $\{|z| = 1\}$, the image of the open unit disk
                  $\{|z| < 1\}$, and the image of the imaginary axis. Illustrate
                  with a sketch.
                  
      \textbf{Solution.} Let $w$ be the fractional linear transformation such
      that $1 \mapsto i$, $0 \mapsto 1 + i$, and $-1 \mapsto 1$. By Theorem on
      page 64, it follows that the fractional linear transformation
      $$f(z) = \frac{-z + 1}{z+1}$$
      maps 1 to 0, 0 to 1, and $-1$ to $\infty$. Also the fractional linear
      transformation
      $$g(z) = \frac{iz + 1}{z-1}$$
      maps $i$ to 0, $1 + i$ to 1, and 1 to $\infty$; thus
      $$g^{-1}(z) = \frac{z + 1}{z-i}$$
      maps 0 to $i$, 1 to $1 + i$, and $\infty$ to 1. That is,
      $$w = (g^{-1} \circ f) = \frac{i-1}{z+i}.$$
      To determine the  image of the following sets under $w$, we shall follow
      the same line of reasoning as Gamelin did for his solutions to the Exercises on page 66.
      
      \textbf{Image of unit circle } $|z| = 1$. Since $w(-i) = \infty$, it
      follows that $w$ maps the circle $|z| = 1$ to the straight line that
      passes through $i$ and 1.
      
      \textbf{Image of open disk } $|z| < 1$. Since orientations are preserved,
      it follows that $w$ maps the open disk $|z| < 1$ to the upper left half
      plane of the  straight line that passes through $i$ and 1.
      
      \textbf{Image of imaginary axis.} $w$ maps the imaginary axis to a circle
      that passes through $1 + i$.
%%%%%%%%%%%%%%%%%%%%%%%%%%%%%%%%%%%%%2.7.9%%%%%%%%%%%%%%%%%%%%%%%%%%%%%%%%%%%%%%
   \item[2.7.9.]  Show that the fractional linear transformations that are real
                  on the real axis are precisely those that can be expressed in
                  the form $(az+b)/(cz+d)$, where $a$, $b$, $c$, and $d$ are
                  real.
                  
      \textbf{Proof.} Suppose that $f$ is a linear transformation that is real
      on the real axis. Then there must exist distinct real numbers $r_0$,
      $r_1$, and $r_2$ such that $f(r_0) = 0$, $f(r_1) = 1$, and
      $f(r_2) = \infty$. Now consider the fractional linear transformation
      $$g(z) = \frac{z-r_0}{r_1-r_0}.$$
      Note that that $r_0$, $r_1$, and $r_2$ have the same images under $f$
      and $g$. Thus by Theorem on Page 64, it follows that $f$ and $g$ are
      equal (up to multiplication of $a$, $b$, $c$, and $d$ by the same nonzero
      constant). Thus, for a nonzero constant complex number $C$, we have
      $$f = \frac{C(z - r_0)}{C(r_1 - r_0)}.$$
      That is, $f$ can be expressed in the form $(az+b)/(cz+d)$, where $a$,
      $b$, $c$, and $d$ are real. This form is precisely
      $$\frac{z-r_0}{r_1-r_0}.$$ \qed
\end{enumerate}
\end{document} 
