\documentclass[9pt]{article}

\usepackage{amssymb}
\usepackage{amsmath}
\usepackage{amsfonts}
\usepackage{comment}
\usepackage{fancyhdr}
\usepackage{mathrsfs}
\usepackage{enumitem}
%\usepackage[retainorgcmds]{IEEEtrantools}


\usepackage{tikz}

\voffset = -50pt
%\textheight = 700pt
\addtolength{\textwidth}{60pt}
\addtolength{\evensidemargin}{-30pt}
\addtolength{\oddsidemargin}{-30pt}
%\setlength{\headheight}{44pt}

\pagestyle{fancy}
\fancyhf{} % clear all fields
\fancyhead[R]{%
  \scshape
  \begin{tabular}[t]{@{}r@{}}
  MATH 562, Fall 2015\\Section 1 (9934)\\
  HW \#1, DUE: 2015, September 10
  \end{tabular}}
\fancyhead[L]{%
  \scshape
  \begin{tabular}[t]{@{}r@{}}
  JOSEPH OKONOBOH\\Mathematics\\Cal State Long Beach
  \end{tabular}}
\fancyfoot[C]{\thepage}

\newcommand{\qed}{\hfill \ensuremath{\Box}}


\newcommand*\circled[1]{\tikz[baseline=(char.base)]{
            \node[shape=circle,draw,inner sep=2pt] (char) {#1};}}

\newcommand{\Z}{\mathbb{Z}}
\newcommand{\I}{\mathbb{I}}
\newcommand{\M}{\mathbb{M}}
\newcommand{\R}{\mathbb{R}}
\newcommand{\C}{\mathbb{C}}
\everymath{\displaystyle}
%\setcounter{section}{-1}

\begin{document}
\begin{enumerate}
%%%%%%%%%%%%%%%%%%%%%%%%%%%%%%%%%%%%%2.2.3%%%%%%%%%%%%%%%%%%%%%%%%%%%%%%%%%%%%%%
   \item[2.2.3.]  Show from the definition that the functions $x = \text{Re }z$
                  and $y = \text{Im }z$ are not complex differentiable at any 
                  point.
                  
      \textbf{Proof.} Let $x(z) = \text{Re}(z)$. It follows by definition that
      \begin{align*}
         \frac{dx}{dz} &= \lim_{\Delta z\rightarrow 0}
            \frac{x(z + \Delta z) - x(z)}{\Delta z} \\
            &=  \lim_{\Delta z\rightarrow 0}
            \frac{\text{Re}(z + \Delta z) - \text{Re}(z)}{\Delta z} \\
            &=  \lim_{\Delta z\rightarrow 0}
            \frac{\text{Re}(z) + \text{Re}(\Delta z) - \text{Re}(z)}{\Delta z} \\
            &=  \lim_{\Delta z\rightarrow 0}
            \frac{\text{Re}(\Delta z)}{\Delta z}.
      \end{align*}      
      If we let $\Delta z$ approach 0 along the real axis, then $\Delta z$
      is real, so that 
      $$\lim_{\Delta z\rightarrow 0}\frac{\text{Re}(\Delta z)}{\Delta z}
        = \frac{\Delta z}{\Delta z} = 1.$$     
      However, if we let $\Delta z$ approach 0 along the imaginary axis, then
      $\Delta z$ is pure imaginary, so that 
      $$\lim_{\Delta z\rightarrow 0}\frac{\text{Re}(\Delta z)}{\Delta z}
        = \frac{0}{\Delta z} = 0.$$
      Thus $x$ is not complex differentiable since the limit has two different
      values for two different paths. Similarly, we Let $y(z) = \text{Im}(z)$.
      It follows by definition that
      \begin{align*}
         \frac{dy}{dz} &= \lim_{\Delta z\rightarrow 0}
            \frac{y(z + \Delta z) - y(z)}{\Delta z} \\
            &=  \lim_{\Delta z\rightarrow 0}
            \frac{\text{Im}(z + \Delta z) - \text{Im}(z)}{\Delta z} \\
            &=  \lim_{\Delta z\rightarrow 0}
            \frac{\text{Im}(z) + \text{Im}(\Delta z) - \text{Im}(z)}{\Delta z} \\
            &=  \lim_{\Delta z\rightarrow 0}
            \frac{\text{Im}(\Delta z)}{\Delta z}.
      \end{align*}      
      If we let $\Delta z$ approach 0 along the real axis, then $\Delta z$
      is real, so that 
      $$\lim_{\Delta z\rightarrow 0}\frac{\text{Im}(\Delta z)}{\Delta z}
        = \frac{0}{\Delta z} = 0.$$     
      However, if we let $\Delta z$ approach 0 along the imaginary axis, then
      $\Delta z$ is pure imaginary, so that 
      $$\lim_{\Delta z\rightarrow 0}\frac{\text{Im}(\Delta z)}{\Delta z}
        = \frac{\Delta z}{\Delta z} = 1.$$
      Thus $y$ is not complex differentiable since the limit has two different
      values for two different paths. \qed
%%%%%%%%%%%%%%%%%%%%%%%%%%%%%%%%%%%%%2.2.5%%%%%%%%%%%%%%%%%%%%%%%%%%%%%%%%%%%%%%
   \item[2.2.5.]  Show that if $f$ is analytic on $D$, then
                  $g(z) = \overline{f(\overline{z})}$ is analytic on the
                  reflected domain $D^* = \{\overline{z} : z \in D\}$, and
                  $g'(z) = \overline{f'(\overline{z})}$.
                  
      \textbf{Proof.} Suppose that $f$ is analytic on $D$. Let
      $D^* = \{\overline{z} : z \in D\}$ and
      $$g : D^* \rightarrow \C, \quad z \mapsto\overline{f(\overline{z})}.$$
      For readiblity sake, we let $C(z)$ denote the complex conjugate of a
      complex number $z$. So
      \begin{align*}
         \frac{dg}{dz} &= \lim_{\Delta z \rightarrow 0}
            \frac{g(z + \Delta z) - g(z)}{\Delta z} \\
            &= \lim_{\Delta z \rightarrow 0}
            \frac{C(f(\overline{z + \Delta z})) -
               C(f(\overline{z}))}{\Delta z} \\
            &= \lim_{\Delta z \rightarrow 0}
            \frac{C(f(\overline{z + \Delta z}) -
               f(\overline{z}))}{C(C(\Delta z))} \\
            &= \lim_{\Delta z \rightarrow 0}
            C\left(\frac{f(\overline{z} + \overline{\Delta z}) -
               f(\overline{z})}{\overline{\Delta z}}\right) \\
            &= C\left(\lim_{\Delta z \rightarrow 0}
            \frac{f(\overline{z} + \overline{\Delta z}) -
               f(\overline{z})}{\overline{\Delta z}}\right) \\
            &= C\left(\lim_{\overline{\Delta z} \rightarrow 0}
            \frac{f(\overline{z} + \overline{\Delta z}) -
               f(\overline{z})}{\overline{\Delta z}}\right) \\
            &= C\left(\lim_{\Delta z \rightarrow 0}
            \frac{f(\overline{z} + \Delta z) -
               f(\overline{z})}{\Delta z}\right) \\
            &= C(f'\overline{z}) = \overline{}
      \end{align*}
%%%%%%%%%%%%%%%%%%%%%%%%%%%%%%%%%%%%%2.3.3%%%%%%%%%%%%%%%%%%%%%%%%%%%%%%%%%%%%%%
   \item[2.3.3.]  Show that if $f$ and $\overline{f}$ are both analytic on a
                  domain $D$, then $f$ is constant.
%%%%%%%%%%%%%%%%%%%%%%%%%%%%%%%%%%%%%2.3.4%%%%%%%%%%%%%%%%%%%%%%%%%%%%%%%%%%%%%%
   \item[2.3.4.]  Show that if $f$ is analytic on a domain $D$, and if $|f|$ is
                  constant, then $f$ is constant. \textit{Hint.} Write
                  $\overline{f} = |f|^2/f$.
%%%%%%%%%%%%%%%%%%%%%%%%%%%%%%%%%%%%%2.3.6%%%%%%%%%%%%%%%%%%%%%%%%%%%%%%%%%%%%%%
   \item[2.3.6.]  If $f = u + iv$ is analytic on $D$, then $\nabla v$ is
                  obtained by rotating $\nabla u$ by $90^\circ$. In particular,
                  $\nabla u$ and $\nabla v$ are orthogonal.
\end{enumerate}
\end{document} 
