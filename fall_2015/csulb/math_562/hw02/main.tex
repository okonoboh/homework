\documentclass[9pt]{article}

\usepackage{amssymb}
\usepackage{amsmath}
\usepackage{amsfonts}
\usepackage{comment}
\usepackage{fancyhdr}
\usepackage{mathrsfs}
\usepackage{enumitem}
\usepackage{graphicx}
%\usepackage[retainorgcmds]{IEEEtrantools}


\usepackage{tikz}

\voffset = -50pt
%\textheight = 700pt
\addtolength{\textwidth}{60pt}
\addtolength{\evensidemargin}{-30pt}
\addtolength{\oddsidemargin}{-30pt}
%\setlength{\headheight}{44pt}

\pagestyle{fancy}
\fancyhf{} % clear all fields
\fancyhead[R]{%
  \scshape
  \begin{tabular}[t]{@{}r@{}}
  MATH 562, Fall 2015\\Section 1 (9934)\\
  HW \#2, DUE: 2015, September 24
  \end{tabular}}
\fancyhead[L]{%
  \scshape
  \begin{tabular}[t]{@{}r@{}}
  JOSEPH OKONOBOH\\Mathematics\\Cal State Long Beach
  \end{tabular}}
\fancyfoot[C]{\thepage}

\newcommand{\qed}{\hfill \ensuremath{\Box}}


\newcommand*\circled[1]{\tikz[baseline=(char.base)]{
            \node[shape=circle,draw,inner sep=2pt] (char) {#1};}}

\newcommand{\Z}{\mathbb{Z}}
\newcommand{\I}{\mathbb{I}}
\newcommand{\M}{\mathbb{M}}
\newcommand{\R}{\mathbb{R}}
\newcommand{\C}{\mathbb{C}}
\everymath{\displaystyle}
%\setcounter{section}{-1}

\begin{document}
\begin{enumerate}
%%%%%%%%%%%%%%%%%%%%%%%%%%%%%%%%%%%%%3.1.4%%%%%%%%%%%%%%%%%%%%%%%%%%%%%%%%%%%%%%
   \item[3.1.4.]  Evaluate $\int_\gamma y\;dx$ both directly and using
                  Green's Theorem, where $\gamma$ is the semicircle in the upper
                  half-plane from $R$ to $-R$.
                  
      \textbf{Solution.} Using the following parametrization
      $$x = R \cos(t), y = R \sin(t) \quad 0 \le t \le \pi$$
      for $\gamma$, it follows that
      \begin{align*}
         \int_\gamma y\;dx &= \int_0^\pi R \sin(t) (-R \sin(t))\;dt \\
            &= -R^2 \int_0^\pi \sin^2(t)\; \\
            &= -R^2 \int_0^\pi \frac{1}{2} - \frac{1}{2}\cos(2\theta)\;dt\\
            &= -R^2 \left(\frac{1}{2}\theta -
                  \frac{1}{4}\sin(2\theta)\right)\bigg|_0^\pi \\
            &= -\frac{1}{2}\pi R^2.
      \end{align*}
      
      Let $P(x, y) = y$, $Q(x, y) = 0$, $D$ the region bounded by $\gamma$ and
      the $x$-axis, and $C = \{(x, 0) : -R \le x \le R\}$. Observe that
      $\int_C y\;dx = 0$, so that
      $$\int_{\partial D} P\;dx + Q\;dy = \int_{\gamma} y\;dx + \int_C y\;dx =
      \int_{\gamma} y\;dx.$$
      According to Green's Theorem, it follows that
      \begin{align*}
         \int_{\gamma} y\;dx = \int_{\partial D} P\;dx + Q\;dy &= \iint_D
            \left(\frac{\partial Q}{\partial x} -
            \frac{\partial P}{\partial y}\right)\;dx\;dy \\
            &= -\int_{-R}^R\int_0^{\sqrt{R^2 - x^2}} dy\;dx \\
            &= -\int_0^\pi\int_0^R r\;dr\;d\theta \\
            &= -\int_0^\pi\frac{1}{2}R^2\;d\theta \\
            &= -\frac{1}{2}\pi R^2,
      \end{align*}
      confirming our previous result.
%%%%%%%%%%%%%%%%%%%%%%%%%%%%%%%%%%%%%3.2.3%%%%%%%%%%%%%%%%%%%%%%%%%%%%%%%%%%%%%%
   \item[3.2.3.]  Suppose that $P$ and $Q$ are smooth functions on the annulus
                  $\{a < |z| < b\}$ that satisfy
                  $\partial P/ \partial y =  \partial Q/ \partial x$. Show
                  directly using Green's theorem that
                  $\oint_{|z|=r} P\;dx + Q\;dy$ is independent of the radius
                  $r$, for $a < r < b$.
                  
      \textbf{Solution.} Suppose that $P$ and $Q$ are smooth functions on the
      annulus $\{a < |z| < b\}$ that satisfy
      $\partial P/ \partial y =  \partial Q/ \partial x$. Consider a real number
      $r \in (a, b)$. Let $D_x$ be region bounded by the circle of radius $x$
      centered at the origin. Observe that $D_r = D_a \cup \{a < |z| < r\}$ and
      $D_a \cap \{a < |z| < r\} = \emptyset$. Since
      $\partial P/ \partial y =  \partial Q/ \partial x$ on the annulus
      $\{a < |z| < b\}$ and since $a < r < b$, it must be the case that
      $\partial P/ \partial y =  \partial Q/ \partial x$ on the annulus
      $\{a < |z| < r\}$. It follows by Green's Theorem that
      \begin{align*}
         \oint_{|z|=r} P\;dx + Q\;dy &= \iint_{D_r}
            \left(\frac{\partial Q}{\partial x} -
            \frac{\partial P}{\partial y}\right)\;dx\;dy \\
            &= \iint_{D_a \cup \{a < |z| < r\}}
            \left(\frac{\partial Q}{\partial x} -
            \frac{\partial P}{\partial y}\right)\;dx\;dy \\
            &= \iint_{D_a} \left(\frac{\partial Q}{\partial x} -
            \frac{\partial P}{\partial y}\right)\;dx\;dy +
            \iint_{\{a < |z| < r\}} \left(\frac{\partial Q}{\partial x} -
            \frac{\partial P}{\partial y}\right)\;dx\;dy \\
            &= \iint_{D_a} \left(\frac{\partial Q}{\partial x} -
            \frac{\partial P}{\partial y}\right)\;dx\;dy +
            \iint_{\{a < |z| < r\}} \left(\frac{\partial Q}{\partial x} -
            \frac{\partial Q}{\partial x}\right)\;dx\;dy \\
            &= \iint_{D_a} \left(\frac{\partial Q}{\partial x} -
            \frac{\partial P}{\partial y}\right)\;dx\;dy +
            \iint_{\{a < |z| < r\}} 0 \;dx\;dy  \\
            &= \iint_{D_a} \left(\frac{\partial Q}{\partial x} -
            \frac{\partial P}{\partial y}\right)\;dx\;dy.
      \end{align*}
     Since the double integral $\iint_{D_a} \left(\frac{\partial Q}{\partial x}
     - \frac{\partial P}{\partial y}\right)\;dx\;dy$ does not depend on the $r$,
     it follows that $\oint_{|z|=r} P\;dx + Q\;dy$ also does not depend on $r$.
%%%%%%%%%%%%%%%%%%%%%%%%%%%%%%%%%%%%%3.3.1%%%%%%%%%%%%%%%%%%%%%%%%%%%%%%%%%%%%%%
   \item[3.3.1.]  For each of the following harmonic functions $u$, find $du$,
                  find $dv$, and find $v$, the conjugate harmonic function of
                  $u$.
                  \begin{enumerate}
                     \item[(b)] $u(x, y) = x^3 - 3xy^2$.
                     \item[(d)] $u(x, y) = \frac{y}{x^2+y^2}$.
                  \end{enumerate}
      
      \textbf{Solution.}
      
      \begin{enumerate}
         \item[(b)] Given $u(x, y) = x^3 - 3xy^2$. We know that
                    $$du = \frac{\partial u}{\partial x} dx +
                      \frac{\partial u}{\partial y} dy.$$
                    Thus
                    $$du = (3x^2 - 3y^2)\;dx - 6xy\;dy.$$
                    Using the Cauchy-Riemann equations:
                    $$\frac{\partial u}{\partial x} = 
                      \frac{\partial v}{\partial y} \text{ and }
                      \frac{\partial u}{\partial y} = 
                      -\frac{\partial v}{\partial x}$$
                    it follows that
                    so that $3x^2 - 3y^2 = \frac{\partial v}{\partial y}$, and
                    thus, $v(x, y) = 3x^2y - y^3 + f(x)$. Thus
                    $$6xy + f'(x) = \frac{\partial v}{\partial x} =
                     -\frac{\partial u}{\partial y} = 6xy,$$ so that
                     $f$ is constant. Thus $v(x, y) = 3x^2y - y^3$, so
                     that
                     $$dv = 6xy\;dx + (3x^2 - 3y^2)\;dy.$$
         \item[(d)] Given $u(x, y) = \frac{y}{x^2+y^2}$, it follows that
                    $$\frac{\partial u}{\partial x} = -\frac{2xy}{(x^2+y^2)^2}
                      \text{ and }
                      \frac{\partial u}{\partial y} = \frac{x^2-y^2}{(x^2+y^2)^2},$$
                     so that
                    $$du = -\frac{2xy}{(x^2+y^2)^2}\;dx +  \frac{x^2-y^2}{(x^2+y^2)^2}\;dy.$$
                    Using the Cauchy-Riemann equations, we must have that
                    $$\frac{\partial u}{\partial x} = -\frac{2xy}{(x^2+y^2)^2} = 
                      \frac{\partial v}{\partial y},$$
                    so that
                    $$v(x, y) = \frac{x}{x^2+y^2} + f(x).$$
                    Moreover,
                    $$\frac{\partial u}{\partial y} = \frac{x^2-y^2}{(x^2+y^2)^2} =
                      -\frac{\partial v}{\partial x} = \frac{x^2-y^2}{(x^2+y^2)^2} + f'(x),$$
                   so that $f$ is constant, and we choose $v(x, y) =  \frac{x}{x^2+y^2}$.
                   Thus
                   $$dv = \frac{y^2-x^2}{(x^2+y^2)^2}\;dx -\frac{2xy}{(x^2+y^2)^2}\;dy$$
      \end{enumerate}
%%%%%%%%%%%%%%%%%%%%%%%%%%%%%%%%%%%%%3.3.5%%%%%%%%%%%%%%%%%%%%%%%%%%%%%%%%%%%%%%
   \item[3.3.5.]  The flux of a function $u$ across a curve $\gamma$ is defined
                  to be
                  $$\int_\gamma\frac{\partial u}{\partial n} ds =
                    \int_\gamma\nabla u \cdot \textbf{n}\;ds,$$
                  where $\textbf{n}$ is the unit normal vector to $\gamma$ and
                  $ds$ is arc length. Show that if a harmonic function $u$ on a
                  domain $D$ has a conjugate harmonic function $v$ on $D$, then
                  the integral giving the flux is independent of path in $D$.
                  Further, the flux across a path $\gamma$ in $D$ from $A$ to
                  $B$ is $v(B) - v(A)$.
                  
      \textbf{Proof.} Suppose that $u$ is a harmonic function on a domain $D$
      and that $u$ has a conjugate harmonic function $v$ on $D$. Let $\gamma$ be
      a path in $D$, Let us parametrize $\gamma$ by its arclength; that is,
      $$\gamma(s) = (x(s), y(s)),$$
      so that the unit tangent vector is
      $$\gamma'(s) = \left(\frac{dx}{ds}, \frac{dy}{ds}\right).$$
      It follows that the unit normal vector is given by
      $$n(s) = \left(\frac{dy}{ds}, -\frac{dx}{ds}\right).$$
      Note that since $v$ is the harmonic conjugate of $u$ on $D$, we have that
      $u_x = v_y$ and $u_y = -v_x$, so that
      \begin{align*}
         \int_\gamma\frac{\partial u}{\partial n} ds &=
                    \int_\gamma\nabla u \cdot \textbf{n}\;ds \\
            &= \int_\gamma\left(\frac{\partial u}{\partial x},
               \frac{\partial u}{\partial y}\right) \cdot
               \left(\frac{dy}{ds}, -\frac{dx}{ds}\right) ds \\
            &= \int_\gamma\left(\frac{\partial u}{\partial x}\frac{dy}{ds}
               -\frac{\partial u}{\partial y}\frac{dx}{ds}\right) ds \\
            &= \int_\gamma\frac{\partial u}{\partial x}dy
               -\frac{\partial u}{\partial y}dx \\
            &= \int_\gamma\frac{\partial v}{\partial y}dy
               +\frac{\partial v}{\partial x}dx \\
            &= \int_\gamma dv.
      \end{align*}
      Now let $P = v_x$ and $Q = v_y$. We just showed that
      $$\int_\gamma P\;dx + Q\;dy = \int_\gamma dv.$$
      That is, $\int P\;dx + Q\;dy$ is exact, so it follows by Lemma on
      Page 77 that $\int P\;dx + Q\;dy$ is independent of path. Indeed, if
      $\gamma'$ is any path in $D$, from point $A$ to $B$, then we have
      that
      $$\int_{\gamma'}\frac{\partial u}{\partial n} ds=\int_A^B dv = v(B) - v(A).$$
      
%%%%%%%%%%%%%%%%%%%%%%%%%%%%%%%%%%%%%3.4.1%%%%%%%%%%%%%%%%%%%%%%%%%%%%%%%%%%%%%%
   \item[3.4.1.]  Let $f(z)$ be a continuous function on a domain $D$. Show that
                  if $f(z)$ has the mean value property with respect to circles,
                  as defined above, then $f(z)$ has the mean value property with
                  respect to disks, that is, if $z_0 \in D$ and $D_0$ is a disk
                  centered at $z_0$ with area $A$ and contained in $D$, then
                  $f(z_0) = \frac{1}{A} \iint_{D_0} f(z)\;dx\;dy$.
                  
      \textbf{Proof.} Suppose that $f$ is a continuous function on a domain $D$
      that has the mean value property with respect to circles. Let $z_0 \in D$
      and let $D_0$ be a disk with radius $R$ centered at $z_0$ and contained in
      $D$. It follows that
      \begin{align*}
         \frac{1}{A} \iint_{D_0} f(z)\;dx\;dy &= 
         \frac{1}{A} \int_0^{2\pi}\int_0^R f(z_0 +
            re^{i\theta})\;r\;dr\;d\theta \\ 
         &= \frac{1}{A} \int_0^Rr\;dr\int_0^{2\pi} f(z_0 +
            re^{i\theta})\;d\theta \\ 
         &= \frac{2\pi}{A} \int_0^Rr\;dr\int_0^{2\pi} f(z_0 +
            re^{i\theta})\;\frac{d\theta}{2\pi}\\ 
         &= \frac{2\pi}{A} \int_0^Rr\;dr f(z_0) &[f \text{ has the MVP wrt circles}]\\ 
         &= f(z_0)\frac{2\pi}{\pi R^2} \int_0^Rr\;dr\\ 
         &= f(z_0)\frac{2}{R^2} \int_0^Rr\;dr\\ 
         &= f(z_0)\frac{2}{R^2} \frac{R^2}{2} = f(z_0),
      \end{align*}
      so that $f$ has the mean value property with respect to disks. \qed
%%%%%%%%%%%%%%%%%%%%%%%%%%%%%%%%%%%%%3.4.3%%%%%%%%%%%%%%%%%%%%%%%%%%%%%%%%%%%%%%
   \item[3.4.3.]  A function $f(t)$ on an interval $I = (a, b)$ has the
                  \textbf{mean value property} if
                  $$f\left(\frac{s+t}{2}\right) = \frac{f(s) + f(t)}{2},
                    \quad s, t \in I.$$
                  Show that any affine function $f(t) = At + B$ has the mean
                  value property. Show that any continuous function on $I$ with
                  the mean value property is affine.
                  
      \textbf{Note.} I spent more time on this problem that on any other, but
      I still could not figure it out. I have found a solution to it, but I have
      decided not to write it down because I did not come up with it.
%%%%%%%%%%%%%%%%%%%%%%%%%%%%%%%%%%%%%3.5.1%%%%%%%%%%%%%%%%%%%%%%%%%%%%%%%%%%%%%%
   \item[3.5.1.]  Let $D$ be a bounded domain, and let $u$ be a real-valued
                  harmonic function on $D$ that extends continuously to the
                  boundary $\partial D$. Show that if $a \le u \le b$ on
                  $\partial D$, then $a \le u \le b$ on $D$.
                  
      \textbf{Proof.} Suppose that 
      \begin{equation} \label{1_1}
         a \le u \le b \text{ on  } \partial D.
      \end{equation}
      By subtracting $a$ from the inequalities in \eqref{1_1}, we get
      $0 \le u - a \le b - a$ on $\partial D$, so that $|u - a| \le b - a$ on
      $\partial D$. By the Maximum Principle, it follows that
      $|u - a| \le b - a$ on $D$, so that $2a - b \le u \le b$ on $D$.
      Particularly, we have $u \le b$ on $D$. Similarly, if we subtract $b$
      from \eqref{1_1}, we will get $|b - u| \le b -a$ on $\partial D$, and
      thus $|b - u| \le b - a$ on $D$ by the Maximum Principle. Hence
      $a \le u \le 2b - a$ on $D$. Particularly $a \le u$ on $D$. So we conclude
      that $a \le u \le b$ on $D$. \qed
%%%%%%%%%%%%%%%%%%%%%%%%%%%%%%%%%%%%%3.5.3%%%%%%%%%%%%%%%%%%%%%%%%%%%%%%%%%%%%%%
   \item[3.5.3.]  Use the maximum principle to prove the fundamental theorem of
                  algebra, that any polynomial $p(z)$ of degree $n \ge 1$ has a
                  zero, by applying the maximum principle to $1/p(z)$ on a disk
                  of large radius.
                  
      \textbf{Proof.} Let $p(z)$ be a polynomial of degree $n \ge 1$. Let $C(R)$
      be the circle of radius $R$ centered at the origin and let $D_R$
      be the disk on radius $R$ centered at the origin. Note that as $|z| \rightarrow \infty$, then
      $|p(z)| \rightarrow \infty$. Assume to the contrary that $p$ has no
      zeroes. Then it follows that $|1/p(z)| \rightarrow 0$ as
      $|z| \rightarrow \infty$. That is, if $\varepsilon > 0$, we can
      select some large enough radius $R_\varepsilon$ so that
      $$|1/p(z)| < \varepsilon \text{ for } z \text{ on } C(R_\varepsilon).$$
      By the Maximum Principle, it follows that
      $$|1/p(z)| < \varepsilon \text{ for } z \text{ on } D(R_\varepsilon).$$
      Since $\varepsilon$ was arbitrary, it follows that
      $|1/p(z)| = 0$, a contradiction. Thus $p$ has at least one zero. \qed
%%%%%%%%%%%%%%%%%%%%%%%%%%%%%%%%%%%%%3.5.4%%%%%%%%%%%%%%%%%%%%%%%%%%%%%%%%%%%%%%
   \item[3.5.4.]  Let $f(z)$ be an analytic function on a domain $D$ that has no
                  zeros on $D$.
                  \begin{enumerate}
                     \item Show that if $|f(z)|$ attains its minimum on $D$,
                           then $f(z)$ is constant.
                     \item Show that if $D$ is bounded, and if $f(z)$ extends
                           continuously to the boundary $\partial D$ of $D$,
                           then $|f(z)|$ attains its minimum on $\partial D$.
                  \end{enumerate}
                  
      \textbf{Solution.}
      
      \begin{enumerate}
         \item Suppose $|f(z)|$ attains its minimum in $D$. Since $f$ has no
               zeroes on $D$, it follows that $|1/f(z)|$ attains a maximum on
               $D$; that is, $|1/f(z)| \le M$ on $D$ and $|1/f(z)| = M$ for some
               $z_0 \in D$. Thus, by the Strict Maximum Principle, it follows
               that $1/f(z)$ is constant, so that $f(z)$ is also constant.
         \item Observe that $D \cup \partial D$ is compact, so that $|f(z)|$
               attains a maximum and minimum on $D \cup \partial D$ by Theorem
               on Page 39. Now suppose that $|f(z)|$ does not attain a minumum
               on $\partial D$; then it must attain this minimum on $D$. Thus
               $f(z)$ is constant by part (a), so that $|f(z)|$ attains a 
               minimum on $\partial D$, a contradiction since we assumed that
               this minimum does not occur on $\partial D$; thus $|f(z)|$ must
               attain a minimum on $\partial D$. 
      \end{enumerate} \qed
%%%%%%%%%%%%%%%%%%%%%%%%%%%%%%%%%%%%%3.6.1%%%%%%%%%%%%%%%%%%%%%%%%%%%%%%%%%%%%%%
   \item[3.6.1.]  Consider the fluid flow with constant velocity
                  $\textbf{V} = (2, 1)$. Find the velocity potential $\phi(z)$,
                  the stream function $\psi(z)$, and the complex velocity
                  potential $f(z)$ of the flow. Sketch the streamlines of the
                  flow. Determine the flux of the flow across the interval
                  $[0, 1]$ on the real axis and across the interval $[0, i]$ on
                  the imaginary axis.
                  
      \textbf{Solution.} Consider the fluid flow with constant velocity
      $\textbf{V} = (2, 1)$. The velocity potential $\phi$ is a function $f$
      such that $\nabla \phi = \textbf{V}$. That is,
      $$\frac{\partial \phi}{\partial x} = 2 \text{ and }
        \frac{\partial \phi}{\partial y} = 1.$$
      Since $\frac{\partial \phi}{\partial x} = 2$, it follows that
      $\phi = 2x + f(y)$, so that $f'(y) = 1$, and we conclude that
      $f(y) = y$. Thus $\phi = 2x + y$. The stream function $\psi$ is a
      conjugate harmonic function of $\phi$, so we want
      $$\frac{\partial \phi}{\partial x} = \frac{\partial \psi}{\partial y},$$
      so that $\frac{\partial \psi}{\partial y} = 2$; that is,
      $\psi = 2y + h(x)$. Now $-\frac{\partial \psi}{\partial x} = -h'(x) = 1$,
      so that $h(x) = -x$. That is, $\psi = 2y - x$, and we have that the
      complex velocity potential $f(z)$ is
      $$2x + y + i(2y - x) = 2x + 2yi + y - ix = 2(x + yi) - (x + yi)i = 2z - zi.$$
      The flux across the interval $[0, 1]$ is
      $$\psi(1) - \psi(0) = -1 - 0 = -1,$$
      and the flux across the interval $[0, i]$ is
      $$\psi(i) - \psi(0) = [2(1) - 0] - [2(0) - 0] = 2.$$
      The streamlines of the fluid flow is defined to be the level curves of
      $\psi$; that is, $y = \frac{1}{2}c + \frac{1}{2}x$, for constants $c$.
      These are straight lines with slope $1/2$.
      \begin{center}
      \includegraphics[width=\textwidth]{streamline}
      \end{center}
\end{enumerate}
\end{document} 
