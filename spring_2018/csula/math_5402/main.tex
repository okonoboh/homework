\documentclass[9pt]{article}

\usepackage{amssymb}
\usepackage{amsmath}
\usepackage{amsfonts}
\usepackage{comment}
\usepackage{fancyhdr}
\usepackage{mathrsfs}
\usepackage{enumitem}
%\usepackage[retainorgcmds]{IEEEtrantools}

\everymath{\displaystyle}

\usepackage{tikz}

\voffset = -50pt
%\textheight = 700pt
\addtolength{\textwidth}{60pt}
\addtolength{\evensidemargin}{-30pt}
\addtolength{\oddsidemargin}{-30pt}
%\setlength{\headheight}{44pt}

\pagestyle{fancy}
\fancyhf{} % clear all fields
\fancyhead[R]{%
  \scshape
  \begin{tabular}[t]{@{}r@{}}
  MATH 5402, Spring 2018\\Section 1 (9932)\\
  HW, DUE: N/A
  \end{tabular}}
\fancyhead[L]{%
  \scshape
  \begin{tabular}[t]{@{}r@{}}
  JOSEPH OKONOBOH\\Mathematics\\Cal State Los Angeles
  \end{tabular}}
\fancyfoot[C]{\thepage}

\newcommand{\qed}{\hfill \ensuremath{\Box}}


\newcommand*\circled[1]{\tikz[baseline=(char.base)]{
            \node[shape=circle,draw,inner sep=2pt] (char) {#1};}}


\newcommand{\cyc}[1]{\langle #1 \rangle}
\newcommand{\Z}{\mathbb{Z}}
\newcommand{\I}{\mathbb{I}}
\newcommand{\F}{\mathbb{F}}
\newcommand{\M}{\mathbb{M}}
\newcommand{\R}{\mathbb{R}}
\newcommand{\Q}{\mathbb{Q}}
\newcommand{\D}{\displaystyle}
%\setcounter{section}{-1}

\begin{document}
\begin{enumerate}
%%%%%%%%%%%%%%%%%%%%%%%%%%%%%%%%%%%%%%001%%%%%%%%%%%%%%%%%%%%%%%%%%%%%%%%%%%%%%%
   \item 
%%%%%%%%%%%%%%%%%%%%%%%%%%%%%%%%%%%%%%002%%%%%%%%%%%%%%%%%%%%%%%%%%%%%%%%%%%%%%%
   \item 
%%%%%%%%%%%%%%%%%%%%%%%%%%%%%%%%%%%%%%003%%%%%%%%%%%%%%%%%%%%%%%%%%%%%%%%%%%%%%%
   \item 
%%%%%%%%%%%%%%%%%%%%%%%%%%%%%%%%%%%%%%004%%%%%%%%%%%%%%%%%%%%%%%%%%%%%%%%%%%%%%%
   \item 
%%%%%%%%%%%%%%%%%%%%%%%%%%%%%%%%%%%%%%005%%%%%%%%%%%%%%%%%%%%%%%%%%%%%%%%%%%%%%%
   \item 
%%%%%%%%%%%%%%%%%%%%%%%%%%%%%%%%%%%%%%006%%%%%%%%%%%%%%%%%%%%%%%%%%%%%%%%%%%%%%%
   \item 
%%%%%%%%%%%%%%%%%%%%%%%%%%%%%%%%%%%%%%007%%%%%%%%%%%%%%%%%%%%%%%%%%%%%%%%%%%%%%%
   \item 
%%%%%%%%%%%%%%%%%%%%%%%%%%%%%%%%%%%%%%008%%%%%%%%%%%%%%%%%%%%%%%%%%%%%%%%%%%%%%%
   \item 
%%%%%%%%%%%%%%%%%%%%%%%%%%%%%%%%%%%%%%009%%%%%%%%%%%%%%%%%%%%%%%%%%%%%%%%%%%%%%%
   \item 
%%%%%%%%%%%%%%%%%%%%%%%%%%%%%%%%%%%%%%010%%%%%%%%%%%%%%%%%%%%%%%%%%%%%%%%%%%%%%%
   \item 
%%%%%%%%%%%%%%%%%%%%%%%%%%%%%%%%%%%%%%011%%%%%%%%%%%%%%%%%%%%%%%%%%%%%%%%%%%%%%%
   \item 
%%%%%%%%%%%%%%%%%%%%%%%%%%%%%%%%%%%%%%012%%%%%%%%%%%%%%%%%%%%%%%%%%%%%%%%%%%%%%%
   \item 
%%%%%%%%%%%%%%%%%%%%%%%%%%%%%%%%%%%%%%013%%%%%%%%%%%%%%%%%%%%%%%%%%%%%%%%%%%%%%%
   \item 
%%%%%%%%%%%%%%%%%%%%%%%%%%%%%%%%%%%%%%014%%%%%%%%%%%%%%%%%%%%%%%%%%%%%%%%%%%%%%%
   \item Let $R$ be a commutative ring with $1 \neq 0$. Prove $R$ is an integral
         domain, if and only if, whenever $ab = 0$, then either $a = 0$ or
         $b = 0$.
         
      \textbf{Proof.} Suppose $R$ is a commutative ring with $1 \neq 0$.
      
      $(\Leftarrow)$ First assume that $R$ is an integral domain and suppose that
      $ab = 0$ for some $a, b \in R$. If $a = 0$, we are done, so suppose that
      $a \neq 0$. If $b \neq 0$, then $a$ is a zero divisor, contradicting our
      assumption that $R$ is an integral domain; thus $b = 0$.
      
      $(\Rightarrow)$ Now assume that if $ab = 0$, for some $a, b \in R$, then
      $a = 0$ or $b = 0$. That is, $R$ contains no zero divisors. Conclude that
      $R$ is an integral domain. \qed
%%%%%%%%%%%%%%%%%%%%%%%%%%%%%%%%%%%%%%015%%%%%%%%%%%%%%%%%%%%%%%%%%%%%%%%%%%%%%%
   \item Show that $n\Z = \{nz : z \in \Z\}$ is an ideal of $\Z$, for all
         integers $n$.
         
      \textbf{Proof.} Fix $n \in \Z$. Let $x, y \in n\Z$. So there exist
      integers $a$ and $b$ such that $x = na$ and $y = nb$. Since
      $a - b$ and $nab$ are integers, conclude that $x - y = n(a - b)$, 
      $xy = n(nab) \in n\Z$. It follows that $n\Z$ is an ideal of $\Z$. \qed
%%%%%%%%%%%%%%%%%%%%%%%%%%%%%%%%%%%%%%016%%%%%%%%%%%%%%%%%%%%%%%%%%%%%%%%%%%%%%%
   \item Suppose $\varphi : R \rightarrow S$ is a ring homomorphism. Show that
         the $\textit{kernel}$ of the homomorphism
         $ker(\varphi) = \{a \in R : \varphi(a) = 0\}$ is an ideal of $R$.
         
      \textbf{Proof.} First, we want to show that $\varphi(0) = 0$, so that
      $ker(\varphi)$ is nonempty. Now
      $$\varphi(0) = \varphi(0 + 0) = \varphi(0) + \varphi(0).$$
      Add $-\varphi(0)$ to both sides of the preceding equality to get
      $\varphi(0) = 0$. Note that if $x \in R$, then
      $$0 = \varphi(0) = \varphi(x + (-x)) = \varphi(x) + \varphi(-x),$$
      so that $\varphi(-x) = -\varphi$ for each $x \in R$.
      Let $a, b \in ker(\varphi)$. So
      $$\varphi(a - b) = \varphi(a + (-b)) = \varphi(a) + \varphi(-b) =
      \varphi(a) + (-\varphi(b)) = 0 + (-0) = 0  + 0 = 0.$$
      That is, $a - b \in ker(\varphi)$. Let $r \in R$. Since
      $\varphi(ra) = \varphi(r)\varphi(a) = \varphi(r)\cdot 0 = 0$ and
      $\varphi(ar) = \varphi(a)\varphi(r) = 0 \cdot\varphi(a) = 0$, it follows
      that $ra, ar \in ker(\varphi)$ for each $r \in R$. Conclude that
      $ker(\varphi)$ is an ideal of $R$. \qed
%%%%%%%%%%%%%%%%%%%%%%%%%%%%%%%%%%%%%%017%%%%%%%%%%%%%%%%%%%%%%%%%%%%%%%%%%%%%%%
   \item 
%%%%%%%%%%%%%%%%%%%%%%%%%%%%%%%%%%%%%%018%%%%%%%%%%%%%%%%%%%%%%%%%%%%%%%%%%%%%%%
   \item The intersection of any collection of ideals in $R$, is an ideal of
         $R$.
         
      \textbf{Proof.} Let $\{I_j : j \in J\}$ be a collection of ideals in $R$,
      where $J$ is some nonempty indexing set. We want to show that
      $S = \bigcap_{j \in J}I_j$ is an ideal of $R$. To that end, we let $x, y
      \in S$ and $r \in R$ ($S$ is nonempty because $0 \in I_j$ for each
      $j \in J$, so that $0 \in S$). Let $k \in J$. Then $x, y \in I_k$. Since
      $I_k$ is an ideal of $R$, it follows that $x - y, rx, xr \in I_k$. Since
      $I_k$ was arbitrarily chosen, it follows that $x - y, rx, xr \in I_j$ for
      each $j \in J$. Hence, $x - y, rx, xr \in S$. Conclude that $S$ is an
      ideal of $R$. \qed
%%%%%%%%%%%%%%%%%%%%%%%%%%%%%%%%%%%%%%019%%%%%%%%%%%%%%%%%%%%%%%%%%%%%%%%%%%%%%%
   \item Let $S$ be a subset of a ring $R$. Show that $\cyc{S}$ is the smallest
         ideal containing $S$ (in the sense that $\cyc{S}$ is contained in
         every ideal containing $S$.)
         
      \textbf{Proof.} Let $I_S$ be the set of ideals containing $S$.
      Let $H \in I_S$. By definition
      $$\cyc{S} = \bigcap_{I \in I_S}I \subseteq H,$$
      so that $\cyc{S}$ is the smallest ideal containing $S$. \qed
%%%%%%%%%%%%%%%%%%%%%%%%%%%%%%%%%%%%%%020%%%%%%%%%%%%%%%%%%%%%%%%%%%%%%%%%%%%%%%
   \item Suppose $R$ is a ring with identity 1. Show the only ideal containing
         1 is $R$, therefore $\cyc{1} = R$.
         
      \textbf{Proof.}  Let $I$ be an ideal of $R$. Suppose $1 \in I$ and let
      $r \in R$. Since $I$ is an ideal, we have that $1 \cdot r = r \in I$. That
      is, $R \subseteq I$. By definition $I \subseteq R$, so conclude that
      $I = R$. Particularly, since $1 \in \cyc{1}$, it follows that
      $\cyc{1} = R$. \qed
%%%%%%%%%%%%%%%%%%%%%%%%%%%%%%%%%%%%%%021%%%%%%%%%%%%%%%%%%%%%%%%%%%%%%%%%%%%%%%
   \item In the rings of integers $\Z$ show:
         \begin{enumerate}
            \item $n\Z = \cyc{n}$,
            \item $\cyc{6, 10} = \cyc{2}$,
            \item $\cyc{60} \subsetneq \cyc{2}$,
            \item $\cyc{2} \cup \cyc{3}$ is not an ideal.
         \end{enumerate}
         
      \textbf{Proof.}
      
      \begin{enumerate}
         \item Fix $n \in \Z$. By Exercise 15, $n\Z$ is an ideal of $\Z$. Since
               $n \in n\Z$, it follows by Exercise 19 that
               $\cyc{n} \subseteq n\Z$. Now let $x \in n\Z$, so that $x = nm$
               for some integer $m$. Since $\cyc{n}$ is an ideal containing $n$,
               it follows that $x = n \cdot m \in \cyc{n}$, so that
               $n\Z \subseteq \cyc{n}$. Conclude that $n\Z = \cyc{n}$.
         \item Since $6, 10 \in \cyc{6, 10}$ and $2, 3 \in \Z$, we have that
               $18 = 3 \cdot 6 \in \cyc{6, 10}$ and
               $20 = 2 \cdot 10 \in \cyc{6, 10}$. By definition, we have that
               $2 = 20 - 18 \in \cyc{6, 10}$. Thus, $\cyc{6, 10}$ is an ideal
               containing 2, so that $\cyc{2} \subseteq \cyc{6, 10}$ by Exercise
               15. Similarly, $\cyc{2}$ is an ideal containing 6 and 10 (because
               $6 = 2 \cdot 3$ and $10 = 2 \cdot 5$), so that
               $\cyc{6, 10} \subseteq \cyc{2}$. Conclude that
               $\cyc{6, 10} = \cyc{2}$.
         \item By part(a), we have $\cyc{60} = 60\Z$ and $\cyc{2} = 2\Z$. Now
               $x \in 60\Z$, so that $x = 60y$ for some integer $y$. Thus
               $x = 60y = 2(30y) \in 2\Z$, so that $60\Z \subseteq 2\Z$. Now
               $2 \in 2\Z$ but $2 \notin 60\Z$ because the smallest positive
               integer in $60\Z$ is 60. Thus $\cyc{60} \subsetneq 2\Z$.
         \item By part (a), $I = \cyc{2} \cup \cyc{3}$ is the set of all integers
               that is a multiple of 2 or 3. If $I$ were an ideal, then it would
               contain $1 = -1 \cdot 2 + 3$, a contradiction because $1$ is
               neither a multiple of 2 nor 3.
      \end{enumerate} \qed
%%%%%%%%%%%%%%%%%%%%%%%%%%%%%%%%%%%%%%022%%%%%%%%%%%%%%%%%%%%%%%%%%%%%%%%%%%%%%%
   \item Suppose $R$ is a commutative ring with identity 1 and $a \in R$; show
         that $\cyc{a} = aR = \{ar : r \in R\}$.
         
      \textbf{Proof.} Assume that $R$ is a commutative ring with identity 1 and
      consider $a \in R$. First we want to show that $aR$ is an ideal of $R$.
      Let $x, y \in aR$ ($aR$ is nonempty because $a = a \cdot 1 \in aR$, so
      that $x = ar$ and $y = as$ for some $r, s \in R$. By closure of $R$, we
      have that $r - s \in R$, so that $x - y = ar - as = a(r - s) \in aR$. For
      $t \in R$, we have that
      \begin{align*}
         tx &= xt &[R \text{ is commutative}] \\
            &= (ar)t = a(rt) \in aR, &[\text{Associativity \& Closure}]
      \end{align*}
      so that $aR$ is an ideal of $R$. Let $I$ be an ideal containing $a$. It
      follows that $ar \in I$, for each $r \in R$. That is, $aR \subseteq I$.
      Particularly, we have that $aR \subseteq \cyc{a}$. Since $aR$ is an ideal
      containing $a$, it follows by Exercise 15 that $\cyc{a} \subseteq aR$, so
      that $\cyc{a} =aR$. \qed
%%%%%%%%%%%%%%%%%%%%%%%%%%%%%%%%%%%%%%023%%%%%%%%%%%%%%%%%%%%%%%%%%%%%%%%%%%%%%%
   \item Prove that a division ring has no nonzero proper ideals.
         
      \textbf{Proof.} Let $I \neq 0$ be an ideal of a division ring $R$. Since
      $I$ is nonzero, it contains a nonzero element, say $a$. Since $R$ is a
      division ring, $a^{-1}$ exists, so that $a(a^{-1}) = 1 \in I$. Conclude by
      Exercise 20 that $I = R$. Thus a division ring has no nonzero proper
      ideals. \qed
%%%%%%%%%%%%%%%%%%%%%%%%%%%%%%%%%%%%%%024%%%%%%%%%%%%%%%%%%%%%%%%%%%%%%%%%%%%%%%
   \item Prove if $R$ is a ring with more than one element, such that $aR = R$
         for every nonzero element $a$ of $R$, then $R$ is a division ring.
         
      \textbf{Proof.} Assume that $R$ is a ring with more than one element, such
      that $aR = R$ for every nonzero element $a$ of $R$. Let $a$ be a nonzero
      element of $R$. First, we want to show that $a$ is not a zero divisor. So
      suppose to the contrary that there exists a nonzero element $b \in R$, such
      that $ab = 0$. Since $a \neq 0$, it follows by assumption that $aR = R$,
      so that $ae = b$, for some $e \in R$. Observe that $e \neq 0$, for
      otherwise, $ae$ would be 0. Similarly, $bR = R$, so there exists $c \in R$
      such that $bc = e$. Thus
      \begin{align*}
         a &= ae \\
           &= a(bc) \\
           &= (ab)c &[\text{Associativity of } \cdot] \\
           &= 0 \cdot c = 0,
      \end{align*}
      a contradiction since $a \neq 0$. Thus $a$ is not a zero divisor; since
      $a$ was an arbitrarily chosen nonzero element of $R$, conclude that $R$
      has no zero divisors. Now we want to show that $e$ is the multiplicative
      identity of $R$. To that end, let $x$ be an arbitrary element of $R$.
      Thus
      \begin{align*}
         a(x - ex) &= ax - a(ex) &[\text{Distributivity}] \\
            &= ax - (ae)x &[\text{Associativity of }\cdot] \\
            &= ax - a x = 0,
      \end{align*}
      so that $x - ex = 0$ (because $a$ is not a zero divisor). Thus
      \begin{equation} \label{l1}
         ex = x \quad \text{for each } x \in R
      \end{equation}
      Now there exists $a' \in R$ so that $aa' = e$. If $a' = 0$, then
      $e = aa' = 0$, a contradiction because $e \neq 0$, so $a' \neq 0$. Hence
      \begin{align*}
         (ea - a)a' &= (ea)a' - aa' &[\text{Distributivity}] \\
            &= e(aa') - e &[\text{Associativity}] \\
            &= ee - e \\
            &= e - e &[\text{Let }x = e \text{ in }\eqref{l1}] \\
            &= 0,
      \end{align*}
      so that $ea - a = 0$ ($a'$ is not a zero divisor), or equivalently,
      $ea = a$. Since
      \begin{align*}
         (x - xe)a &= xa - (xe)a &[\text{Distributivity}] \\ 
            &= xa - x(ea) = xa - xa = 0,
      \end{align*}
      it follows that
      \begin{equation} \label{l2}
         xe = x \quad \text{for each } x \in R.
      \end{equation}
      Conclude from \eqref{l1} and \eqref{l2} that $e = 1$. It remains to show
      that every nonzero element of $R$ is a unit. To that end, we have
      \begin{align*}
         a(a'a - e) &= a(a'a) - ae &[\text{Distributivity}] \\ 
            &= (aa')a - a &[\text{Associativity}] \\
            &= ea - a = a - a = 0,
      \end{align*}
      so that $a'a = e = aa'$. That is, $a$ is a unit. So every nonzero element
      of $R$ is a unit. Conclude that $R$ is a division ring. \qed
%%%%%%%%%%%%%%%%%%%%%%%%%%%%%%%%%%%%%%025%%%%%%%%%%%%%%%%%%%%%%%%%%%%%%%%%%%%%%%
   \item If $I$ is an ideal of a ring $R$, then $I[x]$ is an ideal of $R[x]$.
         
      \textbf{Proof.} Suppose that $I$ is an ideal of a ring $R$. Since $I$ is
      an ideal, it is nonempty, so let $c \in I$. It follows that $I[x]$ is also
      nonempty because $c + cx \in I[x]$. Now let $f, g \in I[x]$. Then
      $f = a_0 + \cdots + a_kx^k$ and $g = b_0 + \cdots + b_mx^m$, where the
      coefficients are elements of $I$ and $k$ and $m$ are nonnegative integers.
      Assume without loss that $k \le m$. Since $I$ is an ideal, we have that
      $a_i - b_i \in I$, where $1 \le i \le k$. Thus
      $$f - g = (a_0 - b_0) + \cdots + (a_k - b_k)x^k + b_{k + 1}x^{k + 1} +
        \cdots + b_mx^m \in I[x].$$
      For each $n \ge 0$, let $P(n)$ be the statement that if $h \in R[x]$ is 
      a polynomial of degree $n$, then $h \cdot f, f \cdot h \in I[x]$. Proceed
      by induction on $n$. If $h$ is a polynomial in $R[x]$ of degree 0, then
      $h = r_0 \in R$, $r_0 \neq 0$. We have
      that $r_0a_i$, $a_ir_0 \in I$, where $0 \le i \le k$ because $I$ is an
      ideal. So
      $$hf = r_0(a_0 + \cdots + a_kx^k) =  r_0a_0 + \cdots + r_0a_kx^k\in I[x]$$
      and
      $$fh = (a_0 + \cdots + a_kx^k)r_0 =  a_0r_0 + \cdots +a_kr_0x^k\in I[x].$$
      That is, $P(0)$ holds. Now suppose that for some nonnegative integer $k$
      that $P(0)$, $P(1)$, $\ldots$, $P(k)$ hold and let $h = c_0 + \cdots + 
      c_kx^k + c_{k+1}x^{k+1} \in R[x]$ be a polynomial of degree $k + 1$. Write
      $h = q + r$, where $q = c_0 + \cdots + c_kx^k$ and $r = c_{k+1}x^{k+1}$.

      \textbf{Case 1.} $c_0 = \ldots = c_k = 0$. Thus $h = r$ and it follows
      that
      $$hf = rf = c_{k+1}x^k(a_0 + \cdots + a_kx^k) = (c_{k+1}a_0)x^k + \cdots +
         (c_{k+1}a_k)x^{2k} \in I[x]$$
      and
      $$fh = fr = (a_0 + \cdots + a_kx^k)c_{k+1}x^k = (a_0c_{k+1})x^k + \cdots +
         (a_kc_{k+1})x^{2k} \in I[x]$$
      because, as $I$ is an ideal, we have
      $c_{k+1}a_i$, $a_ic_{k+1} \in I$, $0 \le i \le k$.

      \textbf{Case 2.} There exists $0 \le i \le k$ such that $c_i \neq 0$. That 
      is, $0 \le \text{degree}(q) \le k$. By our Inductive Hypothesis, we have
      that $fq, qf \in I[x]$. Conclude from Case 1 and closure of $I[x]$ under
      addition that $hf = (q + r)f = qf +  rf \in I[x]$ and
      $fh = f(q + r) = fq + fr \in I[x]$, so that $P(k + 1)$ holds. Conclude
      that $P(n)$ holds for each $n \ge 0$. Clearly
      $0 \cdot f = f \cdot 0 = 0 \in I[x]$. Thus $I[x]$ is an ideal of $R[x]$.
      \qed
%%%%%%%%%%%%%%%%%%%%%%%%%%%%%%%%%%%%%%026%%%%%%%%%%%%%%%%%%%%%%%%%%%%%%%%%%%%%%%
   \item Suppose $I$ and $J$ are ideals of the ring $R$. Show:
         \begin{enumerate}
            \item If $R$ is commutative, then $R/I$ is also commutative.
            \item If $R$ has an identity 1, then $R/I$ has an identity $1 + I$.
            \item The set $I + J = \{a + b: a \in I, b \in J\}$ is an ideal of
                  $R$.
         \end{enumerate}
         
      \textbf{Proof.}
      
      \begin{enumerate}
         \item Suppose $R$ is commutative. Let $x, y \in R/I$. Then
               $x = a + I$ and $y = b + I$ for some $a, b \in R$. Since $R$ is
               commutative, we have that $ab = ba$, so that
               $$xy = (a + I)(b + I) = (ab) + I = (ba) + I =
                 (b + I)(a + I) = yx;$$
               that is, $R/I$ is commutative.
         \item Suppose $R$ has an identity 1. Let $x \in R/I$, so that
               $x = a + I$ for some $a \in R$. Since
               $$x \cdot (1 + I) = (a + I)(1 + I) = (a\cdot 1) + I = a + I =x$$
               and
               $$(1 + I) \cdot x = (1 + I)(a + I) = (1\cdot a) + I = a + I =x,$$
               it follows that $1 + I$ is an identity of $R/I$.
         \item Since $I$ and $J$ are ideals, they are both nonempty, so let
               $i \in I$ and $j \in J$, so that $i + j \in I + J$; that is,
               $I + J$ is nonempty. Now let $x, y \in I + J$. Then there exist
               $i_1, i_2 \in I$ and $j_1, j_2 \in J$ such that $x = i_1 + j_1$
               and $y = i_2 + j_2$. By closure $i_1 - 1_2 \in I$ and
               $j_1 - j_2 \in J$, so it follows that
               $$x - y = (i_1 + j_1) - (i_2 + j_2) =
                 (i_1 - i_2) + (j_1 - j_2) \in I + J.$$
               For each $r \in R$, we have that $ri_1, i_1r \in I$ and
               $rj_1, j_1r \in J$ because $I$ and $J$ are ideals, so
               $$rx = r(i_1 + j_1) = ri_1 + rj_1 \in I + J$$
               and
               $$xr = (i_1 + j_1)r = i_1r + j_1r \in I + J.$$
               Conclude that $I + J$ is an ideal of $R$.
      \end{enumerate} \qed
\end{enumerate}
\end{document}
