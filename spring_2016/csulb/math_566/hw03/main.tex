\documentclass[9pt]{article}

\usepackage{amssymb}
\usepackage{amsmath}
\usepackage{amsfonts}
\usepackage{comment}
\usepackage{fancyhdr}
\usepackage{mathrsfs}
\usepackage{enumitem}
%\usepackage[retainorgcmds]{IEEEtrantools}


\usepackage{tikz}

\voffset = -50pt
%\textheight = 700pt
\addtolength{\textwidth}{60pt}
\addtolength{\evensidemargin}{-30pt}
\addtolength{\oddsidemargin}{-30pt}
%\setlength{\headheight}{44pt}

\pagestyle{fancy}
\fancyhf{} % clear all fields
\fancyhead[R]{%
  \scshape
  \begin{tabular}[t]{@{}r@{}}
  MATH 566, Spring 2016\\Section 1 (9519)\\
  HW \#3, DUE: 2016, February 18
  \end{tabular}}
\fancyhead[L]{%
  \scshape
  \begin{tabular}[t]{@{}r@{}}
  JOSEPH OKONOBOH\\Mathematics\\Cal State Long Beach
  \end{tabular}}
\fancyfoot[C]{\thepage}

\newcommand{\qed}{\hfill \ensuremath{\Box}}


\newcommand*\circled[1]{\tikz[baseline=(char.base)]{
            \node[shape=circle,draw,inner sep=2pt] (char) {#1};}}

\newcommand{\Z}{\mathbb{Z}}
\newcommand{\I}{\mathbb{I}}
\newcommand{\M}{\mathbb{M}}
\newcommand{\R}{\mathbb{R}}
\newcommand{\C}{\mathbb{C}}
\everymath{\displaystyle}
%\setcounter{section}{-1}

\begin{document}
\begin{enumerate}
%%%%%%%%%%%%%%%%%%%%%%%%%%%%%%%%%%%%%8.5.1%%%%%%%%%%%%%%%%%%%%%%%%%%%%%%%%%%%%%%
   \item[8.5.1.]  Find the critical points and critical values of
                  $f(z) = z + 1/z$. Sketch the curves where $f(z)$ is real.
                  Sketch the regions where Im $f(z) > 0$ and where
                  Im $f(z) < 0$.
                  
      \textbf{Solution.} We have $f'(z) = 1 - 1/z^2$. Since $f'(z) = 0$ if and
      only if $z = \pm1$, it follows that the critical points of $f$ are
      $1$ and $-1$, with critical values $f(1) = 2$ and $f(-1) = -2$. Write
      $z = x + yi$, so that
      $$f(x + iy) = x + yi + \frac{1}{x+yi} = x + yi + \frac{x-yi}{x^2+y^2} =
        x + \frac{x}{x^2+y^2} + \left(y - \frac{y}{x^2+y^2}\right)i.$$
        
      \begin{itemize}
         \item \textbf{Curves where $f(z)$ is real.} So $f(x+yi)$ is real if and
               only if $y - \frac{y}{x^2+y^2} = 0$ if and only if
               $y\left(1 - \frac{1}{x^2+y^2}\right) = 0$ if and only if
               $y = 0 (x \neq 0)$ or $x^2 + y^2 = 1$. Thus $f(z)$ is
               real if and only if $z$ is real and nonzero or $|z| = 1$. \\ \\
               \\ \\ \\ \\ \\ \\ \\ \\ \\ \\ \\
         \item \textbf{Region where } Im $f(z) > 0$\textbf{.} Im $f(z) > 0$
               if and only if $y - \frac{y}{x^2 + y^2} > 0$. Solving this
               inequality will give us
               $$\{x + iy : y > 0 \text{ and }
               x^2 + y^2 > 1\} \cup \{x + iy : y < 0 \text{ and }
               x^2 + y^2 < 1\}$$ \\ \\
               \\ \\ \\ \\ \\ \\ \\ \\ \\ \\ \\
         \item \textbf{Region where } Im $f(z) < 0$\textbf{.} We simply take
               the complement of the region above to get \\ \\
               \\ \\ \\ \\ \\ \\ \\ \\ \\ \\ \\
      \end{itemize}
%%%%%%%%%%%%%%%%%%%%%%%%%%%%%%%%%%%%%8.5.10%%%%%%%%%%%%%%%%%%%%%%%%%%%%%%%%%%%%%
   \item[8.5.10.] Locate the critical points and critical values in the
                  extended complex plane of the polynomial $f(z) = z^4 - 2z^2$.
                  Determine the order of each critical point. Sketch the set of
                  points $z$ such that Im $f(z) \ge 0$.
                  
      \textbf{Solution.}  We have $f'(z) = 4z^3 - 4z = 4z(z^2 - 1) =
      4z(z-1)(z+1)$, so that 0, 1, and $-1$ are critical points of $f(z)$ with
      critical values $f(0) = 0$, $f(1) = f(-1) = -1$. If we let
      $g(z) = 4(z^2-1)$, $h(z) = 4z(z+1)$, and $r(z) = 4z(z-1)$, then since
      $$f'(z) = (z-0)g(z) = (z - 1)h(z) = (z + 1)r(z),$$
      where $g(z)$ is entire and $g(0) \neq 0$, $h(z)$ is entire and
      $h(1) \neq 0$, and $r(z)$ is entire and $r(-1) \neq 0$, it follows that
      the orders of 0, 1, and $-1$ are all 1. Now we want to find the set of
      points $z$ such that Im $(f(z) \ge 0$. Let $z = x + yi$, so that
      $$f(z) = z^2(z^2 - 2) = (x+yi)^2((x+yi)^2 - 2) = (x^2-y^2 + 2xyi)
       (x^2-y^2-2 + 2xyi).$$
      Thus Im $f(x+yi) = 2xy(x^2-y^2) + 2xy(x^2-y^2-2) = 4xy(x^2 - y^2 -1)$.
      That is, Im $f(z) \ge 0$ if and only if $z \in \{x + yi : y \ge 1/x
      \text{ or } x^2 - y^2 \ge 1\}$. We sketch the results below: \\ \\ \\ \\ \\ \\ \\ \\ \\ \\ \\ \\ \\
   
                  
%%%%%%%%%%%%%%%%%%%%%%%%%%%%%%%%%%%%%8.6.3%%%%%%%%%%%%%%%%%%%%%%%%%%%%%%%%%%%%%%
   \item[8.6.3.]  Let $f(z)$ be analytic on an open set containing a closed
                  path $\gamma$, and suppose $f(z) \neq 0$ on $\gamma$. Show
                  that the increase in arg $f(z)$ around $\gamma$ is
                  $2\pi W(f \circ \gamma, 0)$.
                  
      \textbf{Proof.} Since $\gamma$ is a closed path, it follows by the
      discussion in Section 1 that
      $$\frac{1}{2\pi i}\int_\gamma\frac{f'(z)}{f(z)}dz =
        \frac{1}{2\pi}\int_\gamma d\text{ arg}(f(z)),$$
      where $d\text{ arg}(f(z))$ is the increase in arg $f(z)$ around $\gamma$.
      If we make the substitution $w = f(z)$, so that $dw = f'(z)dz$, we get
      \begin{align*}
         \frac{1}{2\pi}\int_\gamma d\text{ arg}(f(z)) &=
            \frac{1}{2\pi i}\int_\gamma\frac{f'(z)}{f(z)}dz \\
            &= \frac{1}{2\pi i}\int_{f(\gamma)}\frac{dw}{w - 0} \\
            &= \frac{1}{2\pi}\int_{f(\gamma)}d \text{ arg}(w) \\
            &= W(f(\gamma), 0) = W(f \circ \gamma, 0),
      \end{align*}
      so that $\int_\gamma d\text{ arg}(f(z)) = 2\pi W(f \circ\gamma, 0)$. \qed
%%%%%%%%%%%%%%%%%%%%%%%%%%%%%%%%%%%%%8.6.5%%%%%%%%%%%%%%%%%%%%%%%%%%%%%%%%%%%%%%
   \item[8.6.5.]  Show that if $\gamma$ is a piecewise smooth closed curve in
                  the complex plane, with trace $\Gamma$, and if
                  $z_0 \notin \Gamma$, then
                  $$\int_\gamma\frac{1}{(z-z_0)^n}dz = 0,\qquad n \ge 2.$$
                  
      \textbf{Proof.} Let $f(z) = z - z_0$. We differentiate the formula on the Theorem on page 244
      to get
      $$\frac{1}{2\pi i }\int_\gamma\frac{f(z)}{(z-z_0)^{n+1}}dz = \frac{1}{n!}
      W(\gamma, z_0)f^{n}(z_0) = 0$$
      since $f^n = 0$, for $n \ge 2$. Thus
      $$\int_\gamma\frac{1}{(z-z_0)^n}dz = \frac{1}{2\pi i }\int_\gamma\frac{f(z)}{(z-z_0)^{n+1}} = 0,$$
      for $n \ge 2$.
%%%%%%%%%%%%%%%%%%%%%%%%%%%%%%%%%%%%%8.6.7%%%%%%%%%%%%%%%%%%%%%%%%%%%%%%%%%%%%%%
   \item[8.6.7.]  Evaluate
                  $$\frac{1}{2\pi i}\int_\gamma \frac{dz}{z(z^2-1)},$$
                  where $\gamma$ is the curve shown on page 246.
                 
      \textbf{Solution.} Splitting the integrand into partial fractions, we
      obtain
      \begin{align*}
         \frac{1}{2\pi i}\int_\gamma \frac{dz}{z(z^2-1)} &=
            \frac{1}{2\pi i}\int_\gamma\left(\frac{1}{z} + \frac{1}{2(z+1)} +
            \frac{1}{2(z-1)}\right)dz \\
            &= \frac{1}{2\pi i}\int_\gamma \frac{dz}{z-0} +
               \frac{1}{2}\left(\frac{1}{2\pi i}
               \int_\gamma \frac{dz}{z-(-1)}\right) +               
               \frac{1}{2}\left(\frac{1}{2\pi i}
               \int_\gamma \frac{dz}{z-1}\right) \\
            &= W(\gamma, 0) + \frac{1}{2}W(\gamma, -1)+\frac{1}{2}W(\gamma, 1)\\
            &= 0 + \frac{1}{2} \cdot 2 + \frac{1}{2} \cdot 1 \\
            &= \frac{3}{2}.
      \end{align*}
\end{enumerate}
\end{document} 
