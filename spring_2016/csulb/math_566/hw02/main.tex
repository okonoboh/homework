\documentclass[9pt]{article}

\usepackage{amssymb}
\usepackage{amsmath}
\usepackage{amsfonts}
\usepackage{comment}
\usepackage{fancyhdr}
\usepackage{mathrsfs}
\usepackage{enumitem}
%\usepackage[retainorgcmds]{IEEEtrantools}


\usepackage{tikz}

\voffset = -50pt
%\textheight = 700pt
\addtolength{\textwidth}{60pt}
\addtolength{\evensidemargin}{-30pt}
\addtolength{\oddsidemargin}{-30pt}
%\setlength{\headheight}{44pt}

\pagestyle{fancy}
\fancyhf{} % clear all fields
\fancyhead[R]{%
  \scshape
  \begin{tabular}[t]{@{}r@{}}
  MATH 566, Spring 2016\\Section 1 (9519)\\
  HW \#2, DUE: 2016, February 11
  \end{tabular}}
\fancyhead[L]{%
  \scshape
  \begin{tabular}[t]{@{}r@{}}
  JOSEPH OKONOBOH\\Mathematics\\Cal State Long Beach
  \end{tabular}}
\fancyfoot[C]{\thepage}

\newcommand{\qed}{\hfill \ensuremath{\Box}}


\newcommand*\circled[1]{\tikz[baseline=(char.base)]{
            \node[shape=circle,draw,inner sep=2pt] (char) {#1};}}

\newcommand{\Z}{\mathbb{Z}}
\newcommand{\I}{\mathbb{I}}
\newcommand{\M}{\mathbb{M}}
\newcommand{\R}{\mathbb{R}}
\newcommand{\C}{\mathbb{C}}
\everymath{\displaystyle}
%\setcounter{section}{-1}

\begin{document}
\begin{enumerate}
%%%%%%%%%%%%%%%%%%%%%%%%%%%%%%%%%%%%%8.3.1%%%%%%%%%%%%%%%%%%%%%%%%%%%%%%%%%%%%%%
   \item[8.3.1.]  Let $\{f_k(z)\}$ be a sequence of analytic functions on $D$
                  that converges normally to $f(z)$, and suppose that $f(z)$ has
                  a zero of order $N$ at $z_0 \in D$. Use Rouch\'{e}'s theorem
                  to show that there exists $\rho > 0$ such that for $k$ large,
                  $f_k(z)$ has exactly $N$ zeros counting multiplicity on the
                  disk $\{z \in \C : |z - z_0| < \rho\}$.

      \textbf{Proof.} Since the sequence of functions converges normally to
      $f(z)$, it follows that this sequence converges uniformly to $f(z)$ on
      some compact set $E \subseteq D$ (so that $f(z)$ is analytic on $E$)
      containing $z_0$. Recall that the zeros of an analytic function are
      isolated. So there exists $\rho > 0$ such that $f(z) \neq 0$ on
      $\{z \in \C : 0 < |z - z_0| \le \rho\} \subseteq E$. We can use the
      Minimum Modulus Principle to find a $\delta > 0$ such that
      $|f(z)| \ge \delta > 0$ on the circle $\{z \in \C : |z - z_0| = \rho\}$.
      Since $\{f_k(z)\}$ converges  uniformly to $f(z)$ on $E$, it follows by
      definition that there exists a positive integer $k'$ such that
      $$|f_k(z) - f(z)| < \delta \text{ for all } k \ge k' \text{ and }
         z \in E.$$
      Particularly, we have
      $$|f_k(z) - f(z)| < \delta \le |f(z)|\text{ for all } k \ge k'\text{ and }
         |z - z_0| = \rho.$$
      It follows by Rouch\'{e}'s theorem that, for $k \ge k'$,
      $[f_k(z) - f(z)] + f(z) = f_k(z)$  must have the same number of zeros
      (counting multiplicity) on the disk $\{z \in \C : |z - z_0| < \rho\}$ as
      $f(z)$. Thus $f_k(z)$ has $N$ zeros on the disk
      $\{z \in \C : |z - z_0| < \rho\}$, for all $k \ge k'$. \qed
%%%%%%%%%%%%%%%%%%%%%%%%%%%%%%%%%%%%%8.4.3%%%%%%%%%%%%%%%%%%%%%%%%%%%%%%%%%%%%%%
   \item[8.4.3.]  Let $\{f_k(z)\}$ be a sequence of analytic functions on a
                  domain $D$ that converges normally to $f(z)$. Suppose that
                  $f_k(z)$ attains each value $w$ at most $m$ times (counting
                  multiplicity) in $D$. Show that either $f(z)$ is constant, or
                  $f(z)$ attains each value $w$ at most $m$ times in $D$.

      \textbf{Proof.} Let $E \subseteq D$ be a compact set. Suppose that $f(z)$
      is not constant. We want to first show that $f(z)$ attains each value $w$
      at most $m$ times in $E$. So let $w \in D$. For each positive integer $k$,
      let
      $$a_k = \frac{1}{2\pi i}\int_{\delta E}\frac{f_k'(z)}{f_k(z) - w} dz.$$
      Observe that $a_k$ is the number of zeros of $f_k - w$ in $E$; thus, we
      have $a_k \le m$ for all $k \ge 1$. Now let
      $$a = \frac{1}{2\pi i}\int_{\delta E}\frac{f(z)}{f(z) - w} dz,$$
      so that $a$ is the number of zeros of $f(z) - w$ in $E$. Since
      $\{f_k(z)\}$ converges normally to $f(z)$ on $D$, it follows that
      $\{f_k(z)\}$ converges uniformally to $f(z)$ on $E$. Thus
      $\lim_{k\rightarrow\infty}a_k = a$. But each $a_k \le m$, so that
      $a \le m$. That is, $f(z)$ attains $w$ at most $m$ times in $E$. Now
      suppose to the contrary that $f(z)$ attains $w$ at least $m + 1$ times in
      $D$. Then there exists compact set $E' \subseteq D$(by extending the
      boundary of $E$) containing all the $m+1$ zeros of $f(z) - w$. But this is
      a contradiction since we just showed that every compact set in $D$ must
      contain at most $m$ zeros of $f(z) - w$. Thus $f(z)$ attains $w$ at most
      $m$ times in $D$. Since $w$ was arbitrarily chosen, the result follows.
      \qed
\end{enumerate}
\end{document} 
