\documentclass[9pt]{article}

\usepackage{amssymb}
\usepackage{amsmath}
\usepackage{amsfonts}
\usepackage{comment}
\usepackage{fancyhdr}
\usepackage{mathrsfs}
\usepackage{enumitem}
%\usepackage[retainorgcmds]{IEEEtrantools}


\usepackage{tikz}

\voffset = -50pt
%\textheight = 700pt
\addtolength{\textwidth}{60pt}
\addtolength{\evensidemargin}{-30pt}
\addtolength{\oddsidemargin}{-30pt}
%\setlength{\headheight}{44pt}

\pagestyle{fancy}
\fancyhf{} % clear all fields
\fancyhead[R]{%
  \scshape
  \begin{tabular}[t]{@{}r@{}}
  MATH 566, Spring 2016\\Section 1 (9519)\\
  HW \#1, DUE: 2016, February 02
  \end{tabular}}
\fancyhead[L]{%
  \scshape
  \begin{tabular}[t]{@{}r@{}}
  JOSEPH OKONOBOH\\Mathematics\\Cal State Long Beach
  \end{tabular}}
\fancyfoot[C]{\thepage}

\newcommand{\qed}{\hfill \ensuremath{\Box}}


\newcommand*\circled[1]{\tikz[baseline=(char.base)]{
            \node[shape=circle,draw,inner sep=2pt] (char) {#1};}}

\newcommand{\Z}{\mathbb{Z}}
\newcommand{\I}{\mathbb{I}}
\newcommand{\M}{\mathbb{M}}
\newcommand{\R}{\mathbb{R}}
\newcommand{\C}{\mathbb{C}}
\everymath{\displaystyle}
%\setcounter{section}{-1}

\begin{document}
\begin{enumerate}
%%%%%%%%%%%%%%%%%%%%%%%%%%%%%%%%%%%%%8.1.3%%%%%%%%%%%%%%%%%%%%%%%%%%%%%%%%%%%%%%
   \item[8.1.3.]  Find the number of zeros of the polynomial
                  $p(z) = z^6 + 4z^4 + z^3 + 2z^2 + z + 5$ in the first quadrant
                  $\{\text{Re }z > 0, \text{Im }z > 0\}$.
%%%%%%%%%%%%%%%%%%%%%%%%%%%%%%%%%%%%%8.1.7%%%%%%%%%%%%%%%%%%%%%%%%%%%%%%%%%%%%%%
   \item[8.1.7.]  For a fixed complex number $\lambda$, show that if $m$ and $n$
                  are large integers, then the equation $e^z = z + \lambda$ has
                  exactly $m + n$ solutions in the horizontal strip
                  $\{-2\pi m < \text{Im }z < 2\pi n\}$.
%%%%%%%%%%%%%%%%%%%%%%%%%%%%%%%%%%%%%8.1.8%%%%%%%%%%%%%%%%%%%%%%%%%%%%%%%%%%%%%%
   \item[8.1.8.]  Show that if $\text{Re }\lambda > 1$, then the equation
                  $e^z = z + \lambda$ has exactly one solution in the left
                  half-plane.
%%%%%%%%%%%%%%%%%%%%%%%%%%%%%%%%%%%%%8.2.2%%%%%%%%%%%%%%%%%%%%%%%%%%%%%%%%%%%%%%
   \item[8.2.2.]  How many roots does $z^9 + z^5 - 8z^3 + 2z + 1$ have between
                  the circles $\{|z| = 1\}$ and $\{|z| = 2\}$?

      \textbf{Solution.} Let $p(z) = z^9 + z^5 - 8z^3 + 2z + 1$. We want to find
      the number of zeros of $p(z)$ in the annulus $1 < |z| < 2$. Let
      $$D_1 := \{z \in \C : |z| < 1\} \text{ and }
        D_2 := \{z \in \C : |z| < 2\}.$$
      Now write $p(z) = f_1(z) + h_1(z)$, where $f_1(z) = -8z^3$ and
      $h_1(z) = z^9 + z^5 + 2z + 1$. For $z \in \delta D_1$, we have
      \begin{align*}
         |h_1(z)| &= |z^9 + z^5 + 2z + 1| \\
            &\le |z^9| + |z^5| + |2z| + |1| &[\text{Triangle Inequality}] \\
            &= |z|^9 + |z|^5 + 2|z| + 1 \\
            &= 1^9 + 1^5 + 2 \cdot 1 + 1 \\
            &= 5 < 8 = 8 \cdot 1^3 = 8|z|^3 = |-8z^3| = |f_1(z)|.      
      \end{align*}
      That is,
      $$|h_1(z)| < |f_1(z)| \qquad \text{ for all } z \in \delta D_1.$$
      And since $f_1(z)$ has three zeroes in $D_1$, it follows by Rouch\'{e}'s
      Theorem that $p(z) = f_1(z) + h_1(z)$ has three zeroes in $D_1$. Next, we
      write $p(z) = f_2(z) + h_2(z)$, where $f_2(z) = z^9$ and
      $h_2(z) = z^5 - 8z^3 + 2z + 1$. For $z \in \delta D_2$, we have
      \begin{align*}
         |h_2(z)| &= |z^5 - 8z^3 + 2z + 1| \\
            &\le |z^5| + |-8z^3| + |2z| + |1| &[\text{Triangle Inequality}] \\
            &= |z|^5 + 8|z|^3 + 2|z| + 1 \\
            &= 2^5 + 8(2^3) + 2 \cdot 2 + 1 \\
            &= 101 < 512 \cdot 2^9 = |z|^9 = |f_2(z)|
      \end{align*}
      so that
      $$|h_2(z)| < |f_2(z)| \qquad \text{ for all } z \in \delta D_2,$$
      and since $f_2(z)$ has nine zeroes in $D_2$, it follows by Rouch\'{e}'s
      Theorem that $p(z) = f_2(z) + h_2(z)$ has nine zeroes in $D_2$. Conclude
      that $p(z)$ has $9 - 3 = 6$ zeros in the annulus $1 < |z| < 2$.
%%%%%%%%%%%%%%%%%%%%%%%%%%%%%%%%%%%%%8.2.2%%%%%%%%%%%%%%%%%%%%%%%%%%%%%%%%%%%%%%
   \item[8.2.6.]  Let $p(z) = z^6 + 9z^4 + z^3 + 2z + 4$ be the polynomial
                  treated in the example in this section.
                  \begin{enumerate}
                     \item Determine which quadrants contain the four zeros of
                           $p(z)$ that lie inside the unit circle.
                     \item Determine which quadrants contain the two zeros of
                           $p(z)$ that lie outside the unit circle.
                     \item Show that the two zeros of $p(z)$ that lie outside
                           the unit circle satisfy $\{|z \pm 3i| < 1/10\}$.
                  \end{enumerate}

      \textbf{Solution.}

      \begin{enumerate}
         \item Consider the domain
               $$D := \{z \in \C : |z| < 1, \text{Re }z > 0,
                 \text{Im }z > 0\}.$$
               We want to show that $p(z)$ has exactly 1 zero in $D$. Write
               $p(z) = f(z) + h(z)$, where $f(z) = 9z^4 + 4$ and
               $h(z) = z^6 + z^3 + 2z$, and consider the curves
               $B_1 := \{x \in \R : 0 \le x \le 1\}$,
               $B_2 := \{z \in \C : |z| = 1, \text{Re }z > 0,
                 \text{Im }z > 0\}, \text{ and }$
                $B_3 := \{yi \in \C : y \in \R, 0 < y \le 1\}$. That is,
               $\delta D = B_1 \cup B_2 \cup B_3$. First we observe that
               $f(z) \ge 4$ for all real $z$, so that $|f(z)| = f(z) \ge 4$ for
               all $z \in B_1$. Also $h(z) \le 4$ for all $z\in B_1$ because the
               maximum value of $z^6$, $z^3$, and $2z$ in $B_1$ are 1, 1, and 2 
               respectively. Note that $f(z) = 4$ if and only if $z = 0$ and
               since $h(0) = 0 < f(0)$, we conclude that
               $$h(z) = |h(z)| < |f(z)| = f(z)$$
               for all $z \in B_1$. Now, for $z \in B_2$ (so that $|z| = 1$), it 
               follows that
               \begin{align*}
                  |h(z)| &= |z^6 + z^3 + 2z| \\
                     &\le |z^6| + |z^3| + |2z| &[\text{Triangle Inequality}] \\
                     &= |z|^6 + |z|^3 + 2|z| \\
                     &= 1 + 1 + 2 = 4 < 9 = 9(1^4)= 9|z|^4 <
                     |9z^4 + 4| = |f(z)|.
               \end{align*}
               Hence $|f(z)|$ also dominates $|h(z)|$ in $B_2$. Finally let
               $z \in B_3$. Then $z = yi$ for some real number $y \in (0, 1]$.
               So $|f(z)| = |f(yi)| = |9(yi)^4 + 4| = |9y^4 + 4| \ge 4$. Since
               the maximum modulus of $z \in B_3$ is 1, it follows by the
               triangle inequality that $|h(z)| \le |z|^6 + |z|^3 + 2|z| \le 4$ 
               for all $z \in B_3$. That is, if $z \in B_3$, then
               $$|h(z)| \le 4 \le |f(z)|.$$
               But, as we previously argued, $f(z) = 4$ if and only if $z = 0$,
               and since $|h(0)| = 0 < 4 = |f(0)|$, it follows that
               $$|h(z)| \le 4 \le |f(z)|$$
               for all $z \in B_3$. We conclude that $|f(z)|$ dominates $|h(z)|$
               on $\delta D$. Observe that
               $$f(z) = 9z^4 + 4 = (3z^2)^2 - (2i)^2 = (3z^2 + 2i)(3z^2 - 2i),$$
               so that the roots of $f(z)$ are
               $$z = \sqrt{\frac{2}{3}}e^{-\pi/4} \text{ and }
                 z = \sqrt{\frac{2}{3}}e^{\pi/4}.$$
               The former root is not in $D$ while the latter is in $D$. Thus
               $f(z)$ has one root in $D$. It follows by Rouch\'{e}'s Theorem
               that $p(z) = f(z) + h(z)$ also has one root in $D$. Recall from
               the Example on Page 227 that $p(z)$ has two roots in the first
               quadrant. We have shown that exactly one of these two roots, say
               $z_1$, is in $D$; also, $\overline{z_1}$ must be in the fourth
               quadrant. Since $|z_1| < 1$, it follows that
               $|\overline{z_1}| < 1$, so that $\overline{z_1}$ is in the unit
               circle centered at the origin. Since $p(z)$ has four roots in
               this circle, the remaining two must necessarily be the roots of
               $p(z)$ (found in the Example on Page 227) in the second quadrant 
               and its conjugate in the third quadrant.
         \item From our discussion in (a), we conclude that the roots of $p(z)$
               that lie outside the unit circle must lie in the first and fourth
               conjugates, and these roots are conjugates.
      \end{enumerate}
\end{enumerate}
\end{document} 
