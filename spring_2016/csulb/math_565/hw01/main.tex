\documentclass[9pt]{article}

\usepackage{amssymb}
\usepackage{amsmath}
\usepackage{amsfonts}
\usepackage{comment}
\usepackage{fancyhdr}
\usepackage{mathrsfs}
\usepackage{enumitem}
%\usepackage[retainorgcmds]{IEEEtrantools}


\usepackage{tikz}

\voffset = -50pt
%\textheight = 700pt
\addtolength{\textwidth}{60pt}
\addtolength{\evensidemargin}{-30pt}
\addtolength{\oddsidemargin}{-30pt}
%\setlength{\headheight}{44pt}

\pagestyle{fancy}
\fancyhf{} % clear all fields
\fancyhead[R]{%
  \scshape
  \begin{tabular}[t]{@{}r@{}}
  MATH 566, Spring 2016\\Section 1 (9519)\\
  HW \#1, DUE: 2016, February 02
  \end{tabular}}
\fancyhead[L]{%
  \scshape
  \begin{tabular}[t]{@{}r@{}}
  JOSEPH OKONOBOH\\Mathematics\\Cal State Long Beach
  \end{tabular}}
\fancyfoot[C]{\thepage}

\newcommand{\qed}{\hfill \ensuremath{\Box}}


\newcommand*\circled[1]{\tikz[baseline=(char.base)]{
            \node[shape=circle,draw,inner sep=2pt] (char) {#1};}}

\newcommand{\Z}{\mathbb{Z}}
\newcommand{\I}{\mathbb{I}}
\newcommand{\M}{\mathbb{M}}
\newcommand{\R}{\mathbb{R}}
\newcommand{\C}{\mathbb{C}}
\everymath{\displaystyle}
%\setcounter{section}{-1}

\begin{document}
\begin{enumerate}
%%%%%%%%%%%%%%%%%%%%%%%%%%%%%%%%%%%%%8.2.2%%%%%%%%%%%%%%%%%%%%%%%%%%%%%%%%%%%%%%
   \item[8.2.2.]  How many roots does $z^9 + z^5 - 8z^3 + 2z + 1$ have between
                  the circles $\{|z| = 1\}$ and $\{|z| = 2\}$?

      \textbf{Solution.} Let $p(z) = z^9 + z^5 - 8z^3 + 2z + 1$. We want to find
      the number of zeros of $p(z)$ in the annulus $1 < |z| < 2$. Let
      $$D_1 := \{z \in \C : |z| < 1\} \text{ and }
        D_2 := \{z \in \C : |z| < 2\}.$$
      Now write $p(z) = f_1(z) + h_1(z)$, where $f_1(z) = -8z^3$ and
      $h_1(z) = z^9 + z^5 + 2z + 1$. For $z \in \delta D_1$, we have
      \begin{align*}
         |h_1(z)| &= |z^9 + z^5 + 2z + 1| \\
            &\le |z^9| + |z^5| + |2z| + |1| &[\text{Triangle Inequality}] \\
            &= |z|^9 + |z|^5 + 2|z| + 1 \\
            &= 1^9 + 1^5 + 2 \cdot 1 + 1 \\
            &= 5 < 8 = 8 \cdot 1^3 = 8|z|^3 = |-8z^3| = |f_1(z)|.      
      \end{align*}
      That is,
      $$|h_1(z)| < |f_1(z)| \qquad \text{ for all } z \in \delta D_1.$$
      And since $f_1(z)$ has three zeroes in $D_1$, it follows by Rouch\'{e}'s
      Theorem that $p(z) = f_1(z) + h_1(z)$ has three zeroes in $D_1$. Next, we
      write $p(z) = f_2(z) + h_2(z)$, where $f_2(z) = z^9$ and
      $h_2(z) = z^5 - 8z^3 + 2z + 1$. For $z \in \delta D_2$, we have
      \begin{align*}
         |h_2(z)| &= |z^5 - 8z^3 + 2z + 1| \\
            &\le |z^5| + |-8z^3| + |2z| + |1| &[\text{Triangle Inequality}] \\
            &= |z|^5 + 8|z|^3 + 2|z| + 1 \\
            &= 2^5 + 8(2^3) + 2 \cdot 2 + 1 \\
            &= 101 < 512 \cdot 2^9 = |z|^9 = |f_2(z)|
      \end{align*}
      so that
      $$|h_2(z)| < |f_2(z)| \qquad \text{ for all } z \in \delta D_2,$$
      and since $f_2(z)$ has nine zeroes in $D_2$, it follows by Rouch\'{e}'s
      Theorem that $p(z) = f_2(z) + h_2(z)$ has nine zeroes in $D_2$. Conclude
      that $p(z)$ has $9 - 3 = 6$ zeros in the annulus $1 < |z| < 2$.
%%%%%%%%%%%%%%%%%%%%%%%%%%%%%%%%%%%%%8.2.2%%%%%%%%%%%%%%%%%%%%%%%%%%%%%%%%%%%%%%
   \item[8.2.6.]  Let $p(z) = z^6 + 9z^4 + z^3 + 2z + 4$ be the polynomial
                  treated in the example in this section.
                  \begin{enumerate}
                     \item Determine which quadrants contain the four zeros of
                           $p(z)$ that lie inside the unit circle.
                     \item Determine which quadrants contain the two zeros of
                           $p(z)$ that lie outside the unit circle.
                     \item Show that the two zeros of $p(z)$ that lie outside
                           the unit circle satisfy $\{|z \pm 3i| < 1/10\}$.
                  \end{enumerate}
\end{enumerate}
\end{document} 
