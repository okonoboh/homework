\documentclass[9pt]{article}

\usepackage{amssymb}
\usepackage{amsmath}
\usepackage{amsfonts}
\usepackage{comment}
\usepackage{fancyhdr}
\usepackage{mathrsfs}
\usepackage{enumitem}
%\usepackage[retainorgcmds]{IEEEtrantools}

\everymath{\displaystyle}

\usepackage{tikz}

\voffset = -50pt
%\textheight = 700pt
\addtolength{\textwidth}{60pt}
\addtolength{\evensidemargin}{-30pt}
\addtolength{\oddsidemargin}{-30pt}
%\setlength{\headheight}{44pt}

\pagestyle{fancy}
\fancyhf{} % clear all fields
\fancyhead[R]{%
  \scshape
  \begin{tabular}[t]{@{}r@{}}
  MATH 545, Spring 2016\\Section 1 (9515)\\
  HW \#2, DUE: 2016, February 08
  \end{tabular}}
\fancyhead[L]{%
  \scshape
  \begin{tabular}[t]{@{}r@{}}
  JOSEPH OKONOBOH\\Mathematics\\Cal State Long Beach
  \end{tabular}}
\fancyfoot[C]{\thepage}

\newcommand{\qed}{\hfill \ensuremath{\Box}}


\newcommand*\circled[1]{\tikz[baseline=(char.base)]{
            \node[shape=circle,draw,inner sep=2pt] (char) {#1};}}


\newcommand{\cyc}[1]{\langle #1 \rangle}
\newcommand{\Z}{\mathbb{Z}}
\newcommand{\I}{\mathbb{I}}
\newcommand{\F}{\mathbb{F}}
\newcommand{\M}{\mathbb{M}}
\newcommand{\R}{\mathbb{R}}
\newcommand{\Q}{\mathbb{Q}}
\newcommand{\D}{\displaystyle}
\DeclareTextCommand{\_}{OT1}{%
  \leavevmode \kern.06em\vbox{\hrule width.6em}}
%\setcounter{section}{-1}

\begin{document}
Let $R$ be a ring.
\begin{enumerate}
%%%%%%%%%%%%%%%%%%%%%%%%%%%%%%%%%%%%%%%03%%%%%%%%%%%%%%%%%%%%%%%%%%%%%%%%%%%%%%%
   \item[3.]   Prove that if $M$ is an $R$-module, then
               $$r(-m) = -r(m) = -(rm)$$
               for all $r \in R$ and $m \in M$.

      \textbf{Proof.} Let $M$ be an $R$-module and consider $r \in R$ and
      $m \in M$. We will first show that for every $x \in R$, we have
      $x \cdot 0 = 0$, where $0 \in M$. So let $x \in R$. Then since
      \begin{align*}
         0 + x \cdot 0 &= x \cdot 0 \\
                   &= x \cdot (0 + 0) &[0 \text{ is additive identity in }M] \\
                   &= x \cdot 0 + x \cdot 0, &[\text{Distributivity}]
      \end{align*}
      it follows by cancellation that $x \cdot 0 = 0$. Thus it follows that
      \begin{align*}
         -(rm) + rm &= 0 \\
           &= r \cdot 0 \\
           &=  r((-m) + m) \\
           &= r(-m) + rm. &[\text{Distributivity}]
      \end{align*}
      That is, $-(rm) + rm = r(-m) + rm$, and conclude by cancellation that
      $-(rm) = r(-m)$. Next, we will show that $(-1)y = -y$ for every $y \in R$.
      So let $y \in R$. Since
      \begin{align*}
         (-1)y + y &= (-1)y + (1)y \\
            &= ((-1) + 1)y &[\text{Distributivity}] \\
            &= 0 \cdot y \\
            &= 0 & [\text{From }1.1]\\
            &= -y + y,
      \end{align*}
      it follows by cancellation that $(-1)y = -y$. Finally, we have that
      \begin{align*}
         -r(m) &= (-r)(m) \\
               &= [(-1)r](m) \\
               &= (-1)(rm) &[\text{Associativity}] \\
               &= -(rm),
      \end{align*}
      so that
      $$-r(m) = -(rm) = r(-m),$$
      as desired. \qed
%%%%%%%%%%%%%%%%%%%%%%%%%%%%%%%%%%%%%%%04%%%%%%%%%%%%%%%%%%%%%%%%%%%%%%%%%%%%%%%
   \item[4.]   Show that if $N$ is a nonempty subset of an $R$-module $M$, then
               $$\text{Ann}({N}) = \{r \in R : rn = 0
                 \text{ for all }n \in N\}$$
               is an ideal of $R$.

      \textbf{Proof.} Let $N$ be a nonempty subset of an $R$-module $M$. Since
      $0 \cdot n = 0$ for all $n \in N$, it follows immediately that
      $0 \in \text{Ann}(N)$, so that Ann($N$) is nonempty. Now let
      $x, y \in \text{Ann}(N)$ and $t \in N$. It suffices to show that Ann($N$)
      is closed under subtraction and multiplication by $R$. 
      \begin{itemize}
         \item \textbf{closure under subtraction.} Since
               $x, y \in \text{Ann}(N)$, we have $xt = yt = 0$. That is,
               $(x - y)t = xt - yt = 0 - 0 = 0$, so that
               $x - y \in \text{Ann}(N)$. Hence, Ann($N$) is closed under
               subtraction.
         \item \textbf{closure under multiplication by $R$.} Let $r \in R$. So
               we have that $(rx)t = r(xt) = r\cdot 0 = 0$, so that
               $rx \in \text{Ann}(N)$; i.e., Ann($N$) is closed under
               multiplication by $R$.
      \end{itemize}
      Conclude that Ann($N$) is an ideal of $R$. Let $m \in M$. The statement
      preceding 4.2 (Page 14 in A\&K) follows because we just showed that
      Ann($m$) (which is equal to Ann($\{m\}$) and Ann($M$) are ideals of $R$.
      \qed
%%%%%%%%%%%%%%%%%%%%%%%%%%%%%%%%%%%%%%%05%%%%%%%%%%%%%%%%%%%%%%%%%%%%%%%%%%%%%%%
   \item[5.]   Prove that if $M$ and $N$ are $R$-modules then
               $M \oplus N$ is an $R$-module.

      \textbf{Proof.} Assume that $M$ and $N$ are $R$-modules. By Exercises 28
      and 29 from Section 1.1 in Dummit and Foote, it follows that $M \oplus N$
      is an abelian group. It remains to show that $M \oplus N$ is closed under
      scalar multiplication by $R$, distributive, associative, and unitary. So
      let $x, y \in R$ and $a, b \in M \oplus N$. Then $a = (m, n)$ and
      $b = (m', n')$ for some $m, m' \in M$ and $n, n' \in N$. So 
      \begin{itemize}
         \item \textbf{closed under scalar multiplication.} Since $M$ and $N$
               are $R$-modules, it follows that $xm \in M$ and $xn \in N$; thus,
               $xa = x(m, n) = (xm, xn) \in M \oplus N$, so that $M \oplus N$
               is closed under scalar multiplication by $R$.
         \item \textbf{distributive.} This follows because
               \begin{align*}
                  x(a + b) &= x((m, n) + (m', n')) \\
                     &= x(m + m', n + n') &[\text{Definition}] \\
                     &= (x(m + m'), x(n + n')) &[\text{Definition}] \\
                     &= (xm + xm', xn + xn')
                        &[M\text{ and }N \text{ are } R\text{-modules}] \\
                     &= (xm, xn) + (xm', xn') &[\text{Definition}] \\
                     &= x(m, n) + x(m', n') &[\text{Definition}] \\
                     &= xa + xb
               \end{align*}
               and
               \begin{align*}
                  (x + y)a &= (x + y)(m, n) \\
                     &= ((x + y)m, (x + y)n) &[\text{Definition}] \\
                     &= (xm + ym, xn + yn)
                        &[M\text{ and }N \text{ are } R\text{-modules}] \\
                     &= (xm, xn) + (ym, yn) &[\text{Definition}] \\
                     &= x(m, n) + y(m, n) &[\text{Definition}] \\
                     &= xa + ya.
               \end{align*}
         \item \textbf{associative.} This follows because
               \begin{align*}
                  x(ya) &= x(y(m, n)) \\
                     &= x(ym, yn) &[\text{Definition}] \\
                     &= (x(ym), x(yn)) &[\text{Definition}] \\
                     &= ((xy)m, (xy)n) 
                        &[M\text{ and }N \text{ are } R\text{-modules}] \\
                     &= (xy)(m, n) &[\text{Definition}] \\
                     &= (xy)a.
               \end{align*}
         \item \textbf{unitary.} This follows since
               $$1\cdot a = 1(m, n) = (1 \cdot m, 1 \cdot n) = (m, n) = a.$$
      \end{itemize}
      So conclude that $M \oplus N$ is an $R$-module. \qed
\end{enumerate}
\end{document}
