\documentclass[9pt]{article}

\usepackage{amssymb}
\usepackage{amsmath}
\usepackage{amsfonts}
\usepackage{comment}
\usepackage{fancyhdr}
\usepackage{mathrsfs}
\usepackage{amscd}
%\usepackage[retainorgcmds]{IEEEtrantools}

\everymath{\displaystyle}

\usepackage{tikz}

\voffset = -50pt
%\textheight = 700pt
\addtolength{\textwidth}{60pt}
\addtolength{\evensidemargin}{-30pt}
\addtolength{\oddsidemargin}{-30pt}
%\setlength{\headheight}{44pt}

\pagestyle{fancy}
\fancyhf{} % clear all fields
\fancyhead[R]{%
  \scshape
  \begin{tabular}[t]{@{}r@{}}
  MATH 545, Spring 2016\\Section 1 (9515)\\
  Exam, DUE: 2016, March 14
  \end{tabular}}
\fancyhead[L]{%
  \scshape
  \begin{tabular}[t]{@{}r@{}}
  JOSEPH OKONOBOH\\Mathematics\\Cal State Long Beach
  \end{tabular}}
\fancyfoot[C]{\thepage}

\newcommand{\qed}{\hfill \ensuremath{\Box}}


\newcommand*\circled[1]{\tikz[baseline=(char.base)]{
            \node[shape=circle,draw,inner sep=2pt] (char) {#1};}}


\newcommand{\cyc}[1]{\langle #1 \rangle}
\newcommand{\Z}{\mathbb{Z}}
\newcommand{\I}{\mathbb{I}}
\newcommand{\F}{\mathbb{F}}
\newcommand{\M}{\mathbb{M}}
\newcommand{\R}{\mathbb{R}}
\newcommand{\Q}{\mathbb{Q}}
\newcommand{\Ker}{\text{Ker}}
\newcommand{\Coker}{\text{Coker}}
\newcommand{\im}{\text{Im}}
\newcommand{\D}{\displaystyle}
\DeclareTextCommand{\_}{OT1}{%
  \leavevmode \kern.06em\vbox{\hrule width.6em}}
%\setcounter{section}{-1}

\begin{document}
\begin{enumerate}
%%%%%%%%%%%%%%%%%%%%%%%%%%%%%%%%%%%%%%%01%%%%%%%%%%%%%%%%%%%%%%%%%%%%%%%%%%%%%%%
   \item[1.]   Let $R$ be a local ring that is an integral domain, and let $I$
               be its maximal ideal. Assume further that $I$ is principal with
               generator $s$ and that $\bigcap_{n\in\Z^+}I^n = (0)$.
               \begin{enumerate}
                  \item Show that the set $\Z_{((3))}$, which is the set of all
                        rational numbers whose lowest-term denominators are not
                        divisible by 3, is much a ring. (Prove everything ---
                        starting with the fact that it is a ring.)
                  \item Back to the general case. Prove that every nonzero
                        element $r$ of $R$ can be written uniquely as
                        $r = us^n$, where $u$ is a unit and $n$ is a nonnegative
                        integer. (\textit{Hint}: $r$ cannot lie in every power
                        of $I$!)
                  \item For each $r \in R/\{0\}$, define $\omega(r)$ to
                        be the nonnegative integer $n$ that occurs in the
                        factorization $r = us^n$ from part 1b.
                        \begin{enumerate}
                           \item Prove that for $r_1$, $r_2 \in R/\{0\}$,
                                 $\omega(r_1r_2) = \omega(r_1)+\omega(r_2)$.
                           \item Prove that $r_1, r_2 \in R/\{0\}$ with
                                 $\omega(r_1) < \omega(r_2)$,
                                 $\omega(r_1+r_2) = \omega(r_1)$.
                           \item Give an example from $\Z_{((3))}$ to show that
                                 when $\omega(r_1) = \omega(r_2)$,
                                 $\omega(r_1+r_2)$ may be different from
                                 $\omega(r_1)$.
                        \end{enumerate}
                  \item Prove that $R$ is a PID. (\textit{Hint}: For an ideal
                        $J$, consider the set $\{\omega(r) : r \in J\}$.)
               \end{enumerate}

      \textbf{Solution.}
      
      \begin{enumerate}
         \item \textbf{Proof.} Let $a, b \in \Z_{((3))}$, so that $a = q_1/s_1$
               and $b = q_2/s_2$, where $q_1$ and $q_2$ are integers, $s_1$ and
               $s_2$ are nonzero integers, and
               $\gcd(3, s_1) = \gcd(3, s_2) = 1$.

               \begin{itemize}
                  \item \textbf{closure.} We have that
                        $$a + b = \frac{q_1}{s_1} + \frac{q_2}{s_2} =
                          \frac{q_1s_2 + q_2s_1}{s_1s_2} \text{ and }
                          ab = \frac{q_1q_2}{s_1s_2}.$$
         
                        Since $s_1$ and $s_2$ both have no factors of 3, it
                        follows that $s_1s_2$ has no factors of 3; that is, the
                        denominators of the lowest terms of $a + b$ and $ab$
                        have no factors of 3, and thus,
                        $a + b, ab \in \Z_{((3))}$.
                  \item \textbf{identity.} Since $\Z_{((3))} \subseteq \Q$ and
                        $0 = 0/1, 1 = 1/1\in\Z_{((3))}$, it follows that 0 and 1
                        are the additive and multiplicative identity of
                        $\Z_{((3))}$, respectively.
                  \item \textbf{inverse.} Since
                        $1 = \gcd(3, s_1) = \gcd(3, -s_1)$, it follows that
                        $$-a = -\frac{q_1}{s_1} = 
                          \frac{q_1}{-s_1} \in \Z_{((3))}.$$
               \end{itemize}

               The associativity and commutativity of addition and
               multiplicative on $\Z_{((3))}$ and distributivity follow because
               $\Z_{((3))}$ is a subset of $\Q$. That is,
               $\Z_{((3))}$ is a ring. \qed        
         \item \textbf{Proof.} Let $r \in R/\{0\}$. First suppose that $r$ is a
               unit. Then it follows that $r = rs^0$. To show uniqueness,
               suppose that $r = us^i$, for some $u \in R$ and $i \ge 1$. Since
               $r$ is a unit and $r = (us^{i-1})s \in (s) = I$, it follows by
               Proposition 7.9 (DF) that $I = R$, contradicting the maximality
               of $I$. Thus, $i = 0$, so that $u = r$. Now suppose that $r$ is
               not a unit. Then it follows by Theorem 1 that $r \in I$. Let us
               now make the following observations:
               \begin{itemize}
                  \item Since $I = (s)$, it follows that $I^n = (s^n)$.
                  \item If $t \in I^m$, then $t \in I^i$ for $i = 1, \ldots, m$.
                        To see this, suppose $t \in I^m$, with $m \in \Z^+$, so
                        that $t = ws^m$, for some $w \in R$. For
                        $i \in \{1, \ldots, m\}$, we have
                        $$t = (ws^{m-i})s^i \in (s^i) = I^i.$$
                        So if $v \notin I^m$, then $v \notin I^i$ for all
                        $i \ge m$.
               \end{itemize}

               Now if $r \in I^n$ for all $n \in \Z^+$, we would have that
               $r \in \bigcap_{n\in\Z^+}I^n = \{0\}$, a contradiction because
               $r \neq 0$. Thus there exists $p \in \Z^+$ such that
               $r \notin I^p$. Indeed, $p \notin I^i$ for all $i \ge p$. So the
               set $S_r := \{n \in \Z^+ : r \in I^n\}$ is finite. Let
               $x = \max\{S_r\}$. Hence $r \in (s^x) = I^x$, so that $r = ys^x$
               for some $y \in R$. If $y$ is not a unit, then $y \in I$ by
               Theorem 1. By repeating the same process above, we can conclude
               that $y \in I^{x'}$, where $x' \in \Z^+$ and $x'$ is maximal
               (i.e., $y \notin I^i$ for all $i > x'$.) So $y = y's^{x'}$ for
               some $y' \in R$, and thus,
               $r = ys^x = y's^{x'}s^x = y's^{x'+x} \in I^{x'+x}$, so that
               $x' + x \in S_r$, a contradiction since $x' + x > x'$ and $x'$
               was assumed to be the maximum element of $S_r$. Thus $y$ is a
               unit. Finally, for uniqueness, suppose that $r = as^j$ for some
               unit $a$ in $R$ and $j \ge 1$. Assume without loss that
               $j \ge x$. So $r = ys^x = as^j$. Since $R$ is an integral domain,
               we can cancel to get $y = as^{j-x}$, so that $ay^{-1} = s^{j-x}$.
               If $j > x$, then $ay^{-1} \in (s)$, a contradiction since that
               would imply that $I$ contains a unit. Thus $j = x$ and it follows
               that $ay^{-1} = 1$, so that $a = y$. Thus the representation is
               unique. \qed
         \item \begin{enumerate}
                  \item Let $r_1, r_2 \in R/\{0\}$. By 1(b), we can uniquely
                        write $r_1 = u_1s^{n_1}$ and $r_2 = u_2s^{n_2}$, where
                        $u_1$ and $u_2$ are units in $R$ and $n_1$ and $n_2$ are
                        nonnegative integers. That is, $\omega(r_1) = n_1$ and
                        $\omega(r_2) = n_2$. Now $r_1r_2 = u_1u_2s^{n_1+n_2}$.
                        Since $u_1$ and $u_2$ are units, it follows that
                        $u_1u_2$ is also a unit in $R$. By 1(b), the
                        representation of $r_1r_2$ by $u_1u_2s^{n_1+n_2}$ is
                        unique. So
                        $\omega(r_1r_2) = n_1+n_2 = \omega(r_1) + \omega(r_2)$.
               \end{enumerate}
      \end{enumerate}
\end{enumerate}
\end{document}
