\documentclass[9pt]{article}

\usepackage{amssymb}
\usepackage{amsmath}
\usepackage{amsfonts}
\usepackage{comment}
\usepackage{fancyhdr}
\usepackage{mathrsfs}
\usepackage{enumitem}
%\usepackage[retainorgcmds]{IEEEtrantools}

\everymath{\displaystyle}

\usepackage{tikz}

\voffset = -50pt
%\textheight = 700pt
\addtolength{\textwidth}{60pt}
\addtolength{\evensidemargin}{-30pt}
\addtolength{\oddsidemargin}{-30pt}
%\setlength{\headheight}{44pt}

\pagestyle{fancy}
\fancyhf{} % clear all fields
\fancyhead[R]{%
  \scshape
  \begin{tabular}[t]{@{}r@{}}
  MATH 545, Spring 2016\\Section 1 (9515)\\
  HW \#3, DUE: 2016, February 10
  \end{tabular}}
\fancyhead[L]{%
  \scshape
  \begin{tabular}[t]{@{}r@{}}
  JOSEPH OKONOBOH\\Mathematics\\Cal State Long Beach
  \end{tabular}}
\fancyfoot[C]{\thepage}

\newcommand{\qed}{\hfill \ensuremath{\Box}}


\newcommand*\circled[1]{\tikz[baseline=(char.base)]{
            \node[shape=circle,draw,inner sep=2pt] (char) {#1};}}


\newcommand{\cyc}[1]{\langle #1 \rangle}
\newcommand{\Z}{\mathbb{Z}}
\newcommand{\I}{\mathbb{I}}
\newcommand{\F}{\mathbb{F}}
\newcommand{\M}{\mathbb{M}}
\newcommand{\R}{\mathbb{R}}
\newcommand{\Q}{\mathbb{Q}}
\newcommand{\D}{\displaystyle}
\DeclareTextCommand{\_}{OT1}{%
  \leavevmode \kern.06em\vbox{\hrule width.6em}}
%\setcounter{section}{-1}

\begin{document}
Let $R$ be a ring.
\begin{enumerate}
%%%%%%%%%%%%%%%%%%%%%%%%%%%%%%%%%%%%%%%05%%%%%%%%%%%%%%%%%%%%%%%%%%%%%%%%%%%%%%%
   \item[5.]   Prove that if $M$ and $N$ are $R$-modules then
               $M \oplus N$ is an $R$-module.

      \textbf{Proof.} Assume that $M$ and $N$ are $R$-modules. By Exercises 28
      and 29 from Section 1.1 in Dummit and Foote, it follows that $M \oplus N$
      is an abelian group. It remains to show that $M \oplus N$ is closed under
      scalar multiplication by $R$, is distributive, is associative, and is
      unitary. So let $x, y \in R$ and $a, b \in M \oplus N$. Then $a = (m, n)$
      and $b = (m', n')$ for some $m, m' \in M$ and $n, n' \in N$. So 
      \begin{itemize}
         \item \textbf{closed under scalar multiplication.} Since $M$ and $N$
               are $R$-modules, it follows that $xm \in M$ and $xn \in N$; thus,
               $xa = x(m, n) = (xm, xn) \in M \oplus N$, so that $M \oplus N$
               is closed under scalar multiplication by $R$.
         \item \textbf{distributive.} This follows because
               \begin{align*}
                  x(a + b) &= x((m, n) + (m', n')) \\
                     &= x(m + m', n + n') &[\text{Definition}] \\
                     &= (x(m + m'), x(n + n')) &[\text{Definition}] \\
                     &= (xm + xm', xn + xn')
                        &[M\text{ and }N \text{ are } R\text{-modules}] \\
                     &= (xm, xn) + (xm', xn') &[\text{Definition}] \\
                     &= x(m, n) + x(m', n') &[\text{Definition}] \\
                     &= xa + xb
               \end{align*}
               and
               \begin{align*}
                  (x + y)a &= (x + y)(m, n) \\
                     &= ((x + y)m, (x + y)n) &[\text{Definition}] \\
                     &= (xm + ym, xn + yn)
                        &[M\text{ and }N \text{ are } R\text{-modules}] \\
                     &= (xm, xn) + (ym, yn) &[\text{Definition}] \\
                     &= x(m, n) + y(m, n) &[\text{Definition}] \\
                     &= xa + ya.
               \end{align*}
         \item \textbf{associative.} This follows because
               \begin{align*}
                  x(ya) &= x(y(m, n)) \\
                     &= x(ym, yn) &[\text{Definition}] \\
                     &= (x(ym), x(yn)) &[\text{Definition}] \\
                     &= ((xy)m, (xy)n) 
                        &[M\text{ and }N \text{ are } R\text{-modules}] \\
                     &= (xy)(m, n) &[\text{Definition}] \\
                     &= (xy)a.
               \end{align*}
         \item \textbf{unitary.} This follows since
               $$1\cdot a = 1(m, n) = (1 \cdot m, 1 \cdot n) = (m, n) = a.$$
      \end{itemize}
      So conclude that $M \oplus N$ is an $R$-module. \qed
%%%%%%%%%%%%%%%%%%%%%%%%%%%%%%%%%%%%%%%06%%%%%%%%%%%%%%%%%%%%%%%%%%%%%%%%%%%%%%%
   \item[6.]   Prove ``is an isomorphism to" (for modules) is an equivalence
               relation.

      \textbf{Proof.} Let $\mathscr{X}$ be the set of all $R$-modules. The set
      $\mathscr{X}$ is nonempty because $R \in \mathscr{X}$. Now define a
      relation $\mathcal{R}$ on $\mathscr{X}$ such that if
      $M, N \in \mathscr{X}$ then $(M, N) \in \mathcal{R}$ if and only if $M$
      and $N$ are isomorphic. We will now show that $\mathcal{R}$ is an
      equivalence relation on $\mathscr{X}$.
      \begin{itemize}
         \item \textbf{reflexivity.} Let $M \in \mathscr{X}$ and let
               $i : M \rightarrow M$ be the identity function. Since
               $$i(m_1 + m_2) = m_1 + m_2 = i(m_1) + i(m_2)$$
               and
               $i(rm_1) = rm_1 = ri(m_1)$ for all $r \in R$ and
               $m_1, m_2 \in M$, it follows that $i$ is an $R$-module
               homomorphism. Since $i$ is the identity on $M$, it is trivially
               bijective. Thus $i$ is an $R$-module isomorphism, so that $M$ is
               isomorphic to $M$; i.e., $(M, M) \in \mathcal{R}$ and thus
               $\mathcal{R}$ is reflexive.
         \item \textbf{symmetry.} Suppose that $(M, N) \in \mathcal{R}$ for some
               $M, N \in \mathscr{X}$. That is, there exists an $R$-module
               isomorphism $\alpha : M \rightarrow N$. Since $\alpha$ is
               bijective, it has a unique and well defined inverse,
               $\alpha^{-1} : N \rightarrow M$ (where $\alpha^{-1}(a) = b$ if
               and only if $\alpha(b) = a$), which is also bijective. Now it
               suffices to show that $\alpha^{-1}$ is also an $R$-module
               homomorphism. Let $n_1, n_2 \in N$. Since $\alpha$ is bijective,
               there exist unique $m_1, m_2 \in M$ such that $\alpha(m_1) = n_1$
               and $\alpha(m_2) = n_2$.
               \begin{itemize}
                  \item \textbf{Show that } $\alpha^{-1}$
                        \textbf{is a group homomorphism.} We have
                        \begin{align*}
                           n_1 + n_2 &= \alpha(m_1) + \alpha(m_2) \\
                              &= \alpha(m_1 + m_2), &[\alpha \text{ is a
                                 group homomorphism}]
                        \end{align*}
                        and thus 
                        $$\alpha^{-1}(n_1 + n_2) = m_1 + m_2 =
                          \alpha^{-1}(n_1) + \alpha^{-1}(n_2),$$
                        so that $\alpha^{-1}$ is also a group homomorphism.
                  \item \textbf{Show that }
                        $\alpha^{-1}(rn_1) = r\alpha^{-1}(n_1)$ for all
                        $r \in R$. Let $r \in R$. We have
                        \begin{align*}
                           \alpha(rm_1) &= r\alpha(m_1) 
                              &[\alpha \text{ is an }R
                                \text{-module homomorphism}] \\
                              &= rn_1,
                        \end{align*}
                        so that $\alpha^{-1}(rn_1) = rm_1 = r\alpha^{-1}(n_1)$.
               \end{itemize}
               Thus conclude that $\alpha^{-1}$ is an $R$-module isomorphism.
               That is, $(N, M) \in \mathcal{R}$, so that $\mathcal{R}$ is
               symmetric.
         \item \textbf{transitivity.} Suppose that $(M, N) \in \mathcal{R}$ and
               $(N, T) \in \mathcal{R}$ for some $M, N, T \in \mathscr{X}$.
               That is, there exist $R$-module isomorphisms
               $\alpha : M \rightarrow N$ and $\beta : N \rightarrow T$. We
               claim that $\beta\circ\alpha : M \rightarrow T$ is also an
               $R$-module isomorphisms. The map $\beta\circ\alpha$ is bijective
               because the composition of bijective functions is also bijective.
               Now let $m_1, m_2 \in M$.
               \begin{itemize}
                  \item \textbf{Show that } $\beta\circ\alpha$
                        \textbf{is a group homomorphism.} We have
                        \begin{align*}
                           (\beta\circ\alpha)(m_1 + m_2) &=
                              \beta(\alpha(m_1 + m_2)) \\
                              &= \beta(\alpha(m_1) + \alpha(m_2))
                                 &[\alpha \text{ is a group homomorphism}] \\
                              &= \beta(\alpha(m_1)) + \beta(\alpha(m_2))
                                 &[\beta \text{ is a group homomorphism}] \\
                              &= (\beta\circ\alpha)(m_1)) +
                                    (\beta\circ\alpha)(m_2),
                        \end{align*}
                        so that $\beta\circ\alpha$ is a group homomorphism.
                  \item \textbf{Show that }
                        $(\beta\circ\alpha)(rm_1) = r[(\beta\circ\alpha)(m_1)]$
                        for all $r \in R$. Let $r \in R$. We have
                        \begin{align*}
                           (\beta\circ\alpha)(rm_1) &= \beta(\alpha(rm_1)) \\
                              &= \beta(r\alpha(m_1))
                              &[\alpha \text{ is an }R
                                \text{-module homomorphism}] \\
                              &= r\beta(\alpha(m_1))
                              &[\beta \text{ is an }R
                                \text{-module homomorphism}] \\
                              &= r[(\beta\circ\alpha)(m_1)].
                        \end{align*}
               \end{itemize}
               Conclude that $\beta\circ\alpha$ is an $R$-module isomorphism.
               Thus $(M, T) \in \mathcal{R}$, so that $\mathcal{R}$ is
               transitive.
      \end{itemize}
      The above shows us that $\mathcal{R}$ is an equivalence relation on
      $\mathscr{X}$. \qed
%%%%%%%%%%%%%%%%%%%%%%%%%%%%%%%%%%%%%%%07%%%%%%%%%%%%%%%%%%%%%%%%%%%%%%%%%%%%%%%
   \item[7.]   Prove that if $\phi : M \rightarrow N$ is an $R$-module
               homomorphism then
               $$\text{ker}(\phi) = \{m \in M : \phi(m) = 0\}$$
               is a submodule of $M$.

      \textbf{Proof.} Suppose that $\phi : M \rightarrow N$ is an $R$-module
      homomorphism. Since $\phi$ is, in particular, a group homomorphism, it
      follows by Proposition 3.1 (4) (DF) that ker($\phi$) is a subgroup of $M$.
      So to show that ker($\phi$) is a submodule of $M$, it suffices to show
      ker($\phi$) is closed under scalar multiplication by $R$. So let $x \in R$
      and $a \in \text{ker}(\phi)$. Since $a$ is in the kernel of $\phi$, it
      follows that $\phi(a) = 0$. Since $\phi$ is an $R$-module homomorphism, we
      have that $\phi(xa) = x\phi(a) = x \cdot 0 = 0$, so that
      $xa \in \text{ker}(\phi)$, so that ker($\phi$) is an $R$-submodule of $M$.
      \qed
%%%%%%%%%%%%%%%%%%%%%%%%%%%%%%%%%%%%%%%08%%%%%%%%%%%%%%%%%%%%%%%%%%%%%%%%%%%%%%%
   \item[8.]   Prove that if $N$ is an $R$-submodule of an $R$-module $M$, then
               $$\pi : M \rightarrow M/N \text{ defined by } m \mapsto m + N$$
               is a surjective $R$-module homomorphism.
               
      \textbf{Proof.} Assume that $N$ is an $R$-submodule of an $R$-module $M$
      and consider the map
      $$\pi : M \rightarrow M/N \text{ defined by } m \mapsto m + N.$$
      The map $\pi$ is, in particular, the natural projection of the group $M$
      onto the group $M/N$. Thus, $\pi$ is an onto group homomorphism, so it
      suffices to show that $\pi(rm) = r\pi(m)$ for all $r \in R$ and $m \in M$.
      So let $r \in R$ and $m \in M$. It follows that
      \begin{align*}
         \pi(rm) &= rm + N \\
            &= r(m + N) &[\text{Definition}] \\
            &= r\pi(m),
      \end{align*}
      so that $\pi$ is a surjective $R$-module homomorphism. \qed
%%%%%%%%%%%%%%%%%%%%%%%%%%%%%%%%%%%%%%%09%%%%%%%%%%%%%%%%%%%%%%%%%%%%%%%%%%%%%%%
   \item[9.]   Prove that if $M$ and $N$ are $R$-modules, and we endow the set
               $\text{Hom}_R(M, N)$ with addition and scalar multiplication such
               that
               $$(\alpha + \beta)(m) = \alpha(m) + \beta(m) \text{ and }
                 (r \cdot \alpha)(m) = r \cdot(\alpha(m))$$
               for all $\alpha, \beta \in \text{Hom}_R(M, N)$, $m \in M$, and
               $r \in R$, then $\text{Hom}_R(M, N)$ is an $R$-module.

      \textbf{Proof.} Let $M$ and $N$ be $R$-modules and let addition and scalar
      multiplication by $R$ be defined as stated above. First, we will show that
      $\text{Hom}_R(M, N)$ is an abelian group. So let $\alpha$, $\beta$, and
      $\gamma$ be arbitrary elements in $\text{Hom}_R(M, N)$, $x$, $y$,
      $z \in M$, and $r, s \in R$.

      \begin{itemize}
         \item \textbf{closure.} It follows that
               \begin{align*}
                  (\alpha + \beta)(x + y) &= \alpha(x + y) + \beta(x + y)
                     &[\text{Definition}] \\
                     &= \alpha(x) + \alpha(y) + \beta(x) + \beta(y)
                        &[\alpha \text{ and } \beta \text{ are linear maps}] \\
                     &= \alpha(x) + \beta(x) + \alpha(y) + \beta(y) \\
                     &= (\alpha+\beta)(x) + (\alpha+\beta)(y)
                        &[\text{Definition}] 
               \end{align*}
               and
               \begin{align*}
                  (\alpha + \beta)(rx) &= \alpha(rx) + \beta(rx)
                     &[\text{Definition}] \\
                     &= r\alpha(x) + r\beta(x)
                        &[\alpha \text{ and } \beta \text{ are linear maps}] \\
                     &= r(\alpha(x) + \beta(x)) &[\text{Distributivity}] \\
                     &= r[(\alpha+\beta)(x)]. &[\text{Definition}] 
               \end{align*}
               Thus $\alpha+\beta \in \text{Hom}_R(M, N)$, so that
               $\text{Hom}_R(M, N)$ is closed under addition.
         \item \textbf{associativity.} It follows that
               \begin{align*}
                  [(\alpha + \beta) + \gamma](x) &=
                     (\alpha+\beta)(x) + \gamma(x) &[\text{Definition}] \\
                     &= [\alpha(x + y) + \beta(x + y)] + \gamma(x + y)
                        &[\text{Definition}] \\
                     &= \alpha(x + y) + [\beta(x + y) + \gamma(x + y)]
                        &[(N, +) \text{ is associative}] \\
                     &= \alpha(x + y) + (\beta+\gamma)(x + y) \\
                     &= [\alpha + (\beta + \gamma)](x+y), &[\text{Definition}]
               \end{align*}
               so that $\text{Hom}_R(M, N)$ is associative under addition.
         \item \textbf{identity.} The zero $R$-module homomorphism in
               $\text{Hom}_R(M, N)$ is the additive identity for the
               $\text{Hom}_R(M, N)$.
         \item \textbf{inverse.} Since
               $$[\alpha + (-\alpha)](x) = \alpha(x) + (-\alpha)(x) =
                 \alpha(x) - \alpha(x) = 0$$
               for all $x \in M$, it follows that $\alpha + (-\alpha)$ is the
               zero homomorphism. Thus $-\alpha$ is the additive inverse of
               $\alpha$.
         \item \textbf{commutativity.} Because
               \begin{align*}
                  (\alpha+\beta)(x) &= \alpha(x) + \beta(x)
                     &[\text{Definition}] \\
                     &= \beta(x) + \alpha(x) &[(N, +) \text{ is abelian}] \\
                     &= (\beta+\alpha)(x),
               \end{align*}
               we have $\alpha + \beta = \beta + \alpha$, so that addition in
               $\text{Hom}_R(M, N)$ is commutative.
      \end{itemize}
      Conclude from above that $(\text{Hom}_R(M, N), +)$ is an abelian group.
      Next we will show the closure of scalar multiplication, distributivity,
      associativity,
      \begin{itemize}
         \item \textbf{closure of scalar multiplication.} We want to show that
               $r \cdot \alpha \in \text{Hom}_R(M, N)$. So for $x, y \in M$ and
               $s \in R$, we
               have
               \begin{align*}
                  (r\alpha)(x + y) &= r (\alpha(x + y)) &[\text{Definition}] \\
                     &= r(\alpha(x) + \alpha(y)) &[\alpha \text{ is linear}] \\
                     &= r\alpha(x) + r\alpha(y)
                        &[N \text{ is an } R\text{-module}] \\
                     &= (r\alpha)(x) + (r\alpha)(y) &[\text{Definition}]
               \end{align*}
               and
               \begin{align*}
                  (r\alpha)(sx) &= r (\alpha(sx)) &[\text{Definition}] \\
                     &= r(s\alpha(x)) &[\alpha \text{ is linear}] \\
                     &= (rs)(\alpha(x)) &[N \text{ is an } R\text{-module}] \\
                     &= (sr)(\alpha(x)) &[R \text{ is commutative}] \\
                     &= s(r\alpha(x)) &[N \text{ is an } R\text{-module}] \\
                     &= s((r\alpha)(x)). &[\text{Definition}]
               \end{align*}
               Hence $r \cdot \alpha \in \text{Hom}_R(M, N)$, and thus,
               $\text{Hom}_R(M, N)$ is closed under scalar multiplication by
               $R$.
         \item \textbf{distributivity.} Since
               \begin{align*}
                  [r_1(\alpha + \beta)](m_1) &= r_1[(\alpha + \beta)(m_1)]
                     &[\text{Definition}] \\
                     &= r_1(\alpha(m_1) + \beta(m_1)) &[\text{Definition}] \\
                     &= r_1\alpha(m_1) + r_1\beta(m_1)
                        &[N \text{ is an } R\text{-module}] \\
                     &= (r_1\alpha)(m_1) + (r_1\beta)(m_1),
               \end{align*}
               so that $r_1(\alpha + \beta) = r_1\alpha + r_1\beta$, and
               \begin{align*}
                  [(r_1 + r_2)\alpha](m_1) &= (r_1 + r_2)(\alpha(m_1))
                     &[\text{Definition}] \\
                     &= r_1(\alpha(m_1) + r_2(\alpha(m_1) 
                        &[N \text{ is an } R\text{-module}] \\
                     &= (r_1\alpha)(m_1) +(r_2\alpha)(m_1), &[\text{Definition}]
               \end{align*}
               so that $(r_1 + r_2)\alpha = r_1\alpha + r_2\alpha$,
               distributivity follows.
         \item \textbf{associativity.} Since
               \begin{align*}
                  [r_1(r_2\alpha)](m_1) &= r_1[(r_2\alpha)(m_1)]
                     &[\text{Definition}] \\
                     &= r_1[r_2(\alpha(m_1))] &[\text{Definition}] \\
                     &= (r_1r_2)(\alpha(m_1))
                        &[N \text{ is an } R\text{-module}] \\
                     &= [(r_1r_2)\alpha](m_1), &[\text{Definition}]
               \end{align*}
               so that $r_1(r_2\alpha) = (r_1r_2)\alpha$, associativity follows.
         \item It follows trivially from the definition of scalar multiplication
               that $1 \cdot\tau = \tau$ for all $\tau \in \text{Hom}_R(M, N)$.
      \end{itemize}
      Conclude that $\text{Hom}_R(M, N)$ is an $R$-module homomorphism.
\end{enumerate}
\end{document}
