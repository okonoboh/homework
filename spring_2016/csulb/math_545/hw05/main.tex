\documentclass[9pt]{article}

\usepackage{amssymb}
\usepackage{amsmath}
\usepackage{amsfonts}
\usepackage{comment}
\usepackage{fancyhdr}
\usepackage{mathrsfs}
\usepackage{amscd}
%\usepackage[retainorgcmds]{IEEEtrantools}

\everymath{\displaystyle}

\usepackage{tikz}

\voffset = -50pt
%\textheight = 700pt
\addtolength{\textwidth}{60pt}
\addtolength{\evensidemargin}{-30pt}
\addtolength{\oddsidemargin}{-30pt}
%\setlength{\headheight}{44pt}

\pagestyle{fancy}
\fancyhf{} % clear all fields
\fancyhead[R]{%
  \scshape
  \begin{tabular}[t]{@{}r@{}}
  MATH 545, Spring 2016\\Section 1 (9515)\\
  HW \#5, DUE: 2016, February 17
  \end{tabular}}
\fancyhead[L]{%
  \scshape
  \begin{tabular}[t]{@{}r@{}}
  JOSEPH OKONOBOH\\Mathematics\\Cal State Long Beach
  \end{tabular}}
\fancyfoot[C]{\thepage}

\newcommand{\qed}{\hfill \ensuremath{\Box}}


\newcommand*\circled[1]{\tikz[baseline=(char.base)]{
            \node[shape=circle,draw,inner sep=2pt] (char) {#1};}}


\newcommand{\cyc}[1]{\langle #1 \rangle}
\newcommand{\Z}{\mathbb{Z}}
\newcommand{\I}{\mathbb{I}}
\newcommand{\F}{\mathbb{F}}
\newcommand{\M}{\mathbb{M}}
\newcommand{\R}{\mathbb{R}}
\newcommand{\Q}{\mathbb{Q}}
\newcommand{\Ker}{\text{Ker}}
\newcommand{\Coker}{\text{Coker}}
\newcommand{\im}{\text{Im}}
\newcommand{\D}{\displaystyle}
\DeclareTextCommand{\_}{OT1}{%
  \leavevmode \kern.06em\vbox{\hrule width.6em}}
%\setcounter{section}{-1}

\begin{document}
\begin{enumerate}
%%%%%%%%%%%%%%%%%%%%%%%%%%%%%%%%%%%SnakeLemma%%%%%%%%%%%%%%%%%%%%%%%%%%%%%%%%%%%
   \item[\textbf{Snake Lemma.}] If the diagram below is commutative with exact
                                rows, then it yields the following exact
      \begin{equation}\label{5_1}
         \begin{CD}
            @. M' @>\alpha>> M @>\beta>> M'' @>>> 0 \\
            @. @V\gamma'VV @V{\gamma}VV @V\gamma''VV \\
            0 @>>> N' @>\alpha'>> N @>\beta'>> N''
         \end{CD}
      \end{equation}
      sequence:
      \begin{equation} \label{5_2}
         \Ker(\gamma') \stackrel{\varphi}{\rightarrow} \Ker(\gamma)
         \stackrel{\psi}{\rightarrow} \Ker(\gamma'')
         \stackrel{\partial}{\rightarrow} \Coker(\gamma')
         \stackrel{\varphi'}{\rightarrow} \Coker(\gamma)
         \stackrel{\psi'}{\rightarrow} \Coker(\gamma'').
      \end{equation}
      We proved in class that $\varphi'$ and $\psi'$ are well-defined, wherein
      $$\varphi'(n'+\im(\gamma'))=\alpha'(n') + \im(\gamma) \text{ and }
      \psi'(n + \im(\gamma))=\beta'(n') + \im(\gamma'').$$
      Also, $\varphi = \alpha|_{_{\Ker(\gamma')}}$ and
      $\psi = \beta|_{_{\Ker(\gamma)}}$.
%\begin{comment}
\item[\textbf{Diagram From Class.}] We will make use of the following augmented
                                    diagram that was created in class to prove
                                    the Snake Lemma.
      \begin{equation*}
         \begin{CD}
            @. 0             @.          0            @.       0 \\
            @. @VVV                      @VVV                  @VVV \\
            @. \Ker(\gamma') @>\varphi>> \Ker(\gamma) @>\psi>> \Ker(\gamma'') \\
            @. @VVV                      @VVV                  @VVV \\
            @. M'            @>\alpha>>  M            @>\beta>> M'' @>>> 0 \\
            @. @V\gamma'VV                      @V{\gamma}VV   @V\gamma''VV \\
            0 @>>> N' @>\alpha'>> N @>\beta'>> N''\\
            @. @VVV                      @VVV                  @VVV \\
            @. \Coker(\gamma') @>\varphi'>> \Coker(\gamma) @>\psi'>> \Coker(\gamma'') \\
            @. @VVV                      @VVV                  @VVV \\
            @. 0             @.          0            @.       0 \\
         \end{CD}
      \end{equation*}
      where $\varphi = \alpha|_{_{\Ker(\gamma')}}$,
      $\psi = \beta|_{_{\Ker(\gamma)}}$,
      $\varphi'(n' + \im(\gamma')) = \alpha'(m') + \im(\gamma)$, and      
      $\psi'(n + \im(\gamma)) = \beta'(n') + \im(\gamma'')$.
%\end{comment}
      
%%%%%%%%%%%%%%%%%%%%%%%%%%%%%%%%%%%%%%%13%%%%%%%%%%%%%%%%%%%%%%%%%%%%%%%%%%%%%%%
   \item[13.]  Show that the Snake Lemma sequence is exact at $\Coker(\gamma)$.

      \textbf{Proof.} It suffices to show that
      $\im(\varphi') = \Ker(\psi')$. So we will show each set is
      contained in the other.
      \begin{itemize}
         \item ($\subseteq$): Let $x \in \im(\varphi')$. Then
               $x = \varphi'(n' + \im(\gamma'))$, for some $n' \in N'$. So
               \begin{align*}
                  \psi'(x) &= \psi'(\varphi'(n' + \im(\gamma'))) \\
                     &= \psi'(\alpha'(n') + \im(\gamma)) \\
                     &= \beta'(\alpha'(n')) + \im(\gamma'') \\
                     &= 0 + \im(\gamma'')
                        &[\im(\alpha') = \Ker(\beta')] \\
                     &= \im(\gamma'') \\
                     &= 0,
               \end{align*}
               so that $x \in \Ker(\psi')$. Thus 
               $\im(\varphi') \subseteq \Ker(\psi')$.
         \item ($\supseteq$): Let $y \in \Ker(\psi')$. Since
               $\Ker(\psi') \subseteq \Coker(\gamma)$, it follows that
               $y \in \Coker(\gamma)$, so $y = n + \im(\gamma)$ for some
               $n \in N$. Now since $y$ is in the kernel of $\psi'$, we have
               that
               $$0 = \im(\gamma'') = \psi'(y) =
                 \psi'(n + \text{img}(\gamma)) =
                 \beta'(n) + \im(\gamma'').$$
               From the equalities above, we have
               $\beta'(n) + \im(\gamma'') = \im(\gamma'')$, and this
               implies that $\beta'(n) \in \im(\gamma'')$. So
               $\beta'(n) = \gamma''(m'')$ for some $m'' \in M''$. Since the
               horizontal sequence is exact at $M''$, it follows that $\beta$ is
               surjective, so $\beta(m) = m''$ for some $m \in M$. Thus
               $\beta'(n) = \gamma''(m'') = \gamma''(\beta(m)) =
                \beta'(\gamma(m))$, where, in last equality, we used the
               commutativity of \eqref{5_1}. Hence
               $\beta'(n) = \beta'(\gamma(m))$, so that
               $\beta'(n - \gamma(m)) = 0$; i.e.,
               $n - \gamma(m) \in \Ker(\beta') = \im(\alpha')$, and
               thus, $n = \alpha'(t') + \gamma(m)$ for some $t' \in N'$. Since
               $t' \in N'$, we have that
               $t' + \im(\gamma') \in \Coker(\gamma')$. So
               \begin{align*}
                  y &= n + \im(\gamma) \\
                    &= [\alpha'(t') + \gamma(m)] + \im(\gamma) \\
                    &= \alpha'(t') + \im(\gamma) &[\text{Since }\gamma(m)
                          \in \im(\gamma)] \\
                    &= \varphi'(t' + \im(\gamma')) \in \im(\varphi'),
               \end{align*}
               so that $\im(\varphi') \supseteq \Ker(\psi')$.
      \end{itemize}
      Conclude that $\im(\varphi') = \Ker(\psi')$. That is, the
      Snake Lemma sequence is exact at $\Coker(\gamma)$. \qed
   \item[14.]  Prove that if the two rows in \eqref{5_1} are short exact, then
               \eqref{5_2} also has 0s on the outside.

      \textbf{Proof.} Suppose that the two rows in \eqref{5_1} are short exact.
      To show that \eqref{5_2} also has 0s on the outside, it suffices to show
      that $\varphi$ is injective and that $\psi'$ is surjective.

      \begin{itemize}
         \item \textbf{injectivity of $\varphi$.} Suppose that
               $\varphi(x) = \varphi(y)$, for some $x, y \in \Ker(\gamma')$. So
               \begin{align*}
                  0 &= \varphi(x - y) &[\varphi(x) = \varphi(y)] \\
                    &= \alpha(x - y). &[\varphi = \alpha|_{_{\Ker(\gamma')}}]
               \end{align*}
               But $\alpha$ is injective because the sequence $0 \rightarrow M'
               \stackrel{\alpha}{\longrightarrow} M
               \stackrel{\beta}{\longrightarrow}M'' \rightarrow 0$ is exact, so
               we conclude that $x - y = 0$, so that $x = y$. That is, $\varphi$
               is injective.
         \item \textbf{surjectivity of $\psi'$.} Let $c'' \in \Coker(\gamma'')$.
               So $c'' = n'' + \im(\gamma'')$, for some $n'' \in N''$. Since
               $0 \rightarrow N'\stackrel{\alpha'}{\longrightarrow} N
               \stackrel{\beta'}{\longrightarrow}N'' \rightarrow 0$ is exact, it
               follows that $\beta'$ is surjective. So there exists $n \in N$
               such that $\beta'(n) = n''$. Particularly, we have that
               $n + \im(\gamma) \in \Coker(\gamma)$. Since
               $\psi'(n + \im(\gamma)) = \beta'(n) + \im(\gamma'') =
                n'' + \im(\gamma'') = c''$, it follows that $\psi'$ is
               surjective.
      \end{itemize}
      So conclude that \eqref{5_2} also has 0s on the outside. \qed
\end{enumerate}
\end{document}
