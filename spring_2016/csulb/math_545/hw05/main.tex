\documentclass[9pt]{article}

\usepackage{amssymb}
\usepackage{amsmath}
\usepackage{amsfonts}
\usepackage{comment}
\usepackage{fancyhdr}
\usepackage{mathrsfs}
\usepackage{enumitem}
%\usepackage[retainorgcmds]{IEEEtrantools}

\everymath{\displaystyle}

\usepackage{tikz}

\voffset = -50pt
%\textheight = 700pt
\addtolength{\textwidth}{60pt}
\addtolength{\evensidemargin}{-30pt}
\addtolength{\oddsidemargin}{-30pt}
%\setlength{\headheight}{44pt}

\pagestyle{fancy}
\fancyhf{} % clear all fields
\fancyhead[R]{%
  \scshape
  \begin{tabular}[t]{@{}r@{}}
  MATH 545, Spring 2016\\Section 1 (9515)\\
  HW \#5, DUE: 2016, February 17
  \end{tabular}}
\fancyhead[L]{%
  \scshape
  \begin{tabular}[t]{@{}r@{}}
  JOSEPH OKONOBOH\\Mathematics\\Cal State Long Beach
  \end{tabular}}
\fancyfoot[C]{\thepage}

\newcommand{\qed}{\hfill \ensuremath{\Box}}


\newcommand*\circled[1]{\tikz[baseline=(char.base)]{
            \node[shape=circle,draw,inner sep=2pt] (char) {#1};}}


\newcommand{\cyc}[1]{\langle #1 \rangle}
\newcommand{\Z}{\mathbb{Z}}
\newcommand{\I}{\mathbb{I}}
\newcommand{\F}{\mathbb{F}}
\newcommand{\M}{\mathbb{M}}
\newcommand{\R}{\mathbb{R}}
\newcommand{\Q}{\mathbb{Q}}
\newcommand{\D}{\displaystyle}
\DeclareTextCommand{\_}{OT1}{%
  \leavevmode \kern.06em\vbox{\hrule width.6em}}
%\setcounter{section}{-1}

\begin{document}
\begin{enumerate}
%%%%%%%%%%%%%%%%%%%%%%%%%%%%%%%%%%%%%%%13%%%%%%%%%%%%%%%%%%%%%%%%%%%%%%%%%%%%%%%
   \item[13.]  Show that the Snake Lemma sequence is exact at $C$.

      \textbf{Proof.} It suffices to show that
      $\text{im}(\phi') = \text{ker}(\psi')$. So we will show each set is
      contained in the other.
      \begin{itemize}
         \item ($\subseteq$): Let $x \in \text{im}(\phi')$. Then
               $x = \phi'(n' + \text{im}(\gamma'))$, for some $n' \in N'$. So
               \begin{align*}
                  \psi'(x) &= \psi'(\phi'(n' + \text{im}(\gamma'))) \\
                     &= \psi'(\alpha'(n') + \text{im}(\gamma)) \\
                     &= \beta'(\alpha'(n')) + \text{im}(\gamma'') \\
                     &= 0 + \text{im}(\gamma'')
                        &[\text{im}(\alpha') = \text{ker}(\beta')] \\
                     &= \text{im}(\gamma'') \\
                     &= 0,
               \end{align*}
               so that $x \in \text{ker}(\psi')$. Thus 
               $\text{im}(\phi') \subseteq \text{ker}(\psi')$.
         \item ($\supseteq$): Let $y \in \text{ker}(\psi')$. Since
               $\text{ker}(\psi') \subseteq C$, it follows that $y \in C$, so
               $y = n + \text{im}(\gamma)$ for some $n \in N$. Now since $y$ is
               in the kernel of $\psi'$, we have that
               $$0 = \text{im}(\gamma'') = \psi'(y) =
                 \psi'(n + \text{img}(\gamma)) =
                 \beta'(n) + \text{im}(\gamma'').$$
               From the equalities above, we have
               $\beta'(n) + \text{im}(\gamma'') = \text{im}(\gamma'')$, and this
               implies that $\beta'(n) \in \text{im}(\gamma'')$. So
               $\beta'(n) = \gamma''(m'')$ for some $m'' \in M''$. Since the
               horizontal sequence is exact at $M''$, it follows that $\beta$ is
               surjective, so $\beta(m) = m''$ for some $m \in M$. Thus
               $\beta'(n) = \gamma''(m'') = \gamma''(\beta(m)) =
                \beta'(\gamma(m))$, where, in last equality, we used the
               commutativity of the original diagram. Hence
               $\beta'(n) = \beta'(\gamma(m))$, so that
               $\beta'(n - \gamma(m)) = 0$; i.e.,
               $n - \gamma(m) \in \text{ker}(\beta') = \text{im}(\alpha')$, and
               thus, $n = \alpha'(t') + \gamma(m)$ for some $t' \in N'$. Since
               $t' \in N'$, we have that $t' + \text{im}(\gamma') \in C'$. So
               \begin{align*}
                  y &= n + \text{im}(\gamma) \\
                    &= [\alpha'(t') + \gamma(m)] + \text{im}(\gamma) \\
                    &= \alpha'(t') + \text{im}(\gamma) &[\text{Since }\gamma(m)
                          \in \text{im}(\gamma)] \\
                    &= \phi'(t' + \text{im}(\gamma')) \in \text{im}(\phi'),
               \end{align*}
               so that $\text{im}(\phi') \supseteq \text{ker}(\psi')$.
      \end{itemize}
      Conclude that $\text{im}(\phi') = \text{ker}(\psi')$. That is, the Snake
      Lemma sequence is exact at $C$. \qed
   \item[14.]  Prove that if the two rows in Fig 1. are short exact, then the
               long exact sequence in the Snake Lemma also has 0s on the
               outside.

      \textbf{Proof.} Suppose that the two rows in Fig 1 are short exact. To
      show that the long exact sequence in the Snake Lemma also has 0s on the
      outside, it suffices to show that $\phi$ is injective and that $\psi'$ is
      surjective.

      \begin{itemize}
         \item \textbf{injectivity of $\phi$.} Suppose that $\phi(x) = \phi(y)$,
               for some $x, y \in \text{ker}(\gamma')$. So
               \begin{align*}
                  0 &= \phi(x - y) &[\phi(x) = \phi(y)] \\
                    &= \alpha(x - y). &[\phi = \alpha|_{_{K'}}]
               \end{align*}
               But $\alpha$ is injective because the sequence $0 \rightarrow M'
               \stackrel{\alpha}{\longrightarrow} M
               \stackrel{\beta}{\longrightarrow}M'' \rightarrow 0$ is exact, so
               we conclude that $x - y = 0$, so that $x = y$. That is, $\phi$ is
               injective.
         \item \textbf{surjectivity of $\psi'$.} Let $c'' \in C''$. So
               $c'' = n'' + \text{im}(\gamma'')$, for some $n'' \in N''$. Since
               $0 \rightarrow N'\stackrel{\alpha'}{\longrightarrow} N
               \stackrel{\beta'}{\longrightarrow}N'' \rightarrow 0$ is exact, it
               follows that $\beta'$ is surjective. So there exists $n \in N$
               such that $\beta'(n) = n''$. Particularly, we have that
               $n + \text{im}(\gamma) \in C$. Since
               $\psi'(n + \text{im}(\gamma)) = \beta'(n) + \text{im}(\gamma'') =
                n'' + \text{im}(\gamma'') = c''$, it follows that $\psi'$ is
               surjective.
      \end{itemize}
\end{enumerate}
\end{document}
