\documentclass[9pt]{article}

\usepackage{amssymb}
\usepackage{amsmath}
\usepackage{amsfonts}
\usepackage{comment}
\usepackage{fancyhdr}
\usepackage{mathrsfs}
\usepackage{amscd}
%\usepackage[retainorgcmds]{IEEEtrantools}

\everymath{\displaystyle}

\usepackage{tikz}

\voffset = -50pt
%\textheight = 700pt
\addtolength{\textwidth}{60pt}
\addtolength{\evensidemargin}{-30pt}
\addtolength{\oddsidemargin}{-30pt}
%\setlength{\headheight}{44pt}

\pagestyle{fancy}
\fancyhf{} % clear all fields
\fancyhead[R]{%
  \scshape
  \begin{tabular}[t]{@{}r@{}}
  MATH 545, Spring 2016\\Section 1 (9515)\\
  HW \#7, DUE: 2016, February 24
  \end{tabular}}
\fancyhead[L]{%
  \scshape
  \begin{tabular}[t]{@{}r@{}}
  JOSEPH OKONOBOH\\Mathematics\\Cal State Long Beach
  \end{tabular}}
\fancyfoot[C]{\thepage}

\newcommand{\qed}{\hfill \ensuremath{\Box}}


\newcommand*\circled[1]{\tikz[baseline=(char.base)]{
            \node[shape=circle,draw,inner sep=2pt] (char) {#1};}}


\newcommand{\cyc}[1]{\langle #1 \rangle}
\newcommand{\Z}{\mathbb{Z}}
\newcommand{\I}{\mathbb{I}}
\newcommand{\F}{\mathbb{F}}
\newcommand{\M}{\mathbb{M}}
\newcommand{\R}{\mathbb{R}}
\newcommand{\Q}{\mathbb{Q}}
\newcommand{\Ker}{\text{Ker}}
\newcommand{\Coker}{\text{Coker}}
\newcommand{\im}{\text{Im}}
\newcommand{\D}{\displaystyle}
\DeclareTextCommand{\_}{OT1}{%
  \leavevmode \kern.06em\vbox{\hrule width.6em}}
%\setcounter{section}{-1}

\begin{document}
\begin{enumerate}
%%%%%%%%%%%%%%%%%%%%%%%%%%%%%%%%%%%%%%%17%%%%%%%%%%%%%%%%%%%%%%%%%%%%%%%%%%%%%%%
   \item[17.]  Let $M, N$, and $P$ be $R$-modules,
               $\gamma \in \text{Hom}(M, \text{Hom}(N, P))$, and define
               $\tilde{\gamma} : M \times N \rightarrow P$ by
               $(m, n) \mapsto [\gamma(m)](n)$. Show that
               $\tilde{\gamma} \in \text{Bil}(M, N ; P)$.
               
      \textbf{Proof.} Let $m' \in M$ and $n' \in N$. It suffices to show that
      the maps $\alpha : N \rightarrow P$, defined by
      $n \mapsto \tilde{\gamma}(m', n)$, and $\beta : M \rightarrow P$, defined
      by $m \mapsto \tilde{\gamma}(m, n')$, are $R$-linear. That is, it suffices
      to show that
      $$\alpha(xn_1+yn_2)=x\alpha(n_1) + y\alpha(n_2) \text{ and }
        \beta(xm_1+ym_2)=x\beta(m_1) + y\beta(m_2),$$
      for all  $x, y \in R$, $n_1, n_2 \in N$ and $m_1, m_2 \in M$. The
      equalities above follow immediately because for $x, y \in R$,
      $n_1, n_2 \in N$ and $m_1, m_2 \in M$, we have that
      \begin{align*}
         \alpha(xn_1 + yn_2) &= \tilde{\gamma}(m', xn_1+yn_2) \\
            &= [\gamma(m')](xn_1+yn_2) \\
            &= x([\gamma(m')](n_1)) + y([\gamma(m')](n_2))
               &[\gamma(m') \text{ is $R$-linear}] \\
            &= x[\tilde{\gamma}(m', n_1)]+ y[\tilde{\gamma}(m', n_2)]\\
                     &= x\alpha(n_1) + y\alpha(n_2)
      \end{align*}
      and
      \begin{align*}
         \beta(xm_1 + ym_2) &= \tilde{\gamma}(xm_1+ym_2, n') \\
            &= [\gamma(xm_1+ym_2)](n') \\
            &= [x\gamma(m_1)+y\gamma(m_2)](n')
               &[\gamma \text{ is $R$-linear}] \\
            &= x([\gamma(m_1)](n'))+y([\gamma(m_2)](n')) \\
            &= x\tilde{\gamma}(m_1, n')+y\tilde{\gamma}(m_2, n') \\
            &= x\beta(m_1) + y\beta(m_2).
      \end{align*}
      Thus $\tilde{\gamma} \in \text{Bil}(M, N ; P)$. \qed
\end{enumerate}
\end{document}
