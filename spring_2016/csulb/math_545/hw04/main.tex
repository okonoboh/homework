\documentclass[9pt]{article}

\usepackage{amssymb}
\usepackage{amsmath}
\usepackage{amsfonts}
\usepackage{comment}
\usepackage{fancyhdr}
\usepackage{mathrsfs}
\usepackage{enumitem}
%\usepackage[retainorgcmds]{IEEEtrantools}

\everymath{\displaystyle}

\usepackage{tikz}

\voffset = -50pt
%\textheight = 700pt
\addtolength{\textwidth}{60pt}
\addtolength{\evensidemargin}{-30pt}
\addtolength{\oddsidemargin}{-30pt}
%\setlength{\headheight}{44pt}

\pagestyle{fancy}
\fancyhf{} % clear all fields
\fancyhead[R]{%
  \scshape
  \begin{tabular}[t]{@{}r@{}}
  MATH 545, Spring 2016\\Section 1 (9515)\\
  HW \#4, DUE: 2016, February 15
  \end{tabular}}
\fancyhead[L]{%
  \scshape
  \begin{tabular}[t]{@{}r@{}}
  JOSEPH OKONOBOH\\Mathematics\\Cal State Long Beach
  \end{tabular}}
\fancyfoot[C]{\thepage}

\newcommand{\qed}{\hfill \ensuremath{\Box}}


\newcommand*\circled[1]{\tikz[baseline=(char.base)]{
            \node[shape=circle,draw,inner sep=2pt] (char) {#1};}}


\newcommand{\cyc}[1]{\langle #1 \rangle}
\newcommand{\Z}{\mathbb{Z}}
\newcommand{\I}{\mathbb{I}}
\newcommand{\F}{\mathbb{F}}
\newcommand{\M}{\mathbb{M}}
\newcommand{\R}{\mathbb{R}}
\newcommand{\Q}{\mathbb{Q}}
\newcommand{\D}{\displaystyle}
\DeclareTextCommand{\_}{OT1}{%
  \leavevmode \kern.06em\vbox{\hrule width.6em}}
%\setcounter{section}{-1}

\begin{document}
\begin{enumerate}
%%%%%%%%%%%%%%%%%%%%%%%%%%%%%%%%%%%%%%%11%%%%%%%%%%%%%%%%%%%%%%%%%%%%%%%%%%%%%%%
   \item[11.]  Let $0 \rightarrow M' \stackrel{\alpha}{\longrightarrow} M
               \stackrel{\beta}{\longrightarrow} M'' \rightarrow 0$ be a
               short exact sequence, and $N \subset M$ a submodule. Set
               $N' := \alpha^{-1}(N)$ and $N'' := \beta(N)$. Prove that the
               induced sequence
               $$0 \rightarrow N' \rightarrow N \rightarrow N'' \rightarrow 0.$$

      \textbf{Proof.} By definition, we have $N' \subset M'$ and
      $N'' \subset M''$ are submodulues, so it suffices to show that
      $0 \rightarrow N' \stackrel{\alpha'}{\longrightarrow} N
      \stackrel{\beta'}{\longrightarrow}N'' \rightarrow 0$ is exact, where
      $\alpha' = \alpha|_{_{N'}}$ and $\beta' = \beta|_{_{M}}$.

      \begin{itemize}
         \item \textbf{exactness at $N'$.} It suffices to show that
               $\text{ker}(\alpha') = \{0\}$. Let $x \in \text{ker}(\alpha')$.
               So we have $\alpha'(x) = 0 = \alpha(x)$. That is,
               $x \in \text{ker}(\alpha) = \{0\}$ since the original sequence is
               exact at $M'$. And thus, $x = 0$, so that
               $\text{ker}(\alpha') = \{0\}$, and we conclude that the induced
               sequence is exact at $N'$.
         \item \textbf{exactness at $N$.} It suffices to show that
               $\text{im}(\alpha') = \text{ker}(\beta')$.
               \begin{itemize}
                  \item ($\subseteq$): Let $y \in \text{im}(\alpha')$. Then
                        $y = \alpha'(n')$ for some $n' \in N'$. So
                        $$y = \alpha'(n') = \alpha(n') \in \text{im}(\alpha) =
                         \text{ker}(\beta),$$
                        where the last equality follows from the exactness of
                        the original sequence at $M$. Thus $\beta(y) = 0$. Since
                        $N' = \alpha^{-1}(N)$, we have that
                        $y = \alpha(n') \in N$, so $y \in \text{ker}(\beta')$.
                        Thus $\text{im}(\alpha') \subseteq \text{ker}(\beta')$.
                  \item ($\supseteq$): Since $N' = \alpha^{-1}(N)$, it follows
                        that
                        $$\text{im}(\alpha') = \alpha'(N') = \alpha(N') = N
                          \supseteq \text{ker}(\beta').$$
               \end{itemize}
               So we conclude that $\text{im}(\alpha') = \text{ker}(\beta')$;
               i.e., the induced sequence is exact at $N$.
         \item \textbf{exactness at $N''$.} It suffices to show that
               $\text{im}(\beta') = N''$; that is, it suffices to show that
               $\beta'$ is surjective. To that end, we let $n'' \in N''$. Since
               $N'' = \beta(N)$, it follows that $n'' = \beta(n)$ for some
               $n \in N$. That is, $\beta(n) = \beta'(n)$, so that $\beta'$ is
               surjective. Hence, the induced sequence is exact at $N''$.
      \end{itemize}
      Conclude that the induced sequence is short exact. \qed
   \item[12.]  Given a short exact sequence
               $$0 \rightarrow M' \stackrel{\alpha}{\longrightarrow} M
                 \stackrel{\beta}{\longrightarrow}M'' \rightarrow 0.$$
               Show that if there is a homomorphism $\sigma : M''\rightarrow M$,
               with $\beta \circ \sigma = i_{M''}$, then the exact sequence
               splits.

      \textbf{Proof.} Suppose there exists an homomorphism
      $\sigma : M''\rightarrow M$ such that $\beta \circ \sigma = i_{M''}$ and
      consider $\varphi : M' \oplus M'' \rightarrow M$, defined by
      $(m', m'') \mapsto \alpha(m') + \sigma(m'')$. To show that the sequence
      splits, it suffices to show that $\varphi$ is an isomorphism. First, we
      will show that $\varphi$ is an $R$-module homomorphism (where all the
      modules are over some ring $R$). So let $a, b \in M' \oplus M''$. Then
      $a = (x', x'')$ and $b = (y', y'')$ for some $x', y' \in M'$ and
      $x'', y'' \in M''$. 
      \begin{itemize}
         \item \textbf{additivity.} It follows that $\varphi$ is an homomorphism
               of groups because
               \begin{align*}
                  \varphi(a + b) &= \varphi((x', x'') + (y', y'')) \\
                     &= \varphi((x' + y', x'' + y'')) \\
                     &= \alpha(x' + y') + \sigma(x'' + y'')
                        &[\text{Definition}] \\
                     &= \alpha(x') + \alpha(y') + \sigma(x'') + \sigma(y'')
                        &[\alpha \text{ and } \sigma\text{ are homomorphisms}]\\
                     &= \alpha(x') + \sigma(x'') + \alpha(y') + \sigma(y'') \\
                     &= \varphi((x', x'')) + \varphi((y', y'')) \\
                     &= \varphi(a) + \varphi(b).
               \end{align*}
         \item \textbf{show $\varphi(ra) = r\varphi(a)$ for all $r \in R$}. Let
               $r \in R$. Then it follows that
               \begin{align*}
                  \varphi(ra) &= \varphi(r(x', x'')) \\
                     &= \varphi((rx', rx'')) \\
                     &= \alpha(rx') + \sigma(rx'') &[\text{Definition}] \\
                     &= r\alpha(x') + r\sigma(x'') &[\alpha \text{ and }
                        \sigma\text{ are $R$-homomorphisms}] \\
                     &= r(\alpha(x') + \sigma(x'')) \\
                     &= r\varphi((x', x'')) \\
                     &= r\varphi(a),
               \end{align*}
               so that $\varphi(ra) = r\varphi(a)$ for all $r \in R$.
      \end{itemize}
      Conclude that $\varphi$ is an homomorphism. Next we will show that
      $\varphi$ is bijective.
      \begin{itemize}
         \item \textbf{surjectivity.} Let $m \in M$. Using the fact that
               $\beta\circ\sigma$ is the identity on $M''$ and that $\beta$ is a
               homomorphism, we have that
               $$\beta(m - \sigma(\beta(m))) =
                 \beta(m) - \beta(\sigma(\beta(m))) = \beta(m) - \beta(m) = 0,$$
               so that $m - \sigma(\beta(m)) \in\text{ker}(\beta)=
               \text{im}(\alpha)$, where we have used the exactness of the
               sequence in the last equality. So there exists $m' \in M'$ such
               that $\alpha(m') = m - \sigma(\beta(m))$. Now let
               $m'' = \beta(m) \in M''$, so that $(m', m'') \in M' \oplus M''$.
               It follows immediately that $\varphi$ is surjective because
               $$\varphi((m', m'')) = \alpha(m') + \sigma(m'') =
                 m - \sigma(\beta(m)) + \sigma(\beta(m)) = m.$$
         \item \textbf{injectivity.} Suppose $\varphi(p) = \varphi(q)$ for
               some $p, q \in M' \oplus M''$. Then $p = (p', p'')$ and
               $q = (q', q'')$ for some $p', q' \in M'$ and $p'', q'' \in M''$.
               Now
               $\varphi(p) = \varphi(q)$ implies that
               $$\alpha(p') + \sigma(p'') = \varphi(p) =
                 \varphi(q) = \alpha(q') + \sigma(q''),$$
               so that $\alpha(p' - q') = \sigma(q'' - p'')$. Thus
               $$0 = \beta(\alpha(p' - q')) =
                 \beta(\sigma(q'' - p'')) = q'' - p'',$$
               where the first equality follows from the fact that
               $\text{im}(\alpha) = \text{ker}(\beta)$, so that
               $\beta \circ \alpha$ is the zero map on $M'$; thus $q'' = p''$.
               Now since $\alpha(p') + \sigma(p'') = \alpha(q') + \sigma(q'')$
               and $q'' = p''$, it follows that $\alpha(p') = \alpha(q')$. Since
               the given sequence is exact at $M'$, it follows that $\alpha$ is
               injective, so that $p' = q'$. Thus $p = q$ and we conclude that
               $\varphi$ is injective.
      \end{itemize}
      Thus $\varphi$ is bijective.
\end{enumerate}
\end{document}
