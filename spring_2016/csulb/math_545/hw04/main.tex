\documentclass[9pt]{article}

\usepackage{amssymb}
\usepackage{amsmath}
\usepackage{amsfonts}
\usepackage{comment}
\usepackage{fancyhdr}
\usepackage{mathrsfs}
\usepackage{enumitem}
%\usepackage[retainorgcmds]{IEEEtrantools}

\everymath{\displaystyle}

\usepackage{tikz}

\voffset = -50pt
%\textheight = 700pt
\addtolength{\textwidth}{60pt}
\addtolength{\evensidemargin}{-30pt}
\addtolength{\oddsidemargin}{-30pt}
%\setlength{\headheight}{44pt}

\pagestyle{fancy}
\fancyhf{} % clear all fields
\fancyhead[R]{%
  \scshape
  \begin{tabular}[t]{@{}r@{}}
  MATH 545, Spring 2016\\Section 1 (9515)\\
  HW \#4, DUE: 2016, February 15
  \end{tabular}}
\fancyhead[L]{%
  \scshape
  \begin{tabular}[t]{@{}r@{}}
  JOSEPH OKONOBOH\\Mathematics\\Cal State Long Beach
  \end{tabular}}
\fancyfoot[C]{\thepage}

\newcommand{\qed}{\hfill \ensuremath{\Box}}


\newcommand*\circled[1]{\tikz[baseline=(char.base)]{
            \node[shape=circle,draw,inner sep=2pt] (char) {#1};}}


\newcommand{\cyc}[1]{\langle #1 \rangle}
\newcommand{\Z}{\mathbb{Z}}
\newcommand{\I}{\mathbb{I}}
\newcommand{\F}{\mathbb{F}}
\newcommand{\M}{\mathbb{M}}
\newcommand{\R}{\mathbb{R}}
\newcommand{\Q}{\mathbb{Q}}
\newcommand{\D}{\displaystyle}
\DeclareTextCommand{\_}{OT1}{%
  \leavevmode \kern.06em\vbox{\hrule width.6em}}
%\setcounter{section}{-1}

\begin{document}
\begin{enumerate}
%%%%%%%%%%%%%%%%%%%%%%%%%%%%%%%%%%%%%%%11%%%%%%%%%%%%%%%%%%%%%%%%%%%%%%%%%%%%%%%
   \item[11.]  Let $0 \rightarrow M' \stackrel{\alpha}{\longrightarrow} M
               \stackrel{\beta}{\longrightarrow} M'' \rightarrow 0$ be a
               short exact sequence, and $N \subset M$ a submodule. Set
               $N' := \alpha^{-1}(N)$ and $N'' := \beta(N)$. Prove that the
               induced sequence
               $$0 \rightarrow N' \rightarrow N \rightarrow N'' \rightarrow 0.$$

      \textbf{Proof.} By definition, we have $N' \subset M'$ and
      $N'' \subset M''$ are submodulues, so it suffices to show that
      $0 \rightarrow N' \stackrel{\alpha'}{\longrightarrow} N
      \stackrel{\beta'}{\longrightarrow}N'' \rightarrow 0$ is exact, where
      $\alpha' = \alpha|_{_{N'}}$ and $\beta' = \beta|_{_{M}}$.

      \begin{itemize}
         \item \textbf{exactness at $N'$.} It suffices to show that
               $\text{ker}(\alpha') = \{0\}$. Let $x \in \text{ker}(\alpha')$.
               So we have $\alpha'(x) = 0 = \alpha(x)$. That is,
               $x \in \text{ker}(\alpha) = \{0\}$ since the original sequence is
               exact at $M'$. And thus, $x = 0$, so that
               $\text{ker}(\alpha') = \{0\}$, and we conclude that the induced
               sequence is exact at $N'$.
         \item \textbf{exactness at $N$.} It suffices to show that
               $\text{im}(\alpha') = \text{ker}(\beta')$.
               \begin{itemize}
                  \item ($\subseteq$): Let $y \in \text{im}(\alpha')$. Then
                        $y = \alpha'(n')$ for some $n' \in N'$. So
                        $$y = \alpha'(n') = \alpha(n') \in \text{im}(\alpha) =
                         \text{ker}(\beta),$$
                        where the last equality follows from the exactness of
                        the original sequence at $M$. Thus $\beta(y) = 0$. Since
                        $N' = \alpha^{-1}(N)$, we have that
                        $y = \alpha(n') \in N$, so $y \in \text{ker}(\beta')$.
                        Thus $\text{im}(\alpha') \subseteq \text{ker}(\beta')$.
                  \item ($\supseteq$): Since $N' = \alpha^{-1}(N)$, it follows
                        that
                        $$\text{im}(\alpha') = \alpha'(N') = \alpha(N') = N
                          \supseteq \text{ker}(\beta').$$
               \end{itemize}
               So we conclude that $\text{im}(\alpha') = \text{ker}(\beta')$;
               i.e., the induced sequence is exact at $N$.
         \item \textbf{exactness at $N''$.} It suffices to show that
               $\text{im}(\beta') = N''$; that is, it suffices to show that
               $\beta'$ is surjective. To that end, we let $n'' \in N''$. Since
               $N'' = \beta(N)$, it follows that $n'' = \beta(n)$ for some
               $n \in N$. That is, $\beta(n) = \beta'(n)$, so that $\beta'$ is
               surjective. Hence, the induced sequence is exact at $N''$.
      \end{itemize}
      Conclude that the induced sequence is short exact. \qed
   \item[12.]  Given a short exact sequence
               $$0 \rightarrow M' \stackrel{\alpha}{\longrightarrow} M
                 \stackrel{\beta}{\longrightarrow}M'' \rightarrow 0.$$
               Show that if there is a homomorphism $\sigma : M''\rightarrow M$,
               with $\beta \circ \sigma = i_{M''}$, then the exact sequence
               splits.

      \textbf{Proof.} Suppose there exists an homomorphism
      $\sigma : M''\rightarrow M$ such that $\beta \circ \sigma = i_{M''}$ and
      consider $\varphi : M' \oplus M'' \rightarrow M$, defined by
      $(m', m'') \mapsto \alpha(m') + \sigma(m'')$.
\end{enumerate}
\end{document}
