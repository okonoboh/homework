\documentclass[9pt]{article}

\usepackage{amssymb}
\usepackage{amsmath}
\usepackage{amsfonts}
\usepackage{comment}
\usepackage{fancyhdr}
\usepackage{mathrsfs}
\usepackage{enumitem}
%\usepackage[retainorgcmds]{IEEEtrantools}

\everymath{\displaystyle}

\usepackage{tikz}

\voffset = -50pt
%\textheight = 700pt
\addtolength{\textwidth}{60pt}
\addtolength{\evensidemargin}{-30pt}
\addtolength{\oddsidemargin}{-30pt}
%\setlength{\headheight}{44pt}

\pagestyle{fancy}
\fancyhf{} % clear all fields
\fancyhead[R]{%
  \scshape
  \begin{tabular}[t]{@{}r@{}}
  MATH 545, Spring 2016\\Section 1 (9515)\\
  HW \#4, DUE: 2016, February 15
  \end{tabular}}
\fancyhead[L]{%
  \scshape
  \begin{tabular}[t]{@{}r@{}}
  JOSEPH OKONOBOH\\Mathematics\\Cal State Long Beach
  \end{tabular}}
\fancyfoot[C]{\thepage}

\newcommand{\qed}{\hfill \ensuremath{\Box}}


\newcommand*\circled[1]{\tikz[baseline=(char.base)]{
            \node[shape=circle,draw,inner sep=2pt] (char) {#1};}}


\newcommand{\cyc}[1]{\langle #1 \rangle}
\newcommand{\Z}{\mathbb{Z}}
\newcommand{\I}{\mathbb{I}}
\newcommand{\F}{\mathbb{F}}
\newcommand{\M}{\mathbb{M}}
\newcommand{\R}{\mathbb{R}}
\newcommand{\Q}{\mathbb{Q}}
\newcommand{\D}{\displaystyle}
\DeclareTextCommand{\_}{OT1}{%
  \leavevmode \kern.06em\vbox{\hrule width.6em}}
%\setcounter{section}{-1}

\begin{document}
\begin{enumerate}
%%%%%%%%%%%%%%%%%%%%%%%%%%%%%%%%%%%%%%%11%%%%%%%%%%%%%%%%%%%%%%%%%%%%%%%%%%%%%%%
   \item Find the power series representation, interval of convergence, radius
         of convergence, and center of $f(x) = \frac{x}{(3-2x)^3}$.
   
         \textbf{Solution.} We know that
         \begin{equation} \label{1_1}
            \frac{1}{1-x} = \sum_{n=0}^\infty x^n, \quad |x| < 1.
         \end{equation}
         Take the derivative of both sides of \eqref{1_1} twice to get
         
         \begin{equation} \label{1_2}
            \frac{2}{(1-x)^3} = \sum_{n=2}^\infty n(n-1)x^{n-2}, \quad |x| < 1,
         \end{equation}
         and rewrite equation \eqref{1_2} to get
         
         \begin{equation} \label{1_3}
            \frac{1}{(1-x)^3} = \frac{1}{2}\sum_{n=2}^\infty n(n-1)x^{n-2}, \quad |x| < 1.
         \end{equation}
         Now
         \begin{align*}
            f(x) &= \frac{x}{(3-2x)^3} \\
               &= x \cdot \frac{1}{(3-2x)^3} \\
               &= x \cdot \frac{1}{(3\left(1-\frac{2}{3}x\right))^3} \\
               &= \frac{x}{27} \cdot \frac{1}{(1-\frac{2}{3}x)^3} \\
               &= \frac{x}{27} \cdot \frac{1}{2}\sum_{n=2}^\infty n(n-1)\left(\frac{2}{3}x\right)^{n-2}, \quad \left|\frac{2}{3}x\right| < 1 \\
               &= \frac{x}{3^3} \cdot \frac{1}{2}\sum_{n=2}^\infty n(n-1)\frac{2^{n-2}}{3^{n-2}}x^{n-2}, \quad |x| < \frac{3}{2}\\
               &= \sum_{n=2}^\infty n(n-1)\frac{2^{n-3}}{3^{n+1}}x^{n-1}, \quad |x| < \frac{3}{2}.
         \end{align*}
\end{enumerate}
\end{document}
