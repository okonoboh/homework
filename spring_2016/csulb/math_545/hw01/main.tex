\documentclass[9pt]{article}

\usepackage{amssymb}
\usepackage{amsmath}
\usepackage{amsfonts}
\usepackage{comment}
\usepackage{fancyhdr}
\usepackage{mathrsfs}
\usepackage{enumitem}
%\usepackage[retainorgcmds]{IEEEtrantools}

\everymath{\displaystyle}

\usepackage{tikz}

\voffset = -50pt
%\textheight = 700pt
\addtolength{\textwidth}{60pt}
\addtolength{\evensidemargin}{-30pt}
\addtolength{\oddsidemargin}{-30pt}
%\setlength{\headheight}{44pt}

\pagestyle{fancy}
\fancyhf{} % clear all fields
\fancyhead[R]{%
  \scshape
  \begin{tabular}[t]{@{}r@{}}
  MATH 545, Spring 2016\\Section 1 (9515)\\
  HW \#1, DUE: 2016, February 01
  \end{tabular}}
\fancyhead[L]{%
  \scshape
  \begin{tabular}[t]{@{}r@{}}
  JOSEPH OKONOBOH\\Mathematics\\Cal State Long Beach
  \end{tabular}}
\fancyfoot[C]{\thepage}

\newcommand{\qed}{\hfill \ensuremath{\Box}}


\newcommand*\circled[1]{\tikz[baseline=(char.base)]{
            \node[shape=circle,draw,inner sep=2pt] (char) {#1};}}


\newcommand{\cyc}[1]{\langle #1 \rangle}
\newcommand{\Z}{\mathbb{Z}}
\newcommand{\I}{\mathbb{I}}
\newcommand{\F}{\mathbb{F}}
\newcommand{\M}{\mathbb{M}}
\newcommand{\R}{\mathbb{R}}
\newcommand{\Q}{\mathbb{Q}}
\newcommand{\D}{\displaystyle}
\DeclareTextCommand{\_}{OT1}{%
  \leavevmode \kern.06em\vbox{\hrule width.6em}}
%\setcounter{section}{-1}

\begin{document}
We shall make use of the following facts from Linear Algebra
\begin{itemize}
   \item \textbf{F1.} A linearly independent set in a finite dimensional vector
         space can be extended to a basis for that vector space.
   \item \textbf{F2.} If $\{v_1, \ldots, v_n\}$ is a basis for a vector space
         $V$ and $w_1, \ldots, w_n$ are elements in a vector space $W$ (both
         over the same field), then there exists a unique linear transformation
         from $V$ to $W$ that maps $v_i$ to $w_i$, for $i = 1, \ldots, n$.
\end{itemize}
Let $V, W$, and $U$ be finite dimensional vector spaces over some field $k$. For 
vector spaces $X, Y$ over $k$, denote the set of all linear transformations from
$X$ to $Y$ by $\mathcal{L}(X, Y)$. Fix a linear transformation
$\phi : V \rightarrow W$, and consider the maps
$$\phi \circ \_ : \mathcal{L}(U, V) \rightarrow \mathcal{L}(U, W)
  \text{ defined by } \psi \mapsto \phi \circ \psi$$
and
$$\_ \circ \phi : \mathcal{L}(W, U) \rightarrow \mathcal{L}(V, U)
  \text{ defined by } \psi \mapsto \psi \circ \phi.$$
\begin{enumerate}
   \item Prove for all $r \in R$, $m \in M$
         $$r(-m) = -r(m) = -(rm)$$
   \item Read and understand def of annihilator and prove statement in AK
         starting with ``Clearly..." Page 14.
   \item Prove $ \oplus N$ is an $R$-module. 
   \item Prove ``is an isomorphism to" is an equivalence relation.
   \item For $\phi : M \rightarrow N$ module hom, ker $\phi$ is a submodule of
         $M$ (ker $\phi$ is kernel as grp homomorphism.)
   \item $M$ module, $N$ submodule. Show
         $$\pi : M \rightarrow M/N$$ is a module homomorphism.
   \item Prove that Hom($M, N$) is an $R$-module.
   \item Find counterexample to the other one. Class : phi maps Z to Z (n to 2n)
         injective. pi : Z to z/2z.
%%%%%%%%%%%%%%%%%%%%%%%%%%%%%%%%%%%%%%%01%%%%%%%%%%%%%%%%%%%%%%%%%%%%%%%%%%%%%%%
   \item Prove that $\phi \circ \_$ and $\_ \circ \phi$ are linear
         transformations.

      \textbf{Proof.} First we will show that $\phi \circ \_$ is a linear
      transformation. Now let $\alpha$ and $\beta$ be linear transformations in
      $\mathcal{L}(U, V)$.
      
      \begin{itemize}
         \item \textbf{additivity.} It suffices to show that
               \begin{equation} \label{1_1}
                  (\phi\circ\_)(\alpha+\beta) = (\phi\circ\_)(\alpha) +
                  (\phi\circ\_)(\beta).
               \end{equation}
               That is, we want to show that
               $$[(\phi\circ\_)(\alpha+\beta)](u) = [(\phi\circ\_)(\alpha)](u) +
                 [(\phi\circ\_)(\beta)](u)$$
               for all $u \in U$. To that end, we let $u$ be an arbitrary
               element in $U$. Hence
               \begin{align*}
                  [(\phi\circ\_)(\alpha+\beta)](u) &=
                     [\phi\circ(\alpha+\beta)](u)
                        &[\text{Definition of }\phi\circ\_ ] \\
                     &= \phi((\alpha+\beta)(u)) \\
                     &= \phi(\alpha(u)+\beta(u)) \\
                     &= \phi(\alpha(u))+\phi(\beta(u))
                        &[\phi \text{ is additive}] \\
                     &= (\phi\circ\alpha)(u)+(\phi\circ\beta)(u) \\
                     &= [(\phi\circ\_)(\alpha)](u) + [(\phi\circ\_)(\beta)](u),
               \end{align*}
               as desired, and we conclude that \eqref{1_1} holds, so that
               $\phi\circ\_$ is additive.
         \item \textbf{homogeneity.} Let $a \in k$. We want to show that
               \begin{equation} \label{1_2}
                  (\phi\circ\_)(a\alpha) = a[(\phi\circ\_)(\alpha)].
               \end{equation}
               The equality in \eqref{1_2} holds if and only if for every
               $v \in V$, we have
               $$[(\phi\circ\_)(a\alpha)](v) = a([(\phi\circ\_)(\alpha)](v)).$$
               So let $v \in V$. It follows that \eqref{1_2} holds because
               \begin{align*}
                  [(\phi\circ\_)(a\alpha)](v) &= [\phi\circ(a\alpha)](v)
                     &[\text{Definition of }\phi\circ\_ ] \\
                     &= \phi((a\alpha)(v)) \\
                     &= \phi(a\alpha(v)) \\
                     &= a\phi(\alpha(v)) 
                        &[\phi\text{ is a linear transformation}] \\
                     &= a((\phi\circ\alpha)(v)) \\
                     &= a([(\phi\circ\_)(\alpha)](v)).
               \end{align*}
      \end{itemize}
      Thus we conclude from above that $\phi\circ\_$ is a linear transformation.
      Now we will show that $\_\circ\phi$ is also a linear transformation. So
      let $\alpha$ and $\beta$ be linear transformations in $\mathcal{L}(W, U)$.
      
      \begin{itemize}
         \item \textbf{additivity.} We want to show that
               \begin{equation} \label{1_3}
                  (\_\circ\phi)(\alpha+\beta) = (\_\circ\phi)(\alpha) +
                  (\_\circ\phi)(\beta).
               \end{equation}
               So it suffices to show that
               $$[(\_\circ\phi)(\alpha+\beta)](v) = [(\_\circ\phi)(\alpha)](v) +
                 [(\_\circ\phi)(\beta)](v)$$
               for all $v \in V$. To that end, we let $v$ be an arbitrary
               element in $V$. Hence
               \begin{align*}
                  [(\_\circ\phi)(\alpha+\beta)](v) &=
                     [(\alpha+\beta)\circ\phi](v)
                        &[\text{Definition of }\_\circ\phi ] \\
                     &= (\alpha + \beta)(\phi(v)) \\
                     &= \alpha(\phi(v)) + \beta(\phi(v)) \\
                     &= (\alpha\circ\phi)(v) + (\beta\circ\phi)(v) \\
                     &= [(\_\circ\phi)(\alpha)](v) + [(\_\circ\phi)(\beta)](v),
               \end{align*}
               as desired, and we conclude that \eqref{1_3} holds, so that
               $\_\circ\phi$ is additive.
         \item \textbf{homogeneity.} Let $a \in k$. We want to show that
               \begin{equation} \label{1_4}
                  (\_\circ\phi)(a\alpha) = a[(\_\circ\phi)(\alpha)].
               \end{equation}
               The equality in \eqref{1_4} holds if and only if for every
               $v \in V$, we have
               $$[(\_\circ\phi)(a\alpha)](v) = a([(\_\circ\phi)(\alpha)](v)).$$
               So let $v \in V$. It follows that \eqref{1_4} holds because
               \begin{align*}
                  [(\_\circ\phi)(a\alpha)](v) &= [(a\alpha)\circ\phi](v)
                     &[\text{Definition of }\_\circ\phi ] \\
                     &= (a\alpha)(\phi(v)) \\
                     &= a(\alpha(\phi(v))) \\
                     &= a((\alpha\circ\phi)(v)) \\
                     &= a([(\_\circ\phi)(\alpha)](v)).
               \end{align*}
      \end{itemize}
      Thus $\phi\circ\_$ is a linear transformation. \qed
%%%%%%%%%%%%%%%%%%%%%%%%%%%%%%%%%%%%%%%02%%%%%%%%%%%%%%%%%%%%%%%%%%%%%%%%%%%%%%%
   \item Show that
         \begin{enumerate}
            \item if $\phi$ is injective, then $\phi\circ\_$ is injective and
                  $\_\circ\phi$ is surjective, and
            \item if $\phi$ is surjective, then $\phi\circ\_$ is surjective and
                  $\_\circ\phi$ is injective.
         \end{enumerate}
         
      \textbf{Solution.} Let $\{v_1, v_2, \ldots, v_n\}$ be a basis for $V$,
      $\{w_1, w_2, \ldots, w_m\}$ a basis for $W$, and
      $\{u_1, u_2, \ldots, u_s\}$ a basis for $U$, where $n, m$, and $s$ are
      positive integers.
      
      \begin{enumerate}
         \item \textbf{Proof.} Suppose that $\phi$ is injective.
               \begin{itemize}
                  \item \textit{Show that $\phi\circ\_$ is injective}. Suppose
                        that
                        $(\phi\circ\_)(\alpha) = (\phi\circ\_)(\beta)$ for some
                        $\alpha, \beta \in \mathcal{L}(U, V)$. That is,
                        $\phi\circ\alpha = \phi\circ\beta$. Let $u \in U$. Since
                        $\phi\circ\alpha = \phi\circ\beta$, it follows that
                        $(\phi\circ\alpha)(u) = (\phi\circ\beta)(u)$; i.e.,
                        $\phi(\alpha(u)) = \phi(\beta(u)$, so that by the
                        injectivity of $\phi$, we have $\alpha(u) = \beta(u)$.
                        Thus $\alpha(u) = \beta(u)$ for all $u \in U$, or
                        equivalently, $\alpha = \beta$, so that $\phi\circ\_$ is
                        injective.
                  \item \textit{Show that $\_\circ\phi$ is surjective}. Let
                        $\alpha \in \mathcal{L}(V, U)$. We want to find
                        $\psi \in \mathcal{L}(W, U)$ so that
                        $\psi\circ\phi = \alpha$. First we will show that
                        $$\phi(v_1), \phi(v_2), \ldots, \phi(v_n)$$
                        are linearly independent in $W$. To that end, suppose
                        that
                        $$k_1\phi(v_1) + \cdots + k_n\phi(v_n) = 0$$
                        for some scalars $k_1, \ldots, k_n \in k$. Using the
                        linear properties of $\phi$, we have that
                        $\phi(k_1v_1 + \cdots + k_nv_n) = 0$. Now $\phi(0) = 0$
                        because $\phi$ is a homomorphism of groups. But since
                        $\phi$ is injective, we conclude that
                        $k_1v_1 + \cdots + k_nv_n = 0$; i.e.,
                        $k_1 = k_2 = \cdots = k_n = 0$ because
                        $v_1, \ldots, v_n$ are linearly independent in $V$. Thus
                        $\phi(v_1), \ldots, \phi(v_n)$ are linearly independent
                        in $W$. Let $w_i = \phi(v_i)$, for $i = 1, \ldots n$.
                        Using \textbf{F1}, we extend the linearly independent
                        set $\{w_1, \ldots, w_n\}$ to a basis
                        $\{w_1, \ldots, w_n, b_1, b_2, \ldots, b_j\}$ for $W$,
                        where $j \ge 0$ and $n + j = m$. Using \textbf{F2}, we
                        let $\psi \in \mathcal{L}(W, U)$ such that
                        $$\psi(w_i) = \alpha(v_i), \text{ for } i = 1, \ldots, n
                          \text{ and } \psi(b_i) = 0, \text{ for }
                          i = 1, \ldots, j.$$
                        It remains to show that $\psi\circ\phi = \alpha$. So let
                        $v \in  V$. Since $\{v_1, \ldots, v_n\}$ spans $V$,
                        there exist scalars $a_1, \ldots, a_n$ in $k$ such that
                        $$v = a_1v_1 + \cdots + a_nv_n.$$
                        Hence
                        \begin{align*}
                           (\psi\circ\phi)(v) &= 
                              (\psi\circ\phi)(a_1v_1 + \cdots + a_nv_n) \\
                              &= \psi(\phi(a_1v_1 + \cdots + a_nv_n)) \\
                              &= \psi(\phi(a_1v_1) + \cdots + \phi(a_nv_n)) \\
                              &= \psi(a_1\phi(v_1) + \cdots + a_n\phi(v_n)) \\
                              &= \psi(a_1w_1 + \cdots + a_nw_n) \\
                              &= a_1\psi(w_1) + \cdots + a_n\psi(w_n) \\
                              &= a_1\alpha(v_1) + \cdots + a_n\alpha(v_n) \\
                              &= \alpha(a_1v_1) + \cdots + \alpha(a_nv_n) \\
                              &= \alpha(a_1v_1 + \cdots + a_nv_n) \\
                              &= \alpha(v),
                        \end{align*}
                        so that $\psi\circ\phi = \alpha$, and thus,
                        $\_\circ\phi$ is surjective.
               \end{itemize} \qed
         \item \textbf{Proof.} Suppose $\phi$ is surjective.
               \begin{itemize}
                  \item \textit{Show that $\phi\circ\_$ is surjective}. Let
                        $\alpha \in \mathcal{L}(U, W)$. We want to find
                        $\psi \in \mathcal{L}(U, V)$ such that
                        $\phi\circ\psi = \alpha$. Since $\phi$ is surjective,
                        there exist $b_1, \ldots, b_s \in V$ such that
                        $\phi(b_i) = \alpha(u_i)$, for $i = 1, \ldots, s$. By
                        \textbf{F2}, we let $\psi \in \mathcal{L}(U, V)$ such
                        that $\psi(u_i) = b_i$, for $i = 1, \ldots, s$. It
                        remains to show that $\phi\circ\psi = \alpha$. Let
                        $u \in U$. Since $\{u_1, \ldots, u_s\}$ spans $U$, we
                        have that
                        $$u = c_1u_1 + \cdots + c_su_s$$
                        for some scalars $c_1, \ldots, c_s \in k$. Thus
                        \begin{align*}
                           (\phi\circ\psi)(u) &= (\phi\circ\psi)(
                              c_1u_1 + \cdots + c_su_s) \\
                              &= \phi(\psi(c_1u_1 + \cdots + c_su_s)) \\
                              &= \phi(\psi(c_1u_1) + \cdots + \psi(c_su_s)) \\
                              &= \phi(c_1\psi(u_1) + \cdots + c_s\psi(u_s)) \\
                              &= \phi(c_1\psi(u_1))+\cdots+\phi(c_s\psi(u_s)) \\
                              &= c_1\phi(\psi(u_1))+\cdots+c_s\phi(\psi(u_s)) \\
                              &= c_1\phi(b_1)+\cdots+c_s\phi(b_s) \\
                              &= c_1\alpha(u_1)+\cdots+c_s\alpha(u_s) \\
                              &= \alpha(c_1u_1)+\cdots+\alpha(c_su_s) \\
                              &= \alpha(c_1u_1+\cdots+c_su_s) \\
                              &= \alpha(u),
                        \end{align*}
                        so that $\phi\circ\psi = \alpha$, and thus,
                        $\phi\circ\_$ is surjective.
                  \item \textit{Show that $\_\circ\phi$ is injective}. Suppose
                        that $(\_\circ\phi)(\alpha) = (\_\circ\phi)(\beta)$ for
                        some $\alpha, \beta \in \mathcal{L}(W, U)$. That is,
                        $\alpha\circ\phi = \beta\circ\phi$. Let $w \in W$. Since
                        $\phi$ is surjective, there exists $v \in V$ such that
                        $\phi(v) = w$. Since $\alpha\circ\phi = \beta\circ\phi$,
                        it follows that
                        $(\alpha\circ\phi)(v) = (\beta\circ\phi)(v)$, so that
                        $\alpha(\phi(v)) = \beta(\phi(v))$; i.e.,
                        $\alpha(w) = \beta(w)$. Since $w$ was arbitrary, we
                        conclude that $\alpha(w) = \beta(w)$ for all $w \in W$,
                        so that $\alpha = \beta$, and thus, $\_\circ\phi$ is
                        injective.
               \end{itemize} \qed     
      \end{enumerate}
\end{enumerate}
\end{document}
