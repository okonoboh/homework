\documentclass[9pt]{article}

\usepackage{amssymb}
\usepackage{amsmath}
\usepackage{amsfonts}
\usepackage{comment}
\usepackage{fancyhdr}
\usepackage{mathrsfs}
\usepackage{enumitem}
%\usepackage[retainorgcmds]{IEEEtrantools}

\everymath{\displaystyle}

\usepackage{tikz}

\voffset = -50pt
%\textheight = 700pt
\addtolength{\textwidth}{60pt}
\addtolength{\evensidemargin}{-30pt}
\addtolength{\oddsidemargin}{-30pt}
%\setlength{\headheight}{44pt}

\pagestyle{fancy}
\fancyhf{} % clear all fields
\fancyhead[R]{%
  \scshape
  \begin{tabular}[t]{@{}r@{}}
  MATH 545, Spring 2016\\Section 1 (9515)\\
  HW \#1, DUE: 2016, February 01
  \end{tabular}}
\fancyhead[L]{%
  \scshape
  \begin{tabular}[t]{@{}r@{}}
  JOSEPH OKONOBOH\\Mathematics\\Cal State Long Beach
  \end{tabular}}
\fancyfoot[C]{\thepage}

\newcommand{\qed}{\hfill \ensuremath{\Box}}


\newcommand*\circled[1]{\tikz[baseline=(char.base)]{
            \node[shape=circle,draw,inner sep=2pt] (char) {#1};}}


\newcommand{\cyc}[1]{\langle #1 \rangle}
\newcommand{\Z}{\mathbb{Z}}
\newcommand{\I}{\mathbb{I}}
\newcommand{\F}{\mathbb{F}}
\newcommand{\M}{\mathbb{M}}
\newcommand{\R}{\mathbb{R}}
\newcommand{\Q}{\mathbb{Q}}
\newcommand{\D}{\displaystyle}
\DeclareTextCommand{\_}{OT1}{%
  \leavevmode \kern.06em\vbox{\hrule width.6em}}
%\setcounter{section}{-1}

\begin{document}
Let $V, W$, and $U$ be finite dimensional vector spaces over some field $k$. For 
vector spaces $X, Y$ over $k$, denote the set of all linear transformations from
$X$ to $Y$ by $\mathcal{L}(X, Y)$. Fix a linear transformation
$\phi : V \rightarrow W$, and consider the maps
$$\_ \circ \phi : \mathcal{L}(W, U) \rightarrow \mathcal{L}(V, U)
  \text{ defined by } \psi \mapsto \psi \circ \phi$$
and
$$\phi \circ \_ : \mathcal{L}(U, V) \rightarrow \mathcal{L}(U, W)
  \text{ defined by } \psi \mapsto \phi \circ \psi.$$
\begin{enumerate}
%%%%%%%%%%%%%%%%%%%%%%%%%%%%%%%%%%%%%%%01%%%%%%%%%%%%%%%%%%%%%%%%%%%%%%%%%%%%%%%
   \item[1.]   Prove that $\_ \circ \phi$ and $\phi \circ \_$ are linear
               transformations.

      \textbf{Proof.} First we will show that $\_ \circ \phi$ is a linear
      transformation.

      \begin{itemize}
         \item \textbf{additivity.} Let $\alpha$ and $\beta$ be linear
               transformations in $\mathcal{L}(W, U)$. We want to show that
               \begin{equation} \label{1_1}
                  (\_\circ\phi)(\alpha+\beta) = (\_\circ\phi)(\alpha) +
                  (\_\circ\phi)(\beta).
               \end{equation}
               So it suffices to show that
               $$[(\_\circ\phi)(\alpha+\beta)](v) = [(\_\circ\phi)(\alpha)](v) +
                 [(\_\circ\phi)(\beta)](v)$$
               for all $v \in V$. To that end, we let $v$ be an arbitrary
               element in $V$. Hence
               \begin{align*}
                  [(\_\circ\phi)(\alpha+\beta)](v) &=
                     [(\alpha+\beta)\circ\phi](v)
                        &[\text{Definition of }\_\circ\phi ] \\
                     &= (\alpha + \beta)(\phi(v)) \\
                     &= \alpha(\phi(v)) + \beta(\phi(v)) \\
                     &= (\alpha\circ\phi)(v) + (\beta\circ\phi)(v) \\
                     &= [(\_\circ\phi)(\alpha)](v) + [(\_\circ\phi)(\beta)](v),
               \end{align*}
               as desired, and we conclude that \eqref{1_1} holds, so that
               $\_\circ\phi$ is additive.
         \item \textbf{homogeneity.} Let $a \in k$. We want to show that
               \begin{equation} \label{1_2}
                  (\_\circ\phi)(a\alpha) = a(\_\circ\phi)(\alpha).
               \end{equation}
               The equality in \eqref{1_2} holds if and only if for every
               $v \in V$, we have
               $$[(\_\circ\phi)(a\alpha)](v) = a([(\_\circ\phi)(\alpha)](v)).$$
               So let $v \in V$. It follows that \eqref{1_2} holds because
               \begin{align*}
                  [(\_\circ\phi)(a\alpha)](v) &= [(a\alpha)\circ\phi](v)
                     &[\text{Definition of }\_\circ\phi ] \\
                     &= (a\alpha)(\phi(v)) \\
                     &= a(\alpha(\phi(v))) \\
                     &= a((\alpha\circ\phi)(v)) \\
                     &= a([(\_\circ\phi)(\alpha)](v)).
               \end{align*}
      \end{itemize}
      Thus we conclude from above that $\_\circ\phi$ is a linear transformation.
      Now we will show that $\phi\circ\_$ is also a linear transformation.

      \begin{itemize}
         \item \textbf{additivity.} Let $\alpha$ and $\beta$ be linear
               transformations in $\mathcal{L}(U, V)$. We want to show that
               \begin{equation} \label{1_3}
                  (\phi\circ\_)(\alpha+\beta) = (\phi\circ\_)(\alpha) +
                  (\phi\circ\_)(\beta).
               \end{equation}
               So it suffices to show that
               $$[(\phi\circ\_)(\alpha+\beta)](u) = [(\phi\circ\_)(\alpha)](u) +
                 [(\phi\circ\_)(\beta)](u)$$
               for all $u \in U$. To that end, we let $u$ be an arbitrary
               element in $U$. Hence
               \begin{align*}
                  [(\phi\circ\_)(\alpha+\beta)](u) &=
                     [\phi\circ(\alpha+\beta)](u)
                        &[\text{Definition of }\_\circ\phi ] \\
                     &= (\alpha + \beta)(\phi(v)) \\
                     &= \alpha(\phi(v)) + \beta(\phi(v)) \\
                     &= (\alpha\circ\phi)(v) + (\beta\circ\phi)(v) \\
                     &= [(\_\circ\phi)(\alpha)](v) + [(\_\circ\phi)(\beta)](v),
               \end{align*}
               as desired, and we conclude that \eqref{1_1} holds, so that
               $\_\circ\phi$ is additive.
         \item \textbf{homogeneity.} Let $a \in k$. We want to show that
               \begin{equation} \label{1_4}
                  (\_\circ\phi)(a\alpha) = a(\_\circ\phi)(\alpha).
               \end{equation}
               The equality in \eqref{1_2} holds if and only if for every
               $v \in V$, we have
               $$[(\_\circ\phi)(a\alpha)](v) = a([(\_\circ\phi)(\alpha)](v)).$$
               So let $v \in V$. It follows that \eqref{1_2} holds because
               \begin{align*}
                  [(\_\circ\phi)(a\alpha)](v) &= [(a\alpha)\circ\phi](v)
                     &[\text{Definition of }\_\circ\phi ] \\
                     &= (a\alpha)(\phi(v)) \\
                     &= a(\alpha(\phi(v))) \\
                     &= a((\alpha\circ\phi)(v)) \\
                     &= a([(\_\circ\phi)(\alpha)](v)).
               \end{align*}
      \end{itemize}
\end{enumerate}
\end{document}
