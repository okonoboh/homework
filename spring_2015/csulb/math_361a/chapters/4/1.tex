\begin{enumerate}
%%%%%%%%%%%%%%%%%%%%%%%%%%%%%%%%%%%%%%%01%%%%%%%%%%%%%%%%%%%%%%%%%%%%%%%%%%%%%%%
   \item[Exam] \textbf{12\%}: Part of 3.3 (p.86)
               $$\lim_{x\rightarrow\infty}f(x) = \infty,
                 \lim_{x\rightarrow-\infty}f(x) = \infty,
                 \lim_{x\rightarrow\infty}f(x) = -\infty,
                 \lim_{x\rightarrow-\infty}f(x) = -\infty.$$
                 
               \textbf{28\%}: 4.1 Continuous function at a point (both versions:
               $\lim_{x\rightarrow a}f(x) = f(a)$, $\varepsilon-\delta$ version)
               
               Find and classify discontinuities: Removable discontinuity,
               jump discontinuity, discontinuity of $2^{\text{nd}}$ kind.
               
               Theorem 4.1, 4.2, 4.3, right continuous at $x = a$, left
               continuous at $x = a$. Continuity on an interval: $[a, b]$,
               $(a, b)$.
               
               
               \textbf{28\%} 4.2 (p 102-106) Th 4.4, 4.5
               (Extreme Value Theorem), 4.6 (Intermediate Value Theorem),
               Theorem 4,7 Fixed point.
               
               \textbf{32\%} 4.3 Uniform continuous on an interval. Be able to
               use definition to prove uniform continuity on  $I$ or to prove
               not uniform continuity on $I$. Theorem 4.12, 4.13, 4.14, 4.15
               and its corollary, 4.16.
   \item[4.01] Prove that if $f$ is continuous at $x_0$ then $f$ is bounded at
               $x_0$.

      \textbf{Proof.} Suppose that $f : A \rightarrow \R$ is continuous at
      $x_0$. To show that $f$ is bounded at $x_0$, it suffices to show that
      $f$ is bounded on $N_{\delta}(x_0) \cap A$ for some $\delta > 0$. Since
      $f$ is continuous at $x_0$, it follows by definition that there exists
      $\delta_1 > 0$ such that $|f(x) - f(x_0)| < 1$ whenever
      $|x - x_0| < \delta_1$. Using the triangle inequality we have that
      $$||f(x)| - |f(x_0)|| < |f(x) - f(x_0)| < 1,$$
      so that $|f(x)| - |f(x_0)| < 1$, if $|x - x_0| < \delta_1$. Thus
      $|f(x)| < 1 + |f(x_0)|$, if $|x - x_0| < \delta_1$. We have thus shown
      that $f$ is bounded on $N_{\delta_1}(x_0)$ by $1 + |f(x_0)|$, so that
      $f$ is bounded at $x_0$. \qed
%%%%%%%%%%%%%%%%%%%%%%%%%%%%%%%%%%%%%%%02%%%%%%%%%%%%%%%%%%%%%%%%%%%%%%%%%%%%%%%
   \item[4.02] Find all points of discontinuity for the following functions,
               classify the discontinuities as removable, jump, or second kind,
               and determine where the function is right- and left-continuous.
               \begin{enumerate}
                  \item $f(x) = \begin{cases}
                           x^2                 & \text{if } x < -1, \\
                           2x + 3              & \text{if } -1 \le x \le 0,\\
                           |x - 1|             & \text{if } 0 < x < 2, \\
                           x^3 - 7             & \text{if } 2 \le x < 3, \\
                         \D\frac{x - 3}{x - 4} & \text{if } 3 \le x < 4, \\
                           0                   & \text{if } 4 \le x.
                        \end{cases}$
                  \item $f(x) = x + \gi{-x}$.
                  \item $f(x) = x \gi{x}$.
                  \item $f(x) = \text{sgn}\gi{|x|}$.
                  \item $f(x) = \begin{cases}
                           \gi{x + 1}\D\sin\frac{1}{x}
                              & \text{if } x \in (-1, 0) \cup (0, 1) \\
                           0  & \text{otherwise}.
                        \end{cases}$
                  \item $f(x) = \begin{cases}
                           (1 + x)\text{ sgn }x + \text{sgn }|x| - 1
                              & \text{if } x \text{ is rational} \\
                           \text{sgn } x  & \text{if }x \text{ is irrational}.
                        \end{cases}$
               \end{enumerate}
               
      \textbf{Solution.}
      
      \begin{enumerate}
         \item Since
               $$3 = \lim_{x\rightarrow 0^-} f(x) \neq
                     \lim_{x\rightarrow 0^+} f(x) = 1
                 \text{ and }
                 27 = \lim_{x\rightarrow 3^-} f(x) \neq
                      \lim_{x\rightarrow 3^+} f(x) = 0$$
               it follows that $f$ has jump discontinuities at 0 and 3. The
               function $f$ has a discontinuity of the second kind at $4$
               because $\lim_{x\rightarrow 4^-} f(x) = -\infty$; $f$ is not
               right-continuous at 0 because
               $1 = \lim_{x\rightarrow 0^+} f(x) \neq f(0) = 3$, but it is
               right-continuous at every other point; also, $f$ is not
               left-continuous at 3 and 4 because
               $27 = \lim_{x\rightarrow 3^-} f(x) \neq f(3) = 0$ and
               $\lim_{x\rightarrow 4^-} f(x) = -\infty$, but it is
               left-continuous at every other point.
         \item Let $z$ be an integer. Then it follows that
               \begin{align*}
                  f(z) &= z + \gi{-z} \\
                       &= z + (-z) = 0, \\ \\
                  \lim_{x\rightarrow z^+} f(x) &= \lim_{x\rightarrow z^+} x +
                  \lim_{x\rightarrow z^+} \gi{-x} \\
                  &= z + (-z-1) = -1, \text{ and } \\ \\
                  \lim_{x\rightarrow z^-} f(x) &= \lim_{x\rightarrow z^-} x +
                  \lim_{x\rightarrow z^-} \gi{-x} \\
                  &= z + (-z) = 0,
               \end{align*}
               so that $f$ has jump discontinuities at all integers; $f$ is
               left-continuous at all points, and it is right-continuous at all
               points except the integers.
         \item It is clear that $f$ is continuous at all non-integer points. So
               let $z$ be an integer. Observe that $f(z) = z^2$,
               $\lim_{x\rightarrow z^-} f(x) = z(z - 1)$, and
               $\lim_{x\rightarrow z^+} f(x) = z^2$, so that $f$ is continuous
               at 0 and has jump discontinuities at all other integers; also $f$
               is left-continuous at all points except nonzero integers, and it
               is right continuous at all points.
         \item We have that
               \begin{equation*}
                  \text{sgn}\gi{|x|} = \begin{cases}
                      1 & \text{if } x \le -1 \text{ or } x \ge 1 \\
                      0 & \text{if } -1 < x < 1,
                  \end{cases}
               \end{equation*}
               so that $f$ has jump discontinuites at $-1$ and 1, is
               left-continuous at all points except 1, and is right-continuous
               at all points except $-1$.
      \end{enumerate}
%%%%%%%%%%%%%%%%%%%%%%%%%%%%%%%%%%%%%%%03%%%%%%%%%%%%%%%%%%%%%%%%%%%%%%%%%%%%%%%
   \item[4.03] Prove that $f(x) = \cos {x}$ is continuous on $\R$.
%%%%%%%%%%%%%%%%%%%%%%%%%%%%%%%%%%%%%%%04%%%%%%%%%%%%%%%%%%%%%%%%%%%%%%%%%%%%%%%
   \item[4.04] Prove that if $f$ is continuous at $x_0$ and $g$ is discontinuous
               at $x_0$ then $f + g$ must have a discontinuity at $x_0$.

      \textbf{Proof.} Assume that $f$ is continuous at $x_0$ and $g$ is 
      discontinuous at $x_0$. Now suppose to the contrary that $f + g$ is
      continuous at $x_0$. By Theorem 4.2, it follows that
      $(f +\nobreak g) +\nobreak (-f) = g$ is also continuous at $x_0$, a 
      contradiction. Thus $f + g$ has a
      discontinuity at $x_0$. \qed
%%%%%%%%%%%%%%%%%%%%%%%%%%%%%%%%%%%%%%%05%%%%%%%%%%%%%%%%%%%%%%%%%%%%%%%%%%%%%%%
   \item[4.05] Show that $f + g$ can be continuous at $x_0$ even though both $f$
               and $g$ have discontinuities at $x_0$.

      \textbf{Solution.} For every nonzero $x$, let $f(x) = 1/x$; then define
      $f(0) = 0$ and $g(x) = -f(x)$ for all real $x$. It is clear that both 
      functions are not continuous at 0, but $f + g = 0$ is continuous at 0.
%%%%%%%%%%%%%%%%%%%%%%%%%%%%%%%%%%%%%%%06%%%%%%%%%%%%%%%%%%%%%%%%%%%%%%%%%%%%%%%
   \item[4.06] Show that $f \cdot g$ can be continuous at $x_0$ even though 
               both $f$ and $g$ have discontinuities at $x_0$.

      \textbf{Solution.} Define
      \begin{align*}
         f(x) &= \left\{\begin{array}{cl}
            0 & \text{if } x \le 1 \\
            \D\frac{1}{x - 1} & \text{if } x > 1,
         \end{array} \right. \\
         g(x) &= \left\{\begin{array}{cl}
            \D\frac{1}{x - 1} & \text{if } x < 1 \\            
            0 & \text{if } x \ge 1.
         \end{array} \right.
      \end{align*}

      The functions $f$ and $g$ have discontinuities at 1 because the former
      is not right-continuous at 1 and the latter is not left-continuous at 1;
      however $f \cdot g = 0$ is continuous at 1.
%%%%%%%%%%%%%%%%%%%%%%%%%%%%%%%%%%%%%%%07%%%%%%%%%%%%%%%%%%%%%%%%%%%%%%%%%%%%%%%
   \item[4.07] If $f$ is continuous at $x_0$ and $g$ is discontinuous at $x_0$,
               what can be said about continuity of the product $f \cdot g$ at
               $x_0$?

      \textbf{Answer.} Nothing definite can be said about the continuity of the
      product at $x_0$ because if $f(x) = 1$ and $g(x) = 1/x$, then
      $f \cdot g = g$ is not continuous at 0, and if $f(x) = 0$ and
      $g(x) = 1/x$ with $g(0) = 0$ then $f \cdot g = 0$ is continuous at 0. Note
      that in either case $f$ was continuous at 0 and $g$ was not.
%%%%%%%%%%%%%%%%%%%%%%%%%%%%%%%%%%%%%%%08%%%%%%%%%%%%%%%%%%%%%%%%%%%%%%%%%%%%%%%
   \item[4.08] Show that the composition function $g \circ f$ can be continuous
               at $x_0$ even though $f$ or $g$ or both $f$ and $g$ are
               discontinuous at $x_0$.

      \textbf{Solution.} Let $f(x) = 1/x$, with $f(0) = 0$, and let $g = f$.
      Then it follows that $(f \circ g)(x) = x$ so that $f \circ g$ is
      continuous at 0, but $f$ (and thus $g$) is not continuous at 0.
%%%%%%%%%%%%%%%%%%%%%%%%%%%%%%%%%%%%%%%09%%%%%%%%%%%%%%%%%%%%%%%%%%%%%%%%%%%%%%%
   \item[4.09] Prove that the function
               \begin{equation*}
                  f(x) = \begin{cases}
                     x  & \text{if $x$ is rational}, \\
                     0  & \text{if $x$ is irrational}.
                  \end{cases}
               \end{equation*}
               has a discontinuity of the second kind at each nonzero real
               number.

      \textbf{Proof.} Let $a$ be a nonzero real number. It suffices to show that
      $\lim_{x\rightarrow a^+} f(x)$ does not exist. By Theorems 1.9 and 1.10,
      we know that every interval in $\R$ contains a rational number and an
      irrational number. So let $b_n$ be a sequence of rational numbers and
      $c_n$ a sequence of irrational numbers such that
      $$b_n \in (a, a + 1/n) \cap \Q, \qquad
        c_n \in (a, a + 1/n) \cap (\R - \Q).$$
      It is clear that
      $\lim_{n\rightarrow\infty}b_n = \lim_{n\rightarrow\infty}c_n = a$, but
      $$\lim_{n\rightarrow\infty}f(b_n) = \lim_{n\rightarrow\infty}b_n = a
        \neq 0 = \lim_{n\rightarrow\infty}0 = \lim_{n\rightarrow\infty}c_n,$$
      so that by Exercise 3.36, $\lim_{x\rightarrow a^+} f(x)$ does not exist.
      Thus $f$ has a discontinuity of the second kind at each nonzero real
      number. \qed
%%%%%%%%%%%%%%%%%%%%%%%%%%%%%%%%%%%%%%%10%%%%%%%%%%%%%%%%%%%%%%%%%%%%%%%%%%%%%%%
   \item[4.10] Prove that if $f$ is continuous at $x_0$ and $f$ is nonnegative
               then $h(x) = \sqrt{f(x)}$ is continuous at $x_0$.
%%%%%%%%%%%%%%%%%%%%%%%%%%%%%%%%%%%%%%%11%%%%%%%%%%%%%%%%%%%%%%%%%%%%%%%%%%%%%%%
   \item[4.11] Find a function $f$ which has a discontinuity of the second kind
               at every real number although $f \circ f$ is continuous on $\R$.

      \textbf{Solution.} We know from Example 4.4 that
      \begin{equation*}
         f(x) = \begin{cases}
            1    & \text{if $x$ is rational}, \\
            0    & \text{if $x$ is irrational},
         \end{cases}
      \end{equation*}
      has a discontinuity of the second kind at every real number. Now
      \begin{equation*}
         (f \circ f)(x) = f(f(x)) = \begin{cases}
            f(1) = 1    & \text{if $x$ is rational}, \\
            f(0) = 1    & \text{if $x$ is irrational},
         \end{cases}
      \end{equation*}
      so that $f \circ f$ is identically equal to 1. Thus $f \circ f$ is 
      continuous on $\R$.
%%%%%%%%%%%%%%%%%%%%%%%%%%%%%%%%%%%%%%%12%%%%%%%%%%%%%%%%%%%%%%%%%%%%%%%%%%%%%%%
   \item[4.12] If $f$ is continuous on $(0, 1)$ and $f(x) = 1 - x$ for every
               rational number $x \in (0, 1)$, find $f(\pi/4)$. Explain your
               answer.

      \textbf{Solution.} Since $f$ is continuous on (0, 1) and
      $\pi/4 \in (0, 1)$, it follows that
      $$\lim_{x \rightarrow \pi/4} = f(\pi/4).$$
      Consider the sequence of rationals $a_n$ where
      $\D a_n \in \left(\frac{\pi}{4}, \frac{\pi}{4} + \frac{1}{10n}\right)$.
      Since each $a_n \in (0, 1)$ and since $a_n$ converges to $\pi/4$, it 
      follows by Theorem 3.6 that $f(a_n)$ must converge to $f(\pi/4)$. Thus
      \begin{align*}
         f(\pi/4) &= \lim_{n\rightarrow\infty}f(a_n) \\
            &= \lim_{n\rightarrow\infty}(1 - a_n) \\
            &= \lim_{n\rightarrow\infty}1 - \lim_{n\rightarrow\infty}a_n \\
            &= 1 - \pi/4.
      \end{align*}
%%%%%%%%%%%%%%%%%%%%%%%%%%%%%%%%%%%%%%%13%%%%%%%%%%%%%%%%%%%%%%%%%%%%%%%%%%%%%%%
   \item[4.13] Prove that if $f$ and $g$ are each continuous on $(a, b)$ and
               $f(x) = g(x)$ for every rational $x \in (a, b)$ then
               $f(x) = g(x)$ for every $x \in (a, b)$.

      \textbf{Proof.} Assume the hypothesis holds. It suffices to show that
      $f(x) = g(x)$ for all $x \in (a, b) \cap (\R - \Q)$. So let $c$ be an
      irrational number in $(a, b)$. Since $f$ and $g$ are continuous at $c$, we 
      must have that $\lim_{x\rightarrow c}f(x) = f(c)$ and
      $\lim_{x\rightarrow c}g(x) = g(c)$. Let $\{t_n\}$ be a sequence of 
      rationals in $(a, b)$ that converges to $c$. By Theorem 3.6, it follows 
      that $\{f(t_n)\}$ must converge to $f(c)$ and $\{g(t_n)\}$ must converge 
      to $g(c)$. But $f(t_n) = g(t_n)$ for each $n$, so that by the uniqueness 
      of the limit of a sequence, we must have that $f(c) = g(c)$, which is what 
      we wanted to show. \qed
%%%%%%%%%%%%%%%%%%%%%%%%%%%%%%%%%%%%%%%14%%%%%%%%%%%%%%%%%%%%%%%%%%%%%%%%%%%%%%%
   \item[4.14] Prove: $f$ is right-continuous at $x_0$ if and only if
               $f(x_n) \rightarrow f(x_0)$ for every sequence $\{x_n\}$ in the
               domain of $f$ with $x_n \rightarrow x_0$ and $x_n \ge x_0$ for
               $n = 1, 2, 3, ....$

      \textbf{Proof.} Use Exercise 3.36. \qed
%%%%%%%%%%%%%%%%%%%%%%%%%%%%%%%%%%%%%%%15%%%%%%%%%%%%%%%%%%%%%%%%%%%%%%%%%%%%%%%
   \item[4.15] Discuss one-sided continuity for the pie function.
%%%%%%%%%%%%%%%%%%%%%%%%%%%%%%%%%%%%%%%16%%%%%%%%%%%%%%%%%%%%%%%%%%%%%%%%%%%%%%%
   \item[4.16] Prove that if $f$ is defined on $\R$ and continuous at $x_0 = 0$
               and if $f(x_1 + x_2) = f(x_1) + f(x_2)$ for each
               $x_1, x_2 \in \R$ then $f$ is continuous on $\R$.
%%%%%%%%%%%%%%%%%%%%%%%%%%%%%%%%%%%%%%%17%%%%%%%%%%%%%%%%%%%%%%%%%%%%%%%%%%%%%%%
   \item[4.17] Find all functions $f$ which are continuous on $\R$ and which
               satisfy the equation $f(x)^2 = x^2$ for each $x \in \R$.
               \textit{Hint}: There are four possible solutions.
%%%%%%%%%%%%%%%%%%%%%%%%%%%%%%%%%%%%%%%18%%%%%%%%%%%%%%%%%%%%%%%%%%%%%%%%%%%%%%%
   \item[4.18] Prove that if $g$ is continuous at $x_0 = 0$, $g(0) = 0$ and for
               some $\delta > 0$ $|f(x)| \le |g(x)|$ for each
               $x \in N_\delta(0)$ then $f$ is continuous at $x_0 = 0$.
%%%%%%%%%%%%%%%%%%%%%%%%%%%%%%%%%%%%%%%19%%%%%%%%%%%%%%%%%%%%%%%%%%%%%%%%%%%%%%%
   \item[4.19] Prove that if $f$ is continuous on $[a, b]$ then there exists a
               function $g$ continuous on $\R$ such that $g(x) = f(x)$ for each
               $x \in [a, b]$. The function $g$ is called a
               \textit{continuous extension} of $f$ to $\R$.
%%%%%%%%%%%%%%%%%%%%%%%%%%%%%%%%%%%%%%%20%%%%%%%%%%%%%%%%%%%%%%%%%%%%%%%%%%%%%%%
   \item[4.20] The function $f(x) = \tan x$ defined on $(-\pi/2, \pi/2)$ clearly
               has no continuous extension to $\R$. Find a bounded continuous
               function on $(a, b)$ which has no continuous extension to $\R$.
%%%%%%%%%%%%%%%%%%%%%%%%%%%%%%%%%%%%%%%21%%%%%%%%%%%%%%%%%%%%%%%%%%%%%%%%%%%%%%%
   \item[4.21] Assume that $f$ is continuous on $(a, b)$. Prove that $f$ has a
               continuous extension to $\R$ if and only if both limits
               $\lim_{x\rightarrow a^+}f(x)$ and $\lim_{x\rightarrow b^-}f(x)$
               exist.
%%%%%%%%%%%%%%%%%%%%%%%%%%%%%%%%%%%%%%%22%%%%%%%%%%%%%%%%%%%%%%%%%%%%%%%%%%%%%%%
   \item[4.22] Prove that if $f$ is continuous on $(a, b)$ and both
               $\lim_{x\rightarrow a^+}f(x)$ and $\lim_{x\rightarrow b^-}f(x)$
               exist then $f$ is bounded on $(a, b)$.
%%%%%%%%%%%%%%%%%%%%%%%%%%%%%%%%%%%%%%%23%%%%%%%%%%%%%%%%%%%%%%%%%%%%%%%%%%%%%%%
   \item[4.23] Suppose $f$ is one-to-one on $(a, b)$ and satisfies the following
               property: whenever $f(x_1) \neq f(x_2)$ for $x_1 < x_2$, $x_1$,
               $x_2 \in (a, b)$ and $k$ is any number between $f(x_1)$ and
               $f(x_2)$, there exists a $c \in (x_1, x_2)$ with $f(c) = k$.
               Prove that $f$ is continuous on $(a, b)$.
\end{enumerate}
