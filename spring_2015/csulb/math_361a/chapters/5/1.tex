\begin{enumerate}
%%%%%%%%%%%%%%%%%%%%%%%%%%%%%%%%%%%%%%%01%%%%%%%%%%%%%%%%%%%%%%%%%%%%%%%%%%%%%%%
   \item[5.01] If $f(x) = |x^3|$, find $f'(x)$.

      \textbf{Solution.} By definition we have that
      \begin{equation*}
         f(x) = \left\{\begin{array}{rl}
            x^3 & \text{if } x \ge 0 \\
            -x^3 & \text{if } x < 0
         \end{array} \right. 
      \end{equation*}
      
      so that $f'(x) = 3x^2$ if $x > 0$ and $f'(x) = -3x^2$ if $x < 0$. Now we 
      have that
      $$f'(0) = \lim_{h \to 0}\frac{f(0 + h) - f(0)}{h} =
        \lim_{h \to 0}\frac{|h^3|}{h}.
      $$
      And since
      $$\lim_{h \to 0^+}\frac{|h^3|}{h} = \lim_{h \to 0^+}\frac{h^3}{h} = 0 =
        \lim_{h \to 0^-}\frac{-h^3}{h} = \lim_{h \to 0^-}\frac{|h^3|}{h},$$
      it follows by Theorem 3.7 that
      $f'(0) = \D\lim_{h \to 0}\frac{|h^3|}{h} = 0$. Thus $f'(x) = |3x^2|$.
%%%%%%%%%%%%%%%%%%%%%%%%%%%%%%%%%%%%%%%02%%%%%%%%%%%%%%%%%%%%%%%%%%%%%%%%%%%%%%%
   \item[5.02] Let $f(x) = x|x|$; show that
               \begin{equation*}
                  f''(x) = \left\{\begin{array}{rl}
                     2 & \text{if } x > 0 \\
                     -2 & \text{if } x < 0
                  \end{array} \right. 
               \end{equation*}
               and that 0 is not in the domain of $f''(x)$.
               
      \textbf{Proof.} From example 5.1, we know that  $f'(x) = 2|x|$, so that
      $f''(x) = 2$ if $x > 0$ and $f''(x) = -2$ if $x < 0$. We now want to show
      that $f''(0)$ is undefined. To that end, we have that
      $$f''(0) = \lim_{h \to 0}\frac{f'(0 + h) - f'(0)}{h} = 
        \lim_{h \to 0}\frac{2|h|}{h},$$
      which does not exist since the one-sided limits are not equal. Thus 0 is
      not in the domain of $f''$.
%%%%%%%%%%%%%%%%%%%%%%%%%%%%%%%%%%%%%%%03%%%%%%%%%%%%%%%%%%%%%%%%%%%%%%%%%%%%%%%
   \item[5.03] Find $f'(x)$ if
               \begin{equation*}
                  f(x) = \left\{\begin{array}{ll}
                     x^2 & \text{if } x \ge 2 \\
                     4x - 4 & \text{if } x < 2
                  \end{array} \right. 
               \end{equation*}
               
      \textbf{Solution.} It is clear that $f'(x) = 2x$ if $x > 2$ and that
      $f'(x) = 4$ if $x < 2$, so we only need to investigate if $f'(2)$ exists.
      To that end, we have that
      \begin{align*}
         f'(2) &= \lim_{h \to 0} \frac{f(2 + h) - f(2)}{h} \\
               &= \lim_{h \to 0} \frac{f(2 + h) - 4}{h}.
      \end{align*}
      Since
      $$\lim_{h \to 0^-} \frac{f(2 + h) - 4}{h} = \lim_{h \to 0^-} \frac{4(2+h)
         - 4 - 4}{h} = 4,$$
      and
      $$\lim_{h \to 0^+} \frac{f(2 + h) - 4}{h} = \lim_{h \to 0^+} \frac{(2+h)^2
         - 4}{h} = \lim_{h \to 0^+} \frac{h^2 + 4h}{h} = 4,$$
      it follows that $f'(2) = 4$. Thus
      \begin{equation*}
         f'(x) = \left\{\begin{array}{ll}
            2x & \text{if } x \ge 2 \\
            4 & \text{if } x < 2
         \end{array} \right. 
      \end{equation*}
%%%%%%%%%%%%%%%%%%%%%%%%%%%%%%%%%%%%%%%04%%%%%%%%%%%%%%%%%%%%%%%%%%%%%%%%%%%%%%%
   \item[5.04] For what values of $a$ and $b$ is $f(x)$ differentiable at
               $x = 1$ if
               \begin{equation*}
                  f(x) = \left\{\begin{array}{ll}
                     x^3 & \text{if } x < 1 \\
                     ax + b & \text{if } x \ge 1.
                  \end{array} \right. 
               \end{equation*}

      \textbf{Solution.} Suppose that $f$ is differentiable at 1. Then by
      Theorem 5.1, it follows that $f$ is continuous at 1. Particularly, $f$ is
      left-continuous at 1 so that
      $$1^3 = 1 = \lim_{x \to 1^-}f(x) = f(1) = a + b.$$
      Since $f$ is differentiable at 1, it follows by Definition 5.1 that the 
      limit
      $$\lim_{h \to 0}\frac{f(1 + h) - f(1)}{h}$$
      exists, so that
      $$\lim_{h \to 0^+}\frac{f(1 + h) - f(1)}{h} =
        \lim_{h \to 0^-}\frac{f(1 + h) - f(1)}{h}.$$
      Thus
      \begin{align*}
         \lim_{h \to 0^+}\frac{f(1 + h) - f(1)}{h} &=
            \lim_{h \to 0^+}\frac{a(1 + h) + b - a - b}{h} \\
            &= a \\
            &= \lim_{h \to 0^-}\frac{f(1 + h) - f(1)}{h} \\
            &= \lim_{h \to 0^-}\frac{(1 + h)^3 - (a + b)}{h} \\
            &= \lim_{h \to 0^-}\frac{(1 + h)^3 - 1}{h} \\
            &= 3.
      \end{align*}
      Hence $f$ is differentiable at 1 if and only if $a = 3$ and $b = -2$.
%%%%%%%%%%%%%%%%%%%%%%%%%%%%%%%%%%%%%%%11%%%%%%%%%%%%%%%%%%%%%%%%%%%%%%%%%%%%%%%
   \item[5.11] \begin{enumerate}
                  \item Define
                        \begin{equation*}
                           f(x) = \left\{\begin{array}{cl}
                              \D\frac{1}{4^n} & \qquad\text{if }
                              x = \D\frac{1}{2^n} \; (n = 1, 2, 3, \ldots) \\\\
                              0 & \qquad\text{otherwise}.
                           \end{array} \right.
                        \end{equation*}
                        Is $f$ differentiable at $x = 0$? Verify.
                  \item Define
                        \begin{equation*}
                           g(x) = \left\{\begin{array}{cl}
                              \D\frac{1}{2^{n+1}} & \qquad\text{if }
                              x = \D\frac{1}{2^n} \; (n = 1, 2, 3, \ldots) \\\\
                              0 & \qquad\text{otherwise}.
                           \end{array} \right.
                        \end{equation*}
                        Is $g$ differentiable at $x = 0$? Verify.            
               \end{enumerate}

      \textbf{Solution.}

      \begin{enumerate}
         \item By definition we have that
               \begin{align*}
                  f'(0) &= \lim_{h \to 0}\frac{f(0 + h) - f(0)}{h} \\
                        &= \lim_{h \to 0}\frac{f(h) - 0}{h} \\
                        &= \lim_{h \to 0}\frac{f(h)}{h}.
               \end{align*}
               We now claim that $f$ is differentiable at $x = 0$ because
               $$\lim_{h \to 0}\frac{f(h)}{h} = 0.$$

               \textbf{Proof.} Let $\varepsilon > 0$ be given. We want to find
               a corresponding $\delta > 0$ such that if $0 < |x| < \delta$,
               then $\left|\D\frac{f(x)}{x}\right| < \varepsilon$. Choose a 
               large positive integer $N$ such that
               $\D\frac{1}{2^N} < \varepsilon$. This suggests that we choose
               $\delta = \D\frac{1}{2^N}$. Now suppose $x \in N^*_\delta(0)$. We
               have the following possibilities:

               \textbf{Case 1.} $x < 0$ or ($x > 0$ and
               $x \neq \D\frac{1}{2^n} \; (n = 1, 2, 3, \ldots)$). Thus
               $$\left|\frac{f(x)}{x}\right| = \left|\frac{0}{x}\right| =
                 0 < \varepsilon.$$

               \textbf{Case 2.} $x > 0$ and
               $x = \D\frac{1}{2^n}$ for some positive integer $n$. Thus
               $$\left|\frac{f(x)}{x}\right| =
                 \left|\frac{\D\frac{1}{4^n}}{\D\frac{1}{2^n}}\right| =
                 \left|\frac{1}{2^n}\right| = \frac{1}{2^n} < \delta =
                 \frac{1}{2^N} < \varepsilon.$$

               So it follows by definition that
               $$f'(0) = \lim_{h \to 0}\frac{f(h)}{h} = 0.$$
               \qed

               That is, $f$ is differentiable at $x = 0$.
         \item Similarly we have that
               $$g'(0) = \lim_{h \to 0}\frac{g(0 + h) - g(0)}{h} =
                 \lim_{h \to 0}\frac{g(h)}{h}.$$
               We claim that $g$ is not differentiable at $x = 0$ because
               $$\lim_{h \to 0}\frac{g(h)}{h}$$
               does not exist.

               \textbf{Proof.} Consider the sequences $\D x_n = \frac{1}{2^n}$ 
               and $\D y_n = \frac{1}{3^n}$. We know from our previous 
               discussions that $x_n \rightarrow 0$ and $y_n \rightarrow 0$.
               Now
               $$\lim_{n \to \infty}\frac{g(x_n)}{x_n} = \lim_{n \to \infty}
                 \frac{\D\frac{1}{2^{n+1}}}{\D\frac{1}{2^n}} =
                 \lim_{n \to \infty}\frac{1}{2} = \frac{1}{2},$$
               and
               $$\lim_{n \to \infty}\frac{g(y_n)}{y_n} = \lim_{n \to \infty}
                  3^ng(y_n) = \lim_{n \to \infty} 3^n \cdot 0 = 0.$$
               Since $\D\frac{1}{2} \neq 0$, it follows by Theorem 3.6 that
               $$\lim_{h \to 0}\frac{g(h)}{h}$$
               does not exist, so that $g$ is not differentiable at $x = 0$.
               \qed    
      \end{enumerate}
\end{enumerate}
