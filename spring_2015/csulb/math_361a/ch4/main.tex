\documentclass[9pt]{article}

\usepackage{amssymb}
\usepackage{amsmath}
\usepackage{amsfonts}
\usepackage{comment}
\usepackage{fancyhdr}
\usepackage{mathrsfs}
\usepackage{enumitem}


\usepackage{tikz}

\voffset = -50pt
%\textheight = 700pt
\addtolength{\textwidth}{60pt}
\addtolength{\evensidemargin}{-30pt}
\addtolength{\oddsidemargin}{-30pt}
%\setlength{\headheight}{44pt}

\pagestyle{fancy}
\fancyhf{} % clear all fields
\fancyhead[R]{%
  \scshape
  \begin{tabular}[t]{@{}r@{}}
  MATH 361A, Spring 2015\\Section 1 (1873)\\
  \end{tabular}}
\fancyhead[L]{%
  \scshape
  \begin{tabular}[t]{@{}r@{}}
  JOSEPH OKONOBOH\\Mathematics\\Cal State Long Beach
  \end{tabular}}
\fancyfoot[C]{\thepage}

\newcommand{\qed}{\hfill \ensuremath{\Box}}


\newcommand*\circled[1]{\tikz[baseline=(char.base)]{
            \node[shape=circle,draw,inner sep=2pt] (char) {#1};}}

\newcommand{\N}{\mathbb{N}}
\newcommand{\Z}{\mathbb{Z}}
\newcommand{\I}{\mathbb{I}}
\newcommand{\M}{\mathbb{M}}
\newcommand{\R}{\mathbb{R}}
\newcommand{\D}{\displaystyle}
%\setcounter{section}{-1}

\begin{document}
\begin{enumerate}
%%%%%%%%%%%%%%%%%%%%%%%%%%%%%%%%%%%%%Prob10%%%%%%%%%%%%%%%%%%%%%%%%%%%%%%%%%%%%%
   \item Prove that if $f$ is continuous at $x_0$ and $f$ is nonnegative then
         $h(x) = \sqrt{f(x)}$ is continuous at $x_0$.
         
      \textbf{Proof.} Let $\varepsilon > 0$. We want to find $\delta > 0$ such
      that if $|x - x_0| < \delta$, then $|h(x) - h(x_0)| < \varepsilon$. We
      shall now investigate the following two cases: \\
      
      \textbf{Case 1.} $h(x_0) = \sqrt{f(x_0)} = 0$. Now since $f$ is continuous
      at $x_0$, it follows by definition that there exists a $\delta_1 > 0$ such
      that $|f(x) - f(x_0)| < \varepsilon^2$ whenever $|x - x_0| < \delta_1$.
      So let $x \in (x_0 - \delta_1, x_0 + \delta_1)$. Hence
      $$f(x) = |f(x)| = |f(x) - 0| = |f(x) - f(x_0)| < \varepsilon^2$$
      so that $\sqrt{f(x)} < \varepsilon$. Then it follows that
      $$|h(x) - h(x_0)| = |h(x)| = |\sqrt{f(x)}| = \sqrt{f(x)} < \varepsilon,$$
      if $|x - x_0| < \delta_1$.
      
      \textbf{Case 2.} $h(x_0) = \sqrt{f(x_0)} > 0$.  By the continuity of $f$
      at $x_0$, t follows by definition
      that there exists a $\delta_2 > 0$ such that
      $\displaystyle|f(x) - f(x_0)| < \varepsilon \cdot f(x_0)$
      whenever $|x - x_0| < \delta_2$. Let
      $x \in (x_0 - \delta_2, x_0 + \delta_2)$. Then it follows that
      \begin{align*}
         |h(x) - h(x_0)| &= |\sqrt{f(x)} - \sqrt{f(x_0)}| \\
                         &= |\sqrt{f(x)} - \sqrt{f(x_0)}| \cdot
                            \frac{|\sqrt{f(x)} + \sqrt{f(x_0)}|}
                                 {|\sqrt{f(x)} + \sqrt{f(x_0)}|} \\
                         &= \frac{|f(x) - f(x_0)|}
                                 {|\sqrt{f(x)} + \sqrt{f(x_0)}|} \\
                         &= \frac{|f(x) - f(x_0)|}
                                 {\sqrt{f(x)} + \sqrt{f(x_0)}}
                            &[\text{Since }f \text{ is nonnegative}] \\
                         &\le \frac{|f(x) - f(x_0)|}
                                 {\sqrt{f(x_0)}} \\
                         &< \varepsilon \cdot f(x_0) \cdot
                            \frac{1}{\sqrt{f(x_0)}} \\
                         &= \varepsilon,
      \end{align*}
      if $|x - x_0| < \delta_2$. \\
      
      Thus it follows by definition that
      $$\lim_{x\rightarrow x_0}h(x) = h(x_0),$$
      so that $h$ is continuous at $x_0$. \qed
   \item Suppose $f$ is continuous on $(a, b)$. Prove that if
         $\lim_{x\rightarrow a^+} f(x)$ and $\lim_{x\rightarrow b^-} f(x)$ both
         exist then $f$ is bounded on $(a, b)$ but the converse does not hold.
         
      \textbf{Proof.} Assume $f$ is continuous on $(a, b)$. Now suppose that
      $\lim_{x\rightarrow a^+} f(x)$ and $\lim_{x\rightarrow b^-} f(x)$ both
      exist. Let
      $$\lim_{x\rightarrow a^+} f(x) = R \text{ and }
        \lim_{x\rightarrow b^-} f(x) = L.$$
      It follows by definition that there exists a $\delta_1 > 0$ such that if
      $x \in (a, a + \delta_1)$ then $|f(x) - R| < 1$, so that
      $|f(x)| < 1 + |R|$. Similarly, there exists a $\delta_2 > 0$ such that if
      $x \in (b - \delta_2, b)$ then $|f(x) - L| < 1$, so that
      $|f(x)| < 1 + |L|$. Let
      $$\delta = \min\left\{\frac{b - a}{3},\delta_1, \delta_2\right\}.$$
      Now since $f$ is continuous on $(a, b)$ it must also be continuous on
      $[a + \delta, b - \delta]$, so that by Theorem 4.2, there exists $N > 0$
      such that if $x \in [a + \delta, b - \delta]$, then $|f(x)| \le N$. Let
      $$M = \max\{N, 1 + |R|, 1 + |L|\}.$$
      Finally consider any $x \in (a, b)$. It must be the case that $x$ is in
      exactly one of the following intervals: $(a, a + \delta)$,
      $[a + \delta, b - \delta]$, and $(b - \delta, b)$. The first interval is
      bounded by $1 + |R|$, the second by $1 + |L|$, and the third by $N$, so
      that $|f(x)| \le M$. Thus by definition $f$ is bounded on $(a, b)$.
      
      \textbf{Converse.} Suppose that $f$ is bounded on $(a, b)$. Fix some
      positive integer $N$ such that $\D N > \frac{1}{b - a}$, so that
      $\D\frac{1}{N} < b - a$. Let
      $$X_1 = \left\{\left[a + \frac{1}{n+1},\; a + \frac{1}{n}\right) :
      n \in \N, n \ge N \right\} \text{ and } X_2 = \left[a + \frac{1}{N}, \;b\right),$$
      so that $(a, b) = X_1 \cup X_2$. Let $P \in X_1$. Then 
      $$P = \left[a + \frac{1}{N + i + 1},\; a + \frac{1}{N + i}\right)$$
      for some nonnegative integer $i$. We want to map $P$ to the interval
      $[0, 1)$ for even $i$ and $[1, 0)$ for odd $i$ such that $f(P)$ is a
      straight line. If $i$ is even, for every $x \in P$, let
      
      $$f(x) = \left(\frac{1}{\D\frac{1}{N+i} - \frac{1}{N+i+1}}\right)
               \left(x - a - \frac{1}{N+i+1}\right),$$
      and if $i$ is odd, for every $x \in P$, let
      
      $$f(x) = \left(\frac{-1}{\D\frac{1}{N+i} - \frac{1}{N+i+1}}\right)
               \left(x - a - \frac{1}{N+i+1}\right) + 1.$$
      Now if $x \in X_2$, let $f(x) = 1$. Notice that $f$ oscillates between 0
      and 1 for $x$ close to $a$, so that $\lim_{x\rightarrow a^+}$ does not
      exist. \qed
      
      
\end{enumerate}
\end{document}
