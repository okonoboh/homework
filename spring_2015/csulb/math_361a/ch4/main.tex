\documentclass[9pt]{article}

\usepackage{amssymb}
\usepackage{amsmath}
\usepackage{amsfonts}
\usepackage{comment}
\usepackage{fancyhdr}
\usepackage{mathrsfs}
\usepackage{enumitem}


\usepackage{tikz}

\voffset = -50pt
%\textheight = 700pt
\addtolength{\textwidth}{60pt}
\addtolength{\evensidemargin}{-30pt}
\addtolength{\oddsidemargin}{-30pt}
%\setlength{\headheight}{44pt}

\pagestyle{fancy}
\fancyhf{} % clear all fields
\fancyhead[R]{%
  \scshape
  \begin{tabular}[t]{@{}r@{}}
  MATH 361A, Spring 2015\\Section 1 (1873)\\
  \end{tabular}}
\fancyhead[L]{%
  \scshape
  \begin{tabular}[t]{@{}r@{}}
  JOSEPH OKONOBOH\\Mathematics\\Cal State Long Beach
  \end{tabular}}
\fancyfoot[C]{\thepage}

\newcommand{\qed}{\hfill \ensuremath{\Box}}


\newcommand*\circled[1]{\tikz[baseline=(char.base)]{
            \node[shape=circle,draw,inner sep=2pt] (char) {#1};}}

\newcommand{\Z}{\mathbb{Z}}
\newcommand{\I}{\mathbb{I}}
\newcommand{\M}{\mathbb{M}}
\newcommand{\R}{\mathbb{R}}
%\setcounter{section}{-1}

\begin{document}
\begin{enumerate}
%%%%%%%%%%%%%%%%%%%%%%%%%%%%%%%%%%%%%Prob10%%%%%%%%%%%%%%%%%%%%%%%%%%%%%%%%%%%%%
   \item Prove that if $f$ is continuous at $x_0$ and $f$ is nonnegative then
         $h(x) = \sqrt{f(x)}$ is continuous at $x_0$.
         
      \textbf{Proof.} Let $\varepsilon > 0$. We want to find $\delta > 0$ such
      that if $|x - x_0| < \delta$, then $|h(x) - h(x_0)| < \varepsilon$. Now
      since $f$ is continuous at $x_0$, it follows by definition that there
      exists a $\delta_1 > 0$ such that $|f(x) - f(x_0)| < \varepsilon$
      whenever $|x - x_0| < \delta_1$. Let
      $x \in (x_0 - \delta_1, x_0 + \delta_1)$, with
      $h(x) + h(x_0) = \sqrt{f(x)} + \sqrt{f(x_0)}\neq 0$.
      Then it follows that
      \begin{align*}
         |h(x) - h(x_0)| &= |\sqrt{f(x)} - \sqrt{f(x_0)}| \\
                         &= |\sqrt{f(x)} - \sqrt{f(x_0)}| \cdot
                            \frac{|\sqrt{f(x)} + \sqrt{f(x_0)}|}
                                 {|\sqrt{f(x)} + \sqrt{f(x_0)}|} \\
                         &= \frac{|f(x) - f(x_0)|}
                                 {|\sqrt{f(x)} + \sqrt{f(x_0)}|} \\
                         &= \frac{|f(x) - f(x_0)|}
                                 {\sqrt{f(x)} + \sqrt{f(x_0)}}
                            &[\text{Since }f \text{ is nonnegative}] \\
                         &\le \frac{|f(x) - f(x_0)|}
                                 {\sqrt{f(x)} + \sqrt{f(x_0)} + 1}.
      \end{align*}
      If $\sqrt{f(x)} + \sqrt{f(x_0)} = 0$, so that
      $\sqrt{f(x)} = \sqrt{f(x_0)} = 0$, then the inequality above still holds.
      Thus if $x \in (x_0 - \delta_1, x_0 + \delta_1)$, we must have that
      $$|h(x) - h(x_0)| \le \frac{|f(x) - f(x_0)|}
                                 {\sqrt{f(x)} + \sqrt{f(x_0)} + 1}.$$
                                 
      Let $x \in (x_0 - \delta_1, x_0 + \delta_1)$. Since $f$ is nonnegative, it
      follows that $\sqrt{f}$ is also nonnegative
      so that
      $$\sqrt{f(x)} + \sqrt{f(x_0)} + 1 \ge 1, $$
      and thus
      \begin{equation}
         \frac{1}{\sqrt{f(x)} + \sqrt{f(x_0)} + 1} \le 1. \label{eq_a}
      \end{equation}
      
      Multiply the inequality in \eqref{eq_a} by the nonnegative number
      $|f(x) - f(x_0)|$ to get
      \begin{align*}
         \frac{|f(x) - f(x_0)|}{\sqrt{f(x)} + \sqrt{f(x_0)} + 1}
            \le |f(x) - f(x_0)| < \varepsilon.
      \end{align*}
      
      We have thus shown that if $x \in (x_0 - \delta_1, x_0 + \delta_1)$, then
      \begin{align*}
         |h(x) - h(x_0)| \le \frac{|f(x) - f(x_0)|}
            {\sqrt{f(x)} + \sqrt{f(x_0)} + 1} \le |f(x) - f(x_0)| < \varepsilon.
      \end{align*}
      
      Thus it follows by definition that
      $$\lim_{x\rightarrow x_0}h(x) = h(x_0),$$
      so that $h$ is continuous at $x_0$. \qed
\end{enumerate}
\end{document}
