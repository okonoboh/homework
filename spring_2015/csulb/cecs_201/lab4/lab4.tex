\documentclass[9pt]{article}

\usepackage{amssymb}
\usepackage{amsmath}
\usepackage{amsfonts}
\usepackage{comment}
\usepackage{fancyhdr}
\usepackage{mathrsfs}
\usepackage{enumitem}
\usepackage{graphicx}

\usepackage{tikz}

\voffset = -50pt
%\textheight = 700pt
\addtolength{\textwidth}{60pt}
\addtolength{\evensidemargin}{-30pt}
\addtolength{\oddsidemargin}{-30pt}
%\setlength{\headheight}{44pt}

\newcommand{\qed}{\hfill \ensuremath{\Box}}


\newcommand*\circled[1]{\tikz[baseline=(char.base)]{
            \node[shape=circle,draw,inner sep=2pt] (char) {#1};}}

\newcommand{\Z}{\mathbb{Z}}
\newcommand{\I}{\mathbb{I}}
\newcommand{\M}{\mathbb{M}}
\newcommand{\R}{\mathbb{R}}
\newcommand{\C}{\mathbb{C}}
%\setcounter{section}{-1}

\begin{document}
\topskip0pt
\vspace*{\fill}
\begin{center}
{\Huge \begin{tabular}{@{}ll@{}}
   Class: & CECS 201, Section 7 \\ \\ \\
   Lab: & 4 \\ \\ \\
   Title: & Gate Substitution and Boolean Algebra \\ \\ \\
   Student Name: & Barry Joseph Okonoboh \\ \\ \\
   Due Date: & 07:00:00 P.M., 16, March 2015 \\ \\ \\
   Instructor: & Dan Cregg
\end{tabular}}
\end{center}
\vspace*{\fill}
\newpage
\begin{enumerate}
%%%%%%%%%%%%%%%%%%%%%%%%%%%%%%%%%%%%%%%%01%%%%%%%%%%%%%%%%%%%%%%%%%%%%%%%%%%%%%%
   \item[b.] \textbf{Introduction.} This lab involves observing the outputs of
             the AND, OR, NAND, NOR, NOT and XOR logic gates, with varying
             input.
   \item[c.] \textbf{Project Description.} This lab is similar to lab 3 with the
             restriction that the AND gate be built using NOR gates, the OR
             gates using NAND gates, the NAND gate using NOR gates, the NOR gate
             using NAND gates, the XOR gate using NOR gates, and the NOT gates
             using NAND gates.
  	\item[d.] \textbf{Schematic.}
             \begin{center}
                \includegraphics[width=\textwidth]{schematic.png}
             \end{center}
             
             \textbf{Truth Table.}   
   
   \begin{center}
    \begin{tabular}{@{}|c|c|c|c|c|c|c|c|@{}}
   \hline
   $A$ & $B$ & $AB$ & $A+B$ & $\overline{AB}$ & $\overline{A+B}$ & $A \oplus B$ & $\overline{A}$ \\ 
   \hline
   0 & 0 & 0 & 0 & 1 & 1 & 0 & 1 \\
   \hline
   0 & 1 & 0 & 1 & 1 & 0 & 1 & 1 \\
   \hline
   1 & 0 & 0 & 1 & 1 & 0 & 1 & 0 \\
   \hline
   1 & 1 & 1 & 1 & 0 & 0 & 0 & 0 \\
   \hline
\end{tabular}
   \end{center}
\end{enumerate}
\end{document}