\documentclass[9pt]{article}

\usepackage{amssymb}
\usepackage{amsmath}
\usepackage{amsfonts}
\usepackage{comment}
\usepackage{fancyhdr}
\usepackage{mathrsfs}
\usepackage{graphicx}

\voffset = -50pt
%\textheight = 700pt
\addtolength{\textwidth}{60pt}
\addtolength{\evensidemargin}{-30pt}
\addtolength{\oddsidemargin}{-30pt}
%\setlength{\headheight}{44pt}

\pagestyle{fancy}
\fancyhf{} % clear all fields
\fancyhead[R]{%
  \scshape
  \begin{tabular}[t]{@{}r@{}}
  CECS 285, Spring 2015\\Section 1 (4344)\\
  Lab \#7, Due: 2015, March 25
  \end{tabular}}
\fancyhead[L]{%
  \scshape
  \begin{tabular}[t]{@{}r@{}}
  JOSEPH OKONOBOH, 013755064\\Computer Science\\Cal State Long Beach
  \end{tabular}}
\fancyfoot[C]{\thepage}

\begin{document}
\begin{enumerate}
%%%%%%%%%%%%%%%%%%%%%%%%%%%%%%%%%%%%%Prob1%%%%%%%%%%%%%%%%%%%%%%%%%%%%%%%%%%%%%%
   \item \begin{verbatim}; JOSEPH OKONOBOH
; 0137755064
; SWITCH BASED-COMBO LOCK
      
      MASK     EQU      0x0F
      LED      EQU      P0       ;LED port
      switch   EQU      P1       ;switches port
      num1     EQU      0x04     ;1st input
      num2     EQU      0x05     ;2nd input
      num3     EQU      0x07     ;3rd input
      num4     EQU      0x0F     ;final input
      s0       EQU      0x00     ;state 0
      s1       EQU      0x01     ;state 1
      s2       EQU      0x02     ;state 2
      s3       EQU      0x03     ;state 3
      state    EQU      0x07

;----------   Initialization   ----------;
      MOV      LED,  #0x00    ;set Port as output
      MOV      R7,   #s0      ;R7 contains present state
      MOV      LED,  #s0

; read data from switch into accumulator
READ_DATA:
      MOV      A,    switch
      ANL      A,    #MASK
      

STATE0:
      CJNE     R7,   #s0,     STATE1      ;if not in state0 go to state 1
      CJNE     A,    #num1,   READ_DATA   ;if first input is wrong re-read data from switch
      MOV      R7,   #s1                  ;go to state 1
      MOV      LED,  #s1                  ;signal user that state was successfully changed
      SJMP     READ_DATA                  ;read next input
STATE1:
      CJNE     R7,   #s1,     STATE2      ;if not in state1 go to state 2
      CJNE     A,    #num2,   READ_DATA   ;if second input is wrong re-read data from switch
      MOV      R7,   #s2                  ;go to state 2
      MOV      LED,  #s2                  ;signal user that input two was correct
      SJMP     READ_DATA                  ;read next input
STATE2:
      CJNE     R7,   #s2,     STATE3      ;if not in state2 go to state 3
      CJNE     A,    #num3,   READ_DATA   ;if third input is wrong re-read data from switch
      MOV      R7,   #s3                  ;go to state 2
      MOV      LED,  #s3                  ;signal user that input three was correct
      SJMP     READ_DATA                  ;read next input
STATE3:
      CJNE     R7,   #s3,     READ_DATA   ;if not in state3 or if input is wrong read data from switch
      CJNE     A,    #num4,   READ_DATA

;;;;; Repeatedly display LED for 1s and turn off led for 1s
BLINK:
      MOV      LED,  #0xFF      
      ACALL    DELAY
      MOV      LED,  #0x00      
      ACALL    DELAY
      SJMP     BLINK

;;;;; 0.5Hz timing delay (1s ON, 1s OFF)
DELAY:
      MOV      R0,   #175
      
LOOP0:
      MOV      R1,   #250		
LOOP1:
      MOV      R2,   #250
LOOP2:	
      DJNZ     R2,   LOOP2      
      DJNZ     R1,   LOOP1

      DJNZ     R0,   LOOP0
      RET


END
         \end{verbatim}
\end{enumerate}
\end{document}
