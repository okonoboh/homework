\documentclass[9pt]{article}

\usepackage{amssymb}
\usepackage{amsmath}
\usepackage{amsfonts}
\usepackage{comment}
\usepackage{fancyhdr}
\usepackage{mathrsfs}

\voffset = -50pt
%\textheight = 700pt
\addtolength{\textwidth}{60pt}
\addtolength{\evensidemargin}{-30pt}
\addtolength{\oddsidemargin}{-30pt}
%\setlength{\headheight}{44pt}

\pagestyle{fancy}
\fancyhf{} % clear all fields
\fancyhead[R]{%
  \scshape
  \begin{tabular}[t]{@{}r@{}}
  CECS 285, Spring 2015\\Section 1 (4344)\\
  HW \#5, Due: 2015, March 18
  \end{tabular}}
\fancyhead[L]{%
  \scshape
  \begin{tabular}[t]{@{}r@{}}
  JOSEPH OKONOBOH, 013755064\\Computer Science\\Cal State Long Beach
  \end{tabular}}
\fancyfoot[C]{\thepage}

\begin{document}
\begin{enumerate}
%%%%%%%%%%%%%%%%%%%%%%%%%%%%%%%%%%%%%Prob1%%%%%%%%%%%%%%%%%%%%%%%%%%%%%%%%%%%%%%
   \item[Activities 1 \& 2.] \begin{verbatim}
      MOV      DPTR, #MYDATA     ;address of string
      MOV      R0, #30H          ;destination address


LOOP:
      CLR      A                 ;A <- 0
      MOVC     A, @A+DPTR        ;A <- Next Character
      JZ       ACT2              ;Go to Activity two if character is null
      MOV      @R0, A            ;copy to destination
      INC      DPTR              ;address of next character
      INC      R0                ;address of next destination
      SJMP     LOOP              ;try to get next character


ACT2:								
      MOV      R0, #40H          ;R0 <- 0x40
      MOV      R1, #60H          ;R1 <- 0x60
	
LOOP2:
      MOV      A, @R0            ;A <- RAM[R0]
      MOV      @R1, A            ;RAM[R1] <- RAM[R0]
      INC      R0                ;R0 <- R0 + 1
      INC      R1                ;R1 <- R1 + 1
      CJNE     R0, #60H, LOOP2   ;Keep copying RAM[R0] into RAM[R1]
                                 ;unitl R0 == 0x60


      ORG      200H              ;address of string

MYDATA:
      DB       "Joseph Okonoboh", 0

END
         \end{verbatim}
%%%%%%%%%%%%%%%%%%%%%%%%%%%%%%%%%%%%%Prob2%%%%%%%%%%%%%%%%%%%%%%%%%%%%%%%%%%%%%%
   \item[Activity 3.] \begin{verbatim}
      MOV      DPTR, #MYDATA     ;address of y values

      MOV      A, R0             ;A <- x
      MOVC     A, @A+DPTR        ;A <- x^2 + 2x + 9
      MOV      R2, A             ;R2 <- x^2 + 2x + 9


      ORG      200H
MYDATA:
      DB       9, 12, 17, 24, 33, 44, 57, 72, 89, 108
END
		
         \end{verbatim}
%%%%%%%%%%%%%%%%%%%%%%%%%%%%%%%%%%%%%Prob3%%%%%%%%%%%%%%%%%%%%%%%%%%%%%%%%%%%%%%
   \item[1.] Explain the difference between the following two instructions:
   \begin{verbatim}
MOVC  A, @A+DPTR
MOV   A, @R0
         \end{verbatim}
         
\textbf{Answer.} Both instructions are register indirect addressing; however,
the first instruction is capable of accessing data stored in external RAM or
ROM since the \verb|DPTR| register is 16-bits, while the latter instruction is
limited to accessing internal RAM. Additionally, in the first instruction, the
value in the \verb|A| register is used as an offset to the address in the
\verb|DPTR| register.
   \item[2.] The invalid instructions are:
   \begin{verbatim}
MOV   A, @R2
MOVC  A, @R0+DPTR
         \end{verbatim}
   \item[3.] Explain the difference between the following two instructions:
   \begin{verbatim}
MOV   A, 40H
MOV   A, #40H
         \end{verbatim}
         
\textbf{Answer.} The first instruction reads the byte at RAM location 0x40
into register \verb|A|, while the latter instruction stores the decimal number
64 in \verb|A|.
   \item[4.] Explain the difference between the following two instructions:
   \begin{verbatim}
MOV   40H, A
MOV   40H, #0A
         \end{verbatim}
         
\textbf{Answer.} The first instruction copies the value in register \verb|A|
into the RAM location at 0x40, while the latter instruction copies the decimal
value 10 into the same RAM location.
   \item[5.] Give the RAM address for the following registers.
   
             \begin{tabular}{@{}l c l@{}}
                \verb|A| & = & 0xE0 \\
                \verb|B| & = & 0xF0 \\
                \verb|R0| & = & 0x00 \\
                \verb|R2| & = & 0x02 \\
                \verb|PSW| & = & 0xD0 \\
                \verb|SP| & = &  0x81 \\
                \verb|DPL| & = & 0x82 \\
                \verb|DPH| & = & 0x83 \\
             \end{tabular}
\end{enumerate}
\end{document}
