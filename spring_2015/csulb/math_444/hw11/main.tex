\documentclass[9pt]{article}

\usepackage{amssymb}
\usepackage{amsmath}
\usepackage{amsfonts}
\usepackage{comment}
\usepackage{fancyhdr}
\usepackage{mathrsfs}
\usepackage{enumitem}


\usepackage{tikz}

\voffset = -50pt
%\textheight = 700pt
\addtolength{\textwidth}{60pt}
\addtolength{\evensidemargin}{-30pt}
\addtolength{\oddsidemargin}{-30pt}
%\setlength{\headheight}{44pt}

\pagestyle{fancy}
\fancyhf{} % clear all fields
\fancyhead[R]{%
  \scshape
  \begin{tabular}[t]{@{}r@{}}
  MATH 444, Spring 2015\\Section 1 (5562)\\
  HW \#11, DUE: 2015, APRIL 27
  \end{tabular}}
\fancyhead[L]{%
  \scshape
  \begin{tabular}[t]{@{}r@{}}
  JOSEPH OKONOBOH\\Mathematics\\Cal State Long Beach
  \end{tabular}}
\fancyfoot[C]{\thepage}

\newcommand{\qed}{\hfill \ensuremath{\Box}}


\newcommand*\circled[1]{\tikz[baseline=(char.base)]{
            \node[shape=circle,draw,inner sep=2pt] (char) {#1};}}

\newcommand{\Z}{\mathbb{Z}}
\newcommand{\I}{\mathbb{I}}
\newcommand{\M}{\mathbb{M}}
\newcommand{\Q}{\mathbb{Q}}
\newcommand{\R}{\mathbb{R}}
\newcommand{\C}{\mathbb{C}}
\newcommand{\D}{\displaystyle}
%\setcounter{section}{-1}

\begin{document}
\begin{enumerate}
%%%%%%%%%%%%%%%%%%%%%%%%%%%%%%%%%%%%%Prob1%%%%%%%%%%%%%%%%%%%%%%%%%%%%%%%%%%%%%%
   \item Consider the veracity or falsehood of each of the following statements.
         For bonus, argue for those that you believe are true while providing a
         counterexample for those that you believe are false.

         \begin{enumerate}[label=\protect\circled{\arabic*}]
            \item $\Z(\sqrt{2})$ is a UFD.
            \item If $\alpha : R \to S$ is a ring homomorphism, then it is
                  one-to-one if the only $r \in R$ satisfying $\alpha(r) = 0$
                  is $r = 0$.
            \item Every integral domain is a field.
            \item If every element of a ring is an idempotent, then the ring is
                  commutative.
            \item The group of units of $\mathcal{M}_2(\Z_3)$ has 56 elements.
         \end{enumerate}
         
      \textbf{Solution.}

      \begin{enumerate}[label=\protect\circled{\arabic*}]
         \item True.
         \item True.

               \textbf{Proof.} Let $\alpha : R \rightarrow S$ be a ring
               homomorphism with a trivial kernel. Suppose
               $\alpha(a) = \alpha(b)$. Then it follows that
               $\alpha(a - b) = \alpha(a) - \alpha(b) = 0$. Since the kernel of
               $\alpha$ is trivial, we must have that $a - b = 0$; that is,
               $a = b$. Thus $\alpha$ is injective. \qed
         \item False.

               \textbf{Counterexample.} $\Z$ is an integral domain, but it is
               not a field.
         \item True.

               \textbf{Proof.} Suppose that every element of a ring $R$ is
               idempotent. Let $x \in R$. Then we have that
               \begin{align*}
                  x + x  &= (x + x)(x + x) \\
                         &= x^2 + x^2 + x^2 + x^2 \\
                         &= x + x + x + x,
               \end{align*}
               so that $x + x = 0$; thus $x = -x$ for all $x \in R$. Similarly
               we have that
               \begin{align*}
                  x + y  &= (x + y)(x + y) \\
                         &= x^2 + xy + yx + y^2 \\
                         &= x + xy + yx + y,
               \end{align*}
               so that $xy + yx = 0$. That is $xy + yx = xy - yx = 0$. Hence
               $xy = yx$, so that $R$ is commutative. \qed
         \item False. A matrix is in $\mathcal{M}_2(\Z_3)$ if and only if its
               row vectors are linearly independent. But the number of matrices
                in $\mathcal{M}_2(\Z_3)$ with linearly independent rows is
               $$(3^2 - 1)(3^2-3) = 48,$$
               so that $|\mathcal{M}_2(\Z_3)| = 48$.
      \end{enumerate}
%%%%%%%%%%%%%%%%%%%%%%%%%%%%%%%%%%%%%Prob2%%%%%%%%%%%%%%%%%%%%%%%%%%%%%%%%%%%%%%
   \item Let
         $A = \left(\begin{tabular}{@{}cc@{}}
            $a$ & $b$ \\
            $2b$ & $a$
         \end{tabular}\right)$ be an element of $\Z(\sqrt{2})$. Suppose that
         $\det A$ is even. Show there is an element of $\Z(\sqrt{2})$ whose
         determinant is $\frac{1}{2}\det A$. \textbf{Hint.} Find an element of
         determinant 2 and show that you can divide $A$ by it.
         
      \textbf{Proof.} Since $\det A$ is even, we have that
      $\det A = 2k = a^2 - 2b^2$ for some integer $k$. Now we have that
      $a^2 = 2b^2 + 2k = 2(b^2 + k)$, so that $a^2$---and thus $a$---is even.
      Let $B = \left(\begin{tabular}{@{}cc@{}}
         2 & 1 \\
         2 & 2
      \end{tabular}\right)$. Solving the equation $A = BX$ where
      $X = \left(\begin{tabular}{@{}cc@{}}
         $x$ & $y$ \\
         $2y$ & $x$
      \end{tabular}\right)$ will result in $x = a - b$ and
      $y = b - a/2$. Since $a$ is even, it follows that $a/2$ is an integer, so
      that $y$ is an integer. Thus $X \in \Z(\sqrt{2})$. Hence
      $$2k = \det A = \det(BX) = \det(B)\det(X) = 2\det(X),$$
      so that $\det X = k = \frac{1}{2}\det A$, as desired. \qed
%%%%%%%%%%%%%%%%%%%%%%%%%%%%%%%%%%%%%Prob3%%%%%%%%%%%%%%%%%%%%%%%%%%%%%%%%%%%%%%
   \item \textbf{On Factoring.}

         \begin{enumerate}[label=\protect\circled{\arabic*}]
            \item Give all irreducible cubics over $\Z_2$.
            \item Factor $p(x) = x^6 + x^5 + x^4 + x^3 + x^2 + x + 1$ completely
                  in $\Z_2[x]$.
            \item How many monic quadratics are there in $\Z_3[x]$?
            \item Find all irreducible monic quadratics over $\Z_3$.
            \item Factor $q(x) = x^6 + x^3 - x^2 - x$ completely in $\Z_3[x]$.
            \item Consider the mod function from $\Z$ to $\Z_3$, and its natural
                  extension to a homomorphism from $\Z[x]$ to $\Z_3[x]$ that
                  mods out the coefficients, for example $5x$ goes to $2x$. Find
                  the image under this homomorphism of the polynomial
                  $$h(x) = x^6 + 9x^5 + 21x^4 + 37x^3 + 53x^2 + 29x + 15.$$
            \item Do the same as in \circled{6} except now one mods out 2.
            \item Use parts \circled{2} and \circled{5} to discuss the possible
                  factorization of $h(x)$. For example, does it have any integer
                  roots? Or is it irreducible etc. Discuss as much as you can.
         \end{enumerate}
         
      \textbf{Solution.}

      \begin{enumerate}[label=\protect\circled{\arabic*}]
         \item The irreducible cubics over $\Z_2$ are:
               $$x^3 + x + 1 \text{ and } x^3 + x^2 + 1.$$
         \item $p(x) = (x^3 + x + 1)(x^3 + x^2 + 1)$.
         \item There are 9 monic quadratics in $\Z_3[x]$.
         \item The irreducible monic quadratics over $\Z_3[x]$ are:
               $$x^2+1, x^2+x+2, \text{ and }x^2+2x+2.$$
         \item $q(x) = x(x+1)(x+2)^2(x^2+x+2)$.
         \item The image of $h(x)$ under this homomorphism is:
               $$x^6+x^3+2x^2+2x.$$
         \item The image of $h(x)$ under this homomorphism is:
               $$x^6+x^5+x^4+x^3+x^2+x+1.$$
         \item From \circled{2}, we know that $h(x)$ mod 2 doesn't have a root
               in $\Z_2[x]$; thus $h(x)$ doesn't have a root in $\Z$ by Homework 
               10 Problem 1.2. That is $h(x)$ has no linear factors, and by
               extension, no quintic factor. Although we know from \circled{5}
               that $h(x)$ mod 3 is not irreducible in $\Z_3[x]$, we cannot 
               conclude that $h(x)$ is not irreducible in $\Z$ (Homework 10 
               Problem 1.4). Now $p(x)$ does not have a quadratic factor because
               its image in $\Z_2[x]$ does not have a quadratic factor. Thus if
               $p(x)$ has a factorization in $\Z$, then it must necessarily be
               into two cubics.
      \end{enumerate}
%%%%%%%%%%%%%%%%%%%%%%%%%%%%%%%%%%%%%Prob4%%%%%%%%%%%%%%%%%%%%%%%%%%%%%%%%%%%%%%
   \item \textbf{On Factorization.}

         \begin{enumerate}[label=\protect\circled{\arabic*}]
            \item Find the complete factorization of $x^5 - 1$ in $\Q[x]$.
            \item Find real numbers $a$ and $b$ such that
                  $$(x^2+ax+1)(x^2+bx+1)=x^4+x^3+x^2+x+1.$$
            \item Find the complete factorization of $x^5 - 1$ in $\R[x]$.
            \item Find the complete factorization of $x^5 - 1$ in $\C[x]$.
            \item Find the complete factorization of $x^5 - 1$ in $\Z_{11}[x]$.
            \item Find the complete factorization of $x^5 - 1$ in $\Z_{31}[x]$.
            \item Find the complete factorization of $x^{10} - 1$ in $\Q[x]$.
         \end{enumerate}
         
      \textbf{Solution.}

      \begin{enumerate}[label=\protect\circled{\arabic*}]
         \item In $\Q[x]$, we have $x^5 - 1 = (x-1)(x^4+x^3+x^2+x+1)$.
         \item Solving the equation
               $$(x^2+ax+1)(x^2+bx+1)=x^4+x^3+x^2+x+1,$$
               for real numbers $a$ and $b$, we shall get
               $$a = \frac{1+\sqrt{5}}{2} \text{ and }b =\frac{1-\sqrt{5}}{2}.$$
         \item In $\R[x]$, we have
               $x^5 - 1 = (x-1)\left(x^2+\D\frac{1+\sqrt{5}}{2}x+1\right)
                \left(x^2+\D\frac{1-\sqrt{5}}{2}x+1\right)$.
         \item In $\C[x]$, we have
               $$x^5 - 1 = (x-1)(x - (A+Bi))(x - (A-Bi))(x-(C+Di))(x-(C-Di)),$$
               where $A = \D\frac{-1-\sqrt{5}}{4}$, $B = \D\frac{1}{2}
               \sqrt{\frac{5-\sqrt{5}}{2}}$, $C = \D\frac{\sqrt{5}-1}{4}$, and
               $D = \D\frac{1}{2}\sqrt{\frac{5+\sqrt{5}}{2}}$.
         \item In $\Z_{11}[x]$, we have
               $$x^5-1 = (x+2)(x+6)(x+7)(x+8)(x+10).$$
         \item In $\Z_{31}[x]$, we have
               $$x^5-1 = (x+15)(x+23)(x+27)(x+29)(x+30).$$
         \item In $\Q[x]$, we have $x^{10} - 1 = (x-1)(x+1)(x^4-x^3+x^2-x+1)
               (x^4+x^3+x^2+x+1)$.
      \end{enumerate}
\end{enumerate}
\end{document}
