\documentclass[9pt]{article}

\usepackage{amssymb}
\usepackage{amsmath}
\usepackage{amsthm}
\usepackage{amsfonts}
\usepackage{comment}
\usepackage{fancyhdr}
\usepackage{mathrsfs}
\usepackage{enumitem}


\usepackage{tikz}

\voffset = -50pt
%\textheight = 700pt
\addtolength{\textwidth}{60pt}
\addtolength{\evensidemargin}{-30pt}
\addtolength{\oddsidemargin}{-30pt}
%\setlength{\headheight}{44pt}

\pagestyle{fancy}
\fancyhf{} % clear all fields
\fancyhead[R]{%
  \scshape
  \begin{tabular}[t]{@{}r@{}}
  MATH 444, Spring 2015\\Section 1 (5562)\\
  PROJECT, DUE: 2015, MAY 04
  \end{tabular}}
\fancyhead[L]{%
  \scshape
  \begin{tabular}[t]{@{}r@{}}
  JOSEPH OKONOBOH/ KAVEH DADBIN \\Mathematics\\Cal State Long Beach
  \end{tabular}}
\fancyfoot[C]{\thepage}

%\newcommand{\qed}{\hfill \ensuremath{\Box}}


%\newcommand*\circled[1]{\tikz[baseline=(char.base)]{
%            \node[shape=circle,draw,inner sep=2pt] (char) {#1};}}

\newcommand{\Z}{\mathbb{Z}}
\newcommand{\I}{\mathbb{I}}
\newcommand{\M}{\mathbb{M}}
\newcommand{\Q}{\mathbb{Q}}
\newcommand{\R}{\mathbb{R}}
\newcommand{\C}{\mathbb{C}}
\newcommand{\D}{\displaystyle}
\newtheorem {thm}{Theorem}
\newtheorem {prop}{Proposition}
%\setcounter{section}{-1}

\begin{document}

\begin{prop}
\end{prop}
\begin{enumerate}
%%%%%%%%%%%%%%%%%%%%%%%%%%%%%%%%%%%%%Prob1%%%%%%%%%%%%%%%%%%%%%%%%%%%%%%%%%%%%%%
   \item A prime $p$ can be written as a sum of two integer squares, $a^2 + b^2$,
         if and only if $p = 2$ or $p \equiv 1$ mod 4. Except for changing the
         signs of $a$ and $b$ or switching $a$ and $b$, the representation of
         $p$ as a sum of integer squares is unique.
   \item Recall that the units in $\Z[i]$ are
         $$
            \pm\left(\begin{tabular}{@{}rc@{}}
               0 & 1 \\
               $-1$ & 0
            \end{tabular}\right) \text{ and }
            \pm\left(\begin{tabular}{@{}cc@{}}
               1 & 0 \\
               0 & 1
            \end{tabular}\right).
         $$
         Now we have that
         the irreducibles, up to units, in $\Z[i]$ are:
         \begin{enumerate}
            \item $\left(\begin{tabular}{@{}rc@{}}
                     1 & 1 \\
                     $-1$ & 1
                  \end{tabular}\right)$ (with determinant 2),
            \item $\left(\begin{tabular}{@{}cc@{}}
                     $p$ & 0 \\
                     0 & $p$
                  \end{tabular}\right)$ (with determinant $p^2$), where $p$ is a
                  prime in $\Z$ such that $p \equiv 3$ mod 4,
            \item Distinct (i.e., not associates) irreducibles
                  $\left(\begin{tabular}{@{}rc@{}}
                     $a$ & $b$ \\
                     $-b$ & $a$
                  \end{tabular}\right)$, $\left(\begin{tabular}{@{}cr@{}}
                     $a$ & $-b$ \\
                     $b$ & $a$
                  \end{tabular}\right)$, where $a^2 + b^2 = p$ is a prime in
                  $\Z$ such that $p \equiv 1$ mod 4. We shall call these pair
                  of irreducibles conjugates.
         \end{enumerate}   
\end{enumerate}

\begin{thm}
A positive integer $n$ can be written as a sum of two integer squares if and
only if it has an even number of factors of primes $q$, where $q \equiv 3$
mod 4. Moreover if we factor $n$ into primes:
$$n = 2^k{p_1}^{c_1}\cdots {p_r}^{c_r}{q_1}^{d_1}\cdots {q_s}^{d_s},$$
where the $p_i$s are distinct odd primes with $p_i \equiv 1$ mod 4 and the
$q_j$s are distinct odd primes with $q_j \equiv 3$ mod 4, then the number of
representations of $n$ as a sum of squares is
$$4(c_1 + 1) \cdots(c_r + 1).$$
\end{thm}

\noindent\textbf{Proof.} Let $n$ be a positive integer. Since $\Z$ is a UFD, we
can write
$$n = 2^k{p_1}^{c_1}\cdots {p_r}^{c_r}{q_1}^{d_1}\cdots {q_s}^{d_s}$$
where $p_1, \ldots, p_r$ are distinct primes congruent to 1 modulo 4 and
$q_1, \ldots, q_s$ are distinct primes congruent to 3 modulo 4. Suppose first
that $n$
has an even number of factors of primes $q$, where $q \equiv 3$ mod 4. That is,
$d_1, \ldots, d_s$ are even. By Proposition 1, there exist integers $a_i$ and
$b_i$ such that ${a_i}^2 + {b_i}^2 = {p_i}$ for $i = 1, \ldots, r$. Let
$$ X = 
   \left(\begin{tabular}{@{}rc@{}}
      1 & 1 \\
      $-1$ & 1
   \end{tabular}\right)^k
   \left(\begin{tabular}{@{}rc@{}}
      $a_1$ & $b_1$ \\
      $-b_1$ & $a_1$
   \end{tabular}\right)^{c_1}\cdots
   \left(\begin{tabular}{@{}rc@{}}
      $a_r$ & $b_r$ \\
      $-b_r$ & $a_r$
   \end{tabular}\right)^{c_r}
   \left(\begin{tabular}{@{}cc@{}}
      $q_1$ & 0 \\
      0 & $q_1$
   \end{tabular}\right)^{d_1/2}\cdots
   \left(\begin{tabular}{@{}cc@{}}
      $q_s$ & 0 \\
      0 & $q_s$
   \end{tabular}\right)^{d_s/2}.
$$

\noindent Notice that $X \in \Z[i]$ so that $X = \left(\begin{tabular}{@{}rc@{}}
   $x$ & $y$ \\
   $-y$ & $x$
\end{tabular}\right)$ and $x^2 + y^2 = \det(X) = n$, so that $n$ is the sum of
two integer squares. Conversely suppose that $n = a^2 + b^2$. Let
$B = \left(\begin{tabular}{@{}rc@{}}
   $a$ & $b$ \\
   $-b$ & $a$
\end{tabular}\right)$. Since $\Z[i]$ is a UFD we have the following
factorization of $B$ (up to units) into irreducibles from Proposition 1:
$$B = \left(\begin{tabular}{@{}rc@{}}
   1 & 1 \\
   $-1$ & 1
\end{tabular}\right)^u{A_1}^{e_1}{A_1'}^{e_1'}\cdots{A_m}^{e_m}{A_m}^{e_m'}{C_1}^{f_1}\cdots
{C_t}^{f_t},$$
where the $A_i$ and $A_i'$ are conjugate irreducibles, the determinant of $A_i$
(which is equal to the determinant of $A_i'$) is a prime
congruent to 1 modulo 4, and the determinant of $C_j$ is the square of a prime
such that this prime is congruent to 3 modulo 4. Let
$$\det(A_i) = \pi_i \text{ and } \det(C_j) = {\alpha_j}^2.$$
Thus it follows that
$$n = a^2 + b^2 = \det(B)
  = 2^u{\pi_1}^{e_1+e_1'}\cdots{\pi_m}^{e_m+e_m'}{\alpha_1}^{2f_1}\cdots{\alpha_t}^{2f_t}.$$
The above equality
$$n = 2^u{\pi_1}^{e_1}\cdots{\pi_m}^{e_m}{\alpha_1}^{2f_1}\cdots{\alpha_t}^{2f_t}$$
gives us a factorization of $n$ into primes and it is clear that there are an
even number of prime factors $q$ of $n$ such that $q \equiv 3$ mod 4. To
complete the proof, observe the prime factorizations of $n$:
$$2^k{p_1}^{c_1}\cdots {p_r}^{c_r}{q_1}^{d_1}\cdots {q_s}^{d_s} \text{ and }
{2^u\pi_1}^{e_1+e_1'}\cdots{\pi_m}^{e_m+e_m'}{\alpha_1}^{2f_1}\cdots{\alpha_t}^{2f_t}.$$

\noindent Since prime factorizations are unique (up to order and units) in $\Z$,
it follows that $r = m$, $s = t$, and $u = k$. Also we can assume without loss
that $p_i = \pi_i$, $q_j = \alpha_j$, and $c_i = e_i+e_i'$, $\alpha_j = q_j$, so
that $f_j  = d_j/2$. To complete the proof, notice from the equation
$$c_i = e_i + e_i'$$
that since $e_i$ can take on values $0, \ldots, c_i$ it follows that there
are $c_i + 1$ nonnegative pair of solutions for $(e_i, e_i')$. Thus there are
at least $(c_1 + 1)\cdots(c_r + 1)$ choices for $B$. Since there are four units
in $\Z[i]$, it follows that there are $4(c_1 + 1)\cdots(c_r + 1)$ choices for
$B$.
\end{document}