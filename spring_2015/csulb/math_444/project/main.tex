\documentclass[9pt]{article}

\usepackage{amssymb}
\usepackage{amsmath}
\usepackage{amsthm}
\usepackage{amsfonts}
\usepackage{comment}
\usepackage{fancyhdr}
\usepackage{mathrsfs}
\usepackage{enumitem}


\usepackage{tikz}

\voffset = -50pt
%\textheight = 700pt
\addtolength{\textwidth}{60pt}
\addtolength{\evensidemargin}{-30pt}
\addtolength{\oddsidemargin}{-30pt}
%\setlength{\headheight}{44pt}

\pagestyle{fancy}
\fancyhf{} % clear all fields
\fancyhead[R]{%
  \scshape
  \begin{tabular}[t]{@{}r@{}}
  MATH 444, Spring 2015\\Section 1 (5562)\\
  PROJECT, DUE: 2015, MAY 04
  \end{tabular}}
\fancyhead[L]{%
  \scshape
  \begin{tabular}[t]{@{}r@{}}
  JOSEPH OKONOBOH\\Mathematics\\Cal State Long Beach
  \end{tabular}}
\fancyfoot[C]{\thepage}

%\newcommand{\qed}{\hfill \ensuremath{\Box}}


%\newcommand*\circled[1]{\tikz[baseline=(char.base)]{
%            \node[shape=circle,draw,inner sep=2pt] (char) {#1};}}

\newcommand{\Z}{\mathbb{Z}}
\newcommand{\I}{\mathbb{I}}
\newcommand{\M}{\mathbb{M}}
\newcommand{\Q}{\mathbb{Q}}
\newcommand{\R}{\mathbb{R}}
\newcommand{\C}{\mathbb{C}}
\newcommand{\D}{\displaystyle}
\newtheorem {thm}{Theorem}
\newtheorem {prop}{Proposition}
%\setcounter{section}{-1}

\begin{document}

\begin{prop}
\end{prop}
\begin{enumerate}
%%%%%%%%%%%%%%%%%%%%%%%%%%%%%%%%%%%%%Prob1%%%%%%%%%%%%%%%%%%%%%%%%%%%%%%%%%%%%%%
   \item A prime $p$ can be written as a sum of two integer squares if and only
         if $p = 2$ or $p \equiv 1$ mod 4.
   \item The irreducibles in $\Z[i]$ are:
         \begin{enumerate}
            \item $\left(\begin{tabular}{@{}rc@{}}
                     1 & 1 \\
                     $-1$ & 1
                  \end{tabular}\right)$,
            \item $\left(\begin{tabular}{@{}cc@{}}
                     $p$ & 0 \\
                     0 & $p$
                  \end{tabular}\right)$, where $p$ is a prime in $\Z$ such that
                  $p \equiv 3$ mod 4,
            \item Distinct conjugates (i.e., not associates)
                  $\left(\begin{tabular}{@{}rc@{}}
                     $a$ & $b$ \\
                     $-b$ & $a$
                  \end{tabular}\right)$, $\left(\begin{tabular}{@{}cr@{}}
                     $a$ & $-b$ \\
                     $b$ & $a$
                  \end{tabular}\right)$, where $a^2 + b^2 = p$ is a prime in
                  $\Z$ such that $p \equiv 1$ mod 4.
         \end{enumerate}   
\end{enumerate}

\begin{thm}
A positive integer $n$ can be written as a sum of two integer squares if and
only if it has an even number of factors of primes $q$, where $q \equiv 3$
mod 4. Moreover if we factor $n$ into primes:
$$n = 2^k{p_1}^{c_1}\cdots {p_r}^{c_r}{q_1}^{d_1}\cdots {q_s}^{d_s},$$
where the $p_i$s are distinct odd primes with $p_i \equiv 1$ mod 4 and the
$q_j$s are distinct odd primes with $q_j \equiv 3$ mod 4, then the number of
representations of $n$ as a sum of squares is
$$4(c_1 + 1) \cdots(c_r + 1).$$
\end{thm}

\textbf{Proof.} Suppose first that $n$ is an integer that has an even number of
factors of primes $p$, where $p \equiv 3$ mod 4. Thus we can write $n$ as a
product of primes
$$n = 2^k{p_1}^{c_1}\cdots {p_r}^{c_r}{q_1}^{2d_1}\cdots {q_s}^{2d_s}$$
where $p_1, \ldots, p_r$ are distinct primes congruent 1 mod 4 and
$q_1, \ldots, q_s$ are distinct primes congruent 3 mod 4. By Proposition 1,
there exist integers $a_i$ and $b_i$ such that ${a_i}^2 + {b_i}^2 = {p_i}^2$
for $i = 1, \ldots, r$. Let
$$ X = 
   \left(\begin{tabular}{@{}rc@{}}
      1 & 1 \\
      $-1$ & 1
   \end{tabular}\right)^k
   \left(\begin{tabular}{@{}rc@{}}
      $a_1$ & $b_1$ \\
      $-b_1$ & $a_1$
   \end{tabular}\right)^{c_1}\cdots
   \left(\begin{tabular}{@{}rc@{}}
      $a_r$ & $b_r$ \\
      $-b_r$ & $a_r$
   \end{tabular}\right)^{c_r}
   \left(\begin{tabular}{@{}cc@{}}
      $q_1$ & 0 \\
      0 & $q_1$
   \end{tabular}\right)^{d_1}\cdots
   \left(\begin{tabular}{@{}cc@{}}
      $q_s$ & 0 \\
      0 & $q_s$
   \end{tabular}\right)^{d_s}.$$
Notice that $X \in \Z[i]$ and $\det(X) = n$, so that $n$ is the sum of two
integer squares.
\end{document}
