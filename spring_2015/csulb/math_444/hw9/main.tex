\documentclass[9pt]{article}

\usepackage{amssymb}
\usepackage{amsmath}
\usepackage{amsfonts}
\usepackage{comment}
\usepackage{fancyhdr}
\usepackage{mathrsfs}
\usepackage{enumitem}


\usepackage{tikz}

\voffset = -50pt
%\textheight = 700pt
\addtolength{\textwidth}{60pt}
\addtolength{\evensidemargin}{-30pt}
\addtolength{\oddsidemargin}{-30pt}
%\setlength{\headheight}{44pt}

\pagestyle{fancy}
\fancyhf{} % clear all fields
\fancyhead[R]{%
  \scshape
  \begin{tabular}[t]{@{}r@{}}
  MATH 444, Spring 2015\\Section 1 (5562)\\
  HW \#9, DUE: 2015, APRIL 15
  \end{tabular}}
\fancyhead[L]{%
  \scshape
  \begin{tabular}[t]{@{}r@{}}
  JOSEPH OKONOBOH\\Mathematics\\Cal State Long Beach
  \end{tabular}}
\fancyfoot[C]{\thepage}

\newcommand{\qed}{\hfill \ensuremath{\Box}}


\newcommand*\circled[1]{\tikz[baseline=(char.base)]{
            \node[shape=circle,draw,inner sep=2pt] (char) {#1};}}

\newcommand{\Z}{\mathbb{Z}}
\newcommand{\I}{\mathbb{I}}
\newcommand{\M}{\mathbb{M}}
\newcommand{\Q}{\mathbb{Q}}
\newcommand{\R}{\mathbb{R}}
\newcommand{\C}{\mathbb{C}}
\newcommand{\D}{\displaystyle}
%\setcounter{section}{-1}

\begin{document}
\begin{enumerate}
%%%%%%%%%%%%%%%%%%%%%%%%%%%%%%%%%%%%%Prob1%%%%%%%%%%%%%%%%%%%%%%%%%%%%%%%%%%%%%%
   \item Consider the veracity or falsehood of each of the following statements.
         For bonus, argue for those that you believe are true while providing a
         counterexample for those that you believe are false.

         \begin{enumerate}[label=\protect\circled{\arabic*}]
            \item Every non-constant complex polynomial has a complex root.
            \item Conjugation of complex numbers is a field automorphism of the
                  complex numbers.
            \item Let $x, y \in R$, a finite ring. If $x * y = 1$, then
                  $y * x =  1$ also.
            \item There are exactly four quadratics in $\Z_2[x]$.
            \item If $p(x)$ is a real polynomial, then it either has a real root
                  or there is a quadratic polynomial with real coefficients that
                  divides it.
         \end{enumerate}
         
      \textbf{Solution.}

      \begin{enumerate}[label=\protect\circled{\arabic*}]
         \item True.
         
               This follows from the Fundamental Theorem of Algebra.
         \item True.
         
               \textbf{Proof.} Let $\overline{a}$ denote the conjugate of the 
               complex number $a$. We now want to show that
               $$f : \C \rightarrow \C, \; c \mapsto \overline{c}$$
               is a ring isomorphism. Let $a_1$ and $a_2$ be complex numbers. 
               Since
               $\overline{a_1a_2} = \overline{a_1} \cdot \overline{a_2}$, and
               $\overline{a_1 + a_2} = \overline{a_1} + \overline{a_2}$, it
               follows that
               $$f(a_1a_2) = f(a_1)f(a_2) \text{ and }
                 f(a_1 + a_2) = f(a_1) + f(a_2),$$
               so that $f$ is a ring homomorpshim. It now remains to show that
               $f$ is a bijection. The map $f$ must be surjective because
               $f(\overline{a_1}) = a_1$. Also if $f(a_1) = f(a_2)$, then the
               real parts of $a_1$ and $a_2$ must be equal. Similarly,
               their imaginary parts must be equal, so that $a_1 = a_2$. That is
               $f$ is injective and we can conclude that it is a bijection. Thus
               $f$ is a field automorphism. \qed
         \item True.
         
               \textbf{Proof.} Let $R$ be a finite ring, and consider
               $x, y \in R$ such that $x * y = 1$. The map $f : R\rightarrow R$,
               $r \mapsto r * x$ is bijective because for $r_1, r_2 \in R$ with
               $f(r_1) = f(r_2)$, we have that $r_1 * x = r_2 * x$. We then
               cancel $x$ on both sides by multiplying each side on the right
               by $y$ to get $r_1 = r_2$; thus $f$ is injective, and since $R$
               is finite, we can conclude that $f$ is also bijective. Thus
               there exists $r_3 \in R$ such that $r_3 * x = 1$. Mutltiply the
               preceding equality on the right by $y$ to get $r_3 = y$. \qed
         \item True.
         
               There are exactly 8 polynomials in $\Z_2[x]$, and they are
               $$0, 1, x, x + 1, x^2, x^2 + 1, x^2 + x, x^2 + x + 1.$$
               
               It is clear that only four of then are quadratics.
               
         \item If $p(x)$ is 0, then it is trivially true. However, if $p(x)$ is
               a constant non-zero polynomial then it is not true. We shall now
               show that the statement is true if $p(x)$ is a non-constant real
               polynomial.
         
               \textbf{Proof.} Consider the polynomial
               $$p(x) = a_nx^n + a_{n - 1}x^{n-1} + \cdots + a_0,$$
               where each $a_i \in \R$, $a_n \neq 0$, and $n \ge 1$. By
               the Fundamental Theorem of Algebra, $p(x)$ has a root $\lambda$.
               If $\lambda$ is real, then we are done. So assume that $\lambda$
               is a non-real complex number. Observe that the conjugate of
               $\lambda$, $\overline{\lambda}$, is also a root of $p(x)$ since
               \begin{align*}
                  p(\overline{\lambda}) &= a_n\overline{\lambda}^n +
                     a_{n - 1}\overline{\lambda}^{n-1} + \cdots + a_0 \\
                     &= a_n\overline{\lambda^n} +
                     a_{n - 1}\overline{\lambda^{n-1}} + \cdots + a_0 \\
                     &= \overline{a_n}\overline{\lambda^n} +
                     \overline{a_{n - 1}}\overline{\lambda^{n-1}} + \cdots +
                     \overline{a_0} &[\overline{a} = a \;\forall a \in \R] \\
                     &= \overline{a_n\lambda^n} +
                     \overline{a_{n - 1}\lambda^{n-1}} + \cdots +
                     \overline{a_0} \\
                     &= \overline{a_n\lambda^n + a_{n - 1}\lambda^{n-1} +
                     \cdots + a_0} \\
                     &= \overline{0} = 0. &[p(\lambda) = 0]
               \end{align*}
               Since $\lambda$ is not real, we must have that
               $\lambda \neq \overline{\lambda}$. Thus the quadratic polynomial
               $(x - \lambda)(x - \overline{\lambda})$ divides $p(x)$. To
               complete the proof, we must show that this quadratic polynomial
               has real coefficients. Now we
               have that
               $$
                  (x - \lambda)(x - \overline{\lambda}) = x^2 -(\lambda + 
                     \overline{\lambda})x + \lambda\overline{\lambda}
                     = x^2 - 2\cdot\text{Re}(\lambda)x + |\lambda|^2,
               $$
               where Re($c$) and $|c|$ denote the real part and magnitude of a
               complex number $c$. Thus the quadratic polynomial
               $(x - \lambda)(x - \overline{\lambda})$ has real coefficients.
               \qed
      \end{enumerate}
%%%%%%%%%%%%%%%%%%%%%%%%%%%%%%%%%%%%%Prob2%%%%%%%%%%%%%%%%%%%%%%%%%%%%%%%%%%%%%%
   \item \textbf{On Complex \& Real}.

         \begin{enumerate}[label=\protect\circled{\arabic*}]
            \item Find a ring isomorphism (it has to be both additive and
                  multiplicative) between $\C$ and the subring
                  $\mathcal{C} = \left\{\left(\begin{tabular}{@{}rl@{}}
                     $a$ & $b$ \\
                     $-b$ & $a$
                  \end{tabular}\right) : a, b \in \R\right\} \subseteq
                  \mathcal{M}_2(\R)$.
            \item In the notes we gave two descriptions of the quaternions:
                  $$\mathcal{Q} = \left\{\left(\begin{tabular}{@{}rrrr@{}}
                     $a$ & $b$ & $c$ & $d$ \\
                     $-b$ & $a$ & $-d$ & $c$ \\
                     $-c$ & $d$ & $a$ & $-b$ \\
                     $-d$ & $-c$ & $b$ & $a$
                  \end{tabular}\right) : a, b, c, d \in \R\right\} \text{ and }
                  \mathcal{H} = \left\{\left(\begin{tabular}{@{}rl@{}}
                     $\alpha$ & $\beta$ \\
                     $-\overline{\beta}$ & $\overline{\alpha}$
                  \end{tabular}\right) : \alpha, \beta \in \C\right\}.$$
                  Find an isomorphism between these two rings (it has to be both
                  additive and multiplicative).
         \end{enumerate}
         
      \textbf{Solution.}

      \begin{enumerate}[label=\protect\circled{\arabic*}]
         \item We claim that the map
               $$f : \mathcal{C} \rightarrow \C, \left(\begin{tabular}{@{}rl@{}}
                  $a$ & $b$ \\
                  $-b$ & $a$
               \end{tabular}\right) \mapsto a + bi$$
               is a ring isomorphism.

               \textbf{Proof.} Let $\left(\begin{tabular}{@{}rl@{}}
                  $a$ & $b$ \\
                  $-b$ & $a$
               \end{tabular}\right), \left(\begin{tabular}{@{}rl@{}}
                  $c$ & $d$ \\
                  $-d$ & $c$
               \end{tabular}\right) \in \mathcal{C}$, so that
               \begin{align*}
                  f\left(\left(\begin{tabular}{@{}rl@{}}
                     $a$ & $b$ \\
                     $-b$ & $a$
                  \end{tabular}\right)\left(\begin{tabular}{@{}rl@{}}
                     $c$ & $d$ \\
                     $-d$ & $c$
                  \end{tabular}\right)\right) &=
                  f\left(\left(\begin{tabular}{@{}rl@{}}
                     $ac - bd$ & $ad + bc$ \\
                     $-(ac + bd)$ & $ac - bd$
                  \end{tabular}\right)\right) \\
                  &= (ac - bd) + (ad + bc)i \\
                  &= (a + bi)(c + di) \\
                  &= f\left(\left(\begin{tabular}{@{}rl@{}}
                        $a$ & $b$ \\
                        $-b$ & $a$
                  \end{tabular}\right)\right)
                  f\left(\left(\begin{tabular}{@{}rl@{}}
                        $c$ & $d$ \\
                        $-d$ & $c$
                  \end{tabular}\right)\right)
               \end{align*}
               and
               \begin{align*}
                  f\left(\left(\begin{tabular}{@{}rl@{}}
                     $a$ & $b$ \\
                     $-b$ & $a$
                  \end{tabular}\right) + \left(\begin{tabular}{@{}rl@{}}
                     $c$ & $d$ \\
                     $-d$ & $c$
                  \end{tabular}\right)\right) &=
                  f\left(\left(\begin{tabular}{@{}rl@{}}
                     $a+c$ & $b+d$ \\
                     $-(b+d)$ & $a+c$
                  \end{tabular}\right)\right) \\
                  &= (a + c) + (b + d)i \\
                  &= (a + bi) + (c + di) \\
                  &= f\left(\left(\begin{tabular}{@{}rl@{}}
                        $a$ & $b$ \\
                        $-b$ & $a$
                  \end{tabular}\right)\right) + 
                  f\left(\left(\begin{tabular}{@{}rl@{}}
                        $c$ & $d$ \\
                        $-d$ & $c$
                  \end{tabular}\right)\right).
               \end{align*}
               Hence $f$ is a ring homomorphism. It is clear that $f$ is
               surjective since if ${a_1 + b_1i \in \C}$, then we must have that
               $f\left(\left(\begin{tabular}{@{}rl@{}}
                  $a_1$ & $b_1$ \\
                  $-b_1$ & $a_1$
               \end{tabular}\right)\right) = a_1 + b_1i$. Now suppose that
               $$f\left(\left(\begin{tabular}{@{}rl@{}}
                        $a$ & $b$ \\
                        $-b$ & $a$
                  \end{tabular}\right)\right) =
                  f\left(\left(\begin{tabular}{@{}rl@{}}
                        $c$ & $d$ \\
                        $-d$ & $c$
                  \end{tabular}\right)\right).$$ Then we must have that
                  $a + bi = c + di$ so that $a =  b$ and $c = d$. That is, $f$
                  is injective. We can now conclude that $f$ is a ring 
                  isomorphism. \qed
         \item The map
               $$g : \mathcal{Q} \rightarrow \mathcal{H}, \left(\begin{tabular}{@{}rrrr@{}}
                     $a$ & $b$ & $c$ & $d$ \\
                     $-b$ & $a$ & $-d$ & $c$ \\
                     $-c$ & $d$ & $a$ & $-b$ \\
                     $-d$ & $-c$ & $b$ & $a$
                  \end{tabular}\right) \mapsto \left(\begin{tabular}{@{}rl@{}}
                     $a+bi$ & $c+di$ \\
                     $-c+di$ & $a-bi$
                  \end{tabular}\right)$$
               is clearly bijective. For $$A = \left(\begin{tabular}{@{}rrrr@{}}
                  $a$ & $b$ & $c$ & $d$ \\
                  $-b$ & $a$ & $-d$ & $c$ \\
                  $-c$ & $d$ & $a$ & $-b$ \\
                  $-d$ & $-c$ & $b$ & $a$
               \end{tabular}\right) \text{ and }
               B = \left(\begin{tabular}{@{}cccc@{}}
                  $k$ & $l$ & $m$ & $n$ \\
                  $-l$ & $k$ & $-n$ & $m$ \\
                  $-m$ & $n$ & $k$ & $-l$ \\
                  $-n$ & $-m$ & $l$ & $k$
               \end{tabular}\right) \in \mathcal{Q},$$ we have that
               \begin{align*}
                  g(A + B) &= \left(\begin{tabular}{@{}cccc@{}}
                  $a+k$ & $b+l$ & $c+m$ & $d+n$ \\
                  $-(b+l)$ & $a+k$ & $-(d+n)$ & $c+m$ \\
                  $-(c+m)$ & $d+n$ & $a+k$ & $-(b+l)$ \\
                  $-(d+n)$ & $-(c+m)$ & $b+l$ & $a+k$
               \end{tabular}\right) \\
                  &= \left(\begin{tabular}{@{}rl@{}}
                     $(a+k) + (b+l)i$ & $(c+m)+(d+n)i$ \\
                     $-(c+m)+(d+n)i$ & $(a+k)-(b+l)i$
                  \end{tabular}\right) \\
                  &= \left(\begin{tabular}{@{}rl@{}}
                     $a+bi$ & $c+di$ \\
                     $-c+di$ & $a-bi$
                  \end{tabular}\right) + \left(\begin{tabular}{@{}rl@{}}
                     $k+li$ & $m+ni$ \\
                     $-m+ni$ & $k-li$
                  \end{tabular}\right) \\
                  &= g(A) + g(B), \text{ and } \\
                  g(AB) &= g\left(\left(\begin{tabular}{@{}cccc@{}}
                  $y_1$ & $y_2$ & $y_3$ & $y_4$ \\
                  $-y_2$ & $y_1$ & $-y_4$ & $y_3$ \\
                  $-y_3$ & $y_4$ & $y_1$ & $-y_2$ \\
                  $-y_4$ & $-y_3$ & $y_2$ & $y_1$
               \end{tabular}\right)\right) \\
                  &= \left(\begin{tabular}{@{}rl@{}}
                     $y_1 + y_2i$ & $y_3+y_4i$ \\
                     $-y_3+y_4i$ & $y_1-y_2i$
                  \end{tabular}\right) \\
                  &= \left(\begin{tabular}{@{}rl@{}}
                     $a+bi$ & $c+di$ \\
                     $-c+di$ & $a-bi$
                  \end{tabular}\right)\left(\begin{tabular}{@{}rl@{}}
                     $k+li$ & $m+ni$ \\
                     $-m+ni$ & $k-li$
                  \end{tabular}\right) \\
                  &= g(A)g(B), \text{ where } \\
                  y_1 &= ak - bl - mc - nd \\
                  y_2 &= al + bk - md + nc \\
                  y_3 &= kc + dl + am - bn \\
                  y_4 &= dk - lc + bm + an,
               \end{align*}
               so that $g$ is a ring isomorphism.
      \end{enumerate}      
%%%%%%%%%%%%%%%%%%%%%%%%%%%%%%%%%%%%%Prob3%%%%%%%%%%%%%%%%%%%%%%%%%%%%%%%%%%%%%%
   \item Let $F$ be a field and consider the set $R$ of all matrices of the form
         $\left(\begin{tabular}{@{}rc@{}}
            $a$ & $b$ \\
            $-b$ & $a - b$
         \end{tabular}\right)$ where $a, b \in F$. Do the following:

         \begin{enumerate}[label=\protect\circled{\arabic*}]
            \item Show $R$ is closed under addition, subtraction and
                  multiplication so it is a subring of $\mathcal{M}_2(F)$, the
                  $2 \times 2$ matrices with entries in $F$.
            \item Find a positive integer $n$ so that if we let the field
                  $F = \Z_n$, then $R$ will be an integral domain.
            \item Find a positive integer $n$ so that if we let the field
                  $F = \Z_n$, then $R$ will \textbf{NOT} be an integral domain.
            \item Find a positive integer $n$ so that if we let the field
                  $F = \Z_n$, then $R$ will be a field.
            \item In any one of the situations \circled{2}, \circled{3}, or
                  \circled{4}, find a unit of order bigger than 2. Just do one.
            \item Suppose now that instead of $F$, we take $a, b \in \Z$, the
                  integers. Prove it is an integral domain.
            \item[\textbf{Bonus.}] 
                  Find $G(R)$, the group of units, in the case when the entries
                  are integers (last situation), and find all elements of finite
                  order in that group.
         \end{enumerate}

      \textbf{Solution.}

      \begin{enumerate}[label=\protect\circled{\arabic*}]
         \item \textbf{Proof.} Let $A = \left(\begin{tabular}{@{}rc@{}}
                  $a$ & $b$ \\
                  $-b$ & $a-b$
               \end{tabular}\right), B = \left(\begin{tabular}{@{}rc@{}}
                  $c$ & $d$ \\
                  $-d$ & $c-d$
               \end{tabular}\right) \in R$. Then we have that
               \begin{align*}
                  A + B &= \left(\begin{tabular}{@{}rc@{}}
                     $a+c$ & $b+d$ \\
                     $-(b+d)$ & $(a+c)-(b+d)$
                  \end{tabular}\right) \\
                  AB &= \left(\begin{tabular}{@{}rc@{}}
                     $ac-bd$ & $ad+bc-bd$ \\
                     $-(ad+bc-bd)$ & $ac-ad-bc$
                  \end{tabular}\right), \text{ and } \\
                  -A &= \left(\begin{tabular}{@{}rc@{}}
                     $-a$ & $-b$ \\
                     $b$ & $b-a$
                  \end{tabular}\right),
               \end{align*}
               so that $R$ is closed under addition, multiplication, and
               negation. The set $R$ clearly contains the identity (by letting
               $a = 1$ and $b = 0$). Thus $R$ is a subring of
               $\mathcal{M}_2(F)$. Note that $R$ is also closed under 
               subtraction since it is closed under addition and negation. \qed
         \item Claim that $R$ is an integral domain if $F = \Z_2$.

               \textbf{Proof.} By \circled{4} below, $R$ is commutative. Suppose
               that $AB = 0$ where
               $$A = \left(\begin{tabular}{@{}rc@{}}
                  $a$ & $b$ \\
                  $-b$ & $a-b$
               \end{tabular}\right) \text{ and }
               B = \left(\begin{tabular}{@{}rc@{}}
                  $c$ & $d$ \\
                  $-d$ & $c-d$
               \end{tabular}\right) \in R.$$
               Then we must have that $\det(A)\det(B) = 0$. Since $F$ is an
               integral domain, we can assume without loss that $\det(A) = 0$.
               That is, $a^2 + b^2 - ab = 0$. Since $F = \Z_2$, we observe that
               of the four choices for $a$ and $b$, $\det(A) = 0$ if and only if
               $a = b = 0$ if and only if $A = 0$. Thus $R$ is an integral
               domain if $F = \Z_2$. \qed
         \item Now let $F = \Z_3$. Notice that although
               $$\left(\begin{tabular}{@{}rr@{}}
                  1 & 2 \\
                  $-2$ & $-1$
               \end{tabular}\right) \neq 0, \text{ we have that }
               \left(\begin{tabular}{@{}rr@{}}
                  1 & 2 \\
                  $-2$ & $-1$
               \end{tabular}\right)^2 = 0,$$
               so that $R$ is not an integral domain if $F = \Z_3$.
         \item Let $F = \Z_2$. Then the elements of $R$ are
               $$A = 0, B = 1, C = \left(\begin{tabular}{@{}rc@{}}
                  1 & 1 \\
                  1 & 0
               \end{tabular}\right), \text{ and }
               D = \left(\begin{tabular}{@{}rc@{}}                  
                  0 & 1 \\
                  1 & 1
               \end{tabular}\right).$$
               By inspection we can see that $R$ is commutative under 
               multiplication. Also we have that $B^{-1} = B$, $C^{-1} = D$, so
               that $R$ is a field if $F = \Z_2$.
         \item From \circled{4}, we have that $|C| = 3$.
         \item We shall follow the same line of thought as we did in
               \circled{2}. So to show that $R$ is an integral domain, it 
               suffices to show that the equation $a^2 + b^2 - ab = 0$ has only 
               the trivial solution in $\Z$. Since
               $$a^2 + b^2 - ab = \left(a - \frac{b}{2}\right)^2 +
                 \frac{3b^2}{4},$$
               it is clear that $a^2 + b^2 - ab$ is positive if $a$ or $b$ is 
               nonzero; hence we must have that $a = b = 0$, so that $R$ is an
               integral domain.
         \item[\textbf{Bonus.}] We notice that an element
               $\left(\begin{tabular}{@{}rc@{}}
                  $a$ & $b$ \\
                  $-b$ & $a - b$
               \end{tabular}\right)$ is a unit in $R$ if and only if its
               determinant is a unit in $\Z$. The determinant of this matrix is
               $a^2 + b^2 - ab$. As per our discussion in \circled{6}, we know
               that it cannot be negative, so we want integers $a$ and $b$ such
               that $a^2 + b^2 - ab = 1$. By completing the square we get that
               $$a^2 + b^2 - ab = 1 \text{ iff } a = \frac{b}{2} \pm
                 \sqrt{\frac{4 - 3b^2}{4}}.$$
               For the discrimant to be positive, we must have that $b = 0$ or
               $|b| = 1$. It follows that $(a, b)$ is an integral solution of
               $a^2 + b^2 - ab = 1$ iff
               $$(a, b) \in \{(-1, 0), (1, 0), (0, 1), (1, 1),
                 (0, -1), (-1, -1)\}.$$
               Thus the group of units is
               $$\left\{I, -I, \left(\begin{tabular}{@{}rr@{}}
                  0 & 1 \\
                  $-1$ & $-1$
               \end{tabular}\right), \left(\begin{tabular}{@{}rr@{}}
                  1 & 1 \\
                  $-1$ & 0
               \end{tabular}\right), \left(\begin{tabular}{@{}rr@{}}
                  0 & $-1$ \\
                  1 & 1
               \end{tabular}\right), \left(\begin{tabular}{@{}rr@{}}
                  $-1$ & $-1$ \\
                  1 & 0
               \end{tabular}\right)\right\}.$$
               This group is cyclic because $\left(\begin{tabular}{@{}rr@{}}
                  1 & 1 \\
                  $-1$ & 0
               \end{tabular}\right)$ generates it. Thus all the elements in this
               group is of finite order.
      \end{enumerate}      
%%%%%%%%%%%%%%%%%%%%%%%%%%%%%%%%%%%%%Prob4%%%%%%%%%%%%%%%%%%%%%%%%%%%%%%%%%%%%%%
   \item Consider the set $R$ of matrices of the form
         $\D\frac{1}{2}\left(\begin{tabular}{@{}rl@{}}
            $a$ & $b$ \\
            $5b$ & $a$
         \end{tabular}\right), a, b \in \Z, a \equiv b$ mod 2.

         \begin{enumerate}[label=\protect\circled{\arabic*}]
            \item Show $I_2 \in R$.
            \item Show $R$ is closed under addition, negation and multiplication
                  so it is a subring of $\mathcal{M}_2(\Q)$.
            \item Compute the characteristic polynomial of any such matrix, and
                  observe it is monic with integer coefficients.
            \item Show there are infinitely many units in $R$.
            \item[\textbf{Bonus.}]
                  Find $\I(R)$, the group of units of $R$.
         \end{enumerate}

      \textbf{Solution.}

      \begin{enumerate}[label=\protect\circled{\arabic*}]
         \item Setting $a = 2$ and $b = 0$ will show us that $R$ has the 
               identity.
         \item Let $A = \D\frac{1}{2}\left(\begin{tabular}{@{}rc@{}}
                  $a$ & $b$ \\
                  $5b$ & $a$
               \end{tabular}\right) \text{ and }
               B = \D\frac{1}{2}\left(\begin{tabular}{@{}rc@{}}
                  $c$ & $d$ \\
                  $5d$ & $c$
               \end{tabular}\right) \in R$. Then it follows that
               \begin{align*}
                  A + B &= \frac{1}{2}\left(\begin{tabular}{@{}rc@{}}
                  $a+c$ & $b+d$ \\
                  $5(b+d)$ & $a+c$
               \end{tabular}\right) \\
                  AB &= \frac{1}{2}\left(\begin{tabular}{@{}rc@{}}
                  $\D\frac{ac+5bd}{2}$ & $\D\frac{ad+bc}{2}$ \\ \\
                  $5\left(\D\frac{ad+bc}{2}\right)$ & $\D\frac{ac+5bd}{2}$
               \end{tabular}\right), \text{ and } \\
                  -A &= \frac{1}{2}\left(\begin{tabular}{@{}cc@{}}
                  $-a$ & $-b$ \\
                  $5(-b)$ & $-a$
               \end{tabular}\right).
               \end{align*}
               By membership in $R$, we must have that $a \equiv b$ mod 2 and
               $c \equiv d$ mod 2. Thus $a + c \equiv b + d$ mod 2 and
               $-a \equiv -b$ mod 2, so that $R$ is closed under addition and
               negation. To show that $R$ is closed under multiplication, we
               must now show that 
               \begin{equation} \label{9_1}
                  \frac{ac+5bd}{2} \equiv \frac{ad+bc}{2} \text{ mod }2.
               \end{equation}
               
               Notice that since $a \equiv b$ mod 2 and $c \equiv d$ mod 2, it
               follows that $a - b$ and $c - d$ are both even, so that 4 divides
               $(a - b)(c - d)$. Now
               \begin{align*}
                  ac + 5bd - (ac + bd) &\equiv (a - b)(c - d) \\
                                       &= ac + bd - (ad + bc) \\
                                       &\equiv 0 \text{ mod }4.
               \end{align*}
               
               That is, $ac + 5bd - (ac + bd)$ is divisible by 4, so that
               $\D\frac{ac + 5bd - (ac + bd)}{2}$ is divisible by 2. In other
               words (\ref{9_1}) holds; hence  $R$ is a subring of $M_2(\Q)$.
         \item Let $A = \D\frac{1}{2}\left(\begin{tabular}{@{}rc@{}}
                  $a$ & $b$ \\
                  $5b$ & $a$
               \end{tabular}\right) \in R$. It follows that the characteristic
               polynomial of $A$ is
               $$x^2 - \left(\frac{a}{2} + \frac{a}{2}\right)x +
                 \frac{a^2-5b^2}{4} = x^2 - ax + \frac{a^2-5b^2}{4}.$$
               Let $[y]_n$ denote $y$ reduced modulo $n$. To complete the proof,
               we must now show that $\D\frac{a^2-5b^2}{4} \in \Z$; that is, we
               want to show that $[a^2 - 5b^2]_4 = 0$. Note that $a$ and $b$
               have the same parity since $[a]_2 = [b]_2$. Thus for odd $a$ and
               $b$, we have that
               $$1 = [a^2]_4 = [b^2]_4 = [1]_4[b^2]_4=[5]_4[b^2]_4 = [5b^2]_4;$$
               for even $a$ and $b$, we have that $[a^2]_4 = [5b^2]_4 = 0$.
               Thus, in either case, it follows that $[a^2 - 5b^2]_4 = 0$, so
               that 4 divides $a^2 - 5b^2$. That is,
               $\D\frac{a^2-5b^2}{4} \in \Z$. So the characteristic polynomial
               of the matrices in $R$ are monic with integer coefficients.
         \item Let $A = \D\frac{1}{2}\left(\begin{tabular}{@{}rc@{}}
                  1 & 1 \\
                  5 & 1
               \end{tabular}\right) \in R$. Observe that $A$ is a unit in $R$
               because
               $A^{-1} = \D\frac{1}{2}\left(\begin{tabular}{@{}rc@{}}
                  $-1$ & 1 \\
                  5 & $-1$
               \end{tabular}\right)$, an element in $R$; since $|A| = \infty$,
               it follows that the set of all integral powers of $A$ is a set of
               infinitely many units.
         \item[\textbf{Bonus.}]
               Let $A = \D\frac{1}{2}\left(\begin{tabular}{@{}rc@{}}
                  $a$ & $b$ \\
                  $5b$ & $a$
               \end{tabular}\right)$ be a unit in $R$. Then we must have that
               $$A^{-1} = \D\frac{2}{a^2 - 5b^2}\left(\begin{tabular}{@{}rc@{}}
                  $a$ & $-b$ \\
                  $-5b$ & $a$
               \end{tabular}\right).$$
               We now observe that problem is reduced to solving the
               diophantine equations $a^2 - \nobreak5b^2 = \pm4$. These are
               are called Pell Equations. \\
               
               \textbf{NB:} I am still researching this problem. I have skimmed
                through a paper by H.W. Lenstra Jr : \textit{Solving the Pell
                Equation.} I think this will be a good problem for the class
                project.
      \end{enumerate}

\end{enumerate}
\end{document}
