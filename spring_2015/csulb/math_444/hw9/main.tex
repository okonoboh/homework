\documentclass[9pt]{article}

\usepackage{amssymb}
\usepackage{amsmath}
\usepackage{amsfonts}
\usepackage{comment}
\usepackage{fancyhdr}
\usepackage{mathrsfs}
\usepackage{enumitem}


\usepackage{tikz}

\voffset = -50pt
%\textheight = 700pt
\addtolength{\textwidth}{60pt}
\addtolength{\evensidemargin}{-30pt}
\addtolength{\oddsidemargin}{-30pt}
%\setlength{\headheight}{44pt}

\pagestyle{fancy}
\fancyhf{} % clear all fields
\fancyhead[R]{%
  \scshape
  \begin{tabular}[t]{@{}r@{}}
  MATH 444, Spring 2015\\Section 1 (5562)\\
  HW \#9, DUE: 2015, APRIL 13
  \end{tabular}}
\fancyhead[L]{%
  \scshape
  \begin{tabular}[t]{@{}r@{}}
  JOSEPH OKONOBOH\\Mathematics\\Cal State Long Beach
  \end{tabular}}
\fancyfoot[C]{\thepage}

\newcommand{\qed}{\hfill \ensuremath{\Box}}


\newcommand*\circled[1]{\tikz[baseline=(char.base)]{
            \node[shape=circle,draw,inner sep=2pt] (char) {#1};}}

\newcommand{\Z}{\mathbb{Z}}
\newcommand{\I}{\mathbb{I}}
\newcommand{\M}{\mathbb{M}}
\newcommand{\Q}{\mathbb{Q}}
\newcommand{\R}{\mathbb{R}}
\newcommand{\C}{\mathbb{C}}
\newcommand{\D}{\displaystyle}
%\setcounter{section}{-1}

\begin{document}
\begin{enumerate}
%%%%%%%%%%%%%%%%%%%%%%%%%%%%%%%%%%%%%Prob1%%%%%%%%%%%%%%%%%%%%%%%%%%%%%%%%%%%%%%
   \item Consider the veracity or falsehood of each of the following statements.
         For bonus, argue for those that you believe are true while providing a
         counterexample for those that you believe are false.

         \begin{enumerate}[label=\protect\circled{\arabic*}]
            \item Every non-constant complex polynomial has a complex root.
            \item Conjugation of complex numbers is a field automorphism of the
                  complex numbers.
            \item Let $x, y \in R$, a finite ring. If $x * y = 1$, then
                  $y * x =  1$ also.
            \item There are exactly four quadratics in $\Z_2[x]$.
            \item If $p(x)$ is a real polynomial, then it either has a real root
                  or there is a quadratic polynomial with real coefficients that
                  divides it.
         \end{enumerate}
         
      \textbf{Solution.}

      \begin{enumerate}[label=\protect\circled{\arabic*}]
         \item True.
         
               This follows from the Fundamental Theorem of Algebra.
         \item True.
         
               \textbf{Proof.} Let $\overline{a}$ denote the conjugate of the 
               complex number $a$. We now want to show that
               $$f : \C \rightarrow \C, \; c \mapsto \overline{c}$$
               is an isomorphism. Let $a_1$ and $a_2$ be complex numbers. Since
               $\overline{a_1a_2} = \overline{a_1} \cdot \overline{a_2}$, and
               $\overline{a_1 + a_2} = \overline{a_1} + \overline{a_2}$, it
               follows that
               $$f(a_1a_2) = f(a_1)f(a_2) \text{ and }
                 f(a_1 + a_2) = f(a_1) + f(a_2),$$
               so that $f$ is an homomorpshim. It now remains to show that
               $f$ is a bijection. The map $f$ must be surjective because
               $f(\overline{a_1}) = a_1$. Also if $f(a_1) = f(a_2)$, then the
               real parts of $a_1$ and $a_2$ must be equal. Similarly,
               their imaginary parts must be equal, so that $a_1 = a_2$. That is
               $f$ is injective and we can conclude that it is a bijection. Thus
               $f$ is a field automorphism. \qed
         \item True.
         
               \textbf{Proof.} Let $R$ be a finite ring, and consider
               $x, y \in R$ such that $x * y = 1$. The map $f : R\rightarrow R$,
               $r \mapsto r * x$ is bijective because for $r_1, r_2 \in R$ with
               $f(r_1) = f(r_2)$, we have that $r_1 * x = r_2 * x$. We then
               cancel $x$ on both sides by multiplying each side on the right
               by $y$ to get $r_1 = r_2$; thus $f$ is injective, and since $R$
               is finite, we can conclude that $f$ is also bijective. Thus
               there exists $r_3 \in R$ such that $r_3 * x = 1$. Mutltiply the
               preceding equality on the right by $y$ to get $r_3 = y$. \qed
         \item False.
         
               There are exactly 8 quadratics in $\Z_2[x]$, and they are
               $$0, 1, x, x + 1, x^2, x^2 + 1, x^2 + x, x^2 + x + 1.$$
               
         \item If $p(x)$ is 0, then it is trivially true. However, if $p(x)$ is
               a constant non-zero polynomial then it is not true. We shall now
               show that the statement is true if $p(x)$ is a non-constant real
               polynomial.
         
               \textbf{Proof.} Consider the polynomial
               $$p(x) = a_nx^n + a_{n - 1}x^{n-1} + \cdots + a_0,$$
               where each $a_i \in \R$, $a_n \neq 0$, and $n \ge 1$. By
               the Fundamental Theorem of Algebra, $p(x)$ has a root $\lambda$.
               If $\lambda$ is real, then we are done. So assume that $\lambda$
               is a non-real complex number. Observe that the conjugate of
               $\lambda$, $\overline{\lambda}$, is also a root of $p(x)$ since
               \begin{align*}
                  p(\overline{\lambda}) &= a_n\overline{\lambda}^n +
                     a_{n - 1}\overline{\lambda}^{n-1} + \cdots + a_0 \\
                     &= a_n\overline{\lambda^n} +
                     a_{n - 1}\overline{\lambda^{n-1}} + \cdots + a_0 \\
                     &= \overline{a_n}\overline{\lambda^n} +
                     \overline{a_{n - 1}}\overline{\lambda^{n-1}} + \cdots +
                     \overline{a_0} &[\overline{a} = a \;\forall a \in \R] \\
                     &= \overline{a_n\lambda^n} +
                     \overline{a_{n - 1}\lambda^{n-1}} + \cdots +
                     \overline{a_0} \\
                     &= \overline{a_n\lambda^n + a_{n - 1}\lambda^{n-1} +
                     \cdots + a_0} \\
                     &= \overline{0} = 0. &[p(\lambda) = 0]
               \end{align*}
               Since $\lambda$ is not real, we must have that
               $\lambda \neq \overline{\lambda}$. Thus the quadratic polynomial
               $(x - \lambda)(x - \overline{\lambda})$ divides $p(x)$. To
               complete the proof, we must show that this quadratic polynomial
               has real coefficients. Now we
               have that
               $$
                  (x - \lambda)(x - \overline{\lambda}) = x^2 -(\lambda + 
                     \overline{\lambda})x + \lambda\overline{\lambda}
                     = x^2 - 2\cdot\text{Re}(\lambda)x + |\lambda|^2,
               $$
               where Re($c$) and $|c|$ denote the real part and magnitude of a
               complex number $c$. Thus the quadratic polynomial
               $(x - \lambda)(x - \overline{\lambda})$ has real coefficients.
               \qed
      \end{enumerate}
%%%%%%%%%%%%%%%%%%%%%%%%%%%%%%%%%%%%%Prob2%%%%%%%%%%%%%%%%%%%%%%%%%%%%%%%%%%%%%%
   \item \textbf{On Complex \& Real}.

         \begin{enumerate}[label=\protect\circled{\arabic*}]
            \item Find a ring isomorphism (it has to be both additive and
                  multiplicative) between $\C$ and the subring
                  $\mathcal{C} = \left\{\left(\begin{tabular}{@{}rl@{}}
                     $a$ & $b$ \\
                     $-b$ & $a$
                  \end{tabular}\right) : a, b \in \R\right\} \subseteq
                  \mathcal{M}_2(\R)$.
            \item In the notes we gave two descriptions of the quaternions:
                  $$\mathcal{Q} = \left\{\left(\begin{tabular}{@{}rrrr@{}}
                     $a$ & $b$ & $c$ & $d$ \\
                     $-b$ & $a$ & $-d$ & $c$ \\
                     $-c$ & $d$ & $a$ & $-b$ \\
                     $-d$ & $-c$ & $b$ & $a$
                  \end{tabular}\right) : a, b, c, d \in \R\right\} \text{ and }
                  \mathcal{H} = \left\{\left(\begin{tabular}{@{}rl@{}}
                     $\alpha$ & $\beta$ \\
                     $-\overline{\beta}$ & $\overline{\alpha}$
                  \end{tabular}\right) : \alpha, \beta \in \C\right\}.$$
                  Find an isomorphism between these two rings (it has to be both
                  additive and multiplicative).
         \end{enumerate}
         
      \textbf{Solution.}
%%%%%%%%%%%%%%%%%%%%%%%%%%%%%%%%%%%%%Prob3%%%%%%%%%%%%%%%%%%%%%%%%%%%%%%%%%%%%%%
   \item Let $F$ be a field and consider the set $R$ of all matrices of the form
         $\left(\begin{tabular}{@{}rc@{}}
          $a$ & $b$ \\
          $-b$ & $a - b$
          \end{tabular}\right)$ where $a, b \in F$. Do the following:

         \begin{enumerate}[label=\protect\circled{\arabic*}]
            \item Show $R$ is closed under addition, subtraction and
                  multiplication so it is a subring of $\mathcal{M}_2(F)$, the
                  $2 \times 2$ matrices with entries in $F$.
            \item Find a positive integer $n$ so that if we let the field
                  $F = \Z_n$, then $R$ will be an integral domain.
            \item Find a positive integer $n$ so that if we let the field
                  $F = \Z_n$, then $R$ will \textbf{NOT} be an integral domain.
            \item Find a positive integer $n$ so that if we let the field
                  $F = \Z_n$, then $R$ will be a field.
            \item In any one of the situations \circled{2}, \circled{3}, or
                  \circled{4}, find a unit of order bigger than 2. Just do one.
            \item Suppose now that instead of $F$, we take $a, b \in \Z$, the
                  integers. Prove it is an integral domain.
            \item[\textbf{Bonus.}] 
                  Find $G(R)$, the group of units, in the case when the entries
                  are integers (last situation), and find all elements of finite
                  order in that group.
         \end{enumerate}
%%%%%%%%%%%%%%%%%%%%%%%%%%%%%%%%%%%%%Prob4%%%%%%%%%%%%%%%%%%%%%%%%%%%%%%%%%%%%%%
   \item Consider the set $R$ of matrices of the form
         $\D\frac{1}{2}\left(\begin{tabular}{@{}rl@{}}
            $a$ & $b$ \\
            $5b$ & $a$
         \end{tabular}\right), a, b \in \Z, a \equiv b$ mod 2.

         \begin{enumerate}[label=\protect\circled{\arabic*}]
            \item Show $I_2 \in R$.
            \item Show $R$ is closed under addition, negation and multiplication
                  so it is a subring of $\mathcal{M}_2(\Q)$.
            \item Compute the characteristic polynomial of any such matrix, and
                  observe it is monic with integer coefficients.
            \item Show there are infinitely many units in $R$.
            \item[\textbf{Bonus.}] 
                  Find $\I(R)$, the group of units of $R$.
         \end{enumerate}

\end{enumerate}
\end{document}
