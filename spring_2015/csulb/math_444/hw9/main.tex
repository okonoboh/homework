\documentclass[9pt]{article}

\usepackage{amssymb}
\usepackage{amsmath}
\usepackage{amsfonts}
\usepackage{comment}
\usepackage{fancyhdr}
\usepackage{mathrsfs}
\usepackage{enumitem}


\usepackage{tikz}

\voffset = -50pt
%\textheight = 700pt
\addtolength{\textwidth}{60pt}
\addtolength{\evensidemargin}{-30pt}
\addtolength{\oddsidemargin}{-30pt}
%\setlength{\headheight}{44pt}

\pagestyle{fancy}
\fancyhf{} % clear all fields
\fancyhead[R]{%
  \scshape
  \begin{tabular}[t]{@{}r@{}}
  MATH 444, Spring 2015\\Section 1 (5562)\\
  HW \#9, DUE: 2015, APRIL 13
  \end{tabular}}
\fancyhead[L]{%
  \scshape
  \begin{tabular}[t]{@{}r@{}}
  JOSEPH OKONOBOH\\Mathematics\\Cal State Long Beach
  \end{tabular}}
\fancyfoot[C]{\thepage}

\newcommand{\qed}{\hfill \ensuremath{\Box}}


\newcommand*\circled[1]{\tikz[baseline=(char.base)]{
            \node[shape=circle,draw,inner sep=2pt] (char) {#1};}}

\newcommand{\Z}{\mathbb{Z}}
\newcommand{\I}{\mathbb{I}}
\newcommand{\M}{\mathbb{M}}
\newcommand{\R}{\mathbb{R}}
\newcommand{\C}{\mathbb{C}}
\newcommand{\D}{\displaystyle}
%\setcounter{section}{-1}

\begin{document}
\begin{enumerate}
%%%%%%%%%%%%%%%%%%%%%%%%%%%%%%%%%%%%%Prob1%%%%%%%%%%%%%%%%%%%%%%%%%%%%%%%%%%%%%%
   \item Consider the veracity or falsehood of each of the following statements.
         For bonus, argue for those that you believe are true while providing a
         counterexample for those that you believe are false.

         \begin{enumerate}[label=\protect\circled{\arabic*}]
            \item Every non-constant complex polynomial has a complex root.
            \item Conjugation of complex numbers is a field automorphism of the
                  complex numbers.
            \item Let $x, y \in R$, a finite ring. If $x * y = 1$, then
                  $y * x =  1$ also.
            \item There are exactly four quadratics in $\Z_2[x]$.
            \item If $p(x)$ is a real polynomial, then it either has a real root
                  or there is a quadratic polynomial with real coefficients that
                  divides it.
         \end{enumerate}
         
      \textbf{Solution.}

      \begin{enumerate}[label=\protect\circled{\arabic*}]
         \item True.
         
               This follows from the Fundamental Theorem of Algebra.
         \item True.
         
               \textbf{Proof.} We want to show that
               $$f : \C \rightarrow \C, \; a + bi \mapsto a - bi$$
               is an isomorphism. So we have that
               \begin{align*}
                  f((a + bi)(c + di)) &= f(ac - bd + (ad + bc)i) \\
                                         &= ac - bd - (ad + bc)i \\
                                         &= ac - adi - bci - bd \\
                                         &= a(c - di) - bi(c - di) \\
                                         &= (a - bi)(c - di) \\
                                         &= f(a + bi)f(c + di), \text{ and }\\\\
                  f((a + bi) + (c + di)) &= f((a + c) + (b + d)i) \\
                     &= (a + c) - (b + d)i \\
                     &= a - bi + c - di \\
                     &= f(a + bi) + f(c + di).
               \end{align*}
               Thus conjugation of complex numbers is a field automorphism.
         \item True.
         
               \textbf{Proof.} Let $R$ be a finite ring, and consider
               $x, y \in R$ such that $x * y = 1$. The map $f : R\rightarrow R$,
               $r \mapsto r * x$ is bijective because for $r_1, r_2 \in R$ with
               $f(r_1) = f(r_2)$, we have that $r_1 * x = r_2 * x$. We then
               cancel $x$ on both sides by multiplying each side on the right
               by $y$ to get $r_1 = r_2$; thus $f$ is injective, and since $R$
               is finite, we can conclude that $f$ is also bijective. Thus
               there exists $r_3 \in R$ such that $r_3 * x = 1$. Mutltiply the
               preceding equality on the right by $y$ to get $r_3 = y$. \qed
         \item False.
         
               There are exactly 8 quadratics in $\Z_2[x]$, and they are
               $$0, 1, x, x + 1, x^2, x^2 + 1, x^2 + x, x^2 + x + 1.$$
               
         \item True.
         
               \textbf{Proof.} Consider the polynomial
               $$px = a_1$$
         \item alicehuynh90 g com
      \end{enumerate}
      
      
\end{enumerate}
\end{document}
