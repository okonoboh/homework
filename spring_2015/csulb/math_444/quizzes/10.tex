\documentclass[9pt]{article}

\usepackage{amssymb}
\usepackage{amsmath}
\usepackage{amsfonts}
\usepackage{comment}
\usepackage{fancyhdr}
\usepackage{mathrsfs}
\usepackage{enumitem}


\usepackage{tikz}

\voffset = -50pt
%\textheight = 700pt
\addtolength{\textwidth}{60pt}
\addtolength{\evensidemargin}{-30pt}
\addtolength{\oddsidemargin}{-30pt}
%\setlength{\headheight}{44pt}

\pagestyle{fancy}
\fancyhf{} % clear all fields
\fancyhead[R]{%
  \scshape
  \begin{tabular}[t]{@{}r@{}}
  MATH 444, Spring 2015\\Section 1 (5562)\\
  Quiz \#10
  \end{tabular}}
\fancyhead[L]{%
  \scshape
  \begin{tabular}[t]{@{}r@{}}
  JOSEPH OKONOBOH\\Mathematics\\Cal State Long Beach
  \end{tabular}}
\fancyfoot[C]{\thepage}

\newcommand{\qed}{\hfill \ensuremath{\Box}}


\newcommand*\circled[1]{\tikz[baseline=(char.base)]{
            \node[shape=circle,draw,inner sep=2pt] (char) {#1};}}

\newcommand{\Z}{\mathbb{Z}}
\newcommand{\I}{\mathbb{I}}
\newcommand{\M}{\mathbb{M}}
\newcommand{\R}{\mathbb{R}}
\newcommand{\D}{\displaystyle}
%\setcounter{section}{-1}

\begin{document}
Let $R = \Z_6$, the integers mod 6. Answer the following:
\begin{enumerate}[label=\protect\circled{\arabic*}]
%%%%%%%%%%%%%%%%%%%%%%%%%%%%%%%%%%%%%Prob1%%%%%%%%%%%%%%%%%%%%%%%%%%%%%%%%%%%%%%
   \item $R$ is a commutative ring. Tell me why.
%%%%%%%%%%%%%%%%%%%%%%%%%%%%%%%%%%%%%Prob2%%%%%%%%%%%%%%%%%%%%%%%%%%%%%%%%%%%%%%   
   \item Is $R$ an integral domain? Why or why not?
%%%%%%%%%%%%%%%%%%%%%%%%%%%%%%%%%%%%%Prob3%%%%%%%%%%%%%%%%%%%%%%%%%%%%%%%%%%%%%%
   \item Find the group of units of $R$.
%%%%%%%%%%%%%%%%%%%%%%%%%%%%%%%%%%%%%Prob4%%%%%%%%%%%%%%%%%%%%%%%%%%%%%%%%%%%%%%
   \item Count the number of fixed point of the matrix 
         $\left(\begin{tabular}{@{}cc@{}}
            1 & 3 \\
            0 & 1
         \end{tabular}\right)$ (with entries in $R$) when it acts (by
         multiplication) on the vectors $\left(\begin{tabular}{@{}c@{}}
            $x$ \\
            $y$
         \end{tabular}\right)$ with entries in $\R$.
%%%%%%%%%%%%%%%%%%%%%%%%%%%%%%%%%%%%%Prob5%%%%%%%%%%%%%%%%%%%%%%%%%%%%%%%%%%%%%%
   \item Is $R$ a field? Why or why not?
   \item[\textbf{Bonus.}]  Count the group of units of $\mathcal{M}_2(R)$, which
                           is the ring of $2 \times 2$ matrices with entries in
                           $R$.
\end{enumerate}

\textbf{Solution.}

\begin{enumerate}[label=\protect\circled{\arabic*}]
   \item Yes, $R$ is a commutative ring because
         \begin{itemize}
            \item $(R, +)$ is an abelian group, and
            \item $R$ is associative and closed under multiplication, has the
                  element 1 as its multiplicative identity, and multiplication
                  distributes over addition.
         \end{itemize}
   \item $R$ is not an integral domain because for $2, 3 \in R$ with $2 \neq 0$
         and $3 \neq 0$, we have $2 \cdot 3 = 0$; i.e., $R$ has zero divisors so
         that it cannot be an integral domain.
   \item The group of units of $R$ are the elements of $R$ with a multiplicative
         identity; thus the group of units of $R$ is $\{1, 5\}$.
   \item Suppose that the matrix $\left(\begin{tabular}{@{}cc@{}}
            1 & 3 \\
            0 & 1
         \end{tabular}\right)$ (with entries in $R$) fixes some vector
         $\left(\begin{tabular}{@{}c@{}}
            $x$ \\
            $y$
         \end{tabular}\right)\left(\begin{tabular}{@{}c@{}}
            $x$ \\
            $y$
         \end{tabular}\right) \in R^2$. Then we must have that
         $$\left(\begin{tabular}{@{}c@{}}
            $x$ \\
            $y$
         \end{tabular}\right) = \left(\begin{tabular}{@{}cc@{}}
            1 & 3 \\
            0 & 1
         \end{tabular}\right)\left(\begin{tabular}{@{}c@{}}
            $x$ \\
            $y$
         \end{tabular}\right) = \left(\begin{tabular}{@{}c@{}}
            $x + 3y$ \\
            $y$
         \end{tabular}\right),$$
         so that $x = x + 3y$; that is, $3y = 0$. The solutions are $y$ = 0, 2,
         or 4. So we have 6 choices for $x$ and 3 choices for $y$, and thus
         18 choices for the fixed vectors.
   \item No, $R$ is not a field because 2 has no multiplicative inverse in $R$.
         Or we can conclude from \circled{2} that $R$ is not a field because it
         is not an integral domain.
         
\end{enumerate}
\end{document}
