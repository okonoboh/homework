\documentclass[9pt]{article}

\usepackage{amssymb}
\usepackage{amsmath}
\usepackage{amsfonts}
\usepackage{comment}
\usepackage{fancyhdr}
\usepackage{mathrsfs}
\usepackage{enumitem}


\usepackage{tikz}

\voffset = -50pt
%\textheight = 700pt
\addtolength{\textwidth}{60pt}
\addtolength{\evensidemargin}{-30pt}
\addtolength{\oddsidemargin}{-30pt}
%\setlength{\headheight}{44pt}

\pagestyle{fancy}
\fancyhf{} % clear all fields
\fancyhead[R]{%
  \scshape
  \begin{tabular}[t]{@{}r@{}}
  MATH 444, Spring 2015\\Section 1 (5562)\\
  Quiz \#11
  \end{tabular}}
\fancyhead[L]{%
  \scshape
  \begin{tabular}[t]{@{}r@{}}
  JOSEPH OKONOBOH\\Mathematics\\Cal State Long Beach
  \end{tabular}}
\fancyfoot[C]{\thepage}

\newcommand{\qed}{\hfill \ensuremath{\Box}}


\newcommand*\circled[1]{\tikz[baseline=(char.base)]{
            \node[shape=circle,draw,inner sep=2pt] (char) {#1};}}

\newcommand{\Z}{\mathbb{Z}}
\newcommand{\I}{\mathbb{I}}
\newcommand{\M}{\mathbb{M}}
\newcommand{\R}{\mathbb{R}}
\newcommand{\C}{\mathbb{C}}
\newcommand{\D}{\displaystyle}
%\setcounter{section}{-1}

\begin{document}
Consider the polynomial $p(x) = x^3 + x^2 + x + 1$. In each of the following
situations, decide whether it is irreducible or not. If not, factor it as much
as you can.
\begin{enumerate}[label=\protect\circled{\arabic*}]
%%%%%%%%%%%%%%%%%%%%%%%%%%%%%%%%%%%%%Prob1%%%%%%%%%%%%%%%%%%%%%%%%%%%%%%%%%%%%%%
   \item $p(x) \in \Z[x]$.
%%%%%%%%%%%%%%%%%%%%%%%%%%%%%%%%%%%%%Prob2%%%%%%%%%%%%%%%%%%%%%%%%%%%%%%%%%%%%%%   
   \item $p(x) \in \Z_2[x]$.
%%%%%%%%%%%%%%%%%%%%%%%%%%%%%%%%%%%%%Prob3%%%%%%%%%%%%%%%%%%%%%%%%%%%%%%%%%%%%%%
   \item $p(x) \in \Z_3[x]$.
%%%%%%%%%%%%%%%%%%%%%%%%%%%%%%%%%%%%%Prob4%%%%%%%%%%%%%%%%%%%%%%%%%%%%%%%%%%%%%%
   \item $p(x) \in \R[x]$.
%%%%%%%%%%%%%%%%%%%%%%%%%%%%%%%%%%%%%Prob5%%%%%%%%%%%%%%%%%%%%%%%%%%%%%%%%%%%%%%
   \item $p(x) \in \C[x]$.
\end{enumerate}

\textbf{Solution.}

\begin{enumerate}[label=\protect\circled{\arabic*}]
   \item In $\Z$, observe that $x = -1$ is a root of $p(x)$, so that $x + 1$
         is a factor of $p(x)$. Thus it follows that
         $$p(x) = (x + 1)(x^2 + 1).$$
   \item In $\Z_2[x]$, $x = -1$ is a root of $x^2 + 1$. Thus we have that
         $p(x) = (x + 1)^3$. 
   \item In $\Z_3[x]$, $x^2 + 1$ has no roots. Thus
         $$p(x) = (x + 1)(x^2 + 1).$$
   \item As is the case with \circled{1} and \circled{3}, we have that
         $$p(x) = (x + 1)(x^2 + 1)$$
         in $\R[x]$.
   \item In $\C[x]$, it follows that
         $$p(x) = (x + 1)(x - i)(x + i).$$
         
\end{enumerate}
\end{document}
