\documentclass[9pt]{article}

\usepackage{amssymb}
\usepackage{amsmath}
\usepackage{amsfonts}
\usepackage{comment}
\usepackage{fancyhdr}
\usepackage{mathrsfs}
\usepackage{enumitem}


\usepackage{tikz}

\voffset = -50pt
%\textheight = 700pt
\addtolength{\textwidth}{60pt}
\addtolength{\evensidemargin}{-30pt}
\addtolength{\oddsidemargin}{-30pt}
%\setlength{\headheight}{44pt}

\pagestyle{fancy}
\fancyhf{} % clear all fields
\fancyhead[R]{%
  \scshape
  \begin{tabular}[t]{@{}r@{}}
  MATH 444, Spring 2015\\Section 1 (5562)\\
  Quiz \#12
  \end{tabular}}
\fancyhead[L]{%
  \scshape
  \begin{tabular}[t]{@{}r@{}}
  JOSEPH OKONOBOH\\Mathematics\\Cal State Long Beach
  \end{tabular}}
\fancyfoot[C]{\thepage}

\newcommand{\qed}{\hfill \ensuremath{\Box}}


\newcommand*\circled[1]{\tikz[baseline=(char.base)]{
            \node[shape=circle,draw,inner sep=2pt] (char) {#1};}}

\newcommand{\Z}{\mathbb{Z}}
\newcommand{\I}{\mathbb{I}}
\newcommand{\M}{\mathbb{M}}
\newcommand{\R}{\mathbb{R}}
\newcommand{\C}{\mathbb{C}}
\newcommand{\Q}{\mathbb{Q}}
\newcommand{\D}{\displaystyle}
%\setcounter{section}{-1}

\begin{document}
Find the group of units in each of the following integral domains:
\begin{enumerate}[label=\protect\circled{\arabic*}]
%%%%%%%%%%%%%%%%%%%%%%%%%%%%%%%%%%%%%Prob1%%%%%%%%%%%%%%%%%%%%%%%%%%%%%%%%%%%%%%
   \item $\Z[x]$.
%%%%%%%%%%%%%%%%%%%%%%%%%%%%%%%%%%%%%Prob2%%%%%%%%%%%%%%%%%%%%%%%%%%%%%%%%%%%%%%   
   \item $\Q[x]$.
%%%%%%%%%%%%%%%%%%%%%%%%%%%%%%%%%%%%%Prob3%%%%%%%%%%%%%%%%%%%%%%%%%%%%%%%%%%%%%%
   \item $\Z_5[x]$.
%%%%%%%%%%%%%%%%%%%%%%%%%%%%%%%%%%%%%Prob4%%%%%%%%%%%%%%%%%%%%%%%%%%%%%%%%%%%%%%
   \item $\Z_8[x]$.
%%%%%%%%%%%%%%%%%%%%%%%%%%%%%%%%%%%%%Prob5%%%%%%%%%%%%%%%%%%%%%%%%%%%%%%%%%%%%%%
   \item $
            S = \left\{\left(\begin{tabular}{@{}rc@{}}
               $a$ & $b$ \\
               $-2b$ & $a$
            \end{tabular}\right) : a, b \in \Z\right\}.
         $
\end{enumerate}

\textbf{Solution.} The group of units simply consists of all the elements that
have a multiplicative inverse. Let $\I(G)$ denote the group of units of a
domain $G$. Recall that the units in $G[x]$ are exactly the units in $G$. Thus

\begin{enumerate}[label=\protect\circled{\arabic*}]
   \item $\I(\Z[x]) = \{-1, 1\}$.
   \item $\I(\Q[x]) = \Q - \{0\}$.
   \item $\I(\Z_5[x]) = \{1, 2, 3, 4\}$.
   \item $\I(\Z_8[x]) = \{1, 3, 5, 7\}$.
   \item An matrix in $S$ is a unit if and only if its determinant is $\pm1$. So
         we require $a^2 + 2b^2 = \pm1$. The only possibilities are $a = \pm1$
         and $b = 0$. Thus
         $$\I(S) = \left\{\left(\begin{tabular}{@{}rc@{}}
               1 & 0 \\
               0 & 1
            \end{tabular}\right), \left(\begin{tabular}{@{}rc@{}}
               $-1$ & 0 \\
               0 & $-1$
            \end{tabular}\right)\right\}.$$       
\end{enumerate}
\end{document}
