\documentclass[9pt]{article}

\usepackage{amssymb}
\usepackage{amsmath}
\usepackage{amsfonts}
\usepackage{comment}
\usepackage{fancyhdr}
\usepackage{mathrsfs}
\usepackage{enumitem}


\usepackage{tikz}

\voffset = -50pt
%\textheight = 700pt
\addtolength{\textwidth}{60pt}
\addtolength{\evensidemargin}{-30pt}
\addtolength{\oddsidemargin}{-30pt}
%\setlength{\headheight}{44pt}

\pagestyle{fancy}
\fancyhf{} % clear all fields
\fancyhead[R]{%
  \scshape
  \begin{tabular}[t]{@{}r@{}}
  MATH 444, Spring 2015\\Section 1 (5562)\\
  HW \#10, DUE: 2015, APRIL 20
  \end{tabular}}
\fancyhead[L]{%
  \scshape
  \begin{tabular}[t]{@{}r@{}}
  JOSEPH OKONOBOH\\Mathematics\\Cal State Long Beach
  \end{tabular}}
\fancyfoot[C]{\thepage}

\newcommand{\qed}{\hfill \ensuremath{\Box}}


\newcommand*\circled[1]{\tikz[baseline=(char.base)]{
            \node[shape=circle,draw,inner sep=2pt] (char) {#1};}}

\newcommand{\Z}{\mathbb{Z}}
\newcommand{\I}{\mathbb{I}}
\newcommand{\M}{\mathbb{M}}
\newcommand{\Q}{\mathbb{Q}}
\newcommand{\R}{\mathbb{R}}
\newcommand{\C}{\mathbb{C}}
\newcommand{\D}{\displaystyle}
%\setcounter{section}{-1}

\begin{document}
\begin{enumerate}
%%%%%%%%%%%%%%%%%%%%%%%%%%%%%%%%%%%%%Prob1%%%%%%%%%%%%%%%%%%%%%%%%%%%%%%%%%%%%%%
   \item Consider the veracity or falsehood of each of the following statements.
         For bonus, argue for those that you believe are true while providing a
         counterexample for those that you believe are false.

         \begin{enumerate}[label=\protect\circled{\arabic*}]
            \item There is an integral domain with 6 elements.\\

                  Let $k$ be a positive integer. Let $\;\bar{} : \Z \to \Z_k$
                  be the mod function. Thus, $e.g.,$ if $k = 7$, then
                  $\overline{25} = 4$. This leads naturally to a homomorphism
                  $\bar{} : \Z[x] \to \Z_k[x]$. Thus, $e.g.,$ if $k = 7$, then
                  $\overline{25x^2 + 12} = 4x^2 + 5 = -3x^2 - 2$. Consider the
                  veracity or falsehood of each of the following statements. For
                  those that are true give an argument, for those that are
                  false, give a counterexample. Let $p(x) \in \Z[x]$ be monic.
            \item If $p(x)$ has a root in $\Z$, then $\overline{p(x)}$ has a
                  root in $\Z_k$.
            \item If $\overline{p(x)}$ has a root in $\Z_k$, then $p(x)$ has a
                  root in $\Z$.
            \item If $p(x)$ is irreducible, then so is $\overline{p(x)}$.
            \item If $\overline{p(x)}$ is irreducible, then so is $p(x)$.
         \end{enumerate}
         
      \textbf{Solution.}

      \begin{enumerate}[label=\protect\circled{\arabic*}]
         \item a
      \end{enumerate}
%%%%%%%%%%%%%%%%%%%%%%%%%%%%%%%%%%%%%Prob2%%%%%%%%%%%%%%%%%%%%%%%%%%%%%%%%%%%%%%
   \item Consider the integral domain $R = \Z[\sqrt{3}]$. Let
         $A = \left(\begin{tabular}{@{}cc@{}}
                  5 & 3 \\
                  9 & 5
               \end{tabular}\right)$.

         \begin{enumerate}[label=\protect\circled{\arabic*}]
            \item Find a nontrivial unit, and show it has infinite order.
            \item Compute $\D\frac{A}{\left(\begin{tabular}{@{}cc@{}}
                     20 & 6 \\
                     18 & 20
                  \end{tabular}\right)}$ and its reciprocal
                  $\D\frac{\left(\begin{tabular}{@{}cc@{}}
                     20 & 6 \\
                     18 & 20
                  \end{tabular}\right)}{A}$. These elements may not be in the
                  domain, but they are certainly in the field of quotients.
            \item Decide if $A$ and $\left(\begin{tabular}{@{}cc@{}}
                     19 & 11 \\
                     33 & 19
                  \end{tabular}\right)$ are associates.
            \item Is $\left(\begin{tabular}{@{}cc@{}}
                     7789 & 4488 \\
                     13464 & 7789
                  \end{tabular}\right) \equiv \left(\begin{tabular}{@{}cc@{}}
                     52 & 24 \\
                     72 & 57
                  \end{tabular}\right) \text{ mod } A$? Give reasons for your
                  answer.
         \end{enumerate}
         
      \textbf{Solution.}

      \begin{enumerate}[label=\protect\circled{\arabic*}]
         \item a
      \end{enumerate}
%%%%%%%%%%%%%%%%%%%%%%%%%%%%%%%%%%%%%Prob3%%%%%%%%%%%%%%%%%%%%%%%%%%%%%%%%%%%%%%
   \item Consider the following element $\left(\begin{tabular}{@{}ccc@{}}
            0 & 1 & 0 \\
            0 & 0 & 1 \\
            1 & 1 & 0
         \end{tabular}\right)$ of $GL(3, \Z_2)$.

         \begin{enumerate}[label=\protect\circled{\arabic*}]
            \item Compute all of its powers.
            \item How many elements would you have to add for this set of
                  powers to be closed under addition?
            \item Find the characteristic polynomial of each of the powers.
            \item Find the lowest degree polynomial that all of the powers
                  satify.
            \item Have you constructed a field?
            \item[\textbf{Bonus.}]
                  Show that every irreducible cubic over $\Z_2$ has a root among
                  these powers.
         \end{enumerate}
         
      \textbf{Solution.}

      \begin{enumerate}[label=\protect\circled{\arabic*}]
         \item a
      \end{enumerate}
%%%%%%%%%%%%%%%%%%%%%%%%%%%%%%%%%%%%%Prob4%%%%%%%%%%%%%%%%%%%%%%%%%%%%%%%%%%%%%%
   \item On $\Z_2[x]$. Consider the ring of polynomials $\Z_2[x]$ with
         coefficients in $\Z_2$,
         $$p(x) = a_0 + a_1x + \cdots + a_nx^n.$$

         \begin{enumerate}[label=\protect\circled{\arabic*}]
            \item How many polynomials of degree $n$ are there? \textbf{Hint.}
                  Consider $n = 1, 2, 3, \ldots$.
            \item Consdier the function $E : \Z_2[x] \to \Z_2$ that sends any
                  polynomial $p(x)$ to $p(1)$. Decide if it is a (ring)
                  homomorphism or not. Decide if it is one-to-one and onto.
                  Argue your case.
            \item Consider the function $S : \Z_2[x] \to \Z_2[x]$ that sends any
                  polynomial $p(x)$ to $p^2(x)$, it square. Decide if it is a
                  (ring) homomorphism or not. Decide if it is one-to-one and 
                  onto. Argue your case.
            \item Count the number of irreducible quadratics in $\Z_2[x]$.
            \item Count the number of irreducible cubics in $\Z_2[x]$.
            \item Count the number of irreducible quartics in $\Z_2[x]$.
         \end{enumerate}
         
      \textbf{Solution.}

      \begin{enumerate}[label=\protect\circled{\arabic*}]
         \item a
      \end{enumerate}
\end{enumerate}
\end{document}
