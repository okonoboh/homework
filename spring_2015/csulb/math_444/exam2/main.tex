\documentclass[9pt]{article}

\usepackage{amssymb}
\usepackage{amsmath}
\usepackage{amsfonts}
\usepackage{comment}
\usepackage{fancyhdr}
\usepackage{mathrsfs}
\usepackage{enumitem}

\usepackage{tikz}

\voffset = -50pt
%\textheight = 700pt
\addtolength{\textwidth}{60pt}
\addtolength{\evensidemargin}{-30pt}
\addtolength{\oddsidemargin}{-30pt}
%\setlength{\headheight}{44pt}

\pagestyle{fancy}
\fancyhf{} % clear all fields
\fancyhead[R]{%
  \scshape
  \begin{tabular}[t]{@{}r@{}}
  MATH 444, Spring 2015\\Section 1 (5562)\\
  Exam \#2, 
  \end{tabular}}
\fancyhead[L]{%
  \scshape
  \begin{tabular}[t]{@{}r@{}}
  JOSEPH OKONOBOH\\Mathematics\\Cal State Long Beach
  \end{tabular}}
\fancyfoot[C]{\thepage}

\newcommand{\qed}{\hfill \ensuremath{\Box}}


\newcommand*\circled[1]{\tikz[baseline=(char.base)]{
            \node[shape=circle,draw,inner sep=2pt] (char) {#1};}}

\newcommand{\Z}{\mathbb{Z}}
\newcommand{\I}{\mathbb{I}}
\newcommand{\M}{\mathbb{M}}
\newcommand{\R}{\mathbb{R}}
\newcommand{\cyc}[1]{\langle #1\rangle}
%\setcounter{section}{-1}

\begin{document}
\begin{enumerate}
%%%%%%%%%%%%%%%%%%%%%%%%%%%%%%%%%%%%%Prob1%%%%%%%%%%%%%%%%%%%%%%%%%%%%%%%%%%%%%%
   \item \textbf{Quickie Queries. It is essential you put down reasons for your
         answers and show your work. 30 points.}
         
         Throughout assume $g, h \in G$, an abelian group, and that the order of
         $g$ is 1000.

         \begin{enumerate}[label=\protect\circled{\arabic*}]
            \item The order of $g^{2120}$.
            \item The smallest $n$ such that $S_n$ has an element of the same
                  order as $g$.
            \item The number of generators of $\cyc{g}$.
            \item The number of subgroups of $\cyc{g}$.
            \item The number of subgroups of $\cyc{g}$ of order 100.
            \item The number of elements of $\cyc{g}$ of order 100.
            \item Given that $h$ is of order 2400, the largest possible order
                  of an element in $G$ (as far as you know).
            \item An element of that largest order (as in \circled{7}).
         \end{enumerate}

      \textbf{Solution.}

      \begin{enumerate}[label=\protect\circled{\arabic*}]
         \item The order of $g^{2120}$ is
               $$\frac{1000}{\gcd(2120, 1000)} = 25.$$
         \item Since $1000 = 2^35^3$, it follows that $n = 2^3 + 5^3 = 133$.
         \item Let $\varphi(n)$ be the number of positive integers relatively
               prime to a positive integer $n$. Then the number of generators
               of $\cyc{g}$ is $\varphi(1000) = \varphi(2^35^3) =
               \varphi(2^3)\varphi(5^3) = 400$.
         \item The number of subgroups of $\cyc{g}$ is the number of positive
               divisors of 1000; since $1000 = 2^35^3$, it follows that we have
               $4 \cdot 4 = 16$ subgroups of $\cyc{g}$.
         \item There is 1 subgroup of $\cyc{g}$ of order 100.
         \item There are $\varphi(100) = \varphi(2^25^2) =
               \varphi(2^2)\varphi(5^2) = 40$ elements of $\cyc{g}$ of order
               100.
         \item The largest possible order of an element as far we know is
               $$\frac{1000 \cdot 2400}{\gcd(1000, 2400)} = 12000.$$
         \item The order of $h^{25}$ is 96 and the order of $g^{8}$ is 125.
               Since $\gcd(96, 125) = 1$, it follows that the order of
               $g^8h^{25}$ is $96 \cdot 125 = 12000$.
      \end{enumerate}
%%%%%%%%%%%%%%%%%%%%%%%%%%%%%%%%%%%%%Prob2%%%%%%%%%%%%%%%%%%%%%%%%%%%%%%%%%%%%%%
   \item \textbf{15 points.} Recall that the centralizer of an element
         $a \in G$ (a group) is given by
         $$C(a) = \{g \in G : ag = ga\}.$$
         Do the following:
         
         \begin{enumerate}[label=\protect\circled{\arabic*}]
            \item Show that $gag^{-1} = hah^{-1}$ if and only if
                  $h^{-1}g \in C(a)$.
            \item Assume $G$ is finite. Show that $|C(a)| \times \# = |G|$
                  where \# is the number of conjugates of $a$.
         \end{enumerate}

      \textbf{Solution.}

      \begin{enumerate}[label=\protect\circled{\arabic*}]
         \item \textbf{Proof.} Suppose $h^{-1}g \in C(a)$. Then
               \begin{align*}
                  h^{-1}ga &= ah^{-1}g &\Longleftrightarrow \\
                  ga &= hah^{-1}g &\Longleftrightarrow \\
                  gag^{-1} &= hah^{-1}.
               \end{align*}
               
               Now suppose $gag^{-1} = hah^{-1}$. Then
               \begin{align*}
                  gag^{-1} &= hah^{-1} &\Longleftrightarrow \\
                  ga &= hah^{-1}g &\Longleftrightarrow \\
                  h^{-1}ga &= ah^{-1}g &\Longleftrightarrow \\
                  h^{-1}g &\in C(a).
               \end{align*} \qed
         \item \textbf{Proof.} Let $a \in G$. We know that
               $$|G_a| \cdot |Ga| = |G|,$$
               where $G_a$ is the stablilizer of $a$ and $Ga$ is the orbit of
               $a$ (note that $\# = |Ga|$). It suffices to show that
               $C(a) = G_a$. Now
               \begin{align*}
                  x &\in C(a) &\Longleftrightarrow \\
                  xa &= ax &\Longleftrightarrow \\
                  xax^{-1} &= a &\Longleftrightarrow \\
                  x &\in Ga,
               \end{align*}
               so that $C(a) = Ga$, and we have that
               $|C(a)| \cdot |Ga| = |G_a| \cdot \# = |G|$. \qed
      \end{enumerate}
%%%%%%%%%%%%%%%%%%%%%%%%%%%%%%%%%%%%%Prob3%%%%%%%%%%%%%%%%%%%%%%%%%%%%%%%%%%%%%%
   \item Let $A$ be an abelian group with identity $e$. \textbf{15 points.}
         
         \begin{enumerate}[label=\protect\circled{\arabic*}]
            \item Show that $\{a \in A : a^3 = e\}$ is a subgroup.
            \item Find the elements of this subgroup when $A$ is the
                  multiplicative group of nozero elements of $\Z_{19}$.
            \item Give necessary and sufficient conditions on the size of $A$ in
                  order for this subgroup to have other elements besides $e$,
                  and give reasons.
         \end{enumerate}

      \textbf{Solution.} Let $G = \{a \in A : a^3 = e\}$.

      \begin{enumerate}[label=\protect\circled{\arabic*}]
         \item \textbf{Proof.} $G$ is clearly associative under the operation of 
               $A$ since it is a subset of $A$, so in order to show that $G$ is
               a subgroup, we need to show that it contains $e$ and that it is 
               closed under the operation of $A$ and taking inverses.
               
               \textbf{Identity.} Clearly $e \in G$ since $e^3 =  e$.
               
               \textbf{Closure.} Suppose $g, h \in G$. Then since $G$ is
               abelian, it follows that $(gh)^3 = g^3h^3 = ee = e$, so that
               $gh \in G$.
               
               \textbf{Inverse.} Suppose $g \in G$. Then it follows that
               $ggg = g^3 = e$. Now
               $$ggg = e \Rightarrow gg = g^{-1} \Rightarrow g = (g^{-1})^2
                 \Rightarrow e = (g^{-1})^3 \Rightarrow g^{-1} \in G,$$
               so that $G$ is closed under taking inverses.
               
               Thus we can conclude that $G$ is a subgroup of $A$. \qed
         \item We want the elements $a$ of $\Z_{19}$ such that $a^3 = 1$. By
               computation we find that the subgroup of $A$ that satisfies this
               condition is $\{1, 7, 13\}$.
         \item If $a^3 = e$, then the order of $a$ divides 3 so that the order
               of $a$ is 1 or 3. So we want the order of $a$ to be 3. Thus we
               must require that $3$ divides $|A|$, so that by Cauchy's Theorem,
               an element of order 3 will be in $G$.
      \end{enumerate}
%%%%%%%%%%%%%%%%%%%%%%%%%%%%%%%%%%%%%Prob4%%%%%%%%%%%%%%%%%%%%%%%%%%%%%%%%%%%%%%
   \item \textbf{On a Group.} Consider
         $G_3 = \left\{\left(\begin{tabular}{@{}cc@{}}
            1 & $m$ \\
            0 & $s$
         \end{tabular}\right) : m \in \Z_8, s \in \{1, 3\}\right\}$. Do
         \circled{1} through \circled{6}. \textbf{20 points.}        
         \begin{enumerate}[label=\protect\circled{\arabic*}]
            \item Show $G_3$ is a subgroup of $GL(2, \Z_8)$ and find its order.
            \item Find the order of $h = \left(\begin{tabular}{@{}rr@{}}
                  1 & 1 \\
                  0 & 1
               \end{tabular}\right)$, and its centralizer, $C(h)$.
            \item Find all the conjugates of $h$.
            \item Show that regardless of what $m$ is, the centralizer of
                  $g_m = \left(\begin{tabular}{@{}cc@{}}
                     1 & $m$ \\
                     0 & 3
               \end{tabular}\right)$, $C(g_m)$, has four elements.
            \item Find the center of $G_3$, $Z(G)$. \textbf{Hint.} You basically
                  already have.
            \item Decide how many elements of each order there are in $G_3$.
         \end{enumerate}

      \textbf{Solution.}

      \begin{enumerate}[label=\protect\circled{\arabic*}]
         \item We have that $|G_3| = 8 \cdot 2 = 16$. To show that $G_3$ is a
               subgroup of $GL(2, \Z_8)$, we need only show that $G_3$ has an
               identity and that it is closed under multiplication since it is
               finite. Notice that $G_3$ is associative under multiplication
               since $\Z_8$ is associative under multiplication. 

               \textbf{Identity.} If we let $m = 0$ and $s = 1$, we shall see
               that $G_3$ contains the identity.
               
               \textbf{Closure.} Let $\left(\begin{tabular}{@{}cc@{}}
                  1 & $m$ \\
                  0 & $s$
               \end{tabular}\right), \left(\begin{tabular}{@{}cc@{}}
                  1 & $n$ \\
                  0 & $t$
               \end{tabular}\right) \in G_3$. Then we have that
               $$\left(\begin{tabular}{@{}cc@{}}
                  1 & $m$ \\
                  0 & $s$
               \end{tabular}\right)\left(\begin{tabular}{@{}cc@{}}
                  1 & $n$ \\
                  0 & $t$
               \end{tabular}\right) = \left(\begin{tabular}{@{}cc@{}}
                  1 & $n + mt$ \\
                  0 & $st$
               \end{tabular}\right) \in G_3$$
               because $\{1, 3\}$ and $\Z_8$ are both closed under
               multiplication. Thus $G_3$ is closed under multiplication. Hence
               $G_3$ is a subgroup of $GL(2, \Z_8)$.
         \item \textbf{Order of } $h$. Since
               $h^n = \left(\begin{tabular}{@{}rr@{}}
                  1 & $n$ \\
                  0 & 1
               \end{tabular}\right)$, it follows that the order of $h$ is 8.

               \textbf{Centralizer of } $h$. Let
               $\left(\begin{tabular}{@{}cc@{}}
                  1 & $m$ \\
                  0 & $s$
               \end{tabular}\right) \in C(h)$. Then it follows that
               $$\left(\begin{tabular}{@{}cc@{}}
                  1 & $m$ \\
                  0 & $s$
               \end{tabular}\right)h = \left(\begin{tabular}{@{}cc@{}}
                  1 & $1+m$ \\
                  0 & $s$
               \end{tabular}\right) = \left(\begin{tabular}{@{}cc@{}}
                  1 & $s+m$ \\
                  0 & $s$
               \end{tabular}\right) = h\left(\begin{tabular}{@{}cc@{}}
                  1 & $m$ \\
                  0 & $s$
               \end{tabular}\right),$$
               so that $1 + m = s + m$. That is, $s = 1$. Thus
               $$C(h) = \left\{\left(\begin{tabular}{@{}cc@{}}
                  1 & $m$ \\
                  0 & 1
               \end{tabular}\right) : m \in \Z_8\right\}.$$
         \item The set of conjugates of $h$ is $\{ghg^{-1} : g \in G_3\}$. So
               let $g = \left(\begin{tabular}{@{}cc@{}}
                  1 & $m$ \\
                  0 & $s$
               \end{tabular}\right) \in G_3$. Then we have that
               $ghg^{-1} = \left(\begin{tabular}{@{}cl@{}}
                  1 & $s^{-1}$ \\
                  0 & 1
               \end{tabular}\right),$ so that the conjugates of $h$ are
               $\left(\begin{tabular}{@{}cc@{}}
                  1 & 1 \\
                  0 & 1
               \end{tabular}\right)$ and $\left(\begin{tabular}{@{}cc@{}}
                  1 & 3 \\
                  0 & 1
               \end{tabular}\right)$.
         \item \textbf{Proof.} Let $m \in \Z_8$ and
               $g_m = \left(\begin{tabular}{@{}cc@{}}
                  1 & $m$ \\
                  0 & 3
               \end{tabular}\right)$.We want to show that
               $|C(g_m)| = 4$. Let $\left(\begin{tabular}{@{}cc@{}}
                  1 & $n$ \\
                  0 & $t$
               \end{tabular}\right) \in C(g_m)$. Then it follows that
               $$\left(\begin{tabular}{@{}cc@{}}
                  1 & $n$ \\
                  0 & $t$
               \end{tabular}\right)g_m = \left(\begin{tabular}{@{}cc@{}}
                  1 & $m+3n$ \\
                  0 & $s$
               \end{tabular}\right) = \left(\begin{tabular}{@{}cc@{}}
                  1 & $mt+n$ \\
                  0 & $s$
               \end{tabular}\right) = g_m\left(\begin{tabular}{@{}cc@{}}
                  1 & $n$ \\
                  0 & $t$
               \end{tabular}\right),$$
               so that $m + 3n = mt + n$; that is, $m + 2n = mt$.

               \textbf{Case 1.} $t = 1$. Then we have that $m + 2n = m$ so that
               $2n = 0$; that is $n \in \{0, 4\}$.

               \textbf{Case 2.} $t = 3$. Then we have that $m + 2n = 3m$ so that
               $2m = 2n$. Notice that since $m \in \Z_8$, then
               $2m \in \{0, 2, 4, 6\}$. For each value of $2m$, the equation
               $2m = 2n$ has exactly two solutions for $n$ and they are

               $$\begin{tabular}{@{}|c|c|@{}} \hline
                  $2m$ & $n$ \\ \hline
                  0 & \{0, 4\} \\ \hline
                  2 & \{1, 5\} \\ \hline
                  4 & \{2, 6\} \\ \hline
                  6 & \{3, 7\} \\ \hline
               \end{tabular}.$$

               Thus for each value of $t$, $n$ has exactly two values, so that
               there are 4 matrices in $C(g_m)$. \qed
         \item The center of $G_3$ is the set of elements of $G_3$ that commute
               with all the elements of $G_3$. Notice that if an element is in
               the center of $G_3$, then it must be in the centralizer of all
               elements of $G_3$. By \circled{4}, we know that
               $$h^8 = \left(\begin{tabular}{@{}cc@{}}
                  1 & 0 \\
                  0 & 1
               \end{tabular}\right) \text{ and }
               h^4 = \left(\begin{tabular}{@{}cc@{}}
                  1 & 4 \\
                  0 & 1
               \end{tabular}\right)$$
               are the only elements in
               $\displaystyle\bigcap_{m \in \Z_8}C(g_m)$ (the
               intersection of the centralizers of matrices of the form
               $\left.\left(\begin{tabular}{@{}cc@{}}
                  1 & $m$ \\
                  0 & 3
               \end{tabular}\right)\right)$. Since the remaining matrices
               (matrices of the form $\left.\left(\begin{tabular}{@{}cc@{}}
                  1 & $y$ \\
                  0 & 1
               \end{tabular}\right)\right)$ are all powers of $h$, it follows
               that $h^8$ and $h^4$ also commute with them. Thus the center of 
               $G_3$ consists of $h^8$ and $h^4$.

         \item $$\begin{tabular}{@{}|c|c|@{}} \hline
                  Matrix & Order \\ \hline
                  $\left(\begin{tabular}{@{}cc@{}}
                  1 & 0 \\
                  0 & 1
               \end{tabular}\right)$ & 1 \\ \hline
                  $\left(\begin{tabular}{@{}cc@{}}
                  1 & 4 \\
                  0 & 1
               \end{tabular}\right)$,$\left(\begin{tabular}{@{}cc@{}}
                  1 & 0 \\
                  0 & 3
               \end{tabular}\right)$,$\left(\begin{tabular}{@{}cc@{}}
                  1 & 2 \\
                  0 & 3
               \end{tabular}\right)$,$\left(\begin{tabular}{@{}cc@{}}
                  1 & 4 \\
                  0 & 3
               \end{tabular}\right)$,$\left(\begin{tabular}{@{}cc@{}}
                  1 & 6 \\
                  0 & 3
               \end{tabular}\right)$ & 2 \\ \hline
                  $\left(\begin{tabular}{@{}cc@{}}
                  1 & 2 \\
                  0 & 1
               \end{tabular}\right)$,$\left(\begin{tabular}{@{}cc@{}}
                  1 & 6 \\
                  0 & 1
               \end{tabular}\right)$,$\left(\begin{tabular}{@{}cc@{}}
                  1 & 1 \\
                  0 & 3
               \end{tabular}\right)$,$\left(\begin{tabular}{@{}cc@{}}
                  1 & 3 \\
                  0 & 3
               \end{tabular}\right)$,$\left(\begin{tabular}{@{}cc@{}}
                  1 & 5 \\
                  0 & 3
               \end{tabular}\right)$,$\left(\begin{tabular}{@{}cc@{}}
                  1 & 7 \\
                  0 & 3
               \end{tabular}\right)$ & 4 \\ \hline
                  $\left(\begin{tabular}{@{}cc@{}}
                  1 & 1 \\
                  0 & 1
               \end{tabular}\right)$,$\left(\begin{tabular}{@{}cc@{}}
                  1 & 3 \\
                  0 & 1
               \end{tabular}\right)$,$\left(\begin{tabular}{@{}cc@{}}
                  1 & 5 \\
                  0 & 1
               \end{tabular}\right)$,$\left(\begin{tabular}{@{}cc@{}}
                  1 & 7 \\
                  0 & 1
               \end{tabular}\right)$ & 8 \\ \hline
               \end{tabular}.$$
      \end{enumerate}
%%%%%%%%%%%%%%%%%%%%%%%%%%%%%%%%%%%%%Prob5%%%%%%%%%%%%%%%%%%%%%%%%%%%%%%%%%%%%%%
   \item \textbf{On the Same Group.} Let $G_3$ be the group from the previous
         exercise. Let it act on the set $X$ of vectors of size 2:
         $\left(\begin{tabular}{@{}c@{}}
            $x$ \\
            $y$
         \end{tabular}\right)$ with entries in $\Z_8$. Do the following:
         \textbf{25 points.}
      
         \begin{enumerate}[label=\protect\circled{\arabic*}]
            \item Count the number of fixed points of
                  $\left(\begin{tabular}{@{}cc@{}}
                     1 & 2 \\
                     0 & 3
                  \end{tabular}\right)$.
            \item Count the number of fixed points of
                  $\left(\begin{tabular}{@{}cc@{}}
                     1 & 4 \\
                     0 & 1
                  \end{tabular}\right)$.
            \item  Count the number of fixed points of
                  $\left(\begin{tabular}{@{}cc@{}}
                     1 & 1 \\
                     0 & 1
                  \end{tabular}\right)$.
            \item Finish filling the table below with the number of fixed points
                  below each of the respective matrices:

                  $\begin{tabular}{@{}|c|c|c|c|c|c|c|c|@{}} \hline
                     $\left(\begin{tabular}{@{}cc@{}}
                        1 & 2 \\
                        0 & 3
                     \end{tabular}\right)$ &
                     $\left(\begin{tabular}{@{}cc@{}}
                        1 & 4 \\
                        0 & 1
                     \end{tabular}\right)$ &
                     $\left(\begin{tabular}{@{}cc@{}}
                        1 & 1 \\
                        0 & 1
                     \end{tabular}\right)$ &
                     $\left(\begin{tabular}{@{}cc@{}}
                        1 & 2 \\
                        0 & 1
                     \end{tabular}\right)$ &
                     $\left(\begin{tabular}{@{}cc@{}}
                        1 & 3 \\
                        0 & 1
                     \end{tabular}\right)$ &
                     $\left(\begin{tabular}{@{}cc@{}}
                        1 & 5 \\
                        0 & 1
                     \end{tabular}\right)$ &
                     $\left(\begin{tabular}{@{}cc@{}}
                        1 & 6 \\
                        0 & 1
                     \end{tabular}\right)$ &
                     $\left(\begin{tabular}{@{}cc@{}}
                        1 & 7 \\
                        0 & 1
                     \end{tabular}\right)$ \\ \hline
                     16 & 32 & 8 & 16 & 8 & 8 & 16 & 8 \\ \hline
                     $\left(\begin{tabular}{@{}cc@{}}
                        1 & 1 \\
                        0 & 3
                     \end{tabular}\right)$ &
                     $\left(\begin{tabular}{@{}cc@{}}
                        1 & 3 \\
                        0 & 3
                     \end{tabular}\right)$ &
                     $\left(\begin{tabular}{@{}cc@{}}
                        1 & 4 \\
                        0 & 3
                     \end{tabular}\right)$ &
                     $\left(\begin{tabular}{@{}cc@{}}
                        1 & 5 \\
                        0 & 3
                     \end{tabular}\right)$ &
                     $\left(\begin{tabular}{@{}cc@{}}
                        1 & 6 \\
                        0 & 3
                     \end{tabular}\right)$ &
                     $\left(\begin{tabular}{@{}cc@{}}
                        1 & 7 \\
                        0 & 3
                     \end{tabular}\right)$ &
                     $\left(\begin{tabular}{@{}cc@{}}
                        1 & 0 \\
                        0 & 3
                     \end{tabular}\right)$ &
                     $\left(\begin{tabular}{@{}cc@{}}
                        1 & 0 \\
                        0 & 1
                     \end{tabular}\right)$ \\ \hline
                     8 & 8 & 16 & 8 & 16 & 8 & 16 & 64 \\ \hline
                  \end{tabular}$.
            \item Use Burnside's Lemma to count the number of orbits.
            \item Find the stabilizer of $\left(\begin{tabular}{@{}c@{}}
                     1 \\
                     0 
                  \end{tabular}\right)$, and use it to find the size of its
                  orbit.
            \item Find the stabilizer of $\left(\begin{tabular}{@{}c@{}}
                     1 \\
                     4 
                  \end{tabular}\right)$, and use it to find the size of its
                  orbit.
            \item Find the stabilizer of $\left(\begin{tabular}{@{}c@{}}
                     2 \\
                     2 
                  \end{tabular}\right)$, and use it to find the size of its
                  orbit.
         \end{enumerate}

      \textbf{Solution.}
      
      \begin{enumerate}[label=\protect\circled{\arabic*}]
         \item We want to find all $x$, $y \in \Z_8$ such that
               $$\left(\begin{tabular}{@{}cc@{}}
                     1 & 2 \\
                     0 & 3
                 \end{tabular}\right)\left(\begin{tabular}{@{}c@{}}
                     $x$ \\
                     $y$
                 \end{tabular}\right) = \left(\begin{tabular}{@{}c@{}}
                     $x$ \\
                     $y$
                 \end{tabular}\right).$$
                  From the above, we shall get
               $$\left(\begin{tabular}{@{}c@{}}
                     $x + 2y$ \\
                     $3y$
                 \end{tabular}\right) = \left(\begin{tabular}{@{}c@{}}
                     $x$ \\
                     $y$
                 \end{tabular}\right),$$
               so that $x + 2y = x$ and $2y = 0$. These equations both simplify
               to $2y = 0$; thus $y \in \{0, 4\}$. So we have 8 choices for $x$
               and 2 choices for $y$ for a total of 16 choices. The number of
               fixed points of $\left(\begin{tabular}{@{}cc@{}}
                  1 & 2 \\
                  0 & 3
               \end{tabular}\right)$ is 32.
         \item We want to find all $x$, $y \in \Z_8$ such that
               $$\left(\begin{tabular}{@{}cc@{}}
                     1 & 4 \\
                     0 & 1
                 \end{tabular}\right)\left(\begin{tabular}{@{}c@{}}
                     $x$ \\
                     $y$
                 \end{tabular}\right) = \left(\begin{tabular}{@{}c@{}}
                     $x$ \\
                     $y$
                 \end{tabular}\right).$$
                  From the above, we shall get
               $$\left(\begin{tabular}{@{}c@{}}
                     $x + 4y$ \\
                     $y$
                 \end{tabular}\right) = \left(\begin{tabular}{@{}c@{}}
                     $x$ \\
                     $y$
                 \end{tabular}\right),$$
               so that $x + 4y = x$. That is $4y = 0$, so that
               $y \in \{0, 2, 4, 6\}$. So we have 8 choices for $x$
               and 4 choices for $y$ for a total of 32 choices. The number of
               fixed points of $\left(\begin{tabular}{@{}cc@{}}
                  1 & 4 \\
                  0 & 1
               \end{tabular}\right)$ is thus 32.
         \item Proceed as in \circled{1} and \circled{2} above to get
               $x + y = x$, so that $y = 0$; so we have 8 choices for $x$
               and 1 choices for $y$ for a total of 8 choices. The number of
               fixed points of $\left(\begin{tabular}{@{}cc@{}}
                  1 & 1 \\
                  0 & 1
               \end{tabular}\right)$ is thus 8.
         \item The table has been filled above.
         \item According to Burnside's Lemma, the number of orbits is
               $$\frac{16+32+8+16+8+8+16+8+8+8+16+8+16+8+16+64}{16} = 16.$$
         \item Let $X_1 = \left(\begin{tabular}{@{}c@{}}
                  1 \\
                  0
                 \end{tabular}\right)$. We want to find all
               $\left(\begin{tabular}{@{}cc@{}}
                  1 & $m$ \\
                  0 & $s$
                 \end{tabular}\right) \in G_3$ such that
               $$\left(\begin{tabular}{@{}cc@{}}
                  1 & $m$ \\
                  0 & $s$
                 \end{tabular}\right)X_1 = X_1.$$
               From the above equation, we shall get $X_1 = X_1$,
               so that all of $G_3$ stabilizes $X_1$. That is, 
               $|\text{Stabilizer}(X_1)| = |G_3|$. Recall that
               $$|\text{Stabilizer}(X_1)| \cdot |\text{Orbit}(X_1)| = |G_3|.$$
               Thus $|\text{Orbit}(X_1)| = 1$.
         \item Similarly let $X_2 = \left(\begin{tabular}{@{}c@{}}
                  1 \\
                  4
                 \end{tabular}\right)$. We want to find all
               $\left(\begin{tabular}{@{}cc@{}}
                  1 & $m$ \\
                  0 & $s$
                 \end{tabular}\right) \in G_3$ such that
               $$\left(\begin{tabular}{@{}cc@{}}
                  1 & $m$ \\
                  0 & $s$
                 \end{tabular}\right)X_2 = X_2.$$
               From the above equation, we shall get 
               $$\left(\begin{tabular}{@{}c@{}}
                  $1+4m$ \\
                  $4s$
                 \end{tabular}\right) = \left(\begin{tabular}{@{}c@{}}
                  1 \\
                  4
                 \end{tabular}\right),$$
               so that all $4m = 0$ and $4s = 4$. Thus $m \in \{0, 2, 4, 6\}$ 
               and $s \in \{1, 3\}$. Thus $|\text{Stabilizer}(X_2)| = 8$. Since
               $$|\text{Stabilizer}(X_2)| \cdot |\text{Orbit}(X_2)| = |G_3|,$$
               it follows that $|\text{Orbit}(X_2)| = 2$.
         \item Similarly let $X_3 = \left(\begin{tabular}{@{}c@{}}
                  2 \\
                  2
                 \end{tabular}\right)$. We want to find all
               $\left(\begin{tabular}{@{}cc@{}}
                  1 & $m$ \\
                  0 & $s$
                 \end{tabular}\right) \in G_3$ such that
               $$\left(\begin{tabular}{@{}cc@{}}
                  1 & $m$ \\
                  0 & $s$
                 \end{tabular}\right)X_3 = X_3.$$
               From the above equation, we shall get 
               $$\left(\begin{tabular}{@{}c@{}}
                  $2+2m$ \\
                  $2s$
                 \end{tabular}\right) = \left(\begin{tabular}{@{}c@{}}
                  2 \\
                  2
                 \end{tabular}\right),$$
               so that all $2m = 0$ and $2s = 2$. Thus $m \in \{0, 4\}$ 
               and $s = 1$. Thus $|\text{Stabilizer}(X_3)| = \nobreak2$. Since
               $$|\text{Stabilizer}(X_3)| \cdot |\text{Orbit}(X_3)| = |G_3|,$$
               it follows that $|\text{Orbit}(X_3)| = 8$.
      \end{enumerate}
%%%%%%%%%%%%%%%%%%%%%%%%%%%%%%%%%%%%%Prob6%%%%%%%%%%%%%%%%%%%%%%%%%%%%%%%%%%%%%%
   \item \textbf{True or False.} Consider the veracity or falsehood of each of
         the following statements, and argue as well as you can for those that
         you believe are true while providing a counterexample for those that
         you believe are false. There is partial credit for just the correct 
         answer. \textbf{20 points}
      
         \begin{enumerate}[label=\protect\circled{\arabic*}]
            \item Every abelian group of order 525 has an element of order 21.
            \item The permutation $(123)(2465)(5674)(12578)(456312)$.
            \item $S = \left\{\left(\begin{tabular}{@{}rc@{}}
                     $a$ & $b$ \\
                     $-3b$ & $a$
                  \end{tabular}\right) : a, b \in \Z\right\}$ is an integral 
                  domain.
            \item A subgroup of a cyclic group is cyclic.
            \item It is possible for an infinite group to have subgroups of
                  finite index.
            \item Any group of order 12 has an element of order 6.
            \item Every group of order 1001 has an element of order 7.
            \item Every field is an integral domain.
            \item Let $G$ and $H$ be groups. Let $f : G \rightarrow H$ be a
                  homomorphism onto $H$. If $G$ is abelian, then so is $H$.
            \item Let $G$ and $H$ be groups. Let $f : G \rightarrow H$ be a
                  homomorphism onto $H$. If $H$ is abelian, then so is $G$.
         \end{enumerate}

      \textbf{Solution.}
      
      \begin{enumerate}[label=\protect\circled{\arabic*}]
         \item True.

               \textbf{Proof.} Let $G$ be an abelian group of order 525.
               Because the primes 3 and 7 both divide $|G|$, it follows by
               Cauchy's Theorem that $G$ has an element $h$ of order 3 and an
               element $g$ of order 7. Since 3 and 7 are relatively prime and
               since $G$ is abelian, it must have an element of order
               $3 \cdot 7 = 21$. \qed
         \item False because the sign of the permutation is
               $$\text{sign}(123) \cdot \text{sign}(2465) \cdot
                 \text{sign}(5674) \cdot \text{sign}(12578) \cdot
                 \text{sign}(456312) = 1 \cdot -1 \cdot -1 \cdot 1 \cdot -1 =
                 -1.$$
         \item  True.

               \textbf{Proof.} First observe that the determinant of a matrix in 
               $S$ is 0 if and only it is the zero matrix. Also we know that $S$
               is a commutative ring. Now let $A_1, A_2 \in S$. Suppose
               $A_1A_2 = \textbf{0}$. Then we must have that
               $$0 = \det(\textbf{0}) = \det(A_1A_2) = \det(A_1)\det(A_2),$$
               so that $\det(A_1) = 0$ or $\det(A_2) = 0$. That is
               $A_1 = \textbf{0}$ or $A_2 = \textbf{0}$, so that $S$ is an
               integral domain. \qed
         \item True. Let $G = \cyc{g}$ be a cyclic group and let $H$ be a 
               subgroup of $G$. Then $H$ is also cyclic since it is generated by
               $g^r$ where $r$ is the smallest positive integer such that
               $g^r \in H$.
         \item True. Although $\Z$ is infinite, its subgroup, $60\Z$, has index
               60.
         \item False because $A_4$ has order 12 but it has no element of order
               6 (since $A_4$ doesn't even have a subgroup of order 6).
         \item  True.

               \textbf{Proof.} Let $G$ be a group of order 1001. Since 7 is
               prime and since 7 divides 1001, it follows at once by Cauchy's 
               Theorem that $G$ has an element of order 7. \qed
         \item True.

               \textbf{Proof.} Let $F$ be a field. Suppose for some
               $a, b \in F$, with $a \neq 0$, we have $ab = 0$. Since $F$ is a   
               field and since $a$ is nonzero, it follows that $a^{-1}$ exists. 
               Thus we have that $0 = ab = a^{-1}(ab) =(a^{-1}a)b = b$, so that
               $F$ is an integral domain. \qed
         \item True.

               \textbf{Proof.} Let $x, y \in H$. Then since $f$ is onto, there
               exist $a, b \in G$ such that $f(a) = x$ and $f(b) = y$. Thus
               $$xy = f(a)f(b) = f(ab) = f(ba) = f(b)f(a) = yx,$$
               so that $H$ is also abelian. \qed
         \item False.

               \textbf{Counterexample.} Consider $f : S_6 \rightarrow \{e\}$.
               This map is onto and $\{e\}$ is trivially abelian. Also $f$ is
               a homomorphism, but $S_6$ is not abelian.
      \end{enumerate}
\end{enumerate}
\end{document}
