\documentclass[9pt]{article}

\usepackage{amssymb}
\usepackage{amsmath}
\usepackage{amsfonts}
\usepackage{comment}
\usepackage{fancyhdr}
\usepackage{mathrsfs}
\usepackage{enumitem}


\usepackage{tikz}

\voffset = -50pt
%\textheight = 700pt
\addtolength{\textwidth}{60pt}
\addtolength{\evensidemargin}{-30pt}
\addtolength{\oddsidemargin}{-30pt}
%\setlength{\headheight}{44pt}

\pagestyle{fancy}
\fancyhf{} % clear all fields
\fancyhead[R]{%
  \scshape
  \begin{tabular}[t]{@{}r@{}}
  MATH 444, Spring 2015\\Section 1 (5562)\\
  HW \#8, DUE: 2015, MARCH 25
  \end{tabular}}
\fancyhead[L]{%
  \scshape
  \begin{tabular}[t]{@{}r@{}}
  JOSEPH OKONOBOH\\Mathematics\\Cal State Long Beach
  \end{tabular}}
\fancyfoot[C]{\thepage}

\newcommand{\qed}{\hfill \ensuremath{\Box}}


\newcommand*\circled[1]{\tikz[baseline=(char.base)]{
            \node[shape=circle,draw,inner sep=2pt] (char) {#1};}}

\newcommand{\Z}{\mathbb{Z}}
\newcommand{\I}{\mathbb{I}}
\newcommand{\M}{\mathbb{M}}
\newcommand{\R}{\mathbb{R}}
%\setcounter{section}{-1}

\begin{document}
\begin{enumerate}
%%%%%%%%%%%%%%%%%%%%%%%%%%%%%%%%%%%%%Prob1%%%%%%%%%%%%%%%%%%%%%%%%%%%%%%%%%%%%%%
   \item Consider the veracity or falsehood of each of the following statements.
         For bonus, argue for those that you believe are true while providing a
         counterexample for those that you believe are false. Throughout $G$ is
         a group.

         \begin{enumerate}[label=\protect\circled{\arabic*}]
            \item If $g \in G$ is the only element of order 2, then
                  $g \in Z(G)$, the center.
            \item The intersection of two subgroups of $G$ is also a subgroup.
            \item The union of two subgroups of $G$ is also a subgroup.
            \item The largest order of an element in $S_{12}$ is 60.
            \item If an Abelian group has an element of order 10 and an element
                  of order 12, then it has an element of order 30.
         \end{enumerate}

      \textbf{Solution.}

      \begin{enumerate}[label=\protect\circled{\arabic*}]
         \item True. 

               \textbf{Proof.} Assume that $g \in G$ is the only element of
               order 2. Let $h$ be an arbitrary element in $G$. It suffices to
               show that $gh = hg$. We claim that $|hgh^{-1}| = 2$. So we have
               that $(hgh^{-1})^2 = hgh^{-1}hgh^{-1} = hg^2h^{-1} = hh^{-1} =e$.
               Now suppose that $hgh^{-1} = e$. Then it must be the case that
               $g = h^{-1}h = e$, a contradiction since $|g| = 2$. Thus we have
               that $|hgh^{-1}| = 2$. But since $g$ is the only element of order
               2, it follows that $hgh^{-1} = g$, so that $hg = gh$; since the
               choice of $h$ was arbitrary, we can conclude that $g \in Z(G)$.
               \qed
         \item True.
         
               \textbf{Proof.} Let $H_1 \le G$, $H_2 \le G$, and
               $H' = H_1 \cap H_2$. Since $e$ is in both $H_1$ and $H_2$, it
               follows that $e \in H'$. The set $H'$ is also associative
               because it is a subset of $G$. Now let $a, b \in H'$. Thus we
               must have that $a, b \in H_1$ and $a, b \in H_2$. Since $H_1$ and
               $H_2$ are groups, it follows that they both contain $ab$ and
               $a^{-1}$ so that $ab, a^{-1} \in H'$. That is, $H$ is closed
               under the operation of $G$ and also closed under taking inverses.
               Thus $H' \le G$. \qed
         \item False.
         
               \textbf{Counterexample:} Consider $2\Z, 3\Z \le \Z$. We have that
               $2 \in 2\Z$ and $3 \in 3\Z$, but $2 + 3 = 5 \notin 2\Z \cup 3\Z$.
         \item True. The permutation
               $$\sigma = (1\;2\;3\;4\;5)(6\;7\;8\;9)(10\;11\;12)$$
               has order 60. Let Suppose $\alpha \in S_{12}$ has order greater
               than 60. Now write $\alpha$ as a product of disjoint cycles
               $$\alpha = \alpha_1\alpha_2\cdots\alpha_n.$$
               Let $j \in \{1, 2, \ldots, n\}$. Then $\alpha_j$ cannot be a
               12-cycle since that would imply that $|\alpha| = 12$. For the
               same reason $\alpha_j$ can neither be an 11-cycle or a 10-cycle.
               If $\alpha_j$ is a 9-cycle, then $|\alpha|$ is either 9 (9+3)or
               18 (9+2+1).
               Now if $\alpha_j$ is a 8-cycle, then $|\alpha|$ is either 8
               (8+4, 8+2+2, 8+2+1+1) or 24(8+3+1).
         \item True.
         
               \textbf{Proof.} Let $g$ and $h$ have orders 10 and 12 in some
               abelian group. The element $g^2$ has order 5 and the element
               $h^2$ has order 6. Since $\gcd(5, 6) = 1$, it follows that
               $|g^2h^2| = 5 \cdot 6 = 30$. \qed
      \end{enumerate}
%%%%%%%%%%%%%%%%%%%%%%%%%%%%%%%%%%%%%Prob2%%%%%%%%%%%%%%%%%%%%%%%%%%%%%%%%%%%%%%
   \item We have beads of four different colors.

         \begin{enumerate}[label=\protect\circled{\arabic*}]
            \item How many distinct four-bead necklaces can we make?
            \item How many distinct five-bead necklaces can we make?
            \item How many distinct six-bead necklaces can we make?
            \item[\textbf{BONUS:}] Answer the same questions if we now have
                  beads of five colors.
         \end{enumerate}

      \textbf{Solution.}

      \begin{enumerate}[label=\protect\circled{\arabic*}]
         \item a               
      \end{enumerate}
%%%%%%%%%%%%%%%%%%%%%%%%%%%%%%%%%%%%%Prob3%%%%%%%%%%%%%%%%%%%%%%%%%%%%%%%%%%%%%%
   \item Consider the following two sets of matrices 
         $$S_1 = \left\{\left(\begin{tabular}{@{}cc@{}}
                     $a$ & $b$ \\
                     $b$ & $a$
                  \end{tabular}\right) : a, b \in \Z\right\} \text{ and }
          S_2 = \left\{\left(\begin{tabular}{@{}rc@{}}
                     $a$ & $b$ \\
                     $-b$ & $a$
                  \end{tabular}\right) : a, b \in \Z\right\}.$$
         Do the following for both:

         \begin{enumerate}[label=\protect\circled{\arabic*}]
            \item Decide if they are rings or not---and give reasons.
            \item Decide if they are integral domains or not---and give reasons.
            \item Can you find a root for the polynomial $x^2 + 1$ in either
                  place? If so find all the roots or give reasons.
         \end{enumerate}

      \textbf{Solution.}

      \begin{enumerate}[label=\protect\circled{\arabic*}]
         \item We claim that $S_1$ and $S_2$ are both commutative rings.

               \textbf{Proof.} First we want to show that $(S_1, +)$ and
               $(S_2, +)$ are abelian groups. 
               
               \textbf{Identity.} It is clear that $S_1$ and $S_2$ both contain
               the zero matrix, which is the identity under addition.
               
               \textbf{Associativity, Closure \& Commutativity under +.} Since
               $\Z$ is associative, closed, and commutative under addition, it
               follows that $S_1$ and $S_2$ are both associative, closed, and
               commutative under addition.
               
               \textbf{Inverse.} For each matrix $A$ in $S_1$ (resp. $S_2$) we
               have that $-A$ is in $S_1$ (resp. $S_2$) so that $S_1$ and $S_2$
               are both closed under taking inverses. \\
               
               Thus we have shown that $S_1$ and $S_2$ are abelian groups.
              
               \textbf{Identity under multiplication.} It is clear that $S_1$
               and $S_2$ both contain the $2\times 2$ identity matrix, which
               serves as the multiplicative identity.
               
               \textbf{Associativity, Closure \& Commutativity under
               multiplication.} Let
               $$\left(\begin{tabular}{@{}c c@{}}
                     $a$ & $b$ \\
                     $b$ & $a$
                  \end{tabular}\right), \left(\begin{tabular}{@{}c c@{}}
                     $c$ & $d$ \\
                     $d$ & $c$
                  \end{tabular}\right) \in S_1 \text{ and }
                  \left(\begin{tabular}{@{}r c@{}}
                     $e$ & $f$ \\
                     $-f$ & $e$
                  \end{tabular}\right), \left(\begin{tabular}{@{}r c@{}}
                     $g$ & $h$ \\
                     $-h$ & $g$
                  \end{tabular}\right) \in S_2.$$
                  Then it follows that
                  $$\left(\begin{tabular}{@{}c c@{}}
                     $a$ & $b$ \\
                     $b$ & $a$
                  \end{tabular}\right)\left(\begin{tabular}{@{}c c@{}}
                     $c$ & $d$ \\
                     $d$ & $c$
                  \end{tabular}\right) = \left(\begin{tabular}{@{}c c@{}}
                     $ac+bd$ & $ad+bc$ \\
                     $ad+bc$ & $ac+bd$
                  \end{tabular}\right) = \left(\begin{tabular}{@{}c c@{}}
                     $c$ & $d$ \\
                     $d$ & $c$
                  \end{tabular}\right)\left(\begin{tabular}{@{}c c@{}}
                     $a$ & $b$ \\
                     $b$ & $a$
                  \end{tabular}\right) \in S_1$$
                  and
                  $$\left(\begin{tabular}{@{}r c@{}}
                     $e$ & $f$ \\
                     $-f$ & $e$
                  \end{tabular}\right)\left(\begin{tabular}{@{}r c@{}}
                     $g$ & $h$ \\
                     $-h$ & $g$
                  \end{tabular}\right) = \left(\begin{tabular}{@{}r c@{}}
                     $eg-fh$ & $eh+fg$ \\
                     $-eh-fg$ & $eg-fh$
                  \end{tabular}\right) = \left(\begin{tabular}{@{}r c@{}}
                     $g$ & $h$ \\
                     $-h$ & $g$
                  \end{tabular}\right)\left(\begin{tabular}{@{}r c@{}}
                     $e$ & $f$ \\
                     $-f$ & $e$
                  \end{tabular}\right) \in S_2$$
               so that $S_1$ and $S_2$ are both closed and commutative under
               multiplication. Associativity follows since matrices are
               associative under multiplication.
               
               \textbf{Distributivity.} For any $n \times n$ matrices $A$, $B$,
               and $C$, we know that
               $$A(B + C) = AB + AC \text{ and that }
                (B + C)A = BA + CA;$$ thus multiplication distributes over
               addition in $S_1$ and $S_2$. \\
               
               From the arguments above, we have thus shown that $S_1$ and $S_2$
               are rings. \qed
         \item $S_1$ is not an integral domain because we have that
               $$A = \left(\begin{tabular}{@{}r r@{}}
                     1 & $-1$ \\
                     $-1$ & 1
                  \end{tabular}\right), B = \left(\begin{tabular}{@{}r c@{}}
                     1 & 1 \\
                     1 & 1
                  \end{tabular}\right) \in S_1,$$
               with $A$ and $B$ nonzero but $AB = 0$. We now claim that $S_2$ is
               an integral domain. 
               
               \textbf{Proof.} Suppose to the contrary that $S_2$ is not an
               integral domain. Then there exist nonzero matrices
               $A, B \in S_2$, where
               $$A = \left(\begin{tabular}{@{}r r@{}}
                     $a$ & $b$ \\
                     $-b$ & $a$
                  \end{tabular}\right), B = \left(\begin{tabular}{@{}r c@{}}
                     $c$ & $d$ \\
                     $-d$ & $c$
                  \end{tabular}\right),$$
                  such that $AB = 0$. Then it follows that
                  $0 = \det(0) = \det(A)\det(B)$. Since $\det(A)$ and $\det(B)$
                  are integers and since $\Z$ is an integral domain, it follows
                  that $\det(A) = 0$ or $\det(B) = 0$. Assume without loss that
                  $\det(A) = 0$. Then it follows that $a^2 + b^2 = 0$, so that
                  $a = b = 0$, a contradiction since $A$ is nonzero. Thus $S_2$
                  is an integral domain. \qed
               
      \end{enumerate}
%%%%%%%%%%%%%%%%%%%%%%%%%%%%%%%%%%%%%Prob4%%%%%%%%%%%%%%%%%%%%%%%%%%%%%%%%%%%%%%
   \item Let $R$ be a ring. An additive subgroup $I$ is called an ideal if
         whenever $r \in R$ and $a \in I$, then $ra, ar \in I$.

         \begin{enumerate}[label=\protect\circled{\arabic*}]
            \item Find two ideals of $\Z$ that are neither 0 nor $\Z$.
            \item Let $I$ be an ideal. Prove the following are true: if $I + x$
                  and $I + y$ are the same coset and $I + m$ and $I + n$ are the
                  same coset, then $I + (x + m)$ and $I + (y + n)$ are the same
                  coset, and so are $I + xm$ and $I + yn$.
            \item Let $S$ be a ring, and let $\alpha : R \rightarrow S$ be a
                  ring homomorphism---this means with respect to both
                  operations. Show
                  $I = \ker(\alpha) = \{a \in R : \alpha(a) = 0\}$ is an ideal.
         \end{enumerate}

      \textbf{Solution.}

      \begin{enumerate}[label=\protect\circled{\arabic*}]
         \item Consider $n\Z < \Z$, with $n > 1$. Let $x \in n\Z$ and let
               $z \in \Z$. Then $x = nm$ for some integer $m$, so that
               $zx = xz = (nm)z = n(mz) \in n\Z$. Thus $n\Z$ is an ideal of
               $\Z$, so that $444\Z$ and $410\Z$ are both nontrivial ideals of
               $\Z$.
         \item \textbf{Proof.} Let $I$ be an ideal. Assume that
               $x, y, m, n \in R$ such that $I + x = I + y$ and $I + m = I + n$.
               We want to show that $I + (x + m) \subseteq I + (y + n)$ and
               $I + xm \subseteq I + yn$. So let $r_1 \in I + (x + m)$ and
               $r_2 \in I + xm$. Thus
               \begin{align*}
                  r_1 &= i_1 + (x + m) &[\text{for some }i_1 \in I] \\
                    &= (i_1 + x) + m \\
                    &= (i_2 + y) + m &[I + x = I + y; i_2 \in I] \\
                    &= (i_2 + m) + y &[(R, +) \text{ is abelian}] \\
                    &= (i_3 + n) + y &[I + m = I + n; i_3 \in I] \\
                    &= i_3 + (y + n) \in I + (y + n)
               \end{align*}
               and $r_2 = i_4 + xm$ for some $i_4 \in I$. Since $I + x = I + y$
               and $I + m = I + n$, it follows that $i_4 + x = i_5 + y$ and
               $i_4 + m = i_6 + n$ for some $i_5, i_6 \in I$. Thus we have that
               $x = y + i_5 - i_4$ and $m = n + i_6 - i_4$, so that
               \begin{align*}
                  r_2 &= i_4 + xm \\
                      &= i_4 + (y + i_5 - i_4)(n + i_6 - i_4) \\
                      &= i_4 + y(i_6 - i_4) + n(i_5 - i_4) +
                         (i_5 - i_4)(i_6 - i_4) + yn.
               \end{align*}
               Since $I$ is an ideal, it must be the case that
               $$i_4 + y(i_6 - i_4) + n(i_5 - i_4) + 
                        (i_5 - i_4)(i_6 - i_4)\in I.$$
               Thus $r_2 \in I + yn$. We have thus shown that
               $I + (x + m) \subseteq I +(y + n)$ and $I + xm \subseteq I + yn$.
               The argument that
               $I + (y + n) \subseteq I + (x + m)$ and $I + yn \subseteq I + xm$
               follows by symmetry. Thus $I + (x + m) = I +(y + n)$ and
               $I + xm = I + yn$. \qed
         \item \textbf{Proof.} Let $\alpha : R \rightarrow S$ be a homomorphism
               of rings. To show that $\ker(\alpha)$ is an ideal of $R$, we have
               to first show that $(\ker(\alpha), +) \le (R, +)$.
               
               \textbf{Identity.} Since $\alpha$ is also a group homomorphism,
               we know from our discussion in group theory that $\alpha(1) = 1$,
               so that $1 \in \ker(\alpha)$.
               
               \textbf{Closure.} Let $a, b \in \ker(\alpha)$. Then we have that
               $\alpha(a + b) = \alpha(a) + \alpha(b) = 0 + 0 = 0$, so that
               $a + b \in \ker(\alpha)$; i.e., $\ker(\alpha)$ is closed under
               addition.
               
               \textbf{Inverse.} Let $a \in \ker(\alpha)$. Then we have that
               \begin{align*}
                  \alpha(-a) &= \alpha(-1 \cdot a) \\
                     &= \alpha(-1) \cdot \alpha(a) \\
                     &= \alpha(-1) \cdot 0 = 0,
               \end{align*}
               so that $-a \in \ker(\alpha)$, and thus $\ker(\alpha)$ is closed
               under taking inverses. \\
               
               It follows from above that $\ker(\alpha)$ is an additive subgroup
               of $R$. Now let ${a \in \ker(\alpha)}$, ${r \in R}$. To complete
               the proof, we must show that $ar \in \ker(\alpha)$ and
               $ra \in \ker(\alpha)$. Thus
               \begin{align*}
                  0 &= 0 \cdot \alpha(r) \\
                    &= \alpha(a) \cdot \alpha(r) &[\alpha(a) \in \ker(\alpha)]\\
                    &= \alpha(ar) &[\text{so that } ar \in \ker(\alpha)] \\
                    &= 0 \\
                    &= \alpha(r) \cdot \alpha(a) \\
                    &= \alpha(ra), &[\text{so that } ra \in \ker(\alpha)]
               \end{align*}
               so that$ra, ar \in \ker(\alpha)$ and $\ker(\alpha)$ is an ideal.
               \qed
               
      \end{enumerate}
\end{enumerate}
\end{document}
