\documentclass[9pt]{article}

\usepackage{amssymb}
\usepackage{amsmath}
\usepackage{amsfonts}
\usepackage{comment}
\usepackage{fancyhdr}
\usepackage{mathrsfs}
\usepackage{enumitem}
\usepackage[retainorgcmds]{IEEEtrantools}


\usepackage{tikz}

\voffset = -50pt
%\textheight = 700pt
\addtolength{\textwidth}{60pt}
\addtolength{\evensidemargin}{-30pt}
\addtolength{\oddsidemargin}{-30pt}
%\setlength{\headheight}{44pt}

\pagestyle{fancy}
\fancyhf{} % clear all fields
\fancyhead[R]{%
  \scshape
  \begin{tabular}[t]{@{}r@{}}
  MATH 444, Spring 2015\\Section 1 (5562)\\
  HW \#8, DUE: 2015, MARCH 25
  \end{tabular}}
\fancyhead[L]{%
  \scshape
  \begin{tabular}[t]{@{}r@{}}
  JOSEPH OKONOBOH\\Mathematics\\Cal State Long Beach
  \end{tabular}}
\fancyfoot[C]{\thepage}

\newcommand{\qed}{\hfill \ensuremath{\Box}}


\newcommand*\circled[1]{\tikz[baseline=(char.base)]{
            \node[shape=circle,draw,inner sep=2pt] (char) {#1};}}

\newcommand{\Z}{\mathbb{Z}}
\newcommand{\I}{\mathbb{I}}
\newcommand{\M}{\mathbb{M}}
\newcommand{\R}{\mathbb{R}}
%\setcounter{section}{-1}

\begin{document}
\begin{enumerate}
%%%%%%%%%%%%%%%%%%%%%%%%%%%%%%%%%%%%%Prob1%%%%%%%%%%%%%%%%%%%%%%%%%%%%%%%%%%%%%%
   \item Consider the veracity or falsehood of each of the following statements.
         For bonus, argue for those that you believe are true while providing a
         counterexample for those that you believe are false. Throughout $G$ is
         a group.

         \begin{enumerate}[label=\protect\circled{\arabic*}]
            \item If $g \in G$ is the only element of order 2, then
                  $g \in Z(G)$, the center.
            \item The intersection of two subgroups of $G$ is also a subgroup.
            \item The union of two subgroups of $G$ is also a subgroup.
            \item The largest order of an element in $S_{12}$ is 60.
            \item If an Abelian group has an element of order 10 and an element
                  of order 12, then it has an element of order 30.
         \end{enumerate}

      \textbf{Solution.}

      \begin{enumerate}[label=\protect\circled{\arabic*}]
         \item a               
      \end{enumerate}
%%%%%%%%%%%%%%%%%%%%%%%%%%%%%%%%%%%%%Prob2%%%%%%%%%%%%%%%%%%%%%%%%%%%%%%%%%%%%%%
   \item We have beads of four different colors.

         \begin{enumerate}[label=\protect\circled{\arabic*}]
            \item How many distinct four-bead necklaces can we make?
            \item How many distinct five-bead necklaces can we make?
            \item How many distinct six-bead necklaces can we make?
            \item[\textbf{BONUS:}] Answer the same questions if we now have
                  beads of five colors.
         \end{enumerate}

      \textbf{Solution.}

      \begin{enumerate}[label=\protect\circled{\arabic*}]
         \item a               
      \end{enumerate}
%%%%%%%%%%%%%%%%%%%%%%%%%%%%%%%%%%%%%Prob3%%%%%%%%%%%%%%%%%%%%%%%%%%%%%%%%%%%%%%
   \item Consider the following two sets of matrices 
         $$S_1 = \left\{\left(\begin{tabular}{@{}cc@{}}
                     $a$ & $b$ \\
                     $b$ & $a$
                  \end{tabular}\right) : a, b \in \Z\right\} \text{ and }
          S_2 = \left\{\left(\begin{tabular}{@{}rc@{}}
                     $a$ & $b$ \\
                     $-b$ & $a$
                  \end{tabular}\right) : a, b \in \Z\right\}.$$
         Do the following for both:

         \begin{enumerate}[label=\protect\circled{\arabic*}]
            \item Decide if they are rings or not---and give reasons.
            \item Decide if they are integral domains or not---and give reasons.
            \item Can you find a root for the polynomial $x^2 + 1$ in either
                  place? If so find all the roots or give reasons.
         \end{enumerate}

      \textbf{Solution.}

      \begin{enumerate}[label=\protect\circled{\arabic*}]
         \item a               
      \end{enumerate}
%%%%%%%%%%%%%%%%%%%%%%%%%%%%%%%%%%%%%Prob4%%%%%%%%%%%%%%%%%%%%%%%%%%%%%%%%%%%%%%
   \item Let $R$ be a ring. An additive subgroup $I$ is called an ideal if
         whenever $r \in R$ and $a \in I$, then $ra, ar \in I$.

         \begin{enumerate}[label=\protect\circled{\arabic*}]
            \item Find two ideals of $\Z$ that are neither 0 nor $\Z$.
            \item Let $I$ be an ideal. Prove the following are true: if $I + x$
                  and $I + y$ are the same coset and $I + m$ and $I + n$ are the
                  same coset, then $I + (x + m)$ and $I + (y + n)$ are the same
                  coset, and so are $I + xm$ and $I + yn$.
            \item Let $S$ be a ring, and let $\alpha : R \rightarrow S$ be a
                  ring homomorphism---this means with respect to both
                  operations. Show
                  $I = \ker(\alpha) = \{a \in \R : \alpha(a) = 0\}$ is an ideal.
         \end{enumerate}

      \textbf{Solution.}

      \begin{enumerate}[label=\protect\circled{\arabic*}]
         \item a               
      \end{enumerate}
\end{enumerate}
\end{document}
