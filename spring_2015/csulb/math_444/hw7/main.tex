\documentclass[9pt]{article}

\usepackage{amssymb}
\usepackage{amsmath}
\usepackage{amsfonts}
\usepackage{comment}
\usepackage{fancyhdr}
\usepackage{mathrsfs}
\usepackage{enumitem}


\usepackage{tikz}

\voffset = -50pt
%\textheight = 700pt
\addtolength{\textwidth}{60pt}
\addtolength{\evensidemargin}{-30pt}
\addtolength{\oddsidemargin}{-30pt}
%\setlength{\headheight}{44pt}

\pagestyle{fancy}
\fancyhf{} % clear all fields
\fancyhead[R]{%
  \scshape
  \begin{tabular}[t]{@{}r@{}}
  MATH 444, Spring 2015\\Section 1 (5562)\\
  HW \#7, DUE: 2015, MARCH 18
  \end{tabular}}
\fancyhead[L]{%
  \scshape
  \begin{tabular}[t]{@{}r@{}}
  JOSEPH OKONOBOH\\Mathematics\\Cal State Long Beach
  \end{tabular}}
\fancyfoot[C]{\thepage}

\newcommand{\qed}{\hfill \ensuremath{\Box}}


\newcommand*\circled[1]{\tikz[baseline=(char.base)]{
            \node[shape=circle,draw,inner sep=2pt] (char) {#1};}}

\newcommand{\Z}{\mathbb{Z}}
\newcommand{\I}{\mathbb{I}}
\newcommand{\M}{\mathbb{M}}
\newcommand{\R}{\mathbb{R}}
%\setcounter{section}{-1}

\begin{document}
\begin{enumerate}
%%%%%%%%%%%%%%%%%%%%%%%%%%%%%%%%%%%%%Prob1%%%%%%%%%%%%%%%%%%%%%%%%%%%%%%%%%%%%%%
   \item Consider the veracity or falsehood of each of the following statements.
         For bonus, argue for those that you believe are true while providing a
         counterexample for those that you believe are false. Let
         $G = \langle g\rangle$ have order 300.

         \begin{enumerate}[label=\protect\circled{\arabic*}]
            \item There are exactly 80 generators of $G$.
            \item $G$ has only one element of order 3.
            \item $G$ can be embedded in $S_{30}$.
            \item $G$ has a subgroup of order 20.
            \item $G$ has a totality of 18 subgroups.
         \end{enumerate}

      \textbf{Solution.}

      \begin{enumerate}[label=\protect\circled{\arabic*}]
         \item True. Since $G = \langle g\rangle$ is cyclic and since
               $|g| = 300$, it follows that the number of generators of $G$ is
               the number of positive integers relatively prime to 300, which is
               80.
         \item False.

               \textbf{Counterexample.} We have that $g^{100} \neq g^{200}$
               (since $|g| = 300)$
               and
               $$|g^{100}| = \frac{300}{\gcd(300, 100)} = 3 = 
                 \frac{300}{\gcd(300, 200)} = |g^{200}|.$$
         \item False.

               \textbf{Proof.} It suffices to show that $S_{30}$ has no element
               of order 300. Suppose to the contrary that $\sigma \in S_{30}$
               has order 300. Then we can write $\sigma$ as a product of
               disjoint cycles (each of length greater than 1)
               $$\sigma = \alpha_1\alpha_2\cdots\alpha_n$$
               so that $|\sigma| = \text{lcm}(|\alpha_1|, |\alpha_2|, \cdots,
               |\alpha_n|) = 300$. Now since $5^2 \mid 300$, it follows that
               $5^2$ must divide the order of at least one of the cycles. We can
               assume without loss that $5^2 \mid |\alpha_1|$. Thus $\alpha_1$
               must be a 25-cycle. By a similar argument, it follows that $2^2$
               must divide the order of at least one of the cycles. Assume
               without loss that $2^2 \mid |\alpha_2|$. Since there are 25
               elements in $\alpha_1$, there can be at most 5 elements in
               $\alpha_2$, so that $\alpha_2$ is a 4-cycle. Thus
               $$\sigma = \alpha_1\alpha_2,$$
               a contradiction since
               $\text{lcm}(|\alpha_1|, |\alpha_2|) = 100 \neq 300$.
               \qed
         \item True. This subgroup of $G$, $\langle g^{15}\rangle$, has 20
               elements.
         \item True. Since the number of positive divisors of 300 is 18, it
               follows that $G$ has exactly 18 subgroups.
               
      \end{enumerate}
%%%%%%%%%%%%%%%%%%%%%%%%%%%%%%%%%%%%%Prob2%%%%%%%%%%%%%%%%%%%%%%%%%%%%%%%%%%%%%%
   \item Let $G$ be an abelian group and let $a, b \in G$ be of order 120 and 72
         respectively. Do the following:

         \begin{enumerate}[label=\protect\circled{\arabic*}]
            \item Find an element of order 15.
            \item What is the order of $b^{10}$?
            \item Find an element of as large an order as you can.
         \end{enumerate}

      \textbf{Solution.}

      \begin{enumerate}[label=\protect\circled{\arabic*}]
         \item The element $a^8$ has order 15.
         \item $\displaystyle|b^{10}| = \frac{72}{\gcd(72, 10)} = 36$.
         \item The element $a^{24}b$ has order 360.
      \end{enumerate}
%%%%%%%%%%%%%%%%%%%%%%%%%%%%%%%%%%%%%Prob3%%%%%%%%%%%%%%%%%%%%%%%%%%%%%%%%%%%%%%
   \item Consider the non-abelian group of order 55 from \textbf{Homework \#4}.
         View this group as acting on all column vectors of size 2 (with entries
         in $\Z_{11}$).

         \begin{enumerate}[label=\protect\circled{\arabic*}]
            \item Find the number of fixed points of
                  $\left(\begin{tabular}{@{}l r@{}}
                     1 & 1 \\
                     0 & 1
                  \end{tabular}\right)$.
            \item Find the number of fixed points of
                  $\left(\begin{tabular}{@{}l r@{}}
                     3 & 0 \\
                     0 & 4
                  \end{tabular}\right)$.
            \item Decide on the number of fixed elements each of the elements of
                  the group has.
            \item Use Burnside's Lemma to count the orbits.
         \end{enumerate}

      \textbf{Solution.}

      \begin{enumerate}[label=\protect\circled{\arabic*}]
         \item Suppose
               $\left(\begin{tabular}{@{}l r@{}}
                  1 & 1 \\
                  0 & 1
               \end{tabular}\right)\left(\begin{tabular}{@{}c@{}}
                  $a$ \\
                  $b$
               \end{tabular}\right) = \left(\begin{tabular}{@{}c@{}}
                  $a$ \\
                  $b$
               \end{tabular}\right)$ for some $a, b \in \Z_{11}$. Then it
               follows that
               $$\left(\begin{tabular}{@{}c@{}}
                  $a + b$ \\
                  $b$
               \end{tabular}\right) = \left(\begin{tabular}{@{}c@{}}
                  $a$ \\
                  $b$
               \end{tabular}\right),$$
               so that $b = 0$. Thus the matrix
               $\left(\begin{tabular}{@{}l r@{}}
                  1 & 1 \\
                  0 & 1
               \end{tabular}\right)$ fixes 11 elements and they are
               $$\left\{\left(\begin{tabular}{@{}c@{}}
                  $a$ \\
                  0
               \end{tabular}\right) : a \in \Z_{11}\right\}.$$
         \item Suppose
               $\left(\begin{tabular}{@{}l r@{}}
                  3 & 0 \\
                  0 & 4
               \end{tabular}\right)\left(\begin{tabular}{@{}c@{}}
                  $a$ \\
                  $b$
               \end{tabular}\right) = \left(\begin{tabular}{@{}c@{}}
                  $a$ \\
                  $b$
               \end{tabular}\right)$ for some $a, b \in \Z_{11}$. Then it
               follows that
               $$\left(\begin{tabular}{@{}c@{}}
                  $3a$ \\
                  $4b$
               \end{tabular}\right) = \left(\begin{tabular}{@{}c@{}}
                  $a$ \\
                  $b$
               \end{tabular}\right),$$
               so that $2a = 0$ and $3b = 0$. Multiply the former equality by 6
               and the latter by 4 to get $a = b = 0$.Thus the matrix
               $\left(\begin{tabular}{@{}l r@{}}
                  3 & 0 \\
                  0 & 4
               \end{tabular}\right)$ fixes only 1 element, the zero vector.
         \item There are 44 matrices of the form
               $\left(\begin{tabular}{@{}l r@{}}
                  $b$ & $x$ \\
                  0 & $b^{-1}$
               \end{tabular}\right)$, with $b \neq 1$, and each only fixes the
               zero vector. The identity matrix fixes all the vectors
               (121 of them), while each of the remaining 10 matrices fixes
               exactly 11 vectors.
         \item Let $n$ be the number of orbits. Using our results from
               \circled{3} and Burnside's Lemma, it follows that
               that
               $$n \cdot 55 = 44 \cdot 1 + 1 \cdot 121 + 10 \cdot 11 = 275,$$
               so that $n = 5$.
      \end{enumerate}
%%%%%%%%%%%%%%%%%%%%%%%%%%%%%%%%%%%%%Prob4%%%%%%%%%%%%%%%%%%%%%%%%%%%%%%%%%%%%%%
   \item Let the vertices of the cube be given as follows:
         $$
            1 = \left(\begin{tabular}{@{}c@{}}
                   1 \\
                   1 \\
                   1
                \end{tabular}\right),
            2 = \left(\begin{tabular}{@{}r@{}}
                   1 \\
                   1 \\
                   $-1$
                \end{tabular}\right),
            3 = \left(\begin{tabular}{@{}r@{}}
                   1 \\
                   $-1$ \\
                   1
                \end{tabular}\right),
            4 = \left(\begin{tabular}{@{}r@{}}
                   $-1$ \\
                   1 \\
                   1
                \end{tabular}\right),
            5 = \left(\begin{tabular}{@{}r@{}}
                   1 \\
                   $-1$ \\
                   $-1$
                \end{tabular}\right),
            6 = \left(\begin{tabular}{@{}r@{}}
                   $-1$ \\
                   1 \\
                   $-1$
                \end{tabular}\right),
            7 = \left(\begin{tabular}{@{}r@{}}
                   $-1$ \\
                   $-1$ \\
                   1
                \end{tabular}\right),
         $$
         and $8 = \left(\begin{tabular}{@{}r@{}}
                     $-1$ \\
                     $-1$ \\
                     $-1$
                  \end{tabular}\right)$.

         \begin{enumerate}[label=\protect\circled{\arabic*}]
            \item Label the faces $A, A', B, B', C,$ and $C'$ (where the prime
                  means opposite), and give each as a set of four vertices. Let
                  $A$ be the intersection with the plane $x = 1$, $B$ with the
                  plane $y = 1$ and $C$ with $z = 1$.
            \item Find 24 $3 \times 3$ matrices of determinant 1 that are
                  isometries of the cube, and write each as a permutation in
                  $S_8$ (of the eight vertices) and also as a permutation of the
                  faces. \textbf{Hint:} Start with the six permutation matrices
                  of size 3. \\ \\
                  Assume these 24 matrices form a group $G$. \textbf{Bonus.}
                  Prove this. \\
                  Assume these $G \simeq S_4$. \textbf{Bonus.}
                  Prove this. \\
            \item Find the number of ways to color a cube with two colors.
            \item Find the number of ways to color a cube with three colors.
            \item[\textbf{Bonus.}]  Find the number of ways to color the cube
                                    with $n$ colors.
         \end{enumerate}

      \textbf{Solution.}

      \begin{enumerate}[label=\protect\circled{\arabic*}]
         \item s
      \end{enumerate}
\end{enumerate}
\end{document}
