\documentclass[9pt]{article}

\usepackage{amssymb}
\usepackage{amsmath}
\usepackage{amsfonts}
\usepackage{comment}
\usepackage{fancyhdr}
\usepackage{mathrsfs}
\usepackage{enumitem}


\usepackage{tikz}

\voffset = -50pt
%\textheight = 700pt
\addtolength{\textwidth}{60pt}
\addtolength{\evensidemargin}{-30pt}
\addtolength{\oddsidemargin}{-30pt}
%\setlength{\headheight}{44pt}

\pagestyle{fancy}
\fancyhf{} % clear all fields
\fancyhead[R]{%
  \scshape
  \begin{tabular}[t]{@{}r@{}}
  MATH 444, Spring 2015\\Section 1 (5562)\\
  HW \#12, DUE: 2015, MAY 04
  \end{tabular}}
\fancyhead[L]{%
  \scshape
  \begin{tabular}[t]{@{}r@{}}
  JOSEPH OKONOBOH\\Mathematics\\Cal State Long Beach
  \end{tabular}}
\fancyfoot[C]{\thepage}

\newcommand{\qed}{\hfill \ensuremath{\Box}}


\newcommand*\circled[1]{\tikz[baseline=(char.base)]{
            \node[shape=circle,draw,inner sep=2pt] (char) {#1};}}

\newcommand{\Z}{\mathbb{Z}}
\newcommand{\I}{\mathbb{I}}
\newcommand{\M}{\mathbb{M}}
\newcommand{\Q}{\mathbb{Q}}
\newcommand{\R}{\mathbb{R}}
\newcommand{\C}{\mathbb{C}}
\newcommand{\D}{\displaystyle}
%\setcounter{section}{-1}

\begin{document}
\begin{enumerate}
%%%%%%%%%%%%%%%%%%%%%%%%%%%%%%%%%%%%%Prob1%%%%%%%%%%%%%%%%%%%%%%%%%%%%%%%%%%%%%%
   \item Consider the veracity or falsehood of each of the following statements.
         For bonus, argue for those that you believe are true while providing a
         counterexample for those that you believe are false.

         \begin{enumerate}[label=\protect\circled{\arabic*}]
            \item There is a field with 16 elements.
            \item In $\Z[\sqrt{7}]$, $\left(\begin{tabular}{@{}cc@{}}
                     9  & 4 \\
                     28 & 9
                  \end{tabular}\right)$ is a prime.
            \item The polynomial $x^3+ax^2+bx+c \in \Z[x]$ where $a$ is odd, $b$
                  is even and $c$ is odd, is always irreducible over $\Z$.
            \item $\Z[\sqrt{-5}]$ is a UFD.
            \item In $\Z[\sqrt{7}]$, $\left(\begin{tabular}{@{}cc@{}}
                     8  & 3 \\
                     21 & 8
                  \end{tabular}\right)$ is a unit.
         \end{enumerate}
         
      \textbf{Solution.}

      \begin{enumerate}[label=\protect\circled{\arabic*}]
         \item True.

               \textbf{Example.} Let $F = \Z_2[x]/(x^4+x+1)$. That is, $F$
               consists of the polynomials in $\Z_2[x]$ mod $x^4+x+1$. Thus $F$
               is the set of all polynomials of degree less than 4 with
               coefficients in $\Z_2[x]$, so that $|F| = 16$. Addition and
               multiplication in $F$ are carried out mod $x^4+x+1$. It is
               clear that $F$ is a commutative ring. Since
               \begin{align*}
                  1 \cdot 1 &= 1 \\
                  x(x^3 + 1) &= 1 \\
                  (x+1)(x^3+x^2+x) &= 1 \\
                  x^2(x^3+x^2+1) &= 1 \\
                  (x^2+1)(x^3+x+1) &= 1 \\
                  (x^2+x)(x^2+x+1) &= 1 \\
                  x^3(x^3+x^2+x+1) &= 1 \\
                  (x^3+x^2)(x^3+x) &= 1,
               \end{align*}
               it follows that every nonzero element of $F$ has a multiplicative
               inverse, so that $F$ is a field.
         \item True. It follows immediately since the determinant is a prime
               that is also co-prime with  7.
         \item True.

               \textbf{Proof.} Let $p(x) = x^3+ax^2+bx+c \in \Z[x]$, where $b$
               is even and $a$ and $c$ are odd. Suppose to the contrary that
               $p(x)$ is not irreducible. Then it follows that $p(x)$ must have
               a root, say $x_0 \in \Z$. So
               $0 = p(x_0) = {x_0}^3 + a{x_0}^2 + bx_0 + c$. That is,
               $x_0(-{x_0}^2 - ax_0 - b) = c$. This says that $x_0$ is a divisor
               of $c$. Then since $c$ is odd, it must be the case that $x_0$ is
               also odd. But then we must have that ${x_0}^3$ is odd, $a{x_0}^2$
               is odd, and $bx_0$ is even, so that
               ${x_0}^3 + a{x_0}^2 + bx_0 + c$ is odd, a contradiction since
               ${x_0}^3 + a{x_0}^2 + bx_0 + c = 0$ is even. Thus $p(x)$ is
               irreducible. \qed
         \item False.

               \textbf{Proof.} First we want to show that the elements
               $$A = \left(\begin{tabular}{@{}cc@{}}
                  2 & 0 \\
                  0 & 2
               \end{tabular}\right), B = \left(\begin{tabular}{@{}cc@{}}
                  3 & 0 \\
                  0 & 3
               \end{tabular}\right), C = \left(\begin{tabular}{@{}rc@{}}
                  1 & 1 \\
                  $-5$ & 1
               \end{tabular}\right), \text{ and }
               D = \left(\begin{tabular}{@{}cr@{}}
                  1 & $-1$ \\
                  5 & 1
               \end{tabular}\right)$$
               are irreducible in $\Z[\sqrt{-5}]$. We shall only show that $A$
               and $C$ are irreducible since the arguments for $B$ and $D$ are
               similar. Suppose $A = A_1A_2$. Then it follows that
               $$4 = \det(A) = \det(A_1)\det(A_2).$$
               Observe that since we are in $\Z[\sqrt{-5}]$, it is impossible
               for the determinant of any matrix to be 2. Moreover, since the
               determinant of every matrix is nonnegative, we must have that
               either $A_1$ or $A_2$ has determinant of 1, so that one of
               $A_1$ and $A_2$ is a unit. Thus $A$ is irreducible. Now suppose
               $C = C_1C_2$. Then it follows that
               $$6 = \det(C) = \det(C_1)\det(C_2).$$
               No matrix has determinat 2 or 3 in $\Z[\sqrt{-5}]$. Thus one of
               $C_1$ or $C_2$ must be a unit, and it follows that $C$ is
               irreducible. Since the units in $\Z[\sqrt{-5}]$ are $\pm1$, it
               follows that none of the irreducibles above are associates. It
               follows immediately that $\Z[\sqrt{-5}]$ is not a UFD because we
               have the following two distinct factorizations into irreducibles:
               $$\left(\begin{tabular}{@{}cc@{}}
                  6 & 0 \\
                  0 & 6
               \end{tabular}\right) = \left(\begin{tabular}{@{}cc@{}}
                  2 & 0 \\
                  0 & 2
               \end{tabular}\right)\left(\begin{tabular}{@{}cc@{}}
                  3 & 0 \\
                  0 & 3
               \end{tabular}\right) = \left(\begin{tabular}{@{}rc@{}}
                  1 & 1 \\
                  $-5$ & 1
               \end{tabular}\right)\left(\begin{tabular}{@{}cr@{}}
                  1 & $-1$ \\
                  5 & 1
               \end{tabular}\right).$$ \qed
         \item True. Since the determinant of the matrix in question is 1, it is
               a unit.              
      \end{enumerate}
%%%%%%%%%%%%%%%%%%%%%%%%%%%%%%%%%%%%%Prob2%%%%%%%%%%%%%%%%%%%%%%%%%%%%%%%%%%%%%%
   \item In a previous homework we encountered the integral domain $R$ of
         $2 \times 2$ matrices of the form
         $A = \left(\begin{tabular}{@{}rc@{}}
            $a$ & $b$ \\
            $-b$ & $a-b$
         \end{tabular}\right)$ where $a, b \in \Z$. Do the following:

         \begin{enumerate}[label=\protect\circled{\arabic*}]
            \item Prove that no element of $R$ can have a negative determinant.
            \item Find a nontrivial unit.
            \item Find all units. Give an argument for your answer.
            \item Find an element whose determinant is a prime.
            \item Decide whether $\left(\begin{tabular}{@{}cc@{}}
                     3  & 0 \\
                     0  & 3
                  \end{tabular}\right)$ is irreducible or not. If not factor it
                  into irreducibles.
            \item Do the same for $\left(\begin{tabular}{@{}cc@{}}
                     5  & 0 \\
                     0  & 5
                  \end{tabular}\right)$.
            \item Do the same for $\left(\begin{tabular}{@{}rr@{}}
                     34  & 41 \\
                     $-41$  & $-7$
                  \end{tabular}\right)$.
            \item Show that the element $A = \left(\begin{tabular}{@{}cc@{}}
                     2 & 0 \\
                     0 & 2
                  \end{tabular}\right)$ is a prime by showing that if
                  $MN \equiv 0$ mod $A$, then either $M \equiv 0$ mod $A$ or
                  $N \equiv 0$ mod $A$.
         \end{enumerate}
         
      \textbf{Solution.}

      \begin{enumerate}[label=\protect\circled{\arabic*}]
         \item \textbf{Proof.} Let $A = \left(\begin{tabular}{@{}rc@{}}
                  $a$ & $b$ \\
                  $-b$ & $a-b$
               \end{tabular}\right) \in R$. Since
               $$\det(A) = a(a-b) + b^2 = \left(a - \frac{b}{2}\right)^2 +
                 \frac{3}{4}b^2 \ge 0,$$
               it follows that no element of $R$ can have a negative
               determinant. \qed
         \item A nontrivial unit in $R$ is $\left(\begin{tabular}{@{}rc@{}}
                  1 & 1 \\
                  $-1$ & 0
               \end{tabular}\right)$.
         \item Let
               $A = \left(\begin{tabular}{@{}rc@{}}
                  $a$ & $b$ \\
                  $-b$ & $a - b$
               \end{tabular}\right) \in R$ be a unit. It follows by \circled{1}
               that
               $$1 = \det(A) = a^2 + b^2 - ab;$$
               thus we want integers $a$ and $b$ such that $a^2 + b^2 - ab = 1$.
               By completing the square we get that
               $$a^2 + b^2 - ab = 1 \text{ iff } a = \frac{b}{2} \pm
                 \sqrt{\frac{4 - 3b^2}{4}}.$$
               For the discrimant to be nonnegative, we must have that $b = 0$
               or $|b| = 1$. It follows that $(a, b)$ is an integral solution of
               $a^2 + b^2 - ab = 1$ iff
               $$(a, b) \in \{(-1, 0), (1, 0), (0, 1), (1, 1),
                 (0, -1), (-1, -1)\}.$$
               Thus the group of units is
               $$\left\{I, -I, \left(\begin{tabular}{@{}rr@{}}
                  0 & 1 \\
                  $-1$ & $-1$
               \end{tabular}\right), \left(\begin{tabular}{@{}rr@{}}
                  1 & 1 \\
                  $-1$ & 0
               \end{tabular}\right), \left(\begin{tabular}{@{}rr@{}}
                  0 & $-1$ \\
                  1 & 1
               \end{tabular}\right), \left(\begin{tabular}{@{}rr@{}}
                  $-1$ & $-1$ \\
                  1 & 0
               \end{tabular}\right)\right\}.$$
         \item The element $\left(\begin{tabular}{@{}rc@{}}
                  2 & 1 \\
                  $-1$ & 1
               \end{tabular}\right)$ has determinant 3.
         \item The matrix $\left(\begin{tabular}{@{}cc@{}}
                  3 & 0 \\
                  0 & 3
               \end{tabular}\right)$ is not irreducible since we have the
               following factorization into irredcibles:
               $$\left(\begin{tabular}{@{}cc@{}}
                     3 & 0 \\
                     0 & 3
                  \end{tabular}\right) = \left(\begin{tabular}{@{}rr@{}}
                     $-1$  & $-2$ \\
                     2 & 1
                  \end{tabular}\right)\left(\begin{tabular}{@{}rr@{}}
                     1 & 2 \\
                     $-2$ & $-1$
                  \end{tabular}\right).$$
               The factors in the factorization above are irreducible since
               their determinants are prime.
         \item Let $B = \left(\begin{tabular}{@{}cc@{}}
                  5 & 0 \\
                  0 & 5
               \end{tabular}\right)$. Claim that $B$ is irreducible.

               \textbf{Proof.} Suppose $B = XY$. Then it follows that
               $$25 = \det(B) = \det(X)\det(Y).$$
               Now suppose that $\det(X) = 5$. Then if we have that
               $X = \left(\begin{tabular}{@{}rc@{}}
                  $x$ & $y$ \\
                  $-y$ & $x - y$
               \end{tabular}\right)$, it follows that $x^2 - xy + y^2 = 5$.
               That is
               $$x = \frac{y}{2} \pm \sqrt{\frac{20 - 3y^2}{4}}.$$
               By observing the discrimant, we see that $y$ can only take on
               values 0, 1, and 2. But $x$ is not an integer for any of these
               values. Thus no matrix in $R$ exists with determinant 5. It
               follows that one of $X$ and $Y$ must have determinant 1, so
               that this matrix is a unit; thus $B$ is irreducible in $R$. \qed
         \item The matrix $\left(\begin{tabular}{@{}rr@{}}
                  34 & 41 \\
                  $-41$ & $-7$
               \end{tabular}\right)$ is not irreducible since we have the
               following factorization into irredcibles:
               $$\left(\begin{tabular}{@{}rr@{}}
                     34 & 41 \\
                     $-41$ & $-7$
                  \end{tabular}\right) = \left(\begin{tabular}{@{}rr@{}}
                     2 & 1 \\
                     $-1$ & 1
                  \end{tabular}\right)\left(\begin{tabular}{@{}rr@{}}
                     4 & 1 \\
                     $-1$ & 3
                  \end{tabular}\right)\left(\begin{tabular}{@{}rr@{}}
                     7 & 3 \\
                     $-3$ & 4
                  \end{tabular}\right).$$
               The factors in the factorization above are irreducible since
               their determinants are prime.
         \item \textbf{Proof.} Suppose $MN \equiv 0$ mod $A$. That is,
               $AX = MN$ for some matrix $X \in R$. Thus
               $$4\det(X) = \det(A)\det(X) = \det(AX) = \det(MN) =
                 \det(M)\det(N).$$
               We can then conclude that the determinants of $M$ and $N$ cannot
               be both odd. So suppose without loss that $\det(M) = 2k$ for some
               integer $k$. Now if $$M = \left(\begin{tabular}{@{}rc@{}}
                  $x$ & $y$ \\
                  $-y$ & $x-y$
               \end{tabular}\right),$$
               then $x^2 - xy + y^2 = 2k$. If $x$ and $y$ are both odd, then
               $x^2 - xy + y^2$ will also be odd, a contradiction. If $x$ is
               odd and $y$ is even (or vice-versa), then $x^2 - xy + y^2$ will
               again be odd. Thus the only viable option is that $x$ and $y$ are
               both even. Now write $x = 2k_1$ and $y = 2k_2$ for some integers
               $k_1$ and $k_2$. Since $X' = \left(\begin{tabular}{@{}rc@{}}
                  $k_1$ & $k_2$ \\
                  $-k_2$ & $k_1-k_2$
               \end{tabular}\right) \in R$ and since $AX' = M$, it follows that
               $M \equiv 0$ mod $A$, so that $A$ is prime. \qed
      \end{enumerate}
%%%%%%%%%%%%%%%%%%%%%%%%%%%%%%%%%%%%%Prob3%%%%%%%%%%%%%%%%%%%%%%%%%%%%%%%%%%%%%%
   \item \textbf{On Nilpotent Elements.}

         \begin{enumerate}[label=\protect\circled{\arabic*}]
            \item Let $R$ be a ring. An element $m \in R$ is called nilpotent if
                  $m^k = 0$ for some positive integer $k$. Let
                  $r = 1 + m + m^2 + \cdots + m^{k-1}$. Show $r$ is invertible
                  by finding its inverse.
            \item Exemplify \circled{1} by using the matrix
                  $\left(\begin{tabular}{@{}ccc@{}}
                     0 & 1 & 0 \\
                     0 & 0 & 1 \\
                     0 & 0 & 0
                  \end{tabular}\right)$.
         \end{enumerate}
         
      \textbf{Solution.}

      \begin{enumerate}[label=\protect\circled{\arabic*}]
         \item We have
               \begin{align*}
                  rm &= (1 + m + m^2 + \cdots + m^{k-1})m \\
                     &= m + m^2 + \cdots + m^{k-1} + m^k \\
                     &= m + m^2 + \cdots + m^{k-1} \\
                     &= r - 1,
               \end{align*}
               so that $r(1 - m) = 1$. Similarly
               \begin{align*}
                  mr &= m(1 + m + m^2 + \cdots + m^{k-1}) \\
                     &= m + m^2 + \cdots + m^{k-1} + m^k \\
                     &= m + m^2 + \cdots + m^{k-1} \\
                     &= r - 1,
               \end{align*}
               so that $(1 - m)r = 1$. We have thus shown that the
               multiplicative inverse of $r$ is $1 - m$. Thus $r$ is invertible.
         \item Let $B = \left(\begin{tabular}{@{}ccc@{}}
                  0 & 1 & 0 \\
                  0 & 0 & 1 \\
                  0 & 0 & 0
               \end{tabular}\right)$. A quick computation will show us that the
               smallest positive integer $k$ for which $B^k = 0$ is 3. Thus if
               $r = I + B + B^2 = \left(\begin{tabular}{@{}ccc@{}}
                     1 & 1 & 1 \\
                     0 & 1 & 1 \\
                     0 & 0 & 1
               \end{tabular}\right)$, we have that
               $$r(I - B) = r\left(\begin{tabular}{@{}crr@{}}
                     1 & $-1$ & 1 \\
                     0 & 1 & $-1$ \\
                     0 & 0 & 1
               \end{tabular}\right) = I,$$
               so that $r$ is invertible.
      \end{enumerate}
%%%%%%%%%%%%%%%%%%%%%%%%%%%%%%%%%%%%%Prob4%%%%%%%%%%%%%%%%%%%%%%%%%%%%%%%%%%%%%%
   \item[\textbf{BONUS.}] Consider the integral domain $R = \Z[\sqrt{3}]$. Let
                          $A = \left(\begin{tabular}{@{}cc@{}}
                             5 & 3 \\
                             9 & 5
                          \end{tabular}\right)$. Decide if
                          \circled{1}-\circled{4} are irreducible or not. Argue
                          your case.

         \begin{enumerate}[label=\protect\circled{\arabic*}]
            \item $A$.
            \item $\left(\begin{tabular}{@{}cc@{}}
                     19 & 11 \\
                     33 & 19
                  \end{tabular}\right)$.
            \item $\left(\begin{tabular}{@{}cc@{}}
                     34 & 20 \\
                     60 & 34
                  \end{tabular}\right)$.
            \item $\left(\begin{tabular}{@{}cc@{}}
                     362 & 209 \\
                     627 & 362
                  \end{tabular}\right)$.
            \item Factor $\left(\begin{tabular}{@{}cc@{}}
                     69 & 0 \\
                     0 & 69
                  \end{tabular}\right)$ into irreducibles.
            \item Show that $A$ is prime by showing that $R_A$ is a field.
                  \textbf{Hint.} Show that if $M \in R$ has even determinant,
                  then $A \mid M$, and if $M$ has odd determinant, then
                  $A \mid (M-I)$.
            \item Find a nontrivial common divisor of
                  $\left(\begin{tabular}{@{}cc@{}}
                     89 & 53 \\
                     159 & 89
                  \end{tabular}\right)$ and $\left(\begin{tabular}{@{}cc@{}}
                     86 & 48 \\
                     144 & 86
                  \end{tabular}\right)$, and show why it is a common divisor.
            \item Find the greatest common divisor of
                  $\left(\begin{tabular}{@{}cc@{}}
                     89 & 53 \\
                     159 & 89
                  \end{tabular}\right)$ and $\left(\begin{tabular}{@{}cc@{}}
                     86 & 48 \\
                     144 & 86
                  \end{tabular}\right)$, and give reasons.
            \item Find the lcm of
                  $\left(\begin{tabular}{@{}cc@{}}
                     89 & 53 \\
                     159 & 89
                  \end{tabular}\right)$ and $\left(\begin{tabular}{@{}cc@{}}
                     86 & 48 \\
                     144 & 86
                  \end{tabular}\right)$.
            \item[\textbf{More Bonus.}] Argue every element is a product of
                                        irreducibles in $R$.
            \item[\textbf{Hard Bonus.}] Argue every irreducible is prime.
         \end{enumerate}
         
      \textbf{Solution.}

      \begin{enumerate}[label=\protect\circled{\arabic*}]
         \item Since the determinant of $A$ is 1, it follows that $A$ is a unit
               so that $A$ is not irreducible.
         \item This matrix is irreducible since its determinant is a prime.
         \item Not irreduible since (use idea in \circled{6} to continually
               factor $A$ out of the given matrix)
               $$\left(\begin{tabular}{@{}cc@{}}
                     34 & 20 \\
                     60 & 34
                  \end{tabular}\right) = A^2\left(\begin{tabular}{@{}rr@{}}
                  $-8$ & 5 \\
                  15 & $-8$
               \end{tabular}\right).$$
         \item This matrix is not irreducible since it is a unit.
         \item 
               $$\left(\begin{tabular}{@{}cc@{}}
                     69 & 0 \\
                     0 & 69
                  \end{tabular}\right) = \left(\begin{tabular}{@{}rr@{}}
                  3 & 0 \\
                  0 & 3
               \end{tabular}\right)\left(\begin{tabular}{@{}rr@{}}
                  23 & 0 \\
                  0 & 23
               \end{tabular}\right)$$
         \item \textbf{Proof.} We want to first prove the hint. So
               suppose $M = \left(\begin{tabular}{@{}cc@{}}
                  $m$ & $n$ \\
                  $3n$ & $m$
               \end{tabular}\right)$ has even determinant. Observe that if $m$
               and $n$ have different parities, then the determinant of $M$ must
               be odd. Thus $m$ and $n$ must either be both odd or both even.
               Now we want to find a matrix $B = \left(\begin{tabular}{@{}cc@{}}
                  $a$ & $b$ \\
                  $3b$ & $a$
               \end{tabular}\right) \in R$ such that $BA = X$. Solving the
               equation $BA = X$ for $a$ and $b$ will give us the following:
               \begin{align*}
                  a &= -\frac{1}{2}(5m - 9n) \\
                  b &= \frac{1}{2}(3m - 5n).
               \end{align*}
               But since $m$ and $n$ have the same parities, it follows that
               $5m - 9n$ and $3m - 5n$ are even so that $a$ and $b$ are
               integers. Thus $B \in R$ and $A \mid M$. Now suppose that $M$ has
               an odd determinant. Thus $m$ and $n$ have different parities. So
               suppose first that $m$ is even and $n$ is odd. Thus it follows
               that $m - 1$ is odd, so that $M - I$ has an even determinant and
               so is divisible by $A$. Similarly if $m$ is odd and $n$ is even,
               then $m - 1$ must be even so that $M - I$ has an even determinant
               and thus is also divisible by $A$. Thus we have proven the hint.
               Now consider $R/(A)$ (the set of elements in
               $\Z[\sqrt{3}]$ mod $A$). Let $C \in R$. If $\det(C)$ is even,
               then $C = AX'$ for some $X' \in R$. This means that
               $C \equiv 0$ mod $A$. Now if $\det(C)$ is odd, then
               $C - I = AX''$ for some $X'' \in R$. Thus $C \equiv I$ mod $A$.
               Hence $R/(A) = \{\textbf{0}, I\}$, which is clearly a field,
               where $\textbf{0}$ is the additive inverse and $I$ is the
               multiplicative inverse. Now we are ready to show that $A$ is
               prime. Suppose for some $S, T \in R$, $A$ divides $ST$, so that
               $AY = ST$ for some $Y \in R$. Viewing this equation $AY = ST$ in
               $R/(A)$, we must have that $ST = AY = \textbf{0}Y = \textbf{0}$.
               Since $R/(A)$ is a field, it must be an integral domain. Hence
               $ST = \textbf{0}$ implies that $S = \textbf{0}$ or
               $T = \textbf{0}$. That is $A$ divides $S$ or $A$ divides $T$, so
               that $A$ is prime. \qed
         \item Since the determinants of both matrices in question are even,
               then we know from \circled{6} that they are both divisible by
               $A$. Thus $A$ is a common divisor of both matrices.         
      \end{enumerate}
\end{enumerate}
\end{document}
