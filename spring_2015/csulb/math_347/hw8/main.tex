\documentclass[9pt]{article}

\usepackage{amssymb}
\usepackage{amsmath}
\usepackage{amsfonts}
\usepackage{cancel}
\usepackage{comment}
\usepackage{fancyhdr}
\usepackage{mathrsfs}

\voffset = -50pt
%\textheight = 700pt
\addtolength{\textwidth}{60pt}
\addtolength{\evensidemargin}{-30pt}
\addtolength{\oddsidemargin}{-30pt}
%\setlength{\headheight}{44pt}

\pagestyle{fancy}
\fancyhf{} % clear all fields
\fancyhead[R]{%
  \scshape
  \begin{tabular}[t]{@{}r@{}}
  MATH 347, Spring 2015\\Section 1 (1872)\\
  HW \#8, Due: 2015, May 5
  \end{tabular}}
\fancyhead[L]{%
  \scshape
  \begin{tabular}[t]{@{}r@{}}
  JOSEPH OKONOBOH\\Mathematics\\Cal State Long Beach
  \end{tabular}}
\fancyfoot[C]{\thepage}

\newcommand{\qed}{\hfill \ensuremath{\Box}}


\newcommand{\C}{\mathbb{C}}
\newcommand{\R}{\mathbb{R}}
\newcommand{\cyc}[1]{\langle #1 \rangle}
%\setcounter{secton}{-1}

\begin{document}
\begin{enumerate}
%%%%%%%%%%%%%%%%%%%%%%%%%%%%%%%%%%Prob8.02%%%%%%%%%%%%%%%%%%%%%%%%%%%%%%%%%%%%%%
   \item[8.02] Define $T \in \mathcal{L}(\C^2)$ by
               $$T(w, z) = (-z, w).$$
               Find all generalized eigenvectors.
               
      \textbf{Solution.} First we find the eigenvalues of $T$. Suppose
      $T(w, z) = \lambda(w, z)$. Then it follows that
      $(-z, w) = (\lambda w, \lambda z)$, so that $\lambda w = -z$ and
      $w = \lambda z$. Solving these equations will gives $\lambda = \pm i$.
      Thus the eigenvalues of $T$ are $i$ and $-i$. Since $\dim \C^2 = 2$, it
      follows that the generalized eignevectors corresponding to $i$ is
      $\text{null }(T - iI)^2$ and the generalized eignevectors corresponding to
      $-i$ is $\text{null }(T + iI)^2$. Suppose
      $(a, b) \in \text{null }(T - iI)^2$. Then we have that
      \begin{align*}
         0 &= (T - iI)^2(a, b) \\
           &= (T - iI)[(T - iI)(a, b)] \\
           &= (T - iI)[T(a, b) - (iI)(a, b)] \\
           &= (T - iI)[(-b, a) + (-ai, -bi)] \\
           &= (T - iI)(-b - ai, a - bi) \\
           &= T(-b - ai, a - bi) - (iI)(-b - ai, a - bi) \\
           &= (-a + bi, -b - ai) + (-a + bi, -b - ai) \\
           &= (-2a + 2bi, -2b - 2ai),
      \end{align*}
      so that $-2a + 2bi = 0$ and $-2b - 2ai = 0$. That is $a = bi$. Similarly 
      if $(c, d) \in \text{null }(T + iI)^2$. Then we have that
      \begin{align*}
         0 &= (T + iI)^2(c, d) \\
           &= (T + iI)[(T + iI)(c, d)] \\
           &= (T + iI)[T(c, d) + (iI)(c, d)] \\
           &= (T + iI)[(-d, c) + (ci, di)] \\
           &= (T + iI)(-d + ci, c + di) \\
           &= T(-d + ci, c + di) + (iI)(-d + ci, c + di) \\
           &= (-c - di, -d + ci) + (-c - di, -d + ci) \\
           &= (-2c - 2di, -2d + 2ci),
      \end{align*}
      so that $-2c - 2di = 0$ and $-2d + 2ci = 0$. That is $c = -di$. Thus
      $$\text{null }(T - iI)^2 = \{(xi, x) : x \in \C\} \text{ and }
        \text{null }(T + iI)^2 = \{(-yi, y) : y \in \C\}.$$
%%%%%%%%%%%%%%%%%%%%%%%%%%%%%%%%%%Prob8.03%%%%%%%%%%%%%%%%%%%%%%%%%%%%%%%%%%%%%%
   \item[8.03] Suppose $T \in \mathcal{L}(V)$, $m$ is a positive integer, and
               $v \in V$ is such that $T^{m-1}v \neq 0$ but $T^mv = 0$. Prove
               that
               $$(v, Tv, T^2v, \ldots, T^{m-1}v)$$
               is linearly independent.

      \textbf{Proof.} Consider the equation
      \begin{equation} \label{l8_3_0}
         a_0v + a_1Tv + a_2T^2v + \cdots + a_{m-1}T^{m-1}v = 0.
      \end{equation}
      
      Applying $T^{m-1}$ to equation \eqref{18_3_0} above will result in
      \begin{equation} \label{l8_3_1}
         a_0T^{m-1}v + a_1T^mv + a_2T^{m+1}v + \cdots + a_{m-1}T^{2m-2}v = 0.
      \end{equation}
      Notice since $T^mv = 0$, we must have $T^{m + i}v = 0$ for all $i \ge 0$.
      Thus equation \eqref{l8_3_1} reduces to $a_0T^{m-1}v = 0$, so that
      $a_0 = 0$ since $T^{m-1}v \neq 0$. Now equation \eqref{l8_3_0} reduces to
      
      \begin{equation} \label{l8_3_2}
         a_1Tv + a_2T^2v + \cdots + a_{m-1}T^{m-1}v = 0.
      \end{equation}
      Now apply $T^{m-2}$ to equation \eqref{l8_3_2} to get $a_1 = 0$. Then
      apply $T^{m - 3}$ to the simplified equation to get $a_2 = 0$, \ldots,
      and so on to get $a_{m-1} = 0$. Thus 
               $$(v, Tv, T^2v, \ldots, T^{m-1}v)$$
               is linearly independent.
%%%%%%%%%%%%%%%%%%%%%%%%%%%%%%%%%%Prob8.06%%%%%%%%%%%%%%%%%%%%%%%%%%%%%%%%%%%%%%
   \item[8.06] Suppose $N \in \mathcal{L}(V)$ is nilpotent. Prove (without
               using 8.26) that 0 is the only eigenvalue of $N$.

      \textbf{Proof.} 
%%%%%%%%%%%%%%%%%%%%%%%%%%%%%%%%%%Prob8.07%%%%%%%%%%%%%%%%%%%%%%%%%%%%%%%%%%%%%%
   \item[8.07] Suppose $V$ is an inner-product space. Prove that if
               $N \in \mathcal{L}(V)$ is self-adjoint and nilpotent, then
               $N = 0$.

      \textbf{Proof.}   
%%%%%%%%%%%%%%%%%%%%%%%%%%%%%%%%%%Prob8.11%%%%%%%%%%%%%%%%%%%%%%%%%%%%%%%%%%%%%%
   \item[8.11] Prove that if $T \in \mathcal{L}(V)$, then
               $$V = \text{null }T^n \oplus \text{range }T^n,$$
               where $n = \dim V$.

      \textbf{Proof.} 
\end{enumerate}
\end{document}
