\documentclass[9pt]{article}

\usepackage{amssymb}
\usepackage{amsmath}
\usepackage{amsfonts}
\usepackage{cancel}
\usepackage{comment}
\usepackage{fancyhdr}
\usepackage{mathrsfs}

\voffset = -50pt
%\textheight = 700pt
\addtolength{\textwidth}{60pt}
\addtolength{\evensidemargin}{-30pt}
\addtolength{\oddsidemargin}{-30pt}
%\setlength{\headheight}{44pt}

\pagestyle{fancy}
\fancyhf{} % clear all fields
\fancyhead[R]{%
  \scshape
  \begin{tabular}[t]{@{}r@{}}
  MATH 347, Spring 2015\\Section 1 (1872)\\
  HW \#6, Due: 2015, March 26
  \end{tabular}}
\fancyhead[L]{%
  \scshape
  \begin{tabular}[t]{@{}r@{}}
  JOSEPH OKONOBOH\\Mathematics\\Cal State Long Beach
  \end{tabular}}
\fancyfoot[C]{\thepage}

\newcommand{\qed}{\hfill \ensuremath{\Box}}


\newcommand{\F}{\mathbb{F}}
\newcommand{\R}{\mathbb{R}}
\newcommand{\cyc}[1]{\langle #1 \rangle}
%\setcounter{section}{-1}

\begin{document}
\begin{enumerate}
%%%%%%%%%%%%%%%%%%%%%%%%%%%%%%%%%%Prob7.1%%%%%%%%%%%%%%%%%%%%%%%%%%%%%%%%%%%%%%%
   \item[\textbf{Lemma 7.1}]  \textit{If $T \in \mathcal{L}(V)$ is normal, then
                              $\text{null }T = \text{null }T^*$}.

      \textbf{Proof.} Assume that $T \in \mathcal{L}(V)$ is normal. Thus
      \begin{align*}
         x \in \text{null }T &\Longleftrightarrow Tx = 0 \\
         &\Longleftrightarrow \cyc{Tx, Tx} = 0 \\
         &\Longleftrightarrow \cyc{T^*x, T^*x} = 0 &[\text{Proposition }7.6] \\
         &\Longleftrightarrow T^*x = 0 \\
         &\Longleftrightarrow x \in \text{null }T^*,
      \end{align*}
      so that $\text{null }T = \text{null }T^*$. \qed
%%%%%%%%%%%%%%%%%%%%%%%%%%%%%%%%%%Prob7.1%%%%%%%%%%%%%%%%%%%%%%%%%%%%%%%%%%%%%%%
   \item[7.1]  Make $\mathcal{P}_2(\R)$ into an inner-product space by defining
               $$\cyc{p, q} = \int_0^1p(x)q(x)\;dx.$$
               Define $T \in \mathcal{L}(\mathcal{P}_2(\R))$ by
               $T(a_0 + a_1x + a_2x^2) = a_1x$.
               \begin{enumerate}
                  \item Show that $T$ is not self-adjoint.
                  \item The matrix of $T$ with respect to the basis
                        $(1, x, x^2)$ is
                        $$\left[\begin{tabular}{@{}c c c@{}}
                           0 & 0 & 0 \\
                           0 & 1 & 0 \\
                           0 & 0 & 0
                        \end{tabular}\right].$$
                        The matrix equals its conjugate transpose, even though
                        $T$ is not self-adjoint. Explain why this is not a
                        contradiction.
               \end{enumerate}
               
      \textbf{Solution.}
      
      \begin{enumerate}
         \item Suppose to the contrary that $T$ is self-adjoint. Consider
               $x, 1 \in \mathcal{P}_2(\R)$. Then we must have that
               $\cyc{T1, x} = \cyc{T^*1, x} = \cyc{1, Tx}$. But
               \begin{align*}
                  \cyc{T1, x} &= \cyc{0, x} \\
                              &= 0,
               \end{align*}
               and
               \begin{align*}
                  \cyc{1, Tx} &= \cyc{1, x} \\
                                &= \int_0^1 x\;dx \\
                                &= \frac{1}{2},
               \end{align*}
               so that $\cyc{T1, x} \neq  \cyc{1, Tx}$; i.e., $T$ is not
               self-adjoint.
         \item Let $N$ be the conjugate transpose of $\mathcal{M}(T)$. It is 
               clear that $N = \mathcal{M}(T)$. We know that $T$ is self-adjoint
               if and only if $\mathcal{M}(T) = \mathcal{M}(T^*)$ if and only
               if $N = \mathcal{M}(T^*)$. Since $T$ is not self-adjoint it
               follows that $N \neq \mathcal{M}(T^*)$. This makes sense because
               the equality $N = \mathcal{M}(T^*)$ is guaranteed to be true
               (by Proposition 6.47) if $(1, x, x^2)$ is an orthonormal basis 
               for $\mathcal{P}_2(\R)$, but this is not the case since
               $\cyc{1, x} \neq 0$. So although we have that the matrix of $T$
               equals its conjugate transpose, it does not equal the matrix of
               $T^*$, so we have no contradiction.
      \end{enumerate}
%%%%%%%%%%%%%%%%%%%%%%%%%%%%%%%%%%Prob7.4%%%%%%%%%%%%%%%%%%%%%%%%%%%%%%%%%%%%%%%
   \item[7.4]  Suppose $P \in \mathcal{L}(V)$ is such that $P^2 = P$. Prove that
               $P$ is an orthogonal projection if and only if $P$ is
               self-adjoint.

      \textbf{Proof.} Let $P \in \mathcal{L}(V)$ such that $P^2 = P$.

      ($\Longrightarrow$) Suppose that $P$ is an orthogonal projection. Then we
      have that $P = P_{U,U^\perp}$ for some subspace $U$ of $V$. To prove that
      $P$ is self-adjoint, we must show that $Px = P^*x$ for all $x \in V$. So
      let $v_1, v_2 \in V$. There exist unique $u_1, u_2 \in U$ and
      $u_1', u_2' \in U^\perp$ such that $v_1 = u_1 + u_1'$ and
      $v_2 = u_2 + u_2'$. Thus
      \begin{align*}
         \cyc{Pv_1 - P^*v_1, v_2} &= \cyc{Pv_1, v_2} - \cyc{P^*v_1, v_2} \\
            &= \cyc{u_1, v_2} - \cyc{v_1, Pv_2} \\
            &= \cyc{u_1, v_2} - \cyc{v_1, u_2} \\
            &= \cyc{u_1, u_2 + u_2'} - \cyc{u_1 + u_1', u_2} \\
            &= \cyc{u_1, u_2} + \cyc{u_1, u_2'} -
               \cyc{u_1, u_2} - \cyc{u_1', u_2} \\
            &= \cyc{u_1, u_2'} - \cyc{u_1', u_2} \\
            &= 0 - 0 = 0. \qquad
               [\text{Since }u_1, u_2 \in U, u_1', u_2' \in U^\perp]
      \end{align*}
      Since $v_2$ was arbitrary, we have that $\cyc{Pv_1 - P^*v_1, v_2} = 0$
      for all $v_2 \in V$. Setting $v_2 = Pv_1 - P^*v_1$ gives us that
      $\cyc{Pv_1 - P^*v_1, Pv_1 - P^*v_1} = 0$, so that $Pv_1 - P^*v_1 = 0$;
      i.e., $Pv_1 = P^*v_1$. Thus $P$ is self-adjoint.

      ($\Longleftarrow$) Suppose that $P$ is self-adjoint. We have that
      \begin{align*}
         \text{null }P &= (\text{range }P*)^\perp &[\text{Proposition }6.46] \\
            &= (\text{range }P)^\perp &[P \text{ is self-adjoint}],
      \end{align*}
      so that every vector in $\text{null }P$ is orthogonal to every vector in
      $\text{range }P$. It follows by Homework 5, Problem 6.17 that $P$ is an
      orthogonal projection of $V$ onto $\text{range }P$, with null space
      $\text{null }P$. \qed
%%%%%%%%%%%%%%%%%%%%%%%%%%%%%%%%%%Prob7.6%%%%%%%%%%%%%%%%%%%%%%%%%%%%%%%%%%%%%%%
   \item[7.6]  Prove that if $T \in \mathcal{L}(V)$ is normal, then
               $$\text{range } T = \text{range } T^*.$$

      \textbf{Proof.} Assume that $T \in \mathcal{L}(V)$ is normal. We have that
      \begin{align*}
         \text{range } T &= (\text{null } T^*)^\perp
            &[\text{Proposition }6.46] \\
            &= (\text{null } T)^\perp  &[\text{Lemma }7.1] \\
            &= \text{range } T^*, &[\text{Proposition }6.46]
      \end{align*}
      which is what we wanted to prove. \qed
%%%%%%%%%%%%%%%%%%%%%%%%%%%%%%%%%%Prob7.7%%%%%%%%%%%%%%%%%%%%%%%%%%%%%%%%%%%%%%%
   \item[7.7]  Prove that if $T \in \mathcal{L}(V)$ is normal, then
               $$
                  \text{null } T^k = \text{null } T \text{ and }
                  \text{range }T^k = \text{range }T$$
               for every positive integer $k$.

      \textbf{Proof.} We shall first proceed by induction on $k$ to show that
      $\text{null } T^k = \text{null } T$.

      \textbf{Base Case.} $k = 1$. It follows trivially that
      $\text{null } T^1 = \text{null } T$.

      \textbf{Inductive Hypothesis.} Suppose that
      $\text{null } T^n = \text{null }T$ for some positive integer $n$. \\

      Now we shall show that $\text{null } T^{n+1} = \text{null }T$. We have 
      that
      \begin{align*}
         x \in \text{null }T &\Longrightarrow x \in \text{null }T^n
                  &[\text{Inductive Hypothesis}] \\
                  &\Longrightarrow T^nx = 0 \\
                  &\Longrightarrow T(T^nx) = T(0) = 0 \\
                  &\Longrightarrow T^{n+1}x = 0 \\
                  &\Longrightarrow x \in \text{null }T^{n+1},
      \end{align*}
      so that $\text{null } T \subseteq \text{null } T^{n+1}$. Now let
      $x \in \text{null }T^{n+1}$. By Problem 7.6, we know that
      $\text{range } T = \text{range } T^*$; thus $Tx = T^*y'$ for some
      $y' \in V$. Now
      \begin{align*}
         x \in \text{null }T^{n+1} &\Longrightarrow T^{n+1}x = 0 \\
                  &\Longrightarrow TT^nx = 0 \\
                  &\Longrightarrow \cyc{TT^nx, T^{n-1}y'} = 0 \\
                  &\Longrightarrow \cyc{T^nx, T^*T^{n-1}y'} = 0 \\
                  &\Longrightarrow \cyc{T^nx, T^{n-1}T^*y'} = 0
                  &[T\text{ is normal}] \\
                  &\Longrightarrow \cyc{T^nx, T^{n-1}Tx} \\
                  &\Longrightarrow \cyc{T^nx, T^nx} = 0 \\
                  &\Longrightarrow T^nx = 0 \\
                  &\Longrightarrow x \in \text{null }T^n \\
                  &\Longrightarrow x \in \text{null }T,
                  &[\text{Inductive Hypothesis}]
      \end{align*}
      so that $\text{null } T^{n+1} \subseteq \text{null } T$. We have thus 
      shown that $\text{null } T = \text{null } T^{n+1}$. It follows by
      Mathematical Induction that $\text{null } T^k = \text{null } T$ for each
      positive integer $k$. Let $m$ be a positive integer. Observe that $T^m$ is
      also normal; thus
      \begin{align*}
         \text{range }T^m &= (\text{null }(T^m)^*)^\perp
            &[\text{Proposition }6.46] \\
            &= (\text{null } T^m)^\perp &[\text{Lemma }7.1] \\
            &= (\text{null } T)^\perp &[\text{null }T^m = \text{null }T] \\
            &= \text{range } T^* &[\text{Proposition }6.46] \\
            &= \text{range } T, &[\text{Problem }7.6]
      \end{align*}
      which is what we wanted to show. \qed
%%%%%%%%%%%%%%%%%%%%%%%%%%%%%%%%%%Prob7.9%%%%%%%%%%%%%%%%%%%%%%%%%%%%%%%%%%%%%%%
   \item[7.9]  Prove that a normal operator on a complex inner-product space is
               self-adjoint if and only if all its eigenvalues are real.

      \textbf{Proof.} Let $T \in \mathcal{L}(V)$ be normal, where $V$ is a 
      complex inner-product space.

      ($\Longrightarrow$) Suppose that $T$ is self-adjoint. It follows by 
      Proposition 7.1 that all the eigenvalues of $T$ are real.

      ($\Longleftarrow$) Suppose that all the eigenvalues of $T$ are real. Note
      that if $u$ is an eigenvector of $T$ with eigenvalue $\lambda$, then by 
      Corollary 7.7, we have that $T^*u = \overline{\lambda}u$. To show that
      $T$ is self-adjoint, we must show that $Tx = T^*x$ for all $x \in V$. So
      let $y \in V$. By the Spectral Theorem, $V$ has an orthonormal basis
      $(v_1, \ldots v_n)$ consisting of eigenvectors of $T$. Let $\lambda_i$
      be the eigenvalue corresponding to $v_i$. We know that
      $y = a_1v_1 + \cdots + a_nv_n$ for some unique scalars. Thus
      \begin{align*}
         Ty &= T(a_1v_1 + \cdots + a_nv_n) \\
            &= a_1Tv_1 + \cdots + a_nTv_n \\
            &= a_1\lambda_1v_1 + \cdots + a_n\lambda_nv_n \\
            &= a_1\overline{\lambda_1}v_1 + \cdots + a_n\overline{\lambda_n}v_n
            &[\text{Eigenvalues of $T$ are real}] \\
            &= a_1T^*v_1 + \cdots + a_nT^*v_n \\
            &= T^*(a_1v_1 + \cdots + a_nv_n) \\
            &= T^*y,
      \end{align*}
      so that $T = T^*$. \qed
\end{enumerate}
\end{document}
