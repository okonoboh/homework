\documentclass[9pt]{article}

\usepackage{amssymb}
\usepackage{amsmath}
\usepackage{amsfonts}
\usepackage{cancel}
\usepackage{comment}
\usepackage{fancyhdr}
\usepackage{mathrsfs}

\voffset = -50pt
%\textheight = 700pt
\addtolength{\textwidth}{60pt}
\addtolength{\evensidemargin}{-30pt}
\addtolength{\oddsidemargin}{-30pt}
%\setlength{\headheight}{44pt}

\pagestyle{fancy}
\fancyhf{} % clear all fields
\fancyhead[R]{%
  \scshape
  \begin{tabular}[t]{@{}r@{}}
  MATH 347, Spring 2015\\Section 1 (1872)\\
  HW \#6, Due: 2015, March 26
  \end{tabular}}
\fancyhead[L]{%
  \scshape
  \begin{tabular}[t]{@{}r@{}}
  JOSEPH OKONOBOH\\Mathematics\\Cal State Long Beach
  \end{tabular}}
\fancyfoot[C]{\thepage}

\newcommand{\qed}{\hfill \ensuremath{\Box}}


\newcommand{\F}{\mathbb{F}}
\newcommand{\R}{\mathbb{R}}
\newcommand{\cyc}[1]{\langle #1 \rangle}
%\setcounter{section}{-1}

\begin{document}
\begin{enumerate}
%%%%%%%%%%%%%%%%%%%%%%%%%%%%%%%%%%Prob7.1%%%%%%%%%%%%%%%%%%%%%%%%%%%%%%%%%%%%%%%
   \item[7.1]  Make $\mathcal{P}_2(\R)$ into an inner-product space by defining
               $$\cyc{p, q} = \int_0^1p(x)q(x)\;dx.$$
               Define $T \in \mathcal{L}(\mathcal{P}_2(\R))$ by
               $T(a_0 + a_1x + a_2x^2) = a_1x$.
               \begin{enumerate}
                  \item Show that $T$ is not self-adjoint.
                  \item The matrix of $T$ with respect to the basis
                        $(1, x, x^2)$ is
                        $$\left[\begin{tabular}{@{}c c c@{}}
                           0 & 0 & 0 \\
                           0 & 1 & 0 \\
                           0 & 0 & 0
                        \end{tabular}\right].$$
                        The matrix equals its conjugate transpose, even though
                        $T$ is not self-adjoint. Explain why this is not a
                        contradiction.
               \end{enumerate}
               
      \textbf{Solution.}
      
      \begin{enumerate}
         \item Suppose to the contrary that $T$ is self-adjoint. Consider
               $x, 1 \in \mathcal{P}_2(\R)$. Then we must have that
               $\cyc{T(1), x} = \cyc{T^*(1), x} = \cyc{1, T(x)}$. But
               \begin{align*}
                  \cyc{T^*(1), x} &= \cyc{T(1), x} \\
                                 &= \cyc{0, x} \\
                                 &= 0,
               \end{align*}
               and
               \begin{align*}
                  \cyc{1, T(x)} &= \cyc{1, x} \\
                                &= \int_0^1 x\;dx \\
                                &= \frac{1}{2},
               \end{align*}
               so that $\cyc{T(1), x} \neq  \cyc{T(x), 1}$; i.e., $T$ is not
               self-adjoint.
      \end{enumerate}
%%%%%%%%%%%%%%%%%%%%%%%%%%%%%%%%%%Prob7.4%%%%%%%%%%%%%%%%%%%%%%%%%%%%%%%%%%%%%%%
   \item[7.4]  Suppose $P \in \mathcal{L}(V)$ is such that $P^2 = P$. Prove that
               $P$ is an orthogonal projection if and only if $P$ is
               self-adjoint.
%%%%%%%%%%%%%%%%%%%%%%%%%%%%%%%%%%Prob7.6%%%%%%%%%%%%%%%%%%%%%%%%%%%%%%%%%%%%%%%
   \item[7.6]  Prove that if $T \in \mathcal{L}(V)$ is normal, then
               $$\text{range } T = \text{range } T^*.$$
%%%%%%%%%%%%%%%%%%%%%%%%%%%%%%%%%%Prob7.7%%%%%%%%%%%%%%%%%%%%%%%%%%%%%%%%%%%%%%%
   \item[7.7]  Prove that if $T \in \mathcal{L}(V)$ is normal, then
               $$\text{null } T^k = \text{null } T \text{ and }
                 \text{range }T^k = \text{range }T$$
               for every positive integer $k$.
%%%%%%%%%%%%%%%%%%%%%%%%%%%%%%%%%%Prob7.9%%%%%%%%%%%%%%%%%%%%%%%%%%%%%%%%%%%%%%%
   \item[7.9]  Prove that a normal operator on a complex inner-product space is
               self-adjoint if and only if all its eigenvalues are real.
\end{enumerate}
\end{document}
