\documentclass[9pt]{article}

\usepackage{amssymb}
\usepackage{amsmath}
\usepackage{amsfonts}
\usepackage{cancel}
\usepackage{comment}
\usepackage{fancyhdr}
\usepackage{mathrsfs}

\voffset = -50pt
%\textheight = 700pt
\addtolength{\textwidth}{60pt}
\addtolength{\evensidemargin}{-30pt}
\addtolength{\oddsidemargin}{-30pt}
%\setlength{\headheight}{44pt}

\pagestyle{fancy}
\fancyhf{} % clear all fields
\fancyhead[R]{%
  \scshape
  \begin{tabular}[t]{@{}r@{}}
  MATH 347, Spring 2015\\Section 1 (1872)\\
  HW \#7, Due: 2015, April 21
  \end{tabular}}
\fancyhead[L]{%
  \scshape
  \begin{tabular}[t]{@{}r@{}}
  JOSEPH OKONOBOH\\Mathematics\\Cal State Long Beach
  \end{tabular}}
\fancyfoot[C]{\thepage}

\newcommand{\qed}{\hfill \ensuremath{\Box}}


\newcommand{\F}{\mathbb{F}}
\newcommand{\R}{\mathbb{R}}
\newcommand{\cyc}[1]{\langle #1 \rangle}
%\setcounter{section}{-1}

\begin{document}
\begin{enumerate}
%%%%%%%%%%%%%%%%%%%%%%%%%%%%%%%%%%Prob7.14%%%%%%%%%%%%%%%%%%%%%%%%%%%%%%%%%%%%%%
   \item[7.14] Suppose $T \in \mathcal{L}(V)$ is self-adjoint, $\lambda \in \F$,
               and $\epsilon > 0$. Prove that if there exists $v \in V$ such
               that $||v|| = 1$ and
               $$||Tv - \lambda v|| < \epsilon,$$
               then $T$ has an eigenvalue $\lambda'$ such that
               $|\lambda - \lambda'| < \epsilon$.
               
      \textbf{Proof.}
%%%%%%%%%%%%%%%%%%%%%%%%%%%%%%%%%%Prob7.16%%%%%%%%%%%%%%%%%%%%%%%%%%%%%%%%%%%%%%
   \item[7.16] Give an example of an operator $T$ on an inner product space such
               that $T$ has an invariant subspace whose orthogonal complement is
               not invariant under $T$.

      \textbf{Answer.}
%%%%%%%%%%%%%%%%%%%%%%%%%%%%%%%%%%Prob7.17%%%%%%%%%%%%%%%%%%%%%%%%%%%%%%%%%%%%%%
   \item[7.17] Prove that the sum of any two positive operators on $V$ is
               positive.

      \textbf{Proof.} Suppose that $S$ and $T$ are positive operators on $V$.
      Since $S$ and $T$ are both self-adjoint, it follows immediately that
      $S + T$ is self-adjoint because
      $$(S+T)^* = S^*+T^* = S + T.$$
      Now let $v \in V$. Thus
      \begin{align*}
         \cyc{(S+T)v,v} &= \cyc{Sv + Tv, v} \\
            &= \cyc{Sv, v} + \cyc{Tv, v} \\
            &\ge 0, &[\text{Since } Sv \ge 0, Tv \ge 0]
      \end{align*}
      so that $S + T$ is a positive operator. \qed
%%%%%%%%%%%%%%%%%%%%%%%%%%%%%%%%%%Prob7.19%%%%%%%%%%%%%%%%%%%%%%%%%%%%%%%%%%%%%%
   \item[7.19] Suppose that $T$ is a positive operator on $V$. Prove that $T$ is
               invertible if and only if
               $$\cyc{Tv, v} > 0$$
               for every $v \in V\;\backslash\;\{0\}$.

      \textbf{Proof.} Suppose first that $T$ is invertible. Then it follows that
      $T$ is injective, so that $\text{null }T = \{0\}$. Let $v \in V$. Now 
      suppose that $\cyc{Tv, v} = 0$. By the Spectral Theorem, there exists an 
      orthonormal basis $(e_1, \ldots, e_n)$  of $V$ consisting of eigenvectors
      of $T$. Let $\lambda_1, \ldots, \lambda_n$ denote the corresponding real
      ($T$ is self-adjoint) and nonnegative (Theorem 7.27 (b)) eigenvalues. 
      Since the eigenvectors are independent, they must be nonzero. Thus
      $e_i \notin \text{null }T$, so that $0 \neq T(e_i) = \lambda_ie_i$. That
      is, all the eigenvalues are positive. Now we have
      $v = a_1e_1 + \cdots + a_ne_n$, so that
      \begin{align*}
         0 &= \cyc{Tv, v} \\
           &= \cyc{T(a_1e_1 + \cdots + a_ne_n), a_1e_1 + \cdots + a_ne_n} \\
           &= \cyc{T(a_1e_1) + \cdots + T(a_ne_n), a_1e_1 + \cdots + a_ne_n} \\
           &= \cyc{a_1\lambda_1e_1 + \cdots + a_n\lambda_ne_n,
                   a_1e_1 + \cdots + a_ne_n} \\
           &= a_1\overline{a_1}\lambda_1\cyc{e_1, e_1} + \cdots +
              a_n\overline{a_n}\lambda_n\cyc{e_n, e_n} \\
           &= |a_1|^2\lambda_1 + \cdots + |a_n|^2\lambda_n.
      \end{align*}

      Since the eigenvalues are all positive, it must be the case
      $a_1 = \cdots = a_n = 0$, so that $v = 0$. So it follows that if $v \in V$
      is nonzero, we must have that $\cyc{Tv, v} > 0$. \\

      Conversely suppose that $\cyc{Tv, v} > 0$ for all nonzero $v \in V$. Let
      $x \in \text{null }T$. Then it follows that $x$ is not nonzero because
      $$0 = \cyc{0, x} = \cyc{Tx, x}.$$
      Thus $x = 0$; that is $\text{null }T = \{0\}$. Thus $T$ is injective
      (and surjective) and thus invertible. \qed      
%%%%%%%%%%%%%%%%%%%%%%%%%%%%%%%%%%Prob7.22%%%%%%%%%%%%%%%%%%%%%%%%%%%%%%%%%%%%%%
   \item[7.22] Prove that if $S \in \mathcal{L}(\R^3)$ is an isometry, then
               there exists a nonzero vector $x \in \R^3$ such that $S^2x = x$.

      \textbf{Proof.} Assume that $S \in \mathcal{L}(\R^3)$ is an isometry.
      Since dim $\R^3$ is odd, it follows by Theorem 7.38 that $S$ has an
      eigenvalue of 1 or $-1$. Suppose first that $S$ has an eigenvalue of 1.
      Then there exists a nonzero vector $x$ such that $Sx = x$. Thus we have
      that
      $$S^2x = S(Sx) = Sx = x.$$
      Now suppose that $S$ has an eigenvalue of $-1$. Then there exists a
      nonzero vector $c$ such that $Sc = -c$. Let $y = -c$. Then we have that
      $$S^2y = S(Sy) = S(S(-c)) = S(-S(c)) = S(c) = -c = y,$$
      as desired. \qed
\end{enumerate}
\end{document}
