\documentclass[9pt]{article}

\usepackage{amssymb}
\usepackage{amsmath}
\usepackage{amsfonts}
\usepackage{cancel}
\usepackage{comment}
\usepackage{fancyhdr}
\usepackage{mathrsfs}

\voffset = -50pt
%\textheight = 700pt
\addtolength{\textwidth}{60pt}
\addtolength{\evensidemargin}{-30pt}
\addtolength{\oddsidemargin}{-30pt}
%\setlength{\headheight}{44pt}

\pagestyle{fancy}
\fancyhf{} % clear all fields
\fancyhead[R]{%
  \scshape
  \begin{tabular}[t]{@{}r@{}}
  MATH 347, Spring 2015\\Section 1 (1872)\\
  HW \#5, Due: 2015, March 17
  \end{tabular}}
\fancyhead[L]{%
  \scshape
  \begin{tabular}[t]{@{}r@{}}
  JOSEPH OKONOBOH\\Mathematics\\Cal State Long Beach
  \end{tabular}}
\fancyfoot[C]{\thepage}

\newcommand{\qed}{\hfill \ensuremath{\Box}}


\newcommand{\F}{\mathbb{F}}
\newcommand{\R}{\mathbb{R}}
\newcommand{\cyc}[1]{\langle #1 \rangle}
%\setcounter{section}{-1}

\begin{document}
\begin{enumerate}
%%%%%%%%%%%%%%%%%%%%%%%%%%%%%%%%%%Prob6.6%%%%%%%%%%%%%%%%%%%%%%%%%%%%%%%%%%%%%%%
   \item[6.6]  Prove that if $V$ is a real inner-product space, then
               $$\cyc{u, v} = \frac{||u + v||^2 - ||u - v||^2}{4}$$
               for all $u. v \in V$.

      \textbf{Proof.} Let $V$ be a real inner-product space and let
      $u, v \in V$. We have that
      \begin{align*}
         \frac{||u + v||^2 - ||u - v||^2}{4}
            &= \frac{\cyc{u + v, u + v} - \cyc{u - v, u - v}}{4} \\
            &= \frac{\cyc{u, u + v} + \cyc{v, u + v} -
                     \cyc{u, u - v} - \cyc{-v, u - v}}{4} \\
            &= \frac{\cyc{u + v, u} + \cyc{u + v, v} -
                     \cyc{u - v, u} + \cyc{v, u - v}}{4} \\
            &= \frac{\cyc{u, u} + \cyc{v, u} + \cyc{u, v} + \cyc{v, v} -
                     \cyc{u, u} + \cyc{v, u} + \cyc{u - v, v}}{4} \\
            &= \frac{\cyc{v, u} + \cyc{u, v} + \cyc{v, v} + 
                     \cyc{v, u} + \cyc{u - v, v}}{4} \\
            &= \frac{\cyc{v, u} + \cyc{u, v} + \cyc{v, v} + 
                     \cyc{v, u} + \cyc{u, v} - \cyc{v, v}}{4} \\
            &= \frac{\cyc{v, u} + \cyc{u, v} + \cyc{v, u} + \cyc{u, v}}{4} \\
            &= \frac{\cyc{u, v} + \cyc{u, v} + \cyc{u, v} + \cyc{u, v}}{4} \\
            &= \frac{4\cyc{u, v}}{4} = \cyc{u, v}.
      \end{align*} \qed
%%%%%%%%%%%%%%%%%%%%%%%%%%%%%%%%%%Prob6.10%%%%%%%%%%%%%%%%%%%%%%%%%%%%%%%%%%%%%%
   \item[6.10] On $\mathcal{P}_2(\R)$, consider the inner product given by
               $$\cyc{p, q} = \int_0^1p(x)q(x) dx.$$

               Apply the Gram-Schmidt procedure to the basis $(1, x, x^2)$ to
               produce an orthonormal basis of $\mathcal{P}_2(\R)$.

      \textbf{Solution.} We want to construct an orthonormal basis
      $(e_1, e_2, e_3)$ for $\mathcal{P}_2(\R)$; so applying the Gram-Schmidt
      procedure to the basis $(1, x, x^2)$, we have
      \begin{align*}
         e_1 &= \frac{1}{||1||} \\
         e_2 &= \frac{x - \cyc{x, e_1}e_1}{||x - \cyc{x, e_1}e_1||} \\
         e_3 &= \frac{x^2 - \cyc{x^2, e_1}e_1 - \cyc{x^2, e_2}e_2}
                     {||x^2 - \cyc{x^2, e_1}e_1 - \cyc{x^2, e_2}e_2||}. \\
      \end{align*}
      So $$||1|| = \sqrt{\cyc{1, 1}} = \sqrt{\int_0^11\;dx} = \sqrt{1} = 1,$$
      so that $e_1 = 1$. Now
      $$\cyc{x, e_1} = \int_0^1x\;dx = \frac{1}{2},$$
      and
      $$
         \left|\left|x - \frac{1}{2}\right|\right| = \sqrt{\int_0^1\left(x - \frac{1}{2}\right)^2\;dx}
                 = \frac{\sqrt{3}}{6}.$$
      Thus $\displaystyle e_2 = (x - \frac{1}{2}) \cdot \frac{6}{\sqrt{3}} = 2x\sqrt{3} - \sqrt{3}.$

      Similarly we find that
      $$\cyc{x^2, e_1}e_1 = \int_0^1 x^2\;dx = \frac{1}{3},$$
      and
      $$\cyc{x^2, e_2}e_2 = \left(\int_0^1(2x^3\sqrt{3} - x^2\sqrt{3}\right)\;dx)(2x\sqrt{3} - \sqrt{3})$$
%%%%%%%%%%%%%%%%%%%%%%%%%%%%%%%%%%Prob6.13%%%%%%%%%%%%%%%%%%%%%%%%%%%%%%%%%%%%%%
   \item[6.13] Suppose $(e_1, \ldots, e_m)$ is an orthonormal list of vectors in
               $V$. Let $v \in V$. Prove that
               $$||v||^2 = |\cyc{v, e_1}|^2 + \cdots + |\cyc{v, e_m}|^2$$
               if and only if $v \in \text{span}(e_1, \ldots, e_m)$.
               
      \textbf{Proof.} Suppose $(e_1, \ldots, e_m)$ is an orthonormal list of
      vectors in $V$ and let $v \in V$.
      
      $(\Leftarrow)$ Assume that $v \in \text{span}(e_1, \ldots, e_m)$.
      Therefore $v = a_1e_1 + \cdots + a_me_m$ for some scalars
      $a_1, \ldots, a_m$. By the orthonomality of $(e_1, \ldots, e_m)$, it
      follows that $\cyc{v, e_j} = a_j$ for all $j \in \{1, 2,  \ldots, m\}$, so
      we have that
      \begin{align*}
         ||v||^2 &= ||a_1e_1 + \cdots + a_me_m||^2 \\
                 &= ||\cyc{v, e_1}e_1 + \cdots + \cyc{v, e_m}e_m||^2 \\
                 &= |\cyc{v, e_1}|^2 + \cdots + |\cyc{v, e_m}|^2
                    &[\text{Proposition }6.15]
      \end{align*}
      
      $(\Rightarrow)$ Now assume that
      $||v||^2 = |\cyc{v, e_1}|^2 + \cdots + |\cyc{v, e_m}|^2$. Extend the
      orthonormal list $(e_1, \ldots, e_m)$ to an orthonormal basis
      $(e_1, \ldots, e_m, e_{m+1}, \ldots, e_{m+n})$ for $V$. Thus there exist
      scalars $b_1, \ldots, b_{m+n}$ such that
      $v = b_1e_1 + \cdots + b_{m+n}e_{m+n}$. Thus
      \begin{align*}
         |b_1|^2 + \cdots + |b_m|^2 &=
         |\cyc{v, e_1}|^2 + \cdots + |\cyc{v, e_m}|^2 \\
            &= ||v||^2 \\
            &= ||b_1e_1 + \cdots + b_{m+n}e_{m+n}|| \\
            &= |b_1|^2 + \cdots + |b_{m+n}|^2 \\
            &= |b_1|^2 + \cdots + |b_m|^2 +
               |b_{m+1}|^2 + \cdots + |b_{m+n}|^2,
      \end{align*}
      
      so that $|b_{m+1}|^2 + \cdots + |b_{m+n}|^2 = 0$. Since
      $|b_{m+i}|$ is nonnegative for all $i \in \{1, \ldots, n\}$, it
      follows that $b_{m+1} = \cdots =  b_{m+n} = 0$, so
      that $v = b_1e_1 + \cdots + b_m$. That is
      $$v \in \text{span}(e_1, \ldots, e_m).$$ \qed
%%%%%%%%%%%%%%%%%%%%%%%%%%%%%%%%%%Prob6.17%%%%%%%%%%%%%%%%%%%%%%%%%%%%%%%%%%%%%%
   \item[6.17] Prove that if $P \in \mathcal{L}(V)$ is such that $P^2 = P$ and
               every vector in null $P$ is orthogonal to every vector in range
               $P$, then $P$ is an orthogonal projection.
               
      \textbf{Proof.}
%%%%%%%%%%%%%%%%%%%%%%%%%%%%%%%%%%Prob6.29%%%%%%%%%%%%%%%%%%%%%%%%%%%%%%%%%%%%%%
   \item[6.29] Suppose $T \in \mathcal{L(V)}$ and $U$ is a subspace of $V$.
               Prove that $U$ is invariant under $T$ if and only if $U^\perp$
               is invariant under $T^*$.
               
      \textbf{Proof.} Suppose $T \in \mathcal{L(V)}$ and $U$ is a subspace of
      $V$. 
      
      ($\Rightarrow$) Assume that $U$ is invariant under $T$. Let
      $v \in U^\perp$. In order to show that $U^\perp$ is invariant under
      $T^*$, it suffices to show that $T^*v \in U^\perp$. That is, we must show
      that $\cyc{u, T^*v} = 0$ for all $u \in U$. So let $u \in U$. Thus
      \begin{align*}
         \cyc{u, T^*v} &= \cyc{Tu, v} &[\text{Definition}] \\
          &= 0 &[Tu \in U \text{ and } v \in U^\perp],
      \end{align*}
      which is what we wanted to show.
      
      ($\Leftarrow$) Now assume that $U^\perp$ is invariant under $T^*$. Let
      $u \in U$. We want to show that $Tu \in U$. Consider any $v \in U^\perp$.
      We have that
      \begin{align*}
         \cyc{Tu, v} = \cyc{u, T^*v} &[\text{Definition}] \\
          &= 0 &[u \in U \text{ and } T^*v \in U^\perp],
      \end{align*}
      so that $Tu$ is orthogonal to every vector in $U^\perp$. That is,
      $Tu \in (U^\perp)^\perp$; but $(U^\perp)^\perp = U$. Thus $Tu \in U$, so
      that $U$ is invariant under $T$. \qed
      
\end{enumerate}
\end{document}
