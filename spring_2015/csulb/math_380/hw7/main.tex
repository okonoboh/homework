\documentclass[9pt]{article}

\usepackage{amssymb}
\usepackage{amsmath}
\usepackage{amsfonts}
\usepackage{fancyhdr}
\usepackage{mathrsfs}

\voffset = -50pt
%\textheight = 700pt
\addtolength{\textwidth}{60pt}
\addtolength{\evensidemargin}{-30pt}
\addtolength{\oddsidemargin}{-30pt}
%\setlength{\headheight}{44pt}

\pagestyle{fancy}
\fancyhf{} % clear all fields
\fancyhead[R]{%
  \scshape
  \begin{tabular}[t]{@{}r@{}}
  MATH 380, Spring 2015\\Section 4 (10519)\\
  HW \#7, Due: 2015, May 6
  \end{tabular}}
\fancyhead[L]{%
  \scshape
  \begin{tabular}[t]{@{}r@{}}
  JOSEPH OKONOBOH\\Mathematics\\Cal State Long Beach
  \end{tabular}}
\fancyfoot[C]{\thepage}

\newcommand{\qed}{\hfill \ensuremath{\Box}}


\newcommand{\Z}{\mathbb{Z}}
\newcommand{\I}{\mathbb{I}}
\newcommand{\D}{\displaystyle}
%\setcounter{section}{-1}


\begin{document}
\begin{enumerate}
%%%%%%%%%%%%%%%%%%%%%%%%%%%%%%%%%%%%%%%01%%%%%%%%%%%%%%%%%%%%%%%%%%%%%%%%%%%%%%%
   \item A box contains 4 red balls, 3 white balls, 1 blue ball, and 2 green
         balls. Seven balls are selected with replacement. Find
         \begin{enumerate}
            \item the probability of selecting 3 red balls, 2 white balls, 1
                  blue ball, and 1 green ball.
            \item the expectation and variance of the number of red balls
                  selected.
            \item the covariance of the number of red balls selected and the
                  number of white balls selected
         \end{enumerate}
         
      \textbf{Solution.}
      
      \begin{enumerate}
         \item $p(3r, 2w, 1b, 1g) = \D\frac{7!}{3!2!1!1!}(0.4)^3(0.3)^2(0.1)^1
               (0.2)^1 = 0.048384$.
         \item $E$[red] = $7 \cdot 0.4 = 2.8$, $V$(red) =
               $7 \cdot 0.4 \cdot 0.6 = 1.68$.
         \item Cov(red, white) = $-7 \cdot 0.4 \cdot 0.3 = -0.84$.
      \end{enumerate}
%%%%%%%%%%%%%%%%%%%%%%%%%%%%%%%%%%%%%%%02%%%%%%%%%%%%%%%%%%%%%%%%%%%%%%%%%%%%%%%
   \item Suppose in a certain (large) community, 40\% of the population is under
         30 years of age, 30\% of the population is between 30 and 50 years of
         age, 20\% of the population is between 50 and 70 years of age, and 10\%
         of the population is over 70. If eight people are randomly selected
         from the community, find the probability that
         \begin{enumerate}
            \item exactly four are under 30, exactly two are between 30 and 50,
                  exactly one is between 50 and 70, and exactly one is over 70.
            \item exactly three are under 30 and exactly two are between 50 and
                  70.
            \item exactly five are under 50.
         \end{enumerate}
         
      \textbf{Solution.}
      
      \begin{enumerate}
         \item $P = \D\frac{8!}{4!2!1!1!}(0.4)^4(0.3)^2(0.2)^1
               (0.1)^1 = 0.0387072$.
         \item $P = \D\frac{8!}{3!2!3!}(0.4)^3(0.2)^2(0.4)^3 = 0.0917504$.
         \item $P = \D\frac{8!}{5!3!}(0.7)^5(0.3)^3 = 0.25412$.
      \end{enumerate}
%%%%%%%%%%%%%%%%%%%%%%%%%%%%%%%%%%%%%%%03%%%%%%%%%%%%%%%%%%%%%%%%%%%%%%%%%%%%%%%
   \item Suppose $Y$ has a normal distribution with parameters $\mu = M$ and
         $\sigma^2 = 1$ where $M$ has a gamma distribution with parameters
         $\alpha = 3$ and $\beta = 2$. Find (a) $E[Y]$ and (b) $V(Y)$.

      \textbf{Solution.}
      
      \begin{enumerate}
         \item $E[Y] = E[E(Y | M)] = E[M] = 3 \cdot 2 = 6$.
         \item $V(Y) = E[V(Y|M)] + V[E(Y|M)] = E[\sigma^2] + V[M] =
                \sigma^2E[1] + \alpha\beta = 13$.
      \end{enumerate}
%%%%%%%%%%%%%%%%%%%%%%%%%%%%%%%%%%%%%%%04%%%%%%%%%%%%%%%%%%%%%%%%%%%%%%%%%%%%%%%
   \item Suppose $Y$ is an exponential random variable with parameter $\Lambda$
         and $\Lambda$ itself is a geometric random variable with parameter
         $p = 0.4$. Find $E[Y]$ and $V(Y)$.

      \textbf{Solution.}
      
      \begin{enumerate}
         \item $E[Y] = E[E(Y | \Lambda)] = E[\Lambda] = 1 / 0.4 = 2.5$.
         \item \begin{align*}
                  V(Y) &= E[V(Y|\Lambda)] + V(E[Y|\Lambda]) \\
                       &= E[\Lambda^2] + V(\Lambda)    \\
                       &= 2V(\Lambda) + E[\Lambda]^2 \\
                       &= 2\frac{1 - 0.4}{0.4^2} + 2.5^2 = 13.75.
               \end{align*}
      \end{enumerate}
%%%%%%%%%%%%%%%%%%%%%%%%%%%%%%%%%%%%%%%05%%%%%%%%%%%%%%%%%%%%%%%%%%%%%%%%%%%%%%%
   \item Let $Y$ be exponentially distributed with mean 3.
         \begin{enumerate}
            \item Use the method of distribution functions to find the density
                  function of $Y^2$.
            \item Find a transformation $G$ such that $G(Y)$ is uniformly
                  distributed on (1, 2).
         \end{enumerate}

      \textbf{Solution.}
      
      \begin{enumerate}
         \item Let $U = Y^2$. We have that $F_U(u) = 0$ if $u \le 0$. Now if
               $u > 0$, we have that 
               \begin{align*}
                  F_U(u) &= P(U \le u) = P(Y^2 \le u) \\
                         &= P(-\sqrt{u} \le Y \le \sqrt{u}) \\
                         &= \frac{1}{3}\int_{-\sqrt{u}}^{\sqrt{u}}e^{-y/3} dy \\
                         &= e^{\sqrt{u}/3} - e^{-\sqrt{u}/3},                     
               \end{align*}
               so that
               \begin{equation*}
                  f_U(u) = \frac{dF_U(u)}{du} = \begin{cases}
                     \D\frac{1}{6\sqrt{u}}\left(e^{\sqrt{u}/3}+
                     e^{-\sqrt{u}/3}\right) & \text{if } u > 0 \\
                     0 & \text{if } \text{otherwise}
                  \end{cases}
               \end{equation*}
         \item The distribution of $Y$ is given by
               \begin{equation*}
                  F_Y(y) = \begin{cases}
                     0 & \text{if } y < 0 \\
                     1 - e^{-y/3} & \text{if } y \ge 0.
                  \end{cases}
               \end{equation*}

               Let $U$ be a uniformly distributed random variable on (1, 2).
               Then the distribution of $U$ is given by
               \begin{equation*}
                  F_U(u) = \begin{cases}
                     0 & \text{if } u < 1, \\
                     u - 1 & \text{if } 1 \le u \le 2, \\
                     1 & \text{if } u > 2.
                  \end{cases}
               \end{equation*}

               Let $1 \le u < 2$. We want to find $H(u)$ such that
               $P(Y \le H(u)) = u - 1$. Thus we have that
               $$u - 1 = P(Y \le H(u)) = 1 - e^{-H(u)/3}$$
               so that $H(u) = -3\ln(2 - u)$. Now
               \begin{align*}
                  u - 1 &= P(Y \le H(u)) \\
                        &= P(Y \le -3\ln(2 - u)) \\
                        &= P\left(-\frac{1}{3}Y \ge \ln(2 - u)\right) \\
                        &= P\left(2 - e^{-1/3Y} \le u\right) \\
               \end{align*}
               if $1 \le u < 2$. So let $G(Y) = 2 - e^{-1/3Y}$.
      \end{enumerate}
%%%%%%%%%%%%%%%%%%%%%%%%%%%%%%%%%%%%%%%06%%%%%%%%%%%%%%%%%%%%%%%%%%%%%%%%%%%%%%%
   \item Suppose $Y$ has a probability density function
         \begin{equation*}
            f(y) = \begin{cases}
               3y^2   & \text{if } 0 \le y \le 1 \\
               y^3 & \text{if } \text{otherwise}
            \end{cases}
         \end{equation*}
         Use the method of transformation to find the probability density
         functions of 

         a) $U_1 = 3Y - 1$ and \qquad b) $U_2 = Y^3$.

      \textbf{Solution.}
      
      \begin{enumerate}
         \item We are interested in the function $h(y) = 3y - 1$, an increasing
               function. Let $u_1 = 3y - 1$ so that
               $$y = h^{-1}(u_1) = \frac{u_1+1}{3} \text{ and }
                 \frac{dh^{-1}}{du_1} = \frac{1}{3}.$$
               Thus
               \begin{align*}
                  f_{U_1}(u_1) &= f_Y[h^{-1}(u_1)]\frac{dh^{-1}}{du_1} \\
                     &= \begin{cases}
                     \D\frac{(u_1+1)^2}{9} & \text{if } -1 \le u_1 \le 2, \\
                     \D\frac{(u_1+1)^3}{81} & \text{otherwise }.                  
                  \end{cases}
               \end{align*}
         \item Let $u_2 = h(y) = y^3$. Thus
               $$y = h^{-1}(u_2) = u_2^{1/3} \text{ and }
                 \frac{dh^{-1}}{du_2} = \frac{1}{3}u_2^{-2/3}.$$
               Thus
               \begin{align*}
                  f_{U_2}(u_2) &= f_Y[h^{-1}(u_2)]\frac{dh^{-1}}{du_2} \\
                     &= \begin{cases}
                     1 & \text{if } 0 \le u_2 \le 1, \\
                     \D\frac{1}{3}u_2^{1/3} & \text{otherwise }.                  
                  \end{cases}
               \end{align*}
      \end{enumerate}
%%%%%%%%%%%%%%%%%%%%%%%%%%%%%%%%%%%%%%%07%%%%%%%%%%%%%%%%%%%%%%%%%%%%%%%%%%%%%%%
   \item Let $Y_1$ and $Y_2$ have joint probability density function         
         \begin{equation*}
            f(y_1, y_2) = \begin{cases}
               6(1 - y_2)   & \text{if } 0 < y_1 \le y_2 \le 1 \\
               0 & \text{if } \text{otherwise}
            \end{cases}
         \end{equation*}
         Use the method of transformation to find the probability density
         functions of $U = Y_1 / Y_2$.

      \textbf{Solution.} Let $Y_1$ be fixed for some $y_1 > 0$. Then
      $U = y_1 / Y_2$, so that $h(y_2) = y_1 / y_2$, a decreasing function. Let
      $g(y_1, u)$ denote the joint density of $Y_1$ and $U$, with
      $$y_2 = y_1 / u = h^{-1}(u).$$
      Thus
      \begin{align*}
         g(y_1, u) &= \begin{cases}
            f[y_1, h^{-1}(u)]\left|\D\frac{dh^{-1}}{du}\right| & \text{if }
               0 < y_1 \le y_1/u \le 1, \\
               0 & \text{otherwise }.                  
            \end{cases} \\ &= \begin{cases}
            6\left(1-\D\frac{y_1}{u}\right)\D\frac{y_1}{u^2}& \text{if }
               0 < y_1 \le u \le 1, \\
               0 & \text{otherwise }.                  
            \end{cases}
      \end{align*}

      It follows that
      \begin{align*}
         f_U(u) &= \int_{-\infty}^{\infty} g(y_1, u) dy_1 \\
                &= \int_0^u 6\left(1-\D\frac{y_1}{u}\right)\D\frac{y_1}{u^2} dy_1 \\
                &= \begin{cases}
            1 & \text{if } 0 < u \le 1, \\
               0 & \text{otherwise }.                  
            \end{cases}
      \end{align*}

         
         
%%%%%%%%%%%%%%%%%%%%%%%%%%%%%%%%%%%%%%%08%%%%%%%%%%%%%%%%%%%%%%%%%%%%%%%%%%%%%%%
   \item Suppose $Y_1$ and $Y_2$ are two independent exponentially distributed
         random variables each with mean $\beta$. Use the method of moment
         generating functions to obtain the probability density function of
         $Y_1 + Y_2$.

      \textbf{Solution.} Let $U = Y_1 + Y_2$. Let $m_U(t)$, $m_{Y_1}(t)$,
      and $m_{Y_2}(t)$ denote the moment generating functions of $U$, $Y_1$ and
      $Y_2$. Since $Y_1$ and $Y_2$ are independent, it follows from Theorem 6.2
      that $m_U(t) = m_{Y_1}(t) \times m_{Y_2}(t)$. We know from discussions in
      class that $m_{Y_1}(t) = m_{Y_2}(t) = (1 - \beta t)^{-1}$. Thus
      $m_U(t) = (1 - \beta t)^{-2}$, so that $U$ has gamma distribution and we
      have that
      \begin{align*}
         f_U(u) = \begin{cases}
            \D\frac{1}{\beta^2}ue^{-u/\beta} & \text{if } 0 < u, \\
               0 & \text{otherwise }.                  
            \end{cases}
      \end{align*}
\end{enumerate}
\end{document}
