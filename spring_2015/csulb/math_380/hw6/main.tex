\documentclass[9pt]{article}

\usepackage{amssymb}
\usepackage{amsmath}
\usepackage{amsfonts}
\usepackage{fancyhdr}
\usepackage{mathrsfs}

\voffset = -50pt
%\textheight = 700pt
\addtolength{\textwidth}{60pt}
\addtolength{\evensidemargin}{-30pt}
\addtolength{\oddsidemargin}{-30pt}
%\setlength{\headheight}{44pt}

\pagestyle{fancy}
\fancyhf{} % clear all fields
\fancyhead[R]{%
  \scshape
  \begin{tabular}[t]{@{}r@{}}
  MATH 380, Spring 2015\\Section 4 (10519)\\
  HW \#6, Due: 2015, April 08
  \end{tabular}}
\fancyhead[L]{%
  \scshape
  \begin{tabular}[t]{@{}r@{}}
  JOSEPH OKONOBOH\\Mathematics\\Cal State Long Beach
  \end{tabular}}
\fancyfoot[C]{\thepage}

\newcommand{\qed}{\hfill \ensuremath{\Box}}


\newcommand{\Z}{\mathbb{Z}}
\newcommand{\I}{\mathbb{I}}
\newcommand{\D}{\displaystyle}
%\setcounter{section}{-1}


\begin{document}
\begin{enumerate}
%%%%%%%%%%%%%%%%%%%%%%%%%%%%%%%%%%%%%%%01%%%%%%%%%%%%%%%%%%%%%%%%%%%%%%%%%%%%%%%
   \item If $Z$ is a standard normal variable, find

         $(a) \; P(Z^2 < 1) \qquad (b) \; P(Z^2 > 3.84146)$.

      \textbf{Solution.}

      \begin{enumerate}
         \item We have that
               $$P(Z^2 < 1) = P(-1 < Z < 1) = 1 - 2 \cdot P(Z > 1) \approx
                 0.6826,$$
               and
         \item $$P(Z^2 > 3.84146) = 2 \cdot P(Z > \sqrt{3.84146}) \approx
                 2 \cdot P(Z > 1.96) \approx 0.05.$$
      \end{enumerate}
%%%%%%%%%%%%%%%%%%%%%%%%%%%%%%%%%%%%%%%02%%%%%%%%%%%%%%%%%%%%%%%%%%%%%%%%%%%%%%%
   \item If $Y$ is a normal random variable with $\mu = 20$ and variance
         $\sigma^2 = 4$, i.e., $Y \sim N(20,4)$, find

         $(a) \; P(16 \le Y \le 22) \qquad (b) \; P(100 < 9Y - 80 < 145)$.

      \textbf{Solution.}

      \begin{enumerate}
         \item We have that
               \begin{align*}
                  P(16 \le Y \le 22) &= P\left(\frac{16 - 20}{2} \le
                     Z \le \frac{22 - 20}{2}\right) \\
                  &= P(-2 \le Z \le 1) \\
                  &= 1 - [P(Z < -2) + P(Z > 1)] \\
                  &= 1 - [P(Z > 2) + P(Z > 1)] \\
                  &\approx 0.8185,
               \end{align*}
               and
         \item \begin{align*}
                  P(100 < 9Y - 80 < 145) &= P(20 < Y < 25) \\
                  &= P\left(\frac{20 - 20}{2} <
                     Z < \frac{25 - 20}{2}\right) \\
                  &= P(0 < Z < 2.5) \\
                  &= P(Z > 0) - P(Z > 2.5) \\
                  &\approx 0.4938.
               \end{align*}
      \end{enumerate}
%%%%%%%%%%%%%%%%%%%%%%%%%%%%%%%%%%%%%%%03%%%%%%%%%%%%%%%%%%%%%%%%%%%%%%%%%%%%%%%
   \item The scores of a pre-employment test are normally distributed with mean
         $\mu = 70$ and standard deviation $\sigma = 5$. If only the top 1.5\%
         of the applicants (based on their score on the pre-employment test) are
         to be considered, find the cut-off score (i.e., the value such that
         only 1.5\% of the applicants score this value or higher).

      \textbf{Solution.} Let $y$ be the cut-off score. Then we have that
      $$0.0015 = P(Y \ge y) = P\left(Z \ge \frac{y - 70}{5}\right),$$
      so that $(y-70)/5 \approx 2.97$; i.e., $y \approx 84.85$.
%%%%%%%%%%%%%%%%%%%%%%%%%%%%%%%%%%%%%%%04%%%%%%%%%%%%%%%%%%%%%%%%%%%%%%%%%%%%%%%
   \item Using the fact that
         $\D\int_0^\infty e^{-y^2/2} dy = \sqrt{\frac{\pi}{2}},$ show that
         $\D\Gamma\left(\frac{1}{2}\right) = \sqrt{\pi}$ by making the 
         transformation $y = \frac{1}{2} x^2$.

      \textbf{Proof.} Using the substitution $y = \frac{1}{2} x^2$ we have that
      \begin{align*}
         \Gamma\left(\frac{1}{2}\right) &=
            \int_0^\infty y^{-\frac{1}{2}}e^{-y} dy \\
         &= \int_0^\infty \frac{\sqrt{2}}{x} e^{-\frac{1}{2}x^2} x \; dx \\
         &= \sqrt{2}\int_0^\infty  e^{-\frac{1}{2}x^2} \; dx \\
         &= \sqrt{2} \sqrt{\frac{\pi}{2}} = \sqrt{\pi}.
      \end{align*}
%%%%%%%%%%%%%%%%%%%%%%%%%%%%%%%%%%%%%%%05%%%%%%%%%%%%%%%%%%%%%%%%%%%%%%%%%%%%%%%
   \item If $Y$ has an exponential distribution with $P(Y < 3) = 0.4512$, find

         $(a)\; E[Y] \qquad (b)\; P(Y \ge 2)$.

      \textbf{Solution.}

      \begin{enumerate}
         \item We have that
               \begin{align*}
                  0.4512 &= P(Y < 3) \\
                         &= P(Y \le 3) \\
                         &= F(3) \\
                         &= \int_{-\infty}^3\frac{1}{\beta}
                            e^{-\frac{y}{\beta}} dy \\
                         &= \int_0^3\frac{1}{\beta}
                            e^{-\frac{y}{\beta}} dy \\
                         &= -e^{-\frac{3}{\beta}} + 1,
               \end{align*}
               so that $e^{-\frac{3}{\beta}} = 0.5488$; i.e., $\beta \approx 5$.
               Thus $E[Y] \approx 5$.
         \item \begin{align*}
                  P(Y \ge 2) &= 1 - P(Y < 2) \\
                     &= 1 - \int_0^2\frac{1}{\beta}e^{-\frac{y}{\beta}} dy \\
                     &= e^{-\frac{2}{\beta}} \\
                     &\approx 0.6703.
               \end{align*}
      \end{enumerate}
%%%%%%%%%%%%%%%%%%%%%%%%%%%%%%%%%%%%%%%06%%%%%%%%%%%%%%%%%%%%%%%%%%%%%%%%%%%%%%%
   \item The length of time $Y$ necessary to complete a key operation in the
         construction of houses has an exponential distribution with mean 10
         hrs. The formula $C = 100 + 40Y + 3Y^2$ gives the cost $C$ of
         completing the operation. Find the mean and variance of $C$.
      
      \textbf{Solution.} First we want to find $E[Y^2]$. So
      \begin{align*}
         E[Y^2] &= \frac{1}{10}\lim_{t\rightarrow\infty}\int_0^t y^2
                   e^{-\frac{y}{10}}\;dy \\
            &= \lim_{t\rightarrow\infty}\left[-y^2e^{-\frac{y}{10}}\Big|_0^t +
               2\int_0^t y e^{-\frac{y}{10}}\;dy\right]
               &[\text{Integration by parts}] \\
            &= 2\lim_{t\rightarrow\infty}\left[\int_0^t y e^{-\frac{y}{10}}\;dy
               \right] \\
            &= 2\lim_{t\rightarrow\infty}\left[-10ye^{-\frac{y}{10}}\Big|_0^t +
               10\int_0^t e^{-\frac{y}{10}}\;dy
               \right] &[\text{Integration by parts}] \\
            &= 200\lim_{t\rightarrow\infty}\left[\frac{1}{10}\int_0^t
               e^{-\frac{y}{10}}\;dy \right] \\
            &= 200 \cdot E[Y] = 2000.
      \end{align*}         

      Now the mean of $C$ is given by $E[C]$ so that
      \begin{align*}
         E[C] &= E[100 + 40Y + 3Y^2] \\
            &= E[100] + 40E[Y] + 3E[Y^2] \\
            &= 100 + 40 \cdot 10 + 3 \cdot 2000 \\
            &= 6500,
      \end{align*}
      and the variance of $C$, $V[Y]$, is
      $E[Y^2] - E[Y]^2 = 2000 - 100 = 1900$.
%%%%%%%%%%%%%%%%%%%%%%%%%%%%%%%%%%%%%%%07%%%%%%%%%%%%%%%%%%%%%%%%%%%%%%%%%%%%%%%
   \item Suppose $Y$ has density function $f(y) = ky^9e^{-y/2}$, $y \ge 0$.
         Find
         \begin{enumerate}
            \item $k$.
            \item $E[Y]$ and $V(Y)$.
            \item $P(Y > 34.1696)$.
            \item A value $b$ such that $P(Y < b) = 0.10$.
         \end{enumerate}
\end{enumerate}
\end{document}
