\documentclass[9pt]{article}

\usepackage{amssymb}
\usepackage{amsmath}
\usepackage{amsfonts}
\usepackage{fancyhdr}
\usepackage{mathrsfs}

\voffset = -50pt
%\textheight = 700pt
\addtolength{\textwidth}{60pt}
\addtolength{\evensidemargin}{-30pt}
\addtolength{\oddsidemargin}{-30pt}
%\setlength{\headheight}{44pt}

\pagestyle{fancy}
\fancyhf{} % clear all fields
\fancyhead[R]{%
  \scshape
  \begin{tabular}[t]{@{}r@{}}
  MATH 380, Spring 2015\\Section 4 (10519)\\
  HW \#6, Due: 2015, April 11
  \end{tabular}}
\fancyhead[L]{%
  \scshape
  \begin{tabular}[t]{@{}r@{}}
  JOSEPH OKONOBOH\\Mathematics\\Cal State Long Beach
  \end{tabular}}
\fancyfoot[C]{\thepage}

\newcommand{\qed}{\hfill \ensuremath{\Box}}


\newcommand{\Z}{\mathbb{Z}}
\newcommand{\I}{\mathbb{I}}
\newcommand{\D}{\displaystyle}
%\setcounter{section}{-1}


\begin{document}
\begin{enumerate}
%%%%%%%%%%%%%%%%%%%%%%%%%%%%%%%%%%%%%%%01%%%%%%%%%%%%%%%%%%%%%%%%%%%%%%%%%%%%%%%
   \item If $Z$ is a standard normal variable, find

         $(a) \; P(Z^2 < 1) \qquad (b) \; P(Z^2 > 3.84146)$.

      \textbf{Solution.}

      \begin{enumerate}
         \item We have that
               $$P(Z^2 < 1) = P(-1 < Z < 1) = 1 - 2 \cdot P(Z > 1) \approx
                 0.6826,$$
               and
         \item $$P(Z^2 > 3.84146) = 2 \cdot P(Z > \sqrt{3.84146}) \approx
                 2 \cdot P(Z > 1.96) \approx 0.05.$$
      \end{enumerate}
%%%%%%%%%%%%%%%%%%%%%%%%%%%%%%%%%%%%%%%02%%%%%%%%%%%%%%%%%%%%%%%%%%%%%%%%%%%%%%%
   \item If $Y$ is a normal random variable with $\mu = 20$ and variance
         $\sigma^2 = 4$, i.e., $Y \sim N(20,4)$, find

         $(a) \; P(16 \le Y \le 22) \qquad (b) \; P(100 < 9Y - 80 < 145)$.

      \textbf{Solution.}

      \begin{enumerate}
         \item We have that
               \begin{align*}
                  P(16 \le Y \le 22) &= P\left(\frac{16 - 20}{2} \le
                     Z \le \frac{22 - 20}{2}\right) \\
                  &= P(-2 \le Z \le 1) \\
                  &= 1 - [P(Z < -2) + P(Z > 1)] \\
                  &= 1 - [P(Z > 2) + P(Z > 1)] \\
                  &\approx 0.8185,
               \end{align*}
               and
         \item \begin{align*}
                  P(100 < 9Y - 80 < 145) &= P(20 < Y < 25) \\
                  &= P\left(\frac{20 - 20}{2} <
                     Z < \frac{25 - 20}{2}\right) \\
                  &= P(0 < Z < 2.5) \\
                  &= P(Z > 0) - P(Z > 2.5) \\
                  &\approx 0.4938.
               \end{align*}
      \end{enumerate}
%%%%%%%%%%%%%%%%%%%%%%%%%%%%%%%%%%%%%%%03%%%%%%%%%%%%%%%%%%%%%%%%%%%%%%%%%%%%%%%
   \item The scores of a pre-employment test are normally distributed with mean
         $\mu = 70$ and standard deviation $\sigma = 5$. If only the top 1.5\%
         of the applicants (based on their score on the pre-employment test) are
         to be considered, find the cut-off score (i.e., the value such that
         only 1.5\% of the applicants score this value or higher).

      \textbf{Solution.} Let $y$ be the cut-off score. Then we have that
      $$0.0015 = P(Y \ge y) = P\left(Z \ge \frac{y - 70}{5}\right),$$
      so that $(y-70)/5 \approx 2.97$; i.e., $y \approx 85$.
%%%%%%%%%%%%%%%%%%%%%%%%%%%%%%%%%%%%%%%04%%%%%%%%%%%%%%%%%%%%%%%%%%%%%%%%%%%%%%%
   \item Using the fact that
         $\D\int_0^\infty e^{-y^2/2} dy = \sqrt{\frac{\pi}{2}},$ show that
         $\D\Gamma\left(\frac{1}{2}\right) = \sqrt{\pi}$ by making the 
         transformation $y = \frac{1}{2} x^2$.

      \textbf{Proof.} Using the transformation $y = \frac{1}{2} x^2$ we have
      that
      \begin{align*}
         \Gamma\left(\frac{1}{2}\right) &=
            \int_0^\infty y^{-\frac{1}{2}}e^{-y} dy \\
         &= \int_0^\infty \frac{\sqrt{2}}{x} e^{-\frac{1}{2}x^2} x \; dx \\
         &= \sqrt{2}\int_0^\infty  e^{-\frac{1}{2}x^2} \; dx \\
         &= \sqrt{2} \sqrt{\frac{\pi}{2}} = \sqrt{\pi},
      \end{align*}
      as desired. \qed
%%%%%%%%%%%%%%%%%%%%%%%%%%%%%%%%%%%%%%%05%%%%%%%%%%%%%%%%%%%%%%%%%%%%%%%%%%%%%%%
   \item If $Y$ has an exponential distribution with $P(Y < 3) = 0.4512$, find

         $(a)\; E[Y] \qquad (b)\; P(Y \ge 2)$.

      \textbf{Solution.}

      \begin{enumerate}
         \item We have that
               \begin{align*}
                  0.4512 &= P(Y < 3) \\
                         &= P(Y \le 3) \\
                         &= F(3) \\
                         &= \int_{-\infty}^3\frac{1}{\beta}
                            e^{-\frac{y}{\beta}} dy \\
                         &= \int_0^3\frac{1}{\beta}
                            e^{-\frac{y}{\beta}} dy \\
                         &= -e^{-\frac{3}{\beta}} + 1,
               \end{align*}
               so that $e^{-\frac{3}{\beta}} = 0.5488$; i.e., $\beta \approx 5$.
               Thus $E[Y] \approx 5$.
         \item \begin{align*}
                  P(Y \ge 2) &= 1 - P(Y < 2) \\
                     &= 1 - \int_0^2\frac{1}{\beta}e^{-\frac{y}{\beta}} dy \\
                     &= e^{-\frac{2}{\beta}} \\
                     &\approx 0.6703.
               \end{align*}
      \end{enumerate}
%%%%%%%%%%%%%%%%%%%%%%%%%%%%%%%%%%%%%%%06%%%%%%%%%%%%%%%%%%%%%%%%%%%%%%%%%%%%%%%
   \item The length of time $Y$ necessary to complete a key operation in the
         construction of houses has an exponential distribution with mean 10
         hrs. The formula $C = 100 + 40Y + 3Y^2$ gives the cost $C$ of
         completing the operation. Find the mean and variance of $C$.
      
      \textbf{Solution.} First we want to find $E[Y^2]$. So
      \begin{align*}
         E[Y^2] &= \frac{1}{10}\lim_{t\rightarrow\infty}\int_0^t y^2
                   e^{-\frac{y}{10}}\;dy \\
            &= \lim_{t\rightarrow\infty}\left[-y^2e^{-\frac{y}{10}}\Big|_0^t +
               2\int_0^t y e^{-\frac{y}{10}}\;dy\right]
               &[\text{Integration by parts}] \\
            &= 2\lim_{t\rightarrow\infty}\left[\int_0^t y e^{-\frac{y}{10}}\;dy
               \right] \\
            &= 2\lim_{t\rightarrow\infty}\left[-10ye^{-\frac{y}{10}}\Big|_0^t +
               10\int_0^t e^{-\frac{y}{10}}\;dy
               \right] &[\text{Integration by parts}] \\
            &= 200\lim_{t\rightarrow\infty}\left[\frac{1}{10}\int_0^t
               e^{-\frac{y}{10}}\;dy \right] \\
            &= 200 \cdot E[Y] = 2000.
      \end{align*}         

      Now the mean of $C$ is given by $E[C]$ so that
      \begin{align*}
         E[C] &= E[100 + 40Y + 3Y^2] \\
            &= E[100] + 40E[Y] + 3E[Y^2] \\
            &= 100 + 40 \cdot 10 + 3 \cdot 2000 \\
            &= 6500,
      \end{align*}
      and the variance of $C$, $V[Y]$, is
      $E[Y^2] - E[Y]^2 = 2000 - 100 = 1900$.
%%%%%%%%%%%%%%%%%%%%%%%%%%%%%%%%%%%%%%%07%%%%%%%%%%%%%%%%%%%%%%%%%%%%%%%%%%%%%%%
   \item Suppose $Y$ has density function $f(y) = ky^9e^{-y/2}$, $y \ge 0$.
         Find
         \begin{enumerate}
            \item $k$.
            \item $E[Y]$ and $V(Y)$.
            \item $P(Y > 34.1696)$.
            \item A value $b$ such that $P(Y < b) = 0.10$.
         \end{enumerate}

      \textbf{Solution.} By inspection we can see that $f$ is the gamma 
      distribution with $\alpha = 10$, $\beta = 2$.
      \begin{enumerate}
         \item $\D k = \frac{1}{2^{10} \cdot \Gamma(10)} =
               \frac{1}{2^{10} \cdot 9!}$.
         \item $E[Y] = \alpha\beta = 20$ and $V(Y) = \alpha\beta^2 = 40$.
         \item \begin{align*}
                  P(Y > 34.1696) &= \frac{1}{2^{10} \cdot 9!}
                     \int_{34.1696}^\infty y^9e^{-y/2} dy \\
                     &= \frac{1}{2^{10} \cdot 9!}
                     \int_{17.0848}^\infty 2^{10}z^9e^{-z} dz
                     &\left[z = \frac{y}{2} \text{ substitution}\right]  \\
                     &= \frac{1}{9!}
                     \int_{17.0848}^\infty z^9e^{-z} dz \\
                     &= \sum_{x=0}^9\frac{17.0848^xe^{-17.0848}}{x!} \\
                     &\approx 0.025.
               \end{align*}
         \item Suppose there exists $b$ with $P(Y < b) = 0.10$, then we must
               have that
               $$0.90 = P(Y \ge b) = P(Z \ge b/2),$$
               and from Appendix 3, Table 3, we get $b/2 \approx 14$, so that
               $b \approx 28$.
      \end{enumerate}
%%%%%%%%%%%%%%%%%%%%%%%%%%%%%%%%%%%%%%%08%%%%%%%%%%%%%%%%%%%%%%%%%%%%%%%%%%%%%%%
   \item The function $B(\alpha, \beta)$ is defined by
         $\D B(\alpha, \beta) = \int_0^1y^{\alpha - 1}(1 - y)^{\beta - 1} dy$.
         \begin{enumerate}
            \item Letting $y = \sin^2\theta$, show that
                  $\D B(\alpha, \beta) = 2\int_0^{\pi/2}\sin^{2\alpha - 1}\theta
                  \cos^{2\beta - 1}\theta\;d\theta$.
            \item Write $\Gamma(\alpha)\Gamma(\beta)$ as a double integral using
                  variables of integration $y$ and $z$, make the transformation
                  $y = r^2\sin^2\theta$ and $z = r^2\cos^2\theta$, and then show 
                  that
                  $$B(\alpha, \beta) = \D\frac{\Gamma(\alpha)\Gamma(\beta)}
                   {\Gamma(\alpha +\beta)}.$$
         \end{enumerate}

      \textbf{Solution.}

      \begin{enumerate}
         \item Let $y = \sin^2\theta$, so that
               $dy = 2\sin\theta\cos\theta\;d\theta$. Thus
               \begin{align*}
                  B(\alpha, \beta)
                     &= \int_0^1y^{\alpha - 1}(1 - y)^{\beta - 1} dy \\
                     &= \int_0^{\pi/2}[(\sin\theta)^2]^{\alpha - 1}
                        (1 - \sin^2\theta)^{\beta - 1}
                        2\sin\theta\cos\theta\;d\theta \\
                     &= \int_0^{\pi/2}(\sin\theta)^{2\alpha - 2}
                        [(\cos\theta)^2]^{\beta - 1}
                        2\sin\theta\cos\theta\;d\theta \\
                     &= 2\int_0^{\pi/2}\sin^{2\alpha - 1}\theta
                     \cos^{2\beta - 1}\theta\;d\theta.                  
               \end{align*}
         \item By definition we have that
               $$\Gamma(\alpha)\Gamma(\beta) = \int_0^\infty y^{\alpha - 1}
                 e^{-y}\;dy\int_0^\infty z^{\beta - 1}e^{-z}\;dz = \int_0^\infty
                 \int_0^\infty y^{\alpha - 1}e^{-(y+z)} z^{\beta - 1}\;dydz.$$
               Consider the transformation $y = r^2\sin^2\theta$ and
               $z = r^2\cos^2\theta$. The Jacobian of this transformation,
               $\D\frac{\partial(y, z)}{\partial(r, \theta)}$, is given by
               $$\left|\begin{tabular}{@{}cc@{}}
                  $\D\frac{\partial y}{\partial r}$ &
                  $\D\frac{\partial y}{\partial\theta}$ \\
                  $\D\frac{\partial z}{\partial r}$ &
                  $\D\frac{\partial z}{\partial\theta}$
               \end{tabular}\right| = -4r^3\sin\theta\cos\theta.$$
               Thus we have that
               \begin{align*}
                  \Gamma(\alpha)\Gamma(\beta) &= \int_0^\infty\int_0^\infty
                     y^{\alpha - 1}e^{-(y+z)} z^{\beta - 1}\;dydz \\
                     &= \int_0^{\pi/2}\int_0^\infty
                        [(r\sin\theta)^2]^{\alpha - 1}e^{-r^2}
                        [(r\cos\theta)^2]^{\beta - 1}\;
                        \left|\frac{\partial(y, z)}
                        {\partial(r, \theta)}\right|drd\theta \\
                     &= \int_0^{\pi/2}\int_0^\infty
                        r^{2\alpha - 2}\sin^{2\alpha - 2}\theta e^{-r^2}
                        r^{2\beta - 2}\cos^{2\beta - 2}\theta\;
                        \left|\frac{\partial(y, z)}
                        {\partial(r, \theta)}\right|drd\theta \\
                     &= \int_0^{\pi/2}\int_0^\infty
                        r^{2\alpha + 2\beta- 4}\sin^{2\alpha - 2}\theta e^{-r^2}
                        \cos^{2\beta - 2}\theta\;
                        (4r^3\sin\theta\cos\theta)\;drd\theta \\
                     &= \int_0^{\pi/2}\int_0^\infty
                        4r^{2\alpha + 2\beta- 1}\sin^{2\alpha -1}\theta e^{-r^2}
                        \cos^{2\beta - 1}\theta\;drd\theta \\
                     &= \left(2\int_0^{\pi/2}\sin^{2\alpha -1}\theta
                        \cos^{2\beta - 1}\theta\;d\theta\right)
                        \left(\int_0^\infty r^{2(\alpha + \beta - 1)}
                        e^{-r^2}2r\;dr\right) \\
                     &= B(\alpha, \beta)\int_0^\infty r^{2(\alpha + \beta - 1)}
                        e^{-r^2}2r\;dr.
               \end{align*}
               Thus using the substitution $x = r^2$ will give us
               $$B(\alpha, \beta)\int_0^\infty r^{2(\alpha + \beta - 1)}
                 e^{-r^2}2r\;dr = B(\alpha, \beta)\int_0^\infty
                 x^{\alpha + \beta - 1} e^{-x}\;dx = B(\alpha, \beta)
                 \Gamma(\alpha + \beta),$$
               so that $\Gamma(\alpha)\Gamma(\beta) = B(\alpha, \beta)
                 \Gamma(\alpha + \beta)$, as desired.
               
      \end{enumerate}
%%%%%%%%%%%%%%%%%%%%%%%%%%%%%%%%%%%%%%%09%%%%%%%%%%%%%%%%%%%%%%%%%%%%%%%%%%%%%%%
   \item Prove that the variance of a beta-distributed random variable with
         parameters $\alpha$ and $\beta$ are given by
         $$\sigma^2 = \frac{\alpha\beta}{(\alpha+\beta)^2(\alpha+\beta+1)}.$$

      \textbf{Proof.} Let $Y$ be a beta-distributed random variable with
      parameters $\alpha$ and $\beta$. We then have that
      \begin{align*}
         E[Y^2] &= \int_0^1y^2\frac{y^{\alpha - 1}(1 - y)^{\beta - 1}}
            {B(\alpha, \beta)} \\
            &= \frac{1}{B(\alpha, \beta)} \int_0^1y^{\alpha + 1}
               (1 - y)^{\beta - 1} \\
            &= \frac{\Gamma(\alpha + \beta)}{\Gamma(\alpha)\Gamma(\beta)} \cdot
               \frac{\Gamma(\alpha + 2)\Gamma(\beta)}
               {\Gamma(\alpha + \beta + 2)}  \\
            &= \frac{\Gamma(\alpha + \beta)}{\Gamma(\alpha)\Gamma(\beta)} \cdot
               \frac{(\alpha+1)\alpha\Gamma(\alpha)\Gamma(\beta)}
               {(\alpha + \beta + 1)(\alpha + \beta)\Gamma(\alpha + \beta)} \\
            &= \frac{(\alpha+1)\alpha}
               {(\alpha + \beta + 1)(\alpha + \beta)}.
      \end{align*}
      By the Proof on Pg 196 of the book we have that
      $\D E[Y] = \frac{\alpha}{\alpha+\beta}$. Thus the variance of $Y$, $V(Y)$,
      is
      \begin{align*}
         E[Y^2] - E[Y]^2 &= \frac{(\alpha+1)\alpha}
            {(\alpha + \beta + 1)(\alpha + \beta)} -
            \frac{\alpha^2}{(\alpha+\beta)^2} \\
            &= \frac{(\alpha+1)(\alpha+\beta)\alpha -
               \alpha^2(\alpha + \beta + 1)}{(\alpha + \beta + 1)
               (\alpha + \beta)^2} \\
            &= \frac{\alpha\beta}{(\alpha + \beta + 1)(\alpha + \beta)^2},
      \end{align*}
      which is what we wanted to prove. \qed
%%%%%%%%%%%%%%%%%%%%%%%%%%%%%%%%%%%%%%%09%%%%%%%%%%%%%%%%%%%%%%%%%%%%%%%%%%%%%%%
   \item Suppose $Y$ has the density function $f(y) = k(y - 2)^4(5 - y)^6$,
         $2 \le y \le 5$. Find $(a)\; k \qquad (b)\; E[Y] \text{ and } V(Y)$.

      \textbf{Solution.}

      \begin{enumerate}
         \item By definition we must have that
               $$k\int_2^5(y - 2)^4(5 - y)^6\;dy = 1.$$
               So make the substitution $x = 5 - y$ to get
               \begin{align*}
                  1 &= k\int_2^5(y - 2)^4(5 - y)^6\;dy \\
                    &= -k\int_3^0(3 - x)^4x^6\;dx \\
                    &= k\int_0^3(3 - x)^4x^6\;dx \\
                    &= k\int_0^3(x^4 - 12x^3 + 54x^2 - 108x + 81)x^6\;dx \\
                    &= k\int_0^3(x^{10} - 12x^9 + 54x^8 - 108x^7 + 81x^6)\;dx \\
                    &= k\left(\frac{1}{11}x^{11} - \frac{6}{5}x^{10} + 6x^9 - 
                       \frac{27}{2}x^8 + \frac{81}{7}x^7\right)\Big|_0^3 \\
                    &= \frac{59049}{770}k,
               \end{align*}
               so that $\D k = \frac{770}{59049}$.
         \item Using the same substitution $x = 5 - y$, it follows that
               \begin{align*}
                  E[Y] &= k\int_2^5y(y - 2)^4(5 - y)^6\;dy \\
                    &= k\int_0^3(5 - x)
                       (x^{10} - 12x^9 + 54x^8 - 108x^7 + 81x^6)\;dx \\
                    &= k\int_0^3(-x^{11} + 17x^{10} - 114x^9 + 378x^8 -
                       621x^7 + 405x^6)\;dx \\
                    &= k\left(-\frac{1}{12}x^{12} + \frac{17}{11}x^{11} -
                       \frac{57}{5}x^{10} + 42x^9 - \frac{621}{8}x^8 +
                       \frac{405}{7}x^7\right)\Big|_0^3 \\
                    &= \frac{770}{59049}\frac{767637}{3080} = \frac{13}{4},
               \end{align*}
               and
               \begin{align*}
                  E[Y^2] &= k\int_2^5y^2(y - 2)^4(5 - y)^6\;dy \\
                    &= k\int_0^3(5 - x)^2
                       (x^{10} - 12x^9 + 54x^8 - 108x^7 + 81x^6)\;dx \\
                    &= k\int_0^3(x^2 - 10x + 25)
                       (x^{10} - 12x^9 + 54x^8 - 108x^7 + 81x^6)\;dx \\
                    &= k\int_0^3(x^{12} - 22x^{11} + 199x^{10} - 948x^9 +
                       2511^8 - 3510x^7 + 2025x^6)\;dx \\
                    &= k\left(\frac{1}{13}x^{13}-\frac{11}{6}x^{12} +
                       \frac{199}{11}x^{11} - \frac{474}{5}x^{10} + 279x^9 - 
                       \frac{1755}{4}x^8 + \frac{2025}{7}x^7\right)\Big|_0^3 \\
                    &= \frac{770}{59049}\frac{49424013}{60060} = \frac{279}{26}.
               \end{align*}

               We can then conclude that
               $$V(Y) = E[Y^2] - E[Y]^2 = \frac{279^2}{26^2} - \frac{13^2}{4^2}
                 = \frac{282803}{2704} \approx 104.59.$$
      \end{enumerate}
\end{enumerate}
\end{document}
