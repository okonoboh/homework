\documentclass[9pt]{article}

\usepackage{amssymb}
\usepackage{amsmath}
\usepackage{amsfonts}
\usepackage{fancyhdr}
\usepackage{mathrsfs}

\voffset = -50pt
%\textheight = 700pt
\addtolength{\textwidth}{60pt}
\addtolength{\evensidemargin}{-30pt}
\addtolength{\oddsidemargin}{-30pt}
%\setlength{\headheight}{44pt}

\pagestyle{fancy}
\fancyhf{} % clear all fields
\fancyhead[R]{%
  \scshape
  \begin{tabular}[t]{@{}r@{}}
  MATH 380, Spring 2015\\Section 4 (10519)\\
  HW \#5, Due: 2015, March 23
  \end{tabular}}
\fancyhead[L]{%
  \scshape
  \begin{tabular}[t]{@{}r@{}}
  JOSEPH OKONOBOH\\Mathematics\\Cal State Long Beach
  \end{tabular}}
\fancyfoot[C]{\thepage}

\newcommand{\qed}{\hfill \ensuremath{\Box}}


\newcommand{\Z}{\mathbb{Z}}
\newcommand{\I}{\mathbb{I}}
\newcommand{\D}{\displaystyle}
%\setcounter{section}{-1}


\begin{document}
\begin{enumerate}
%%%%%%%%%%%%%%%%%%%%%%%%%%%%%%%%%%%%%%%01%%%%%%%%%%%%%%%%%%%%%%%%%%%%%%%%%%%%%%%
   \item Suppose $Y$ is a discrete random variable with probability function
         $p(y) = ky(1/4)^y$, $y = 0, 1, 2, 3, \ldots$. Find

         $(a)\;k \qquad \text{ and } \qquad (b)\;E(Y) \text{ and } V(Y)$.
         
      \textbf{Solution.} Let $p = 1/4$.
      
      \begin{enumerate}
         \item We have that
               \begin{align*}
                  1 &= \sum_{y=0}^\infty kyp^y \\
                    &= \sum_{y=0}^\infty kp\left(\frac{d}{dp}p^y\right) \\
                    &= kp\frac{d}{dp}\sum_{y=0}^\infty p^y \\
                    &= kp\frac{d}{dp}\left(\frac{1}{1 - p}\right) \\
                    &= \frac{kp}{(1 - p)^2}.
               \end{align*}
               It follows that $\D k = \frac{(1 - p)^2}{p} = \frac{9}{4}$.
         \item  We have that
               \begin{align*}
                  E(Y) &= \sum_{y=0}^\infty ky^2p^y \\
                       &= \sum_{y=0}^\infty ky^2p^y - \sum_{y=0}^\infty kyp^y
                          + \sum_{y=0}^\infty kyp^y    \\
                       &= \sum_{y=0}^\infty k(y^2-y)p^y +
                          \sum_{y=0}^\infty kyp^y    \\
                       &= \sum_{y=0}^\infty kp^2\left(\frac{d^2}{dp^2}p^y\right)
                          + \sum_{y=0}^\infty kyp^y \\
                       &= kp^2\frac{d^2}{dp^2}\sum_{y=0}^\infty p^y + 1 \\
                       &= kp^2\frac{d^2}{dp^2}\left(\frac{1}{1 - p}\right) + 1\\
                       &= \frac{2kp^2}{(1-p)^3} + 1 \\
                       &= \frac{5}{3},
               \end{align*}
               and
               \begin{align*}
                  E(Y^2) &= \sum_{y=0}^\infty ky^3p^y \\
                       &= \sum_{y=0}^\infty ky^3p^y - 3\sum_{y=0}^\infty ky^2p^y
                          + 2\sum_{y=0}^\infty kyp^y
                          + 3\sum_{y=0}^\infty ky^2p^y
                          - 2\sum_{y=0}^\infty kyp^y   \\
                       &= \sum_{y=0}^\infty k(y^3-3y^2+2y)p^y + 3E(Y) - 2 \\
                       &= \sum_{y=0}^\infty ky(y-1)(y-2)p^y + 3 \\
                       &= \sum_{y=0}^\infty kp^3\left(\frac{d^3}{dp^3}p^y\right)
                          + 3 \\
                       &= kp^3\frac{d^3}{dp^3}\sum_{y=0}^\infty p^y + 3 \\
                       &= \frac{6kp^3}{(1 - p)^4} + 3.
               \end{align*}
               
               Since $V(Y) = E(Y^2) - E(Y)^2$, it follows that
               \begin{align*}
                  V(Y) &= E(Y^2) - E(Y)^2 \\
                       &= \frac{6kp^3}{(1 - p)^4} + 3 - \frac{25}{9} \\
                       &= \frac{8}{9}.
               \end{align*}
      \end{enumerate}
%%%%%%%%%%%%%%%%%%%%%%%%%%%%%%%%%%%%%%%02%%%%%%%%%%%%%%%%%%%%%%%%%%%%%%%%%%%%%%%
   \item Verify the identity $\D\binom{n}{k} = \frac{n}{k}\binom{n-1}{k-1}$ and
         use it to show that $E[Y^k] = npE[(X+1)^{k-1}]$ where $Y$ is a binomial
         random variable with parameters $n$ and $p$ and $X$ is a binomial
         random variable with parameters $n - 1$ and $p$.

      \textbf{Proof.} We have that
      \begin{align*}
         \binom{n}{k} &= \frac{n!}{k!(n - k)!} \\
            &= \frac{n(n - 1)!}{k(k - 1)!(n - k)!} \\
            &= \frac{n}{k}\frac{(n - 1)!}{(k - 1)!(n - k)!} \\
            &= \frac{n}{k}\frac{(n - 1)!}{(k - 1)!((n - 1) - (k - 1))!} \\
            &= \frac{n}{k}\binom{n - 1}{k - 1}.
      \end{align*}

      Now
      \begin{align*}
         E[Y^k] &= \sum_{y=0}^ny^kp(y) &[\text{Definition}] \\
                &= \sum_{y=0}^ny^k\binom{n}{y}p^y(1 - p)^{n - y} \\
                &= \sum_{y=1}^ny^k\binom{n}{y}p^y(1 - p)^{n - y} \\
                &= \sum_{y=1}^ny^{k-1}n\binom{n - 1}{y - 1}p^y(1 - p)^{n - y} \\
                &= np\sum_{y=1}^ny^{k-1}\binom{n - 1}{y - 1}
                   p^{y-1}(1 - p)^{n - y} \\
                &= np\sum_{x=0}^{n-1}(x + 1)^{k-1}\binom{n - 1}{x}
                   p^{x}(1 - p)^{(n - 1)-x} &[\text{Let }y = x + 1] \\
                &= np\sum_{x=0}^{n-1}(x + 1)^{k-1}p(x) \\
                &= npE[(X + 1)^{k-1}],
      \end{align*}
      which is what we wanted to show. \qed
%%%%%%%%%%%%%%%%%%%%%%%%%%%%%%%%%%%%%%%03%%%%%%%%%%%%%%%%%%%%%%%%%%%%%%%%%%%%%%%
   \item Using the recursion relation found in problem 2 for the binomial random
         variable with parameters $n$ and $p$, find $E[Y^2]$ and then $V(Y)$.

      \textbf{Solution.} If we set $k = 2$ in the formula in problem 2, we get
      \begin{align*}
         E[Y^2] &= npE[(X + 1)] \\
                &= np(E[X] + E[1]) \\
                &= np((n-1)p + 1) \\
                &= (np)^2 - np^2 + np.
      \end{align*}
      Thus
      \begin{align*}
         V(Y) &= E[Y^2] - E[Y]^2 \\
              &= (np)^2 - np^2 + np - (np)^2 \\
              &= np - np^2 \\
              &= np(1 - p).
      \end{align*}
%%%%%%%%%%%%%%%%%%%%%%%%%%%%%%%%%%%%%%%04%%%%%%%%%%%%%%%%%%%%%%%%%%%%%%%%%%%%%%%
   \item Using the identity from problem 2, show that
         \begin{enumerate}
            \item if $Y$ is a negative binomial random variable with parameters
                  $r$ and $p$, then $E[Y^k] = \frac{r}{p}E[(X-1)^{k-1}]$, where
                  $X$ is a negative binomial random variable with parameters
                  $r + 1$ and $p$.
            \item Use the relation in (a) to find $E[Y]$ and $V(Y)$.
         \end{enumerate}
\end{enumerate}
\end{document}
