\documentclass[9pt]{article}

\usepackage{amssymb}
\usepackage{amsmath}
\usepackage{amsfonts}
\usepackage{fancyhdr}
\usepackage{mathrsfs}

\voffset = -50pt
%\textheight = 700pt
\addtolength{\textwidth}{60pt}
\addtolength{\evensidemargin}{-30pt}
\addtolength{\oddsidemargin}{-30pt}
%\setlength{\headheight}{44pt}

\pagestyle{fancy}
\fancyhf{} % clear all fields
\fancyhead[R]{%
  \scshape
  \begin{tabular}[t]{@{}r@{}}
  MATH 380, Spring 2015\\Section 4 (10519)\\
  HW \#5, Due: 2015, March 23
  \end{tabular}}
\fancyhead[L]{%
  \scshape
  \begin{tabular}[t]{@{}r@{}}
  JOSEPH OKONOBOH\\Mathematics\\Cal State Long Beach
  \end{tabular}}
\fancyfoot[C]{\thepage}

\newcommand{\qed}{\hfill \ensuremath{\Box}}


\newcommand{\Z}{\mathbb{Z}}
\newcommand{\I}{\mathbb{I}}
\newcommand{\D}{\displaystyle}
%\setcounter{section}{-1}


\begin{document}
\begin{enumerate}
%%%%%%%%%%%%%%%%%%%%%%%%%%%%%%%%%%%%%%%01%%%%%%%%%%%%%%%%%%%%%%%%%%%%%%%%%%%%%%%
   \item Suppose $Y$ is a discrete random variable with probability function
         $p(y) = ky(1/4)^y$, $y = 0, 1, 2, 3, \ldots$. Find

         $(a)\;k \qquad \text{ and } \qquad (b)\;E(Y) \text{ and } V(Y)$.
         
      \textbf{Solution.} Let $p = 1/4$.
      
      \begin{enumerate}
         \item We have that
               \begin{align*}
                  1 &= \sum_{y=0}^\infty kyp^y \\
                    &= \sum_{y=0}^\infty kp\left(\frac{d}{dp}p^y\right) \\
                    &= kp\frac{d}{dp}\sum_{y=0}^\infty p^y \\
                    &= kp\frac{d}{dp}\left(\frac{1}{1 - p}\right) \\
                    &= \frac{kp}{(1 - p)^2}.
               \end{align*}
               It follows that $\D k = \frac{(1 - p)^2}{p} = \frac{9}{4}$.
         \item  We have that
               \begin{align*}
                  E(Y) &= \sum_{y=0}^\infty ky^2p^y \\
                       &= \sum_{y=0}^\infty ky^2p^y - \sum_{y=0}^\infty kyp^y
                          + \sum_{y=0}^\infty kyp^y    \\
                       &= \sum_{y=0}^\infty k(y^2-y)p^y +
                          \sum_{y=0}^\infty kyp^y    \\
                       &= \sum_{y=0}^\infty kp^2\left(\frac{d^2}{dp^2}p^y\right)
                          + \sum_{y=0}^\infty kyp^y \\
                       &= kp^2\frac{d^2}{dp^2}\sum_{y=0}^\infty p^y + 1 \\
                       &= kp^2\frac{d^2}{dp^2}\left(\frac{1}{1 - p}\right) + 1\\
                       &= \frac{2kp^2}{(1-p)^3} + 1 \\
                       &= \frac{5}{3},
               \end{align*}
               and
               \begin{align*}
                  E(Y^2) &= \sum_{y=0}^\infty ky^3p^y \\
                       &= \sum_{y=0}^\infty ky^3p^y - 3\sum_{y=0}^\infty ky^2p^y
                          + 2\sum_{y=0}^\infty kyp^y
                          + 3\sum_{y=0}^\infty ky^2p^y
                          - 2\sum_{y=0}^\infty kyp^y   \\
                       &= \sum_{y=0}^\infty k(y^3-3y^2+2y)p^y + 3E(Y) - 2 \\
                       &= \sum_{y=0}^\infty ky(y-1)(y-2)p^y + 3 \\
                       &= \sum_{y=0}^\infty kp^3\left(\frac{d^3}{dp^3}p^y\right)
                          + 3 \\
                       &= kp^3\frac{d^3}{dp^3}\sum_{y=0}^\infty p^y + 3 \\
                       &= \frac{6kp^3}{(1 - p)^4} + 3.
               \end{align*}
               
               Since $V(Y) = E(Y^2) - E(Y)^2$, it follows that
               \begin{align*}
                  V(Y) &= E(Y^2) - E(Y)^2 \\
                       &= \frac{6kp^3}{(1 - p)^4} + 3 - \frac{25}{9} \\
                       &= \frac{8}{9}.
               \end{align*}
      \end{enumerate}
%%%%%%%%%%%%%%%%%%%%%%%%%%%%%%%%%%%%%%%02%%%%%%%%%%%%%%%%%%%%%%%%%%%%%%%%%%%%%%%
   \item Verify the identity $\D\binom{n}{k} = \frac{n}{k}\binom{n-1}{k-1}$ and
         use it to show that $E[Y^k] = npE[(X+1)^{k-1}]$ where $Y$ is a binomial
         random variable with parameters $n$ and $p$ and $X$ is a binomial
         random variable with parameters $n - 1$ and $p$.

      \textbf{Proof.} We have that
      \begin{align*}
         \binom{n}{k} &= \frac{n!}{k!(n - k)!} \\
            &= \frac{n(n - 1)!}{k(k - 1)!(n - k)!} \\
            &= \frac{n}{k}\frac{(n - 1)!}{(k - 1)!(n - k)!} \\
            &= \frac{n}{k}\frac{(n - 1)!}{(k - 1)!((n - 1) - (k - 1))!} \\
            &= \frac{n}{k}\binom{n - 1}{k - 1}.
      \end{align*}

      Now
      \begin{align*}
         E[Y^k] &= \sum_{y=0}^ny^kp(y) &[\text{Definition}] \\
                &= \sum_{y=0}^ny^k\binom{n}{y}p^y(1 - p)^{n - y} \\
                &= \sum_{y=1}^ny^k\binom{n}{y}p^y(1 - p)^{n - y} \\
                &= \sum_{y=1}^ny^{k-1}n\binom{n - 1}{y - 1}p^y(1 - p)^{n - y} \\
                &= np\sum_{y=1}^ny^{k-1}\binom{n - 1}{y - 1}
                   p^{y-1}(1 - p)^{n - y} \\
                &= np\sum_{x=0}^{n-1}(x + 1)^{k-1}\binom{n - 1}{x}
                   p^{x}(1 - p)^{(n - 1)-x} &[\text{Let }y = x + 1] \\
                &= np\sum_{x=0}^{n-1}(x + 1)^{k-1}p(x) \\
                &= npE[(X + 1)^{k-1}],
      \end{align*}
      which is what we wanted to show. \qed
%%%%%%%%%%%%%%%%%%%%%%%%%%%%%%%%%%%%%%%03%%%%%%%%%%%%%%%%%%%%%%%%%%%%%%%%%%%%%%%
   \item Using the recursion relation found in problem 2 for the binomial random
         variable with parameters $n$ and $p$, find $E[Y^2]$ and then $V(Y)$.

      \textbf{Solution.} If we set $k = 2$ in the formula in problem 2, we get
      \begin{align*}
         E[Y^2] &= npE[X + 1] \\
                &= np(E[X] + E[1]) \\
                &= np((n-1)p + 1) \\
                &= (np)^2 - np^2 + np.
      \end{align*}
      Thus
      \begin{align*}
         V(Y) &= E[Y^2] - E[Y]^2 \\
              &= (np)^2 - np^2 + np - (np)^2 \\
              &= np - np^2 \\
              &= np(1 - p).
      \end{align*}
%%%%%%%%%%%%%%%%%%%%%%%%%%%%%%%%%%%%%%%04%%%%%%%%%%%%%%%%%%%%%%%%%%%%%%%%%%%%%%%
   \item Using the identity from problem 2, show that
         \begin{enumerate}
            \item if $Y$ is a negative binomial random variable with parameters
                  $r$ and $p$, then
                  $$E[Y^k] = \frac{r}{p}E[(X-1)^{k-1}],$$
                  where $X$ is a negative binomial random variable with 
                  parameters $r + 1$ and $p$.
            \item Use the relation in (a) to find $E[Y]$ and $V(Y)$.
         \end{enumerate}

      \textbf{Solution.}

      \begin{enumerate}
         \item From problem 2, we know that 
               $$\frac{1}{r}\binom{y-1}{r-1} = \frac{1}{y}\binom{y}{r};$$
               thus
               \begin{align*}
                  E[Y^k] &= \sum_{y=r}^\infty y^kp(y) \\
                         &= \sum_{y=r}^\infty y^k \binom{y-1}{r-1}
                            p^r(1-p)^{y-r} \\
                         &= \frac{r}{p}\frac{p}{r}\sum_{y=r}^\infty
                            y^k \binom{y-1}{r-1} p^r(1-p)^{y-r} \\
                         &= \frac{r}{p}\sum_{y=r}^\infty
                            y^k\frac{1}{r}\binom{y-1}{r-1} p^{r+1}(1-p)^{y-r} \\
                         &= \frac{r}{p}\sum_{y=r}^\infty
                            y^k\frac{1}{y}\binom{y}{r} p^{r+1}(1-p)^{y-r} \\
                         &= \frac{r}{p}\sum_{y=r}^\infty
                            y^{k-1}\binom{y}{r} p^{r+1}(1-p)^{y-r} \\
                         &= \frac{r}{p}\sum_{x=r+1}^\infty
                            (x-1)^{k-1}\binom{x-1}{r} p^{r+1}(1-p)^{(x-1)-r} \\
                         &= \frac{r}{p}\sum_{x=r+1}^\infty (x-1)^{k-1}p(x) \\
                         &= \frac{r}{p}E[(X-1)^{k-1}]. \\
               \end{align*}
         \item If we set $k = 1$ in (a), we immediately get that
               $\D E[Y] = \frac{r}{p}$. Now
               \begin{align*}
                  V(Y) &= E[Y^2] - E[Y]^2 \\
                       &= \frac{r}{p}E[X-1] - \frac{r^2}{p^2} \\
                       &= \frac{r}{p}(E[X]- E[1]) - \frac{r^2}{p^2} \\
                       &= \frac{r}{p}\left(\frac{r + 1}{p}- 1\right) -
                          \frac{r^2}{p^2} \\
                       &= \frac{r^2 + r}{p^2}- \frac{r}{p} - \frac{r^2}{p^2} \\
                       &= \frac{r - pr}{p^2} \\
                       &= \frac{r(1 - p)}{p^2}.
               \end{align*}
      \end{enumerate}
%%%%%%%%%%%%%%%%%%%%%%%%%%%%%%%%%%%%%%%05%%%%%%%%%%%%%%%%%%%%%%%%%%%%%%%%%%%%%%%
   \item Using the identity from problem 2, show that if $Y$ is a hypergeometric
         random variable with parameters $N$, $r$, amd $n$, then
         $$E[Y^k] = \frac{nr}{N}E[(X+1)^{k-1}],$$
         where $X$ is a hypergeometric random variable with parameters $N - 1$,
         $r - 1$, and $n - 1$.

      \textbf{Proof.} We have that
      \begin{align*}
         E[Y^k] &= \sum_{y=0}^ny^kp(y) \\
                &= \sum_{y=0}^ny^k \frac{\D\binom{r}{y}\binom{N-r}{n-y}}
                                        {\D\binom{N}{n}} \\
                &= \sum_{y=1}^ny^k \frac{\D\binom{r}{y}\binom{N-r}{n-y}}
                                        {\D\binom{N}{n}} \\
                &= \sum_{y=1}^ny^k \frac{\D\frac{r}{y}\binom{r-1}{y-1}
                   \binom{N-r}{n-y}}{\D\frac{N}{n}\binom{N-1}{n-1}} \\
                &= \frac{nr}{N}\sum_{y=1}^ny^{k-1} \frac{\D\binom{r-1}{y-1}
                   \binom{N-r}{n-y}}{\D\binom{N-1}{n-1}} \\
                &= \frac{nr}{N}\sum_{x=0}^{n-1}(x+1)^{k-1}\frac{\D\binom{r-1}{x}
                   \binom{(N-1)-(r-1)}{(n-1)-x)}}{\D\binom{N-1}{n-1}}
                   &[\text{Let } y = x + 1] \\
                &= \frac{nr}{N}\sum_{x=0}^{n-1}(x+1)^{k-1}p(x) \\
                &= \frac{nr}{N}E[(X+1)^{k-1}],
      \end{align*}
      which is what we wanted to prove. \qed
%%%%%%%%%%%%%%%%%%%%%%%%%%%%%%%%%%%%%%%06%%%%%%%%%%%%%%%%%%%%%%%%%%%%%%%%%%%%%%%
   \item If $Y$ is a hypergeometric random variable with parameters $N$, $r$,
         and $n$, use the recursion relation found in problem 5 to find $E[Y]$
         and $V(Y)$.

         \textbf{Solution.} Plugging in $k = 1$ in the formula from problem 5
         immediately shows us that $\D E[Y] = \frac{nr}{N}$. Using this same
         formula, we have that
         \begin{align*}
            V(Y) &= E[Y^2] - E[Y]^2 \\
                 &= \frac{nr}{N}E[X+1] - \left(\frac{nr}{N}\right)^2 \\
                 &= \frac{nr}{N}(E[X]+E[1]) - \left(\frac{nr}{N}\right)^2 \\
                 &= \frac{nr}{N}\left(\frac{(n-1)(r-1)}{N-1}+1\right) -
                    \left(\frac{nr}{N}\right)^2 \\
                 &= \frac{nr}{N}\left(\frac{(n-1)(r-1)+N-1}{N-1}\right) -
                    \left(\frac{nr}{N}\right)^2 \\
                 &= \frac{nr}{N}\left(\frac{(n-1)(r-1)+N-1}{N-1}-
                    \frac{nr}{N}\right) \\
                 &= \frac{nr}{N}\left(
                    \frac{N(n-1)(r-1)+N^2-N-nr(N-1)}{N(N-1)}\right) \\
                 &= \frac{nr}{N}\left(
                    \frac{Nnr-Nn-Nr+N+N^2-N-Nnr+nr}{N(N-1)}\right) \\
                 &= \frac{nr}{N}\left(\frac{-Nn-Nr+N^2+nr}{N(N-1)}\right) \\
                 &= \frac{nr}{N}\left(\frac{N^2-Nr-Nn+nr}{N(N-1)}\right) \\
                 &= \frac{nr}{N}\left(\frac{N(N-r)-n(N-r)}{N(N-1)}\right) \\
                 &= \frac{nr}{N}\left(\frac{N-r}{N}\right)
                    \left(\frac{N-n}{N-1}\right).
         \end{align*}
%%%%%%%%%%%%%%%%%%%%%%%%%%%%%%%%%%%%%%%07%%%%%%%%%%%%%%%%%%%%%%%%%%%%%%%%%%%%%%%
   \item The number of chocolate chips in 1 cup of chocolate chip ice cream has
         a Poisson distribution with a mean of 10 chips per cup.
         \begin{enumerate}
            \item What is the probability that a cup of chocolate chip ice
                  cream has 9 chocolate chips.
            \item What is the probability that a half cup of chocolate chip ice
                  cream has at least 3 chocolate chips?
         \end{enumerate}

      \textbf{Solution.}

      \begin{enumerate}
         \item We have that $\lambda = 10$, so that
               $$P(Y = 9) = p(9) = \frac{10^9}{9!}e^{-10} \approx 0.12511.$$
         \item Now $\lambda = 5$, so that
               \begin{align*}
                  P(Y \ge 3) &= 1 - P(Y < 3) \\
                             &= 1 - p(0) - p(1) - p(2) \\
                             &= 1 - e^{-5} - 5e^{-5} - \frac{5^2}{2}e^{-5} \\
                             &\approx 0.875348.
               \end{align*} 
      \end{enumerate}
%%%%%%%%%%%%%%%%%%%%%%%%%%%%%%%%%%%%%%%08%%%%%%%%%%%%%%%%%%%%%%%%%%%%%%%%%%%%%%%
   \item Suppose the distribution function of $Y$ is given by
         \begin{equation*}
            F(y) = \begin{cases}
               0   & \text{if } y < 0 \\
               y^3 & \text{if } 0 \le y \le 1 \\
               1   & \text{if } y > 1.
            \end{cases}
         \end{equation*}
         Find (a) $\D P\left(Y > \frac{3}{4}\right)$ \qquad (b) $E[Y]$ \qquad 
                  and \qquad (c) $V(Y)$.

      \textbf{Solution.}

      \begin{enumerate}
         \item We have that
               \begin{align*}
                  P\left(Y > \frac{3}{4}\right) &=
                     1 - P\left(Y \le \frac{3}{4}\right) \\
                     &= 1 - F\left(\frac{3}{4}\right) \\
                     &= 0.578125.
               \end{align*}
         \item The probability density function, $f(y)$, is
               \begin{equation*}
                  \frac{dF(y)}{dy} = \begin{cases}
                     0   & \text{if } y < 0 \\
                     3y^2 & \text{if } 0 \le y < 1 \\
                     0   & \text{if } y > 1,
                  \end{cases}
               \end{equation*}
               so that
               \begin{align*}
                  E[Y] &= \int_{-\infty}^{\infty} y f(y) dy \\
                     &= \int_0^1 y f(y) dy \\
                     &= \int_0^1 3y^3 dy = 0.75.
               \end{align*}
         \item By definition
               \begin{align*}
                  V(Y) &= E[Y^2] - E[Y]^2 \\
                     &= \int_{-\infty}^{\infty} y^2f(y) dy - 0.75^2 \\
                     &= \int_0^1 3y^4 dy - 0.75^2 \\
                     &= 0.0375.
               \end{align*}               
      \end{enumerate}
%%%%%%%%%%%%%%%%%%%%%%%%%%%%%%%%%%%%%%%09%%%%%%%%%%%%%%%%%%%%%%%%%%%%%%%%%%%%%%%
   \item Let $Y$ be a continuous random variable with density function
         $$f(y) = \frac{k}{1 + y^2}, \infty < y < \infty.$$
         Find (a) $k$ \qquad (b) $E[Y]$ \qquad and $V(Y)$, if they exist. (Such 
         a distribution is called a Cauchy distribution.)

      \textbf{Solution.}

      \begin{enumerate}
         \item We must have that
               \begin{align*}
                  1 &= \int_{-\infty}^{\infty} f(y) dy \\
                    &= \int_{-\infty}^{\infty} \frac{k}{1 + y^2} dy \\
                    &= \int_{-\infty}^a \frac{k}{1 + y^2} dy +
                       \int_a^\infty \frac{k}{1 + y^2} dy \\
                    &= \lim_{t\rightarrow-\infty}\int_t^a \frac{k}{1 + y^2} dy +
                       \lim_{s\rightarrow\infty}\int_a^s \frac{k}{1 + y^2} dy \\
                    &= k\lim_{t\rightarrow-\infty}(
                       \text{arctan}(a) - \text{arctan}(t)) +
                       k\lim_{s\rightarrow\infty}(
                       \text{arctan}(s) - \text{arctan}(a)) \\
                    &= k\left(\text{arctan}(a) + \frac{\pi}{2}\right) +
                       k\left(\frac{\pi}{2} - \text{arctan}(a)\right) \\
                    &= k\pi,
               \end{align*}
               so that $\D k = \frac{1}{\pi}$.
         \item \begin{align*}
                  E[Y] &= \int_{-\infty}^{\infty} y f(y) dy \\
                    &= \int_{-\infty}^{\infty} \frac{ky}{1 + y^2} dy \\
                    &= \lim_{t\rightarrow-\infty}\int_t^a
                       \frac{ky}{1 + y^2} dy +
                       \lim_{s\rightarrow\infty}\int_a^s \frac{ky}{1 + y^2} dy\\
                    &= \frac{k}{2}\lim_{t\rightarrow-\infty}(
                       \text{ln}(1 + a^2) - \text{ln}(1+t^2)) +
                       \frac{k}{2}\lim_{s\rightarrow\infty}(
                       \text{ln}(1+s^2) - \text{ln}(1+a^2)) \\
                    &= \text{Does Not Exist}.
               \end{align*}

               Since $E[Y]$ does not exist, it follows that $V(Y)$ does not
               exist.
      \end{enumerate}
   
\end{enumerate}
\end{document}
